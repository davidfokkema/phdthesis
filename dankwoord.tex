\chapter{Dankwoord}
\markboth{Dankwoord}{}

\begin{otherlanguage}{dutch}

Het was een lange weg.  Er zijn niet veel promovendi die hun werk in
deeltijd doen.  Het blijkt, na onderzoek, dat er ook niet veel promovendi
op tijd klaar zijn.  De afgelopen vijf jaar (en daarvoor) heb ik voldoende
mensen ontmoet die uiteindelijk mijn proefschrift mede hebben gevormd.

Allereerst wil ik Gerrit bedanken, die mij, hoewel ik in eerste instantie
koos voor het onderwijs, aangemoedigd heeft verder te blijven kijken en
het onderzoek niet de rug toe te keren.  Je belde mij zodra je de
advertentie onder ogen had gekregen (ik had hem ook al gezien) en je
moedigde me aan contact op te nemen met Bob.  Bob was gek genoeg om een
aanstaande vader aan te nemen die a) ruim drie jaar eerder was
afgestudeerd en in eerste instantie niet had gekozen voor onderzoek, b)
niet meteen kon beginnen vanwege lopende examenklassen en c) enkel in
deeltijd wilde werken.  Bob, ik ben blij dat we er uit gekomen zijn en ik
heb de afgelopen jaren mogen zien hoe je je vol vuur inzette voor
\hisparc.  Gepassioneerd vertel je over natuurkunde, of het nu mij,
studenten, LIO's of VWO-leerlingen betreft.  Je ziet altijd mogelijkheden
en `potjes' en zodoende is \hisparc de afgelopen jaren flink gegroeid.
Als man van het eerste uur is je kennis van \hisparc ongekend.  Charles,
jij was er zelfs nóg eerder bij als mede-oprichter van \textsmaller{NAHSA},
de voorloper van \hisparc.  Ik wil je danken voor je hulp bij het leggen
van de contacten met \kascade en het opzetten van de eerste analyse.  Ook
wil ik je danken voor de nauwgezetheid waarmee je mijn manuscript hebt
gelezen.

I'd like to thank Andreas Haungs and Harald Schieler for their hospitality
and help in setting up the \hisparc station at \kascade, rebooting our
computer from time to time (damn those Windows updates) and keeping the
lawnmowers off the grass lest they be cutting up our cables.  Harald,
thank you for driving me through awful weather and rush hour to catch my
train, because the last bus leaves early on Fridays.  Vitor de Souza and
Jürgen Wochele, thank you for providing us with the data needed for the
analysis in \chref{ch:kascade}.

Jos, hoe kan ik je ooit danken voor de hulp, sturing, het lezen van alle
versies van pamfletjes en hoofdstukken, goede ideeën, inzichten,
weekendmailtjes, bemoedigende woorden en al die andere dingen.  Hoe je er
telkens weer in slaagt om op maandagochtend een pamfletje te produceren
dat een uitgewerkte mini-Monte Carlo beschrijft met figuren en grafieken,
en dat op vrijdag nog niet bestond, zal voor mij wel altijd een raadsel
blijven.  Ik heb ontzettend veel aan onze gesprekken gehad!  De
resultaten van jouw pamfletjes worden her en der in dit proefschrift
gebruikt.

Ook veel anderen die betrokken zijn bij \hisparc ben ik veel dank
verschuldigd. Jan-Willem, dank je voor het lezen van mijn manuscript en je
opmerkingen.  Hans, dank je voor je tijd wanneer ik weer eens vragen had
over de elektronica.  Ik wil alle leraren-in-onderzoek bedanken voor hun
inzet en voor de fijne samenwerking.  In het bijzonder wil ik noemen
Dorrith en Niels.  Dorrith, jouw vasthoudendheid in het leren van Python,
de werking van de detectoren en de onderliggende natuurkunde heb ik altijd
bewonderd.  Je bent een onzettend harde werker en ik heb twee jaar op de
maandagen van jouw gezelschap mogen genieten.  Je hebt veel bereikt in die
twee jaar en jouw resultaten hebben ook in dit proefschrift een plek
gevonden in Hoofdstuk \ref{ch:hisparc-experiment} en Hoofdstuk
\ref{ch:kascade}.  Niels, ook jij hebt je twee jaar flink ingezet voor
\hisparc door het schrijven van simulaties en het uitvoeren van analyses
op de data van het Science Park cluster.  We hebben heel wat code
uitgewisseld en soms gingen we dezelfde richting op, om daarna weer onze
eigen weg in te slaan.  Het was fijn om met je samen te werken!  Jouw
resultaten hebben ook een plek gevonden, en wel in Hoofdstukken
\ref{ch:hisparc-experiment}, \ref{ch:sciencepark} en \ref{ch:conclusions}.
Niek, ook jij bent een trouwe kamergenoot en ik waardeer je inzet om de
\hisparc data dicht bij de leerlingen te brengen door middel van
interactieve analyses in een web browser.  De overige LiO's hebben ook
veel werk verzet voor \hisparc en de afgelopen jaren heb ik verder nog
samen mogen werken met Machiel, Arjan, Wim, Guus, Hans, Remon, Henk,
Daniël, Paul, Wytse en Sjoerd, bedankt voor jullie werk!  Ik heb met nog
veel meer mensen samengewerkt die als student of in een andere
hoedanigheid betrokken zijn of waren bij \hisparc.  In het bijzonder wil
ik nog noemen Aart-Jan (lesmateriaal muonbalken) en natuurlijk Loran
(weerdata, scripts voor leerlingen).  Arne, jij neemt het stokje van mij
over.  In een jaar tijd heb je laten zien dat je niet alleen Python onder
de knie hebt, maar dat je een grotere PEP-8 extremist bent dan ik (hou
vol!).  

Een groot deel van mijn eerste jaren bestond uit het helpen opzetten van
een nieuw softwaresysteem voor \hisparc.  Dan kan ik natuurlijk niet
voorbijgaan aan Tristan en Karel.  Zonder jullie werk gingen we nog elke
dag op de fiets een paar scholen langs om pc's af te regelen.  Jeroen,
onze \emph{resident LabVIEW expert}, heeft over een langere periode aan de
\daq software gewerkt dan ik heb gedaan over mijn promotie.  Bedankt!  Het
OOTI team uit Eindhoven, een groep post-doctorale Master studenten van de
TU/E heeft het beheer van de Windows software sterk vereenvoudigd.  Nog
vele andere studenten hebben hun steen bijgedragen.

Dan de CT-groep: zonder jullie zat \hisparc allang aan de grond.  Dank aan
Robert voor de overname van het beheer van de Windows software en het
vernieuwen van de installer.  Bart, dank voor je werk aan de publieke
database en betere wifi in H343c.  André, jij hebt veel van onze servers
geïnstalleerd en was, vóór de helpdesk-dienst bestond, ons eerste
aanspreekpunt.  Ton, als UN*X administrator heb je niet alleen onze
servers gereanimeerd wanneer dat nodig bleek, maar heb je met veel geduld
onze hardware crashes opgelost.  Schijven, SCSI interfaces, geheugen, we
hebben het geloof ik allemaal wel zien stukgaan.  Je creatieve constructie
via twee servers houdt op dit moment onze RAID al een jaar in de lucht.
Dank!  Wim, bedankt dat je altijd weer iemand wist die voor ons aan de
slag kon.

Ik wil mijn oud-en-nieuw collega's op het Kaj Munk bedanken voor hun
afscheid (vijf jaar terug), aanhoudende interesse, gevoel van welkom
wanneer ik een keer op school was, het rotsvaste vertrouwen (in ieder
geval van sommigen) dat ik geen leukere baan zou kunnen vinden om de
simpele reden dat die niet bestaat, en het warme welkom toen dat het geval
bleek en ik graag weer terugkwam op mijn oude stekkie.  Albert, dank voor
de inspiratie tijdens mijn eigen schooljaren (uiteindelijk mede
resulterend in dit proefschrift) en dat ik nu (weer) jouw collega mag zijn
om te leren hoe je die perfectie van goed voorbereide lessen bereikt.

Mijn vrienden en familie hebben de afgelopen vijf jaar behoorlijk last van
me gehad.  Nooit tijd hebbend en altijd druk, werd mijn wereldje tamelijk
klein: werk en gezin(s-uitbreiding x2).  Jullie steun en vertrouwen is
voor mij al die jaren ontzettend belangrijk geweest.  Ivo, ooit brachten
wij uren per dag in elkaars gezelschap door, nu is dat anders.  Maar je
bent altijd mijn broer gebleven en staat nog steeds voor mij klaar!  Dank
je dat je mijn paranimf wilt zijn.  Bart, de nieuwste aanwinst in de
familie (`eindelijk' heb je Karen ten huwelijk gevraagd!), ik kan het goed
met je vinden en ik ben blij dat ook jij mijn paranimf wilt zijn.

Esther en Hannah, jullie hele leven heb ik aan mijn proefschrift gewerkt.
En jullie zitten allebei al op school!  Twee grote meiden.  Ik ben trots
op jullie!  Ik ben blij dat ik niets van jullie heb hoeven missen.  Twee
dagen per week was ik helemaal alleen met jullie thuis en de flessen
werden pap en fruithapjes, en boterhammen en nu gewoon wat de pot schaft.
Jullie passen niet meer in het kommetje van mijn arm, maar jullie
knuffelen nog steeds even fijn en niets is beter tegen de stress van
alledag dan jullie boekjes voorlezen of samen met de Duplo-trein te
spelen.  Ik hou van jullie!

Papa en mama, jullie hebben mij altijd het gevoel gegeven dat ik bijzonder
was en hebben nooit geaarzeld al mijn vragen te beantwoorden (en dat waren
er nogal wat).  Ik herinner me het verhaal dat jullie ons vergeleken met
een melkfles (met smalle hals).  Giet al die antwoorden er maar in.  Op
jonge leeftijd gaat het grootste deel er naast, maar het deel dat er wél
in komt is mooi meegenomen.  En dus is het beter om al die vragen maar te
beantwoorden (op het vermoeiende af) dan op jonge leeftijd het vragen al
af te leren.  Het is gelukt, en ik ben mijn hele leven vragen blijven
stellen.  Op een aantal heb ik een antwoord gevonden.

\vfill
\renewcommand{\epigraphflush}{flushright}
\epigraph{Voici mon secret. Il est très simple: on ne voit bien qu'avec le
cœur. L'essentiel est invisible pour les yeux.\\[1em]
\textit{Hier is mijn geheim.  Het is zeer eenvoudig: je kunt slechts
werkelijk goed zien met het hart.  Het wezenlijke is onzichtbaar voor het
oog.}}
{Antoine de Saint Exupéry\\
\textit{Le Petit Prince}}

\noindent Sommige mensen zijn nu eenmaal voor elkaar geschapen.  Jij,
Eveline, bent voor mij geschapen.  Je brengt orde in mijn chaos, bent een
onuitputtelijke bron van liefde en steun, en hebt er letterlijk voor
gezorgd dat ik de afgelopen jaren heb kunnen volbrengen.  Er is niemand
die mij beter kent en je neemt mijn zwakheden voor lief.  De verwachtingen
die je van mij hebt maken me een beter mens.  Ik ben blij dat we de zorg
voor onze meiden gelijk verdeelden en dat ik zodoende vijf jaar lang vier
dagen per week met ze doorbracht, waarvan twee met elkaar als gezin.  Ik
had het voor geen goud willen missen!  In de dertien jaar die we samen
zijn hebben we al aardig wat meegemaakt.  Soms sleepte jij mij er door
heen.  Soms sleepte ik jou er door heen.  Meestal sleepten we elkaar er
door heen.  Je hebt je altijd vol overgave aan mij en de meiden gewijd, en
het is heerlijk om te zien hoe dol de meiden op je zijn.  Ik ook!  Ik ken
niemand die zo hard werkt en overal de schouders onder zet.  Je weet wat
je wilt en gaat er voor.  Je daadkracht en zelfvertrouwen zijn dan om
jaloers op te worden.  Ik ben dankbaar dat hoewel we zo verschillend zijn,
we zo ontzettend hetzelfde denken over alles wat er écht toe doet.  Als ik
het even niet meer weet, weet jij het wél.  Ik ben trots op je.  Je bent
mijn maatje, mijn partner, mijn vrouw.  Ik hou van je!


\vspace{1cm}
\flushright
David Fokkema,\\
September 2012

\end{otherlanguage}
