\chapter{Single Station Event Simulation and Reconstruction}
\label{ch:reconstruction}

A \hisparc station consists of two or four detectors. The four-detector stations
have the layout shown in \figref{fig:station-layout}.  An
inclined shower will not reach the detectors within the station simultaneously.
The arrival time differences depend
on the direction of the shower.  From the time measurements of the individual
detectors, it should be possible to reconstruct the direction of the shower.

To calculate the arrival time differences, a model of the shower front has to be
defined.  The shower direction, i.e. the zenith and azimuthal angles, are
parameters in this model.  The conventional method for reconstructing the
shower direction is to fit the results of the model to the arrival times
measured by the stations of a (dense) array.  An accurate estimate for the
direction of the shower can then be determined using a minimization method.
Some of the algorithms developed in the literature do not make use of a
minimization procedure \cite{Mayer:1993}, but require a large number of
detectors.  Both methods work best for extended arrays with knowledge of the
shower core position.

In the case of a single \hisparc station, a method has been developed using
direct calculation of the shower direction based on the arrival time
measurements of three out of the four detectors.

In this chapter, this algorithm will be tested on simulated air shower data.
First, the simulations will be discussed.  Then, the algorithm for the
reconstruction of the direction of the shower will be developed and the
statistical uncertainties will be evaluated.  The uncertainties will be
propagated through the analysis yielding equations for the accuracy of the
reconstructed azimuthal and zenith angles.  Correlations between
experimental errors are not considered and errors on errors are not
determined.  Finally, the reconstruction results of the simulated data
will be discussed and errors in the reconstruction will be compared to the
calculated uncertainties.


\section{Event Simulation}

The simulation contains three parts.
The first part is a full Monte Carlo simulation of the shower development.  The
second part is a simulation which uses the results of the first part to
effectively simulate many showers occurring in the neighborhood of a \hisparc
station.  This part tracks which of the shower particles hit the detectors. The
third part is a simulation of the detector response.


\subsection{Extensive Air Showers}
\label{sec:shower-simulation}

\aires \cite{Sciutto:1999}, such as the \corsika \cite{Corsika} program, is a
simulation package for studying extensive air showers.
Given an input file containing parameters defining the primary particle and
the accuracy of the calculations, \aires simulates interactions and tracks
particles through the atmosphere.

\corsika makes extensive use of third-party interaction models.  This allows
\corsika to be based on proven models.  While \aires also links to high-energy
hadronic interaction models such as SIBYLL and QGSJET, other interactions such
as electrodynamical processes, particle decays and the propagation of charged
particles, is handled by custom algorithms.  This allows \aires to be tuned to
be fast.  In fact, it is about \num{3.5} times faster than \corsika
\cite{Knapp:2002}.  The input file syntax of \aires is less complex than
the one used by \corsika.  A comparison between the results of \aires and
\corsika can be found in \cite{Knapp:2002}.  Differences in the results are
mainly due to different simulation parameters and details in the types
of interactions considered.  When the simulations are rerun with identical
parameters and interactions for both packages, no significant differences are
observed.
Overall, the authors conclude that \corsika and \aires agree to better than
\SI{20}{\percent} for the basic shower parameters observed by the Pierre Auger
Observatory.

Both programs include \emph{thinning algorithms}.
These algorithms are designed to reduce the number of particles that have to be
fully tracked. This reduces computation time, memory usage and the size of a
simulated event considerably. \aires makes use of an extended version of the
\emph{Hillas thinning algorithm}.  \citeauthor{Hillas:1981} \cites{Hillas:1981}{Hillas:1997}
proposed the following procedure to reduce the number of particles.  Consider
the process in which a particle $A$ generates a number of secondary particles:
\begin{equation}
A \to B_1, B_2, \ldots, B_n, \qquad \text{with } n \geq 1.
\end{equation}
Let $E_A$ be the energy of particle $A$, $E_{B_i}$ be the energy of
particle $B_i$, and $E_{th}$ the so-called \emph{thinning energy}.  Then, if
$E_A \geq E_{th}$, the secondaries are considered separately and included in the
remainder of the simulation with probability $P_i$, given by
\begin{equation}
P_i = \begin{cases}
  1 &\text{if $E_{B_i} \geq E_{th}$}, \\[5pt]
  \frac{E_{B_i}}{E_{th}} &\text{if $E_{B_i} < E_{th}$}.
\end{cases}
\end{equation}
That is, if a particle has an energy \emph{greater than} the thinning energy, it
is kept in the simulation.  If the energy is \emph{less than} the thinning
energy, there is a probability that the particle is removed from the simulation.
Particles with a larger energy fraction have a larger probability to be kept in
the simulation.  If the energy $E_A < E_{th}$, then particle $A$ is the result
of a previous thinning operation. Only one of the secondary particles is kept.
In other words, the number of particles with energies below the thinning energy
$E_{th}$ is never increased.  The secondary particle to keep in the simulation
is selected among all $n$ particles with the probability given by
\begin{equation}
P_i = \frac{E_{B_i}}{\sum_{j=1}^n E_{B_j}}.
\end{equation}
The remaining particles are given a statistical weight $w_{B_i} = w_A / P_i$,
with $w_A$ the weight of particle $A$.  This enables faster simulations while
keeping the results statistically correct.
When taking the statistical weights into account during analysis of the shower,
the global shower observables are almost unaffected.  Fluctuations increase,
however.

To reduce the fluctuations, \aires incorporates an extension to the
original splitting algorithm.  The new algorithm monitors the statistical
weights of the particles. If the weights become too large, the thinning
algorithm will keep more particles in the simulation.

It is essential to track individual particles in a detector to simulate detector
responses. To accomplish this, it becomes necessary to not use or reverse the
thinning.
A \emph{resampling} algorithm to obtain individual particle properties for use
in simulations for the Auger Observatory is proposed in \cite{Billoir:2008}.
A similar method for application in the Telescope Array, including validation
studies, is discussed in \cite{Stokes:2011}. However, the validation studies are only
done using the same thinned showers \emph{before} and \emph{after} the
resampling procedure.  Comparison of a \emph{dethinned} shower with a shower
\emph{without} thinning is carried out in \cite{Bruijn:2009}, using the \corsika
program.  A shower induced by a proton with an energy of \SI{e19}{\electronvolt}
is simulated with and without thinning.  The thinned shower is subsequently
resampled to obtain individual particles.  The study shows that fluctuations of
the number of particles in the detectors are significantly larger for dethinned
showers than for showers without thinning.  The fluctuations in the simulation
are compared to the expected Poissonian fluctuations. Showers are simulated with
thinning level $E_{th} = \num{e-6} E_p$, with $E_p$ the primary energy. This is
a commonly used setting. For electrons, the ratio $\sigma /
\sigma_\mathrm{Poisson} \approx 80$ \cite{Bruijn:2009}.

A realistic simulation of the shower front is essential to study the effect of
the accuracy of the time-of-arrival measurements on the reconstruction of the
shower direction.  The simulation of showers with $E_p \leq
\SI{e16}{\electronvolt}$ is still feasible without thinning.  Therefore, to
obtain the most realistic results, thinning and resampling is not performed in
this work.

A single \hisparc station is sensitive to extensive air showers starting from
primary energies of approximately \SI{1}{\peta\electronvolt}, see
\secref{sec:design-criteria-EAS} for details.  Since the number of showers falls
steeply with increasing primary energy, only an energy of
\SI{1}{\peta\electronvolt} is taken.

Showers have been simulated for a series of discrete zenith angles:
\SIlist{0;5;10;15;22.5;30;35;45}{\degree}.  For each angle, \num{10} showers
are generated. All showers are simulated with an incoming proton with an
energy of \SI{1}{\peta\electronvolt}.


\subsection{Detector Response}
\label{sec:detector-response}

Since the shower footprint is very large compared to the size of the
station, it is justified to change the coordinate system such that a single
shower can be used multiple times.  A station is then positioned at various
core distances inside the same shower and is given a random rotation
(\figref{fig:station-in-shower}).

\begin{figure}
\centering
\begin{tikzpicture}[scale=.05,
  detector/.style={fill},
]
\newcommand{\station}[2]{
  \begin{scope}[shift={#1},rotate=#2]
    \draw[detector] (-.25, -.5) rectangle (.25, .5);
    \draw[detector,shift={(0, 5.77)}] (-.25, -.5) rectangle (.25, .5);
    \draw[detector,shift={(-5, -2.89)},rotate=90] (-.25, -.5) rectangle (.25, .5);
    \draw[detector,shift={(5, -2.89)},rotate=-90] (-.25, -.5) rectangle (.25, .5);
    \draw[->,>=stealth',black!50] (0,0) -- (10,0);
    \draw[->,>=stealth',black!50] (0,0) -- (0,10);
  \end{scope}
}
\station{(33,37)}{146}
\station{(50,-27)}{-147}
\station{(29,-32)}{157}
\station{(0,76)}{-135}
\station{(-17,46)}{-23}
\station{(-61,-21)}{89}
\station{(-78,66)}{-35}
\station{(2,13)}{40}
\station{(7,-30)}{-157}
\station{(18,32)}{-160}
\station{(20,74)}{77}
\station{(43,59)}{10}
\station{(0,61)}{-80}
\station{(15,35)}{20}
\station{(5,7)}{134}
\station{(16,4)}{94}
\station{(17,-30)}{81}
\station{(-26,57)}{12}
\station{(56,-51)}{15}
\station{(-26,-6)}{-81}
\station{(65,-59)}{99}
\station{(-22,54)}{-58}
\station{(45,-46)}{83}
\station{(21,39)}{78}
\station{(-3,21)}{-27}
\station{(6,73)}{-81}
\station{(-73,28)}{59}
\station{(-44,-15)}{-41}
\station{(-13,-34)}{39}

\foreach \r in {7, 20, 40, 80}
  %\draw[black!50] (0,0) ellipse (1.5 * \r cm and \r cm);
  \draw[black!50] (0,0) circle (\r cm);
\end{tikzpicture}

\caption{A station is placed at various locations inside an inclined
shower and rotated over random angles.  This is equivalent to showers
arriving at the station from random azimuthal angles.  In this way, a
single simulated shower can be reused many times.}
\label{fig:station-in-shower}
\end{figure}

For each position, the charged leptons that traverse the detectors are
identified.  Leptons are by far the most numerous charged particles in a
shower.  Hadrons appear only at small core distances and are ignored since
even where they appear, they are far outnumbered by electrons.

The (cartesian) $x,\, y$ coordinate system is used by \aires and places the
shower core at the origin $O$ (\figref{fig:coordinates}).  The azimuthal angle of the shower,
$\phi$, is always equal to zero.  For each event, a random position inside the
shower is chosen for the location of the station. The coordinate system $x',\,
y'$ is introduced such that the station is always at the origin $O'$. This
position is defined by a core distance $r$ and polar angle $\Phi$, with respect
to $O$.
An angle $\alpha$ is randomly chosen and the station is rotated over that angle.
The coordinate system $x'',\, y''$ is defined as the rotation of $x', y'$ over
$-\alpha$, such that in the system $x'', y''$ the positions of the detectors are
the same for all events.
This coordinate system describes a stationary station which detects showers with
random core positions and random azimuthal angles $\phi''$.
 The transformation of $x,\, y$ coordinates to $x'',\, y''$ coordinates is given
by
\begin{equation}
r'' = r, \qquad
\Phi'' = \Phi + \pi - \alpha, \qquad
\phi'' = -\alpha.
\end{equation}
The shower core position $x'',\, y''$ and direction $\theta'',\, \phi''$
is given by
\begin{equation}
x'' = -r\cos (\Phi - \alpha), \qquad
y'' = -r\sin (\Phi - \alpha), \qquad
\theta'' = \theta, \qquad
\phi'' = -\alpha.
\end{equation}

\begin{figure}
\centering
\begin{sansmath}
\begin{tikzpicture}[scale=.3,
  vector/.style={->,>=stealth',thick},
  axis/.style={->,>=stealth',thick},
]
\def\sangle{40}
\def\angalpha{20}
\def\sshift{(\sangle:15)}
\begin{scope}[shift={\sshift},rotate=\angalpha]
\draw (-.25, -.5) rectangle (.25, .5);
\draw[shift={(0, 5.77)}] (-.25, -.5) rectangle (.25, .5);
\draw[shift={(-5, -2.89)},rotate=90] (-.25, -.5) rectangle (.25, .5);
\draw[shift={(5, -2.89)},rotate=-90] (-.25, -.5) rectangle (.25, .5);
\draw[fill] (0,0) circle (3pt) node[inner sep=7pt,left] {$O'$};
\draw[axis] (0,0) -- (10,0) node[right] {$x''$};
\draw[axis] (0,0) -- (0,10) node[above] {$y''$};
\end{scope}
\draw[axis] (0,0) -- (20,0) node[right] {$x$};
\draw[axis] (0,0) -- (0,20) node[above] {$y$};
\draw[axis,shift={\sshift}] (0,0) -- (0,10) node[above] {$y'$};
\draw[axis,shift={\sshift}] (0,0) -- (10,0) node[right] {$x'$};
\draw[shift={\sshift}] (2,0) arc (0:\angalpha:2);
\draw[shift={\sshift}] (.5 * \angalpha:3) node {$\alpha$};
\draw[vector] (0,0) -- \sshift node[midway,above] {$r$};
\draw (2,0) arc (0:\sangle:2);
\draw (.5 * \sangle:3) node {$\Phi$};
\draw[fill] (0,0) circle (3pt) node[below left] {$O$};
\end{tikzpicture}
\end{sansmath}

\caption{Coordinate system used in the simulation.  The $x,\, y$ axes are
the natural coordinates for the shower, as used by \aires, with the shower
core at the origin $O$.  The four rectangles are the four \hisparc
detectors.  In polar coordinates, the center of the station is defined as
$O'$ = $(r,\, \Phi)$, with the $x',\, y'$ coordinates defined as a
translation.  The angle $\alpha$ is defined as the angle over which the
station is rotated with respect to the $x',\, y'$ coordinates.  The
$x'',\, y''$ coordinate system is defined as the natural coordinates for
the station in which the detectors are always stationary.  This coordinate
system is used for the simulation output.}
\label{fig:coordinates}
\end{figure}

Photons generated by the charged particles propagate through the
scintillator. A photon may reflect multiple times before it reaches the PMT, or
it may escape the scintillator. In \secref{sec:transmission} a simulation is
discussed in which the propagation of photons is tracked to measure the
transmission efficiency of the scintillator. The simulation is extended to
include the propagation time of photons \cite{Steijger:2010-time}. The transport
time of the photons from the point of impact to the PMT depends on the location
of the particle path in the scintillator, as well as the initial directions of
the photons. If a photon reaches the PMT there is a probability $Q$, the PMT's
\emph{quantum efficiency}, that it will create a photoelectron. The
photoelectron creates an avalanche in the PMT, resulting in a pulse of
$\mathcal{O}(\num{e6})$ electrons.
For a signal above the \SI{70}{\milli\volt} threshold, 15 photoelectrons are required.
\figref{fig:photon-arrival-times} shows the distribution of the time of arrival
of the 15\textsuperscript{th} \emph{detected} photon. The transport time
distribution is approximated by the straight lines.
The mean value of the distribution is \SI{4.4}{\nano\second}, with a standard
deviation of \SI{1.2}{\nano\second}.

In the detector response simulation, for each charged particle traversing the
detector a random propagation time is chosen according to the transport time
distribution.
This value is added to the arrival time of the particle in the scintillator,
resulting in the \emph{measurement} of the arrival time:
\begin{equation}
t_\mathrm{measured} = t_\mathrm{arrival} + \Delta t_\mathrm{transport}.
\end{equation}
More than one particle
may traverse a detector. In this case, the detector simulation is performed for
each particle independently. From the list of measured arrival times, the
smallest value of $t_\mathrm{measured}$ is selected as the time measurement. The
simulation records the total number of particles which are detected in each
scintillator.

\begin{figure}
\centering
% \usepackage{tikz}
% \usetikzlibrary{arrows}
% \usepackage{pgfplots}
% \pgfplotsset{compat=1.3}
% \usepackage[detect-family]{siunitx}
% \usepackage[eulergreek]{sansmath}
% \sisetup{text-sf=\sansmath}
% \usepackage{relsize}
%
\pgfkeysifdefined{/artist/width}
    {\pgfkeysgetvalue{/artist/width}{\defaultwidth}}
    {\def\defaultwidth{ .67\linewidth }}
%
%
\begin{sansmath}
\begin{tikzpicture}[
        font=\sffamily,
        every pin/.style={inner sep=2pt, font={\sffamily\smaller}},
        every label/.style={inner sep=2pt, font={\sffamily\smaller}},
        every pin edge/.style={<-, >=stealth', shorten <=2pt},
        pin distance=2.5ex,
    ]
    \begin{axis}[
            width=\defaultwidth,
            %
            title={  },
            %
            xlabel={ Transport time [\si{\nano\second}] },
            ylabel={ Counts },
            %
            xmin={ 1 },
            xmax={ 8 },
            ymin={ 0 },
            ymax={  },
            %
            xtick={  },
            ytick={  },
            %
            tick align=outside,
            max space between ticks=40,
            every tick/.style={},
        ]

        

        

        
            
            % Draw series plot
            \addplot[no markers,gray,const plot] coordinates {
                (1.0, 0.0)
                (1.1, 0.0)
                (1.2, 1.0)
                (1.3, 3.0)
                (1.4, 2.0)
                (1.5, 4.0)
                (1.6, 3.0)
                (1.7, 2.0)
                (1.8, 5.0)
                (1.9, 19.0)
                (2.0, 22.0)
                (2.1, 26.0)
                (2.2, 34.0)
                (2.3, 58.0)
                (2.4, 78.0)
                (2.5, 123.0)
                (2.6, 201.0)
                (2.7, 296.0)
                (2.8, 414.0)
                (2.9, 445.0)
                (3.0, 435.0)
                (3.1, 460.0)
                (3.2, 552.0)
                (3.3, 579.0)
                (3.4, 304.0)
                (3.5, 196.0)
                (3.6, 198.0)
                (3.7, 185.0)
                (3.8, 191.0)
                (3.9, 193.0)
                (4.0, 165.0)
                (4.1, 203.0)
                (4.2, 177.0)
                (4.3, 218.0)
                (4.4, 239.0)
                (4.5, 199.0)
                (4.6, 205.0)
                (4.7, 192.0)
                (4.8, 218.0)
                (4.9, 165.0)
                (5.0, 230.0)
                (5.1, 183.0)
                (5.2, 216.0)
                (5.3, 177.0)
                (5.4, 226.0)
                (5.5, 208.0)
                (5.6, 191.0)
                (5.7, 188.0)
                (5.8, 153.0)
                (5.9, 178.0)
                (6.0, 165.0)
                (6.1, 197.0)
                (6.2, 149.0)
                (6.3, 130.0)
                (6.4, 117.0)
                (6.5, 89.0)
                (6.6, 75.0)
                (6.7, 55.0)
                (6.8, 35.0)
                (6.9, 20.0)
                (7.0, 5.0)
                (7.1, 2.0)
                (7.2, 1.0)
                (7.3, 0.0)
                (7.4, 0.0)
                (7.5, 0.0)
                (7.6, 0.0)
                (7.7, 0.0)
                (7.8, 0.0)
                (7.9, 0.0)
                (8.0, 0.0)
            };
        
            
            % Draw series plot
            \addplot[no markers,black] coordinates {
                (2.5507, 0)
                (2.5507, 380.96)
                (3.4953, 380.96)
                (3.4953, 186.68)
                (6.463, 186.68)
                (6.463, 0)
            };
        

        

        

        

        

    \end{axis}
\end{tikzpicture}
\end{sansmath}
\caption{Transport time of photons from the point of impact of the
detected charged particles to the PMT (gray).  The time for the
15\textsuperscript{th} photoelectron is recorded.  This is the first
photon over the detection threshold and will result in the arrival time
measurement.  The mean value is \SI{4.4}{\nano\second} and the standard
deviation is \SI{1.2}{\nano\second}.  An approximation for the transport
time distribution, used in the detector simulation, is shown in the same
graph (black).  Redrawn from \cite{Steijger:2010-time}.}
\label{fig:photon-arrival-times}
\end{figure}


\section{Reconstruction of Shower Direction}
\label{sec:direction-reconstruction-method}

To reconstruct the direction of the shower, it is sufficient to determine the
arrival time in three detectors.
The procedure is discussed below.  The
shower front is approximated by a plane.  This means that all particles in a
vertical shower should arrive at the same time, i.e.
there is no curvature of the shower front, and the front is infinitely thin.
These assumptions will be justified in the following discussion.

In \figref{fig:reconstruction-coordinates}
\begin{figure}
\centering
\tdplotsetmaincoords{50}{115}
\pgfmathsetmacro{\axlen}{8}
\pgfmathsetmacro{\slen}{8}
\pgfmathsetmacro{\thetavec}{50}
\pgfmathsetmacro{\phivec}{70}
\pgfmathsetmacro{\phiovec}{30}
\pgfmathsetmacro{\rlen}{\slen * sin(\thetavec) / cos(\phivec - \phiovec)}
\pgfmathsetmacro{\splen}{\slen * sin(\thetavec) * sin(\thetavec)}
\pgfmathsetmacro{\ylen}{\slen * sin(\thetavec) * cos(\thetavec)}
\pgfmathsetmacro{\clen}{.5}

\begin{sansmath}
\begin{tikzpicture}[tdplot_main_coords, node distance=0pt, font=\sffamily]

% Set coordinates
\coordinate (O) at (0, 0, 0);
\tdplotsetcoord{S}{\slen}{\thetavec}{\phivec}
\tdplotsetcoord{D}{\rlen}{90}{\phiovec}
\tdplotsetcoord{S'}{\splen}{\thetavec}{\phivec}

% Draw axes
\draw[thick,->] (0, 0, 0) -- (\axlen, 0, 0) node[anchor=north]{$x$};
\draw[thick,->] (0, 0, 0) -- (0, \axlen, 0) node[anchor=west]{$y$};
\draw[thick,->] (0, 0, 0) -- (0, 0, \axlen) node[anchor=south]{$z$};

% Draw detector points
\fill (O) circle (2pt);
\node at (O) [anchor=east] {Detector 0};
\fill (D) circle (2pt);
\node at (D) [anchor=north] {Detector 1};

% Draw labelled lines
\draw[very thick] (O) -- (S');
\path (O) -- (S') node [midway,below,sloped] {$c \Delta t_1$};
\draw[dashed] (O) -- (Sxy) node [pos=.6,below,sloped] {$r'_1$};
\draw (O) -- (D) node [midway,below,sloped] {$r_1$};

% Draw phi angle arcs
\tdplotdrawarc{(O)}{2}{0}{\phivec}{above left,fill=white}{$\phi$}
\tdplotdrawarc{(O)}{2.5}{0}{\phiovec}{below}{$\phi_1$}
\tdplotdrawarc{(O)}{3}{\phiovec}{\phivec}{below}{$\phi - \phi_1$}

% Draw theta angle arcs and right angle square
\tdplotsetthetaplanecoords{\phivec}
\tdplotdrawarc[tdplot_rotated_coords]{(O)}{1.5}{0}{\thetavec}{above}{$\theta$}
\tdplotdrawarc[tdplot_rotated_coords]{(Sxy)}{1.5}{-90 + \thetavec}{-90}{below right}{$\theta$}

% Draw right angle square in rotated theta plane
\tdplotsetrotatedcoords{\phivec-90}{90}{90-\thetavec}
\draw[tdplot_rotated_coords] ($(S') + (\clen,0)$) -- +(0,-\clen) -- +(-\clen,-\clen);

% Draw right angle on ground
\tdplotsetrotatedcoords{\phivec-90}{0}{0}
\draw[tdplot_rotated_coords] ($(Sxy) + (\clen,0)$) -- +(0,-\clen) -- +(-\clen,-\clen);

% Draw shower front
\tdplotsetrotatedcoords{\phivec}{\thetavec}{0}
\fill[tdplot_rotated_coords,gray!50,opacity=.6] ($(S')-(\ylen,\ylen,0)$) -- +(2*\ylen,0,0) -- +(2*\ylen,2*\ylen,0) -- +(0,2*\ylen,0) -- cycle;
\path[tdplot_rotated_coords] ($(S')-(0,\ylen,0)$) -- +(0,2*\ylen,0) node [midway,sloped,above,inner sep=30] {shower front};

% Draw lines in front of shower front
\draw[very thick] (S') -- ($(S')!2.2!(S)$) node [near end,above,sloped] {shower axis};
\draw[very thick,-stealth] ($(S')!2.2!(S)$) -- ($(S')!1.6!(S)$);
\draw[dashed] (D) -- (Sxy);
\draw[dashed] (Sxy) -- (S');

\end{tikzpicture}
\end{sansmath}

\caption{Three-dimensional representation of the coordinate system used in
reconstructing the direction of a shower. Refer to
\figref{fig:front-at-stations} for a two-dimensional representation.}
\label{fig:reconstruction-coordinates}
\end{figure}
a three-dimensional representation is given of the coordinate system used in the
reconstruction of the direction of a shower. For a slightly less
general two-dimensional representation, refer to \figref{fig:front-at-stations}.
The plane shower front is depicted as a surface traveling in the direction of
the shower axis. The shower axis has an angle $\theta$ with respect to
the vertical direction ($z$-axis), and an angle $\phi$ with respect to the
positive-$x$ direction. In the figure, the shower front is at the location of
detector 1. To reach detector 0, it has to travel a distance $c\Delta t_1$, where the
velocity of the shower front is approximated by $c$, and $\Delta t_1 \equiv t_1
- t_0$ is the arrival time difference between the detectors 0 and 1. The
projection of the shower front on the ground, the $xy$-plane, travels a distance
$r'_1$. It can be seen that
\begin{equation}
\sin \theta = \frac{c \Delta t_1}{r'_1},
\label{eq:sintheta}
\end{equation}
and
\begin{equation}
\cos (\phi - \phi_1) = \frac{r'_1}{r_1}.
\label{eq:cosphi}
\end{equation}
Combining \eqtworef{eq:sintheta}{eq:cosphi} results in
\begin{equation}
c \Delta t_1 = r_1 \cos (\phi - \phi_1) \sin \theta.
\label{eq:time_angle_relation}
\end{equation}
Then $\phi$ and $\theta$ are the two unknowns that can be resolved by two
measurements of the time differences ($\Delta t_1$, $\Delta t_2 \equiv t_2 -
t_0$):
\begin{equation}
\label{eq:system}
\left.\begin{aligned}
c \Delta t_1 &= r_1 \cos (\phi - \phi_1) \sin \theta \\
c \Delta t_2 &= r_2 \cos (\phi - \phi_2) \sin \theta
\end{aligned}\right\},
\end{equation}
where $\phi_2$ is the azimuthal angle of detector 2
with respect to detector 0. From \eqsref{eq:system} one obtains
\begin{equation}
\frac{\Delta t_1}{\Delta t_2} = \frac{r_1 \cos(\phi - \phi_1)}{r_2 \cos(\phi - \phi_2)},
\end{equation}
which, after using the trigonometric identity $\cos(\alpha - \beta) = \cos
\alpha \cos \beta + \sin \alpha \sin \beta$, turns into
\begin{equation}
\frac{r_2 \Delta t_1}{r_1 \Delta t_2}
= \frac{\cos \phi \cos \phi_1 + \sin \phi \sin \phi_1}{\cos \phi \cos \phi_2 + \sin \phi \sin \phi_2}
= \frac{\cos \phi_1 + \tan \phi \sin \phi_1}{\cos \phi_2 + \tan \phi \sin \phi_2}.
\end{equation}
By reordering, one arrives at
\begin{equation}
\label{eq:phi}
\tan \phi = - \frac{r_1 \Delta t_2 \cos \phi_1 - r_2 \Delta t_1 \cos \phi_2}{r_1 \Delta t_2 \sin \phi_1 - r_2 \Delta t_1 \sin \phi_2}.
\end{equation}
Once the angle $\phi$ is determined, either one of the \eqsref{eq:system} can be
solved for $\theta$:
\begin{subequations}
\label{eq:theta}
\begin{align}
\label{eq:theta1}
\sin \theta &= \frac{c \Delta t_1}{r_1 \cos(\phi - \phi_1)}, \\
\label{eq:theta2}
\sin \theta &= \frac{c \Delta t_2}{r_2 \cos(\phi - \phi_2)}.
\end{align}
\end{subequations}


\section{Measurement Uncertainties}
\label{sec:uncertainty-calculations}

\subsection{Timing Uncertainties}
\label{sec:uncertainty-sources}

Experimental uncertainties in the determination of the arrival time are
introduced by the decay time of the scintillator (\secref{sec:scintillator}),
time-of-flight of scintillation photons from the point of impact of the particle
to the PMT, transit time spread of the electrons through the PMT, and the timing
resolution of the ADCs.  The GPS timing system has no influence on the
reconstruction from a station, since it is only used to generate an overall
timestamp for the event.

The decay constant of the scintillator is \SI{2.1}{\nano\second} \cite{BC-408}.
On average, approximately \num{440} photons reach the PMT, while only \num{15}
photoelectrons are required to cross the detection threshold. Most of these
photons will therefore have a vanishingly small delay due to the decay time.

When a charged particle traverses the scintillator, the time for
scintillation photons to reach the phototube depends on the distance
between the phototube and the point of impact.  This is determined by the
geometry of the scintillator and the lightguide.  When the scintillator is
uniformly illuminated, the mean time-of-flight is \SI{4.4}{\nano\second}
with a standard deviation of \SI{1.2}{\nano\second}
\cite{Steijger:2010-time}, see \figref{fig:photon-arrival-times}.  The
sampling time of the ADCs is \SI{2.5}{\nano\second}.


\subsection{Model Uncertainties}

The shower front is not a flat plane.  Also, the shower front
has a certain thickness.
It is therefore unknown whether a particle is measured very near the causal
front\footnote{When the primary cosmic ray particle interacts, information from
that event can at best be transmitted at the speed of light.  All possible light
signals from the event form an expanding sphere.  The part of the sphere in the
direction of the primary particle momentum is called the \emph{causal front}.
The shower front, containing the secondary particles, travels behind the causal
front.} or lagging behind, which leads to an uncertainty in the timing
measurement.
The arrival time distributions of simulated vertical
\SI{1}{\peta\electronvolt} proton showers are shown in
\figref{fig:arrival-time-distributions}.
The arrival time is defined as the amount of time a particle arrives after the
causal front has passed.  Only events that have at least one charged particle in all
corner detectors are considered.  The median arrival time delay
increases with increasing core distance.
\SI{50}{\percent} of the particles are contained within the gray band.  The
width of the band, i.e. the thickness of the shower front, also increases with
core distance.

\begin{figure}
\centering
% \usepackage{tikz}
% \usetikzlibrary{arrows}
% \usepackage{pgfplots}
% \pgfplotsset{compat=1.3}
% \usepackage[detect-family]{siunitx}
% \usepackage[eulergreek]{sansmath}
% \sisetup{text-sf=\sansmath}
% \usepackage{relsize}
%
\pgfkeysifdefined{/artist/width}
    {\pgfkeysgetvalue{/artist/width}{\defaultwidth}}
    {\def\defaultwidth{ .67\linewidth }}
%
%
\begin{sansmath}
\begin{tikzpicture}[
        font=\sffamily,
        every pin/.style={inner sep=2pt, font={\sffamily\smaller}},
        every label/.style={inner sep=2pt, font={\sffamily\smaller}},
        every pin edge/.style={<-, >=stealth', shorten <=2pt},
        pin distance=2.5ex,
    ]
    \begin{axis}[
            width=\defaultwidth,
            axis equal=false,
            %
            title={  },
            %
            xlabel={ Core distance [\si{\meter}] },
            ylabel={ Arrival time difference $|t_2 - t_1|$ [\si{\nano\second}] },
            %
            xmin={ 0 },
            xmax={ 100 },
            ymin={ 0 },
            ymax={  },
            %
            xtick={  },
            ytick={  },
            %
            tick align=outside,
            max space between ticks=40,
            every tick/.style={},
            axis on top,
        ]

        

        
            \addplot[draw=none, fill=lightgray] coordinates {
                (5.0, 0.132635653019)
                (15.0, 0.327908158302)
                (25.0, 0.704736948013)
                (35.0, 1.18435704708)
                (45.0, 1.7541)
                (55.0, 2.51782)
                (65.0, 3.08078861237)
                (75.0, 3.65520644188)
                (85.0, 4.36163759232)
                (95.0, 4.38032484055)
                (95.0, 21.9856824875)
                (85.0, 20.0310139656)
                (75.0, 17.849779129)
                (65.0, 14.7559034824)
                (55.0, 12.8306)
                (45.0, 9.03775)
                (35.0, 5.74408388138)
                (25.0, 3.34901916981)
                (15.0, 1.66783547401)
                (5.0, 0.658097743988)
                } -- cycle;
        

        
            
            % Draw series plot
            \addplot[mark=*,mark options=white,only marks] coordinates {
                (5.0, 0.312565326691)
                (15.0, 0.772647)
                (25.0, 1.62775)
                (35.0, 2.73280000687)
                (45.0, 4.15145)
                (55.0, 5.99931)
                (65.0, 7.31134724617)
                (75.0, 8.86992359161)
                (85.0, 10.2278804779)
                (95.0, 11.1200256348)
            };
        
            
            % Draw series plot
            \addplot[mark=o,only marks] coordinates {
                (5.0, 0.312565326691)
                (15.0, 0.772647)
                (25.0, 1.62775)
                (35.0, 2.73280000687)
                (45.0, 4.15145)
                (55.0, 5.99931)
                (65.0, 7.31134724617)
                (75.0, 8.86992359161)
                (85.0, 10.2278804779)
                (95.0, 11.1200256348)
            };
        

        

        

        

        

    \end{axis}
\end{tikzpicture}
\end{sansmath}
\caption{The measured arrival time distributions of vertical showers. The
difference in arrival time in two detectors is graphed. The showers are
generated by a \SI{1}{\peta\electronvolt} proton.  Only measurements with
at least one charged particle in all three corner detectors are taken into
account. The dots show the median arrival time and the gray bands contain
\SI{50}{\percent} of the events, evenly distributed around the median.
This figure can be compared to \figref{fig:front-arrival-time}, which shows
the actual simulated arrival time distributions.  The data in this graph
is an approximation derived from \emph{measuring} the arrival times in two
detectors.}
\label{fig:arrival-time-distributions}
\end{figure}


\subsection{Propagation through Analysis}

The uncertainty in reconstructing the azimuthal angle introduced by timing
uncertainties is given by
\begin{equation}
\sigma_\phi^2 = \sigma_t^2 \left(\left|\frac{\partial\phi}{\partial t_0}\right|^2 + \left|\frac{\partial\phi}{\partial t_1}\right|^2 + \left|\frac{\partial\phi}{\partial t_2}\right|^2\right),
\end{equation}
with $\sigma_\phi, \sigma_t$ the \emph{standard deviation}, of
$\phi$ and $t$ respectively.  The first-order derivatives of $\phi(t_0, t_1,
t_2)$ are:
\begin{align}
\frac{\partial \phi}{\partial t_0} &= \frac{1}{1 + \tan^2\phi} \cdot \frac{r_2\cos \phi_2 - r_1\cos \phi_1 + (r_2\sin \phi_2 - r_1\sin \phi_1)\tan \phi}{\frac{r_1 r_2}{c}\sin\theta\left(\sin\phi_2\cos(\phi - \phi_1) - \sin\phi_1\cos(\phi - \phi_2)\right)} , \\[10pt]
\frac{\partial \phi}{\partial t_1} &= \frac{1}{1 + \tan^2\phi} \cdot \frac{-r_2(\sin \phi_2 \tan \phi + \cos \phi_2)}{\frac{r_1 r_2}{c}\sin\theta\left(\sin\phi_2\cos(\phi - \phi_1) - \sin\phi_1\cos(\phi - \phi_2)\right)}, \\[10pt]
\frac{\partial \phi}{\partial t_2} &= \frac{1}{1 + \tan^2\phi} \cdot
\frac{r_1(\sin \phi_1 \tan \phi + \cos \phi_1)}{\frac{r_1 r_2}{c}\sin\theta\left(\sin\phi_2\cos(\phi - \phi_1) - \sin\phi_1\cos(\phi - \phi_2)\right)}.
\end{align}
The time differences $\Delta t$ remaining after the differentiation procedure
have been rewritten in terms of $\theta$ and $\phi$ using \eqsref{eq:theta}.  Applying
the trigonometric identities, these equations reduce to
\begin{align}
\frac{\partial \phi}{\partial t_0} &= -\left( \frac{\partial \phi}{\partial t_1} + \frac{\partial \phi}{\partial t_2} \right), \\[10pt]
\frac{\partial \phi}{\partial t_1} &= \frac{-c\, \cos(\phi - \phi_2)}{r_1 \sin\theta \sin(\phi_2 - \phi_1)}, \\[10pt]
\frac{\partial \phi}{\partial t_2} &= \frac{c\, \cos(\phi - \phi_1)}{r_2 \sin\theta \sin(\phi_2 - \phi_1)}.
\end{align}
%
Singularities occur for $r_1 = 0$, $r_2 = 0$, $\theta = 0$, and $\phi_2 -
\phi_1 \in \{-\pi,\, 0,\, \pi\}$. For $r_1 = 0$ or $r_2 = 0$, the direction cannot be reconstructed
since two detectors will be on top of each other.  The same holds
for $\phi_2 - \phi_1 \in \{-\pi,\, 0,\, \pi\}$: the
three detectors are on a straight line.  For $\theta = 0$, the azimuthal angle
$\phi$ is also undefined and the uncertainty becomes infinite.

The full expression for the uncertainty in $\phi$ is then given by
\begin{equation}
\begin{split}
\label{eq:errphi}
\sigma_\phi^2 &= 2 \sigma_t^2 \left(\left|\frac{\partial\phi}{\partial t_1}\right|^2 + \left|\frac{\partial\phi}{\partial t_2}\right|^2 + \frac{\partial\phi}{\partial t_1}\frac{\partial\phi}{\partial t_2} \right) \\
&= \frac{2c^2\sigma_t^2}{\sin^2\theta\sin^2(\phi_2 - \phi_1)} \left( \frac{\cos^2(\phi - \phi_2)}{r_1^2} + \frac{\cos^2(\phi - \phi_1)}{r_2^2} - \frac{\cos(\phi - \phi_1)\cos(\phi - \phi_2)}{r_1r_2} \right).
\end{split}
\end{equation}
%
Similarly, the uncertainty in the zenith angle in \eqref{eq:theta1} becomes:
\begin{equation}
\label{eq:errtheta}
\begin{split}
\sigma_\theta^2 &= \sigma_t^2 \left(\left|\frac{\partial\theta}{\partial t_0}\right|^2 + \left|\frac{\partial\theta}{\partial t_1}\right|^2 + \left|\frac{\partial\theta}{\partial t_2}\right|^2\right) \\
&= \sigma_t^2 \frac{A\sin^2\theta + B\sin\theta + C}{r_1^2(1 - \sin^2\theta)\cos^2(\phi - \phi_1)},
\end{split}
\end{equation}
with
\begin{align}
A &= r_1^2\sin^2(\phi - \phi_1) \left[\left|\frac{\partial\phi}{\partial t_0}\right|^2 + \left|\frac{\partial\phi}{\partial t_1}\right|^2 + \left|\frac{\partial\phi}{\partial t_2}\right|^2\right] ,\\
B &= -2r_1 c \sin(\phi - \phi_1) \left[\frac{\partial\phi}{\partial t_0} - \frac{\partial\phi}{\partial t_1}\right] ,\\
C &= 2c^2 .
\end{align}
%
For horizontal showers ($\theta = \SI{90}{\degree}$) the acceptance becomes zero
and the uncertainty becomes infinite.  Furthermore, when a shower hits detector
0 and detector 1 (\figref{fig:reconstruction-coordinates}) from a direction
perpendicular to the line connecting the two detectors, $\left(\phi - \phi_1 =
\pm\frac{\pi}{2}\right)$, it is impossible to determine the zenith angle since all
angles result in the same time difference $\Delta t_1 = 0$.  In this situation,
it is required to switch from \eqref{eq:theta1} to \eqref{eq:theta2}.  The
uncertainty in that equation can be calculated as shown above.  The end result
is identical for the substitutions $r_1 \to r_2, \phi_1 \to \phi_2, t_1 \to t_2$
and $t_2 \to t_1$.  Denoting the angle and uncertainties from
\eqsref{eq:theta} as $(\theta_1, \sigma_{\theta_1})$ and $(\theta_2,
\sigma_{\theta_2})$, respectively, one obtain an expression for the zenith
angle:
\begin{equation}
\theta = \frac{\frac{1}{\sigma_{\theta_1}} \theta_1 + \frac{1}{\sigma_{\theta_2}} \theta_2}{\frac{1}{\sigma_{\theta_1}} + \frac{1}{\sigma_{\theta_2}}}.
\end{equation}
This will lead to a robust calculation which is only susceptible to large uncertainties
for (near-) horizontal showers. Obviously, for horizontal showers the acceptance
of the detectors is extremely small.


\section{Performance of a Single Station}
\label{sec:direction-reconstruction-results}


\subsection{Reconstruction Efficiency}

Three independent measurements are required to reconstruct the shower direction.
 The best result is obtained when the distance between the detectors is
large. Then the difference between arrival times is large and the uncertainty
is subsequently smaller.
Therefore, the three corner detectors of the station
(\figref{fig:station-layout}) are chosen to reconstruct the shower direction.

Particle densities in a shower are highest near the shower core and fall off
steeply (\figref{fig:ldf-threshold}).  Therefore, the probability that a
detector is traversed by a particle decreases with increasing core distance.
\figref{fig:sim-detection-efficiency} shows the fraction
of simulated showers detected by all three corner detectors.  Three zenith
angles are shown.  If the shower core is close to the station, almost all
showers are detected.  At a distance of approximately \SI{40}{\meter}, the
detection efficiency is \SI{50}{\percent} for vertical showers.  For inclined
showers, the detection efficiency is much lower. Since the distance particles
travel through the atmosphere is larger, fewer particles reach the
ground.  Moreover, the acceptance of the detectors is less for inclined showers,
since the effective detection area is decreased.

\begin{figure}
\centering
% \usepackage{tikz}
% \usetikzlibrary{arrows}
% \usepackage{pgfplots}
% \pgfplotsset{compat=1.3}
% \usepackage[detect-family]{siunitx}
% \usepackage[eulergreek]{sansmath}
% \sisetup{text-sf=\sansmath}
% \usepackage{relsize}
%
\pgfkeysifdefined{/artist/width}
    {\pgfkeysgetvalue{/artist/width}{\defaultwidth}}
    {\def\defaultwidth{ .67\linewidth }}
%
%
\begin{sansmath}
\begin{tikzpicture}[
        font=\sffamily,
        every pin/.style={inner sep=2pt, font={\sffamily\smaller}},
        every label/.style={inner sep=2pt, font={\sffamily\smaller}},
        every pin edge/.style={<-, >=stealth', shorten <=2pt},
        pin distance=2.5ex,
    ]
    \begin{axis}[
            xmode=normal,
            ymode=normal,
            width=\defaultwidth,
            axis equal=false,
            %
            title={  },
            %
            xlabel={ Core distance [\si{\meter}] },
            ylabel={ Detection efficiency },
            %
            xmin={ 0 },
            xmax={ 100 },
            ymin={ 0 },
            ymax={ 1 },
            %
            xtick={  },
            ytick={  },
            %
            tick align=outside,
            max space between ticks=40,
            every tick/.style={},
            axis on top,
        ]

        




    
    % Draw series plot
    \addplot[no markers,solid] coordinates {
        (2.63157894737, 0.999572649573)
        (7.89473684211, 0.994736852186)
        (13.1578947368, 0.980147149984)
        (18.4210526316, 0.938285622317)
        (23.6842105263, 0.865317869453)
        (28.9473684211, 0.763634777099)
        (34.2105263158, 0.651506009017)
        (39.4736842105, 0.531305012336)
        (44.7368421053, 0.424368697968)
        (50.0, 0.32878314729)
        (55.2631578947, 0.24675802482)
        (60.5263157895, 0.179818774083)
        (65.7894736842, 0.127030135242)
        (71.0526315789, 0.0898349046516)
        (76.3157894737, 0.062488601219)
        (81.5789473684, 0.0445459833686)
        (86.8421052632, 0.0301208447679)
        (92.1052631579, 0.0220469018658)
        (97.3684210526, 0.0153106081475)
    };

    
    % Draw series plot
    \addplot[no markers,solid] coordinates {
        (2.63157894737, 0.964215832957)
        (7.89473684211, 0.924255094009)
        (13.1578947368, 0.84598653971)
        (18.4210526316, 0.740531306693)
        (23.6842105263, 0.622435615829)
        (28.9473684211, 0.487076014556)
        (34.2105263158, 0.370596123705)
        (39.4736842105, 0.266876340828)
        (44.7368421053, 0.190959643762)
        (50.0, 0.126909886937)
        (55.2631578947, 0.0833565459828)
        (60.5263157895, 0.0533179703205)
        (65.7894736842, 0.0366989641519)
        (71.0526315789, 0.0243144882107)
        (76.3157894737, 0.0161706656918)
        (81.5789473684, 0.0115216383806)
        (86.8421052632, 0.00792666060904)
        (92.1052631579, 0.004703498086)
        (97.3684210526, 0.00345590489113)
    };

    
    % Draw series plot
    \addplot[no markers,solid] coordinates {
        (2.63157894737, 0.718054615732)
        (7.89473684211, 0.621453371293)
        (13.1578947368, 0.484235953188)
        (18.4210526316, 0.344157833597)
        (23.6842105263, 0.229677794241)
        (28.9473684211, 0.14141080186)
        (34.2105263158, 0.0882802126969)
        (39.4736842105, 0.051315809924)
        (44.7368421053, 0.0317630825135)
        (50.0, 0.0193836001475)
        (55.2631578947, 0.0109395715332)
        (60.5263157895, 0.00710895115761)
        (65.7894736842, 0.00431826316402)
        (71.0526315789, 0.00291676275465)
        (76.3157894737, 0.00215685615789)
        (81.5789473684, 0.00129545950139)
        (86.8421052632, 0.000920976188453)
        (92.1052631579, 0.000606608032522)
        (97.3684210526, 0.000520282590557)
    };







    \node[coordinate,,
          pin={ right:{ \SI{0}{\degree} }}]
        at (axis cs:19.6842105263, 0.92077336163) {};

    \node[coordinate,,
          pin={ left:{ \SI{22.5}{\degree} }}]
        at (axis cs:15.8947368421, 0.791149818541) {};

    \node[coordinate,,
          pin={ below left:{ \SI{35}{\degree} }}]
        at (axis cs:16.8421052632, 0.386181269474) {};




    \end{axis}
\end{tikzpicture}
\end{sansmath}
\caption{Fraction of showers that are detected with at least 1 charged particle
in the three corner detectors as a function of core distance.  The showers
are initiated by \SI{1}{\peta\electronvolt} protons for a series of zenith
angles. If the shower core is close to the station, almost all showers are
detected.  At a distance of approximately \SI{40}{\meter}, the detection
efficiency is \SI{50}{\percent} for vertical showers.  For inclined showers, the
detection efficiency is much lower, as fewer particles reach the ground.}
\label{fig:sim-detection-efficiency}
\end{figure}

A detected shower yields three independent arrival time measurements.  Thus, the
equations given in \secref{sec:direction-reconstruction-method} can be used to
reconstruct the direction of the shower.  However, if the arrival time
differences become too large, the reconstruction results in an unphysical
solution and the reconstruction is rejected. This occurs when the arrival time
difference between two detectors is larger than the \emph{light time} ($\Delta t
= d / c$) between the detectors with distance $d$. Since the velocity of charged
particles in a shower is approximated by the speed of light, the light time
between two detectors is the largest time difference allowed in the
reconstruction.  For a \hisparc station, $d = \SI{10}{\meter}$, thus $\Delta t =
\SI{33}{\nano\second}$.

\figref{fig:sim-reconstruction-efficiency} shows the fraction of
\emph{detected} \SI{1}{\peta\electronvolt} proton showers for which the
direction can be reconstructed as a function of the core distance.
The reconstruction efficiency decreases with increasing core distance, but is
up to \SI{90}{\meter} well above \SI{50}{\percent}.  The shape and thickness of
the shower front differs from the approximated thin, flat plane.  This leads to
larger arrival time delays for increasing core distance
(\figref{fig:arrival-time-distributions}).  Approximately \SI{25}{\percent} of
the vertical EAS events with a core distance of \SI{70}{\meter}, arrive later
than the light time for the distance between two detectors in a \hisparc
station and are thus rejected.  This fraction increases for inclined
showers.

\begin{figure}
\centering
% \usepackage{tikz}
% \usetikzlibrary{arrows}
% \usepackage{pgfplots}
% \pgfplotsset{compat=1.3}
% \usepackage[detect-family]{siunitx}
% \usepackage[eulergreek]{sansmath}
% \sisetup{text-sf=\sansmath}
% \usepackage{relsize}
%
\pgfkeysifdefined{/artist/width}
    {\pgfkeysgetvalue{/artist/width}{\defaultwidth}}
    {\def\defaultwidth{ .67\linewidth }}
%
%
\begin{sansmath}
\begin{tikzpicture}[
        font=\sffamily,
        every pin/.style={inner sep=2pt, font={\sffamily\smaller}},
        every label/.style={inner sep=2pt, font={\sffamily\smaller}},
        every pin edge/.style={<-, >=stealth', shorten <=2pt},
        pin distance=2.5ex,
    ]
    \begin{axis}[
            width=\defaultwidth,
            %
            title={  },
            %
            xlabel={ Core distance [\si{\meter}] },
            ylabel={ Reconstruction efficiency },
            %
            xmin={ 0 },
            xmax={ 100 },
            ymin={  },
            ymax={ 1 },
            %
            xtick={  },
            ytick={  },
            %
            tick align=outside,
            max space between ticks=40,
            every tick/.style={},
        ]

        

        

        
            
            % Draw series plot
            \addplot[no markers,solid] coordinates {
                (5.55555555556, 0.999756986634)
                (16.6666666667, 0.995549382016)
                (27.7777777778, 0.979324445557)
                (38.8888888889, 0.944064901793)
                (50.0, 0.882616093973)
                (61.1111111111, 0.813462514808)
                (72.2222222222, 0.743339055153)
                (83.3333333333, 0.660478547855)
                (94.4444444444, 0.596038114343)
            };
        
            
            % Draw series plot
            \addplot[no markers,solid] coordinates {
                (5.55555555556, 0.996893606006)
                (16.6666666667, 0.983677642248)
                (27.7777777778, 0.950610643549)
                (38.8888888889, 0.893984837124)
                (50.0, 0.813229298468)
                (61.1111111111, 0.761540600667)
                (72.2222222222, 0.697945585786)
                (83.3333333333, 0.643528183716)
                (94.4444444444, 0.567010309278)
            };
        
            
            % Draw series plot
            \addplot[no markers,solid] coordinates {
                (5.55555555556, 0.981039678553)
                (16.6666666667, 0.956066501592)
                (27.7777777778, 0.89586074284)
                (38.8888888889, 0.822180916976)
                (50.0, 0.74479889043)
                (61.1111111111, 0.69394261424)
                (72.2222222222, 0.618937644342)
                (83.3333333333, 0.59649122807)
                (94.4444444444, 0.525423728814)
            };
        

        

        

        
            \node[coordinate,
                  pin={[] above right:{ \SI{0}{\degree} }}]
                at (axis cs:85.5555555556, 0.647590461152) {};
        
            \node[coordinate,
                  pin={[] below left:{ \SI{22.5}{\degree} }}]
                at (axis cs:23.3333333333, 0.963837443028) {};
        
            \node[coordinate,
                  pin={[] below left:{ \SI{35}{\degree} }}]
                at (axis cs:23.3333333333, 0.919943046341) {};
        

        

    \end{axis}
\end{tikzpicture}
\end{sansmath}
\caption{Fraction of \emph{detected} \SI{1}{\peta\electronvolt} proton showers
for which the direction can be reconstructed as function of the core distance.
The requirement for detection is at least 1 charged particle in the three corner
detectors.  The reconstruction is successful if the zenith and azimuthal angles
can be calculated using the equations given in
\secref{sec:direction-reconstruction-method}.  The reconstruction efficiency
decreases with increasing core distance, but is well above \SI{50}{\percent} up
to \SI{90}{\meter}.}
\label{fig:sim-reconstruction-efficiency}
\end{figure}


\subsection{Shower Direction}
\label{sec:shower-direction}

\figref{fig:phi-reconstruction} shows a strong correlation between simulated and
reconstructed azimuthal angles for \SI{1}{\peta\electronvolt} proton showers
with a zenith angle $\theta = \SI{22.5}{\degree}$.
Reconstruction uncertainties are depicted in
\figref{fig:phi-reconstruction-error}.  For the left-hand figure at least one
particle in all three corner detectors is required.
On the right, however, more strict conditions are applied: at least two
particles in each of the three corner detectors, which results in a more
accurate azimuthal angle.

\begin{figure}
\centering
\longprocess{% \usepackage{tikz}
% \usetikzlibrary{arrows}
% \usepackage{pgfplots}
% \pgfplotsset{compat=1.3}
% \usepackage[detect-family]{siunitx}
% \usepackage[eulergreek]{sansmath}
% \sisetup{text-sf=\sansmath}
% \usepackage{relsize}
%
\pgfkeysifdefined{/artist/width}
    {\pgfkeysgetvalue{/artist/width}{\defaultwidth}}
    {\def\defaultwidth{ .67\linewidth }}
%
%
\begin{sansmath}
\begin{tikzpicture}[
        font=\sffamily,
        every pin/.style={inner sep=2pt, font={\sffamily\smaller}},
        every label/.style={inner sep=2pt, font={\sffamily\smaller}},
        every pin edge/.style={<-, >=stealth', shorten <=2pt},
        pin distance=2.5ex,
    ]
    \begin{axis}[
            width=\defaultwidth,
            %
            title={  },
            %
            xlabel={ $\phi_\mathrm{sim}$ [\si{\degree}] },
            ylabel={ $\phi_\mathrm{rec}$ [\si{\degree}] },
            %
            xmin={ -180.0 },
            xmax={ 180.0 },
            ymin={ -180.0 },
            ymax={ 180.0 },
            %
            xtick={ -180, -90, 0, 90, 180 },
            ytick={ -180, -90, 0, 90, 180 },
            %
            tick align=outside,
            max space between ticks=40,
            every tick/.style={},
            axis on top,
        ]

        
            % Draw histogram backbround, if needed
                \path[fill,white]
                    (axis cs:-180.0, -180.0)
                    rectangle (axis cs:180.0, 180.0);

            % Draw non-empty bins in histogram
                        \path[,fill,
                              black!84]
                            (axis cs:-180.0, -180.0)
                            rectangle (axis cs:-175.0, -175.0);
                        \path[,fill,
                              black!51]
                            (axis cs:-180.0, -175.0)
                            rectangle (axis cs:-175.0, -170.0);
                        \path[,fill,
                              black!37]
                            (axis cs:-180.0, -170.0)
                            rectangle (axis cs:-175.0, -165.0);
                        \path[,fill,
                              black!27]
                            (axis cs:-180.0, -165.0)
                            rectangle (axis cs:-175.0, -160.0);
                        \path[,fill,
                              black!22]
                            (axis cs:-180.0, -160.0)
                            rectangle (axis cs:-175.0, -155.0);
                        \path[,fill,
                              black!14]
                            (axis cs:-180.0, -155.0)
                            rectangle (axis cs:-175.0, -150.0);
                        \path[,fill,
                              black!10]
                            (axis cs:-180.0, -150.0)
                            rectangle (axis cs:-175.0, -145.0);
                        \path[,fill,
                              black!9]
                            (axis cs:-180.0, -145.0)
                            rectangle (axis cs:-175.0, -140.0);
                        \path[,fill,
                              black!10]
                            (axis cs:-180.0, -140.0)
                            rectangle (axis cs:-175.0, -135.0);
                        \path[,fill,
                              black!9]
                            (axis cs:-180.0, -135.0)
                            rectangle (axis cs:-175.0, -130.0);
                        \path[,fill,
                              black!5]
                            (axis cs:-180.0, -130.0)
                            rectangle (axis cs:-175.0, -125.0);
                        \path[,fill,
                              black!8]
                            (axis cs:-180.0, -125.0)
                            rectangle (axis cs:-175.0, -120.0);
                        \path[,fill,
                              black!5]
                            (axis cs:-180.0, -120.0)
                            rectangle (axis cs:-175.0, -115.0);
                        \path[,fill,
                              black!4]
                            (axis cs:-180.0, -115.0)
                            rectangle (axis cs:-175.0, -110.0);
                        \path[,fill,
                              black!1]
                            (axis cs:-180.0, -110.0)
                            rectangle (axis cs:-175.0, -105.0);
                        \path[,fill,
                              black!2]
                            (axis cs:-180.0, -105.0)
                            rectangle (axis cs:-175.0, -100.0);
                        \path[,fill,
                              black!1]
                            (axis cs:-180.0, -100.0)
                            rectangle (axis cs:-175.0, -95.0);
                        \path[,fill,
                              black!1]
                            (axis cs:-180.0, -75.0)
                            rectangle (axis cs:-175.0, -70.0);
                        \path[,fill,
                              black!1]
                            (axis cs:-180.0, 15.0)
                            rectangle (axis cs:-175.0, 20.0);
                        \path[,fill,
                              black!1]
                            (axis cs:-180.0, 25.0)
                            rectangle (axis cs:-175.0, 30.0);
                        \path[,fill,
                              black!1]
                            (axis cs:-180.0, 30.0)
                            rectangle (axis cs:-175.0, 35.0);
                        \path[,fill,
                              black!1]
                            (axis cs:-180.0, 40.0)
                            rectangle (axis cs:-175.0, 45.0);
                        \path[,fill,
                              black!4]
                            (axis cs:-180.0, 45.0)
                            rectangle (axis cs:-175.0, 50.0);
                        \path[,fill,
                              black!1]
                            (axis cs:-180.0, 50.0)
                            rectangle (axis cs:-175.0, 55.0);
                        \path[,fill,
                              black!2]
                            (axis cs:-180.0, 55.0)
                            rectangle (axis cs:-175.0, 60.0);
                        \path[,fill,
                              black!1]
                            (axis cs:-180.0, 60.0)
                            rectangle (axis cs:-175.0, 65.0);
                        \path[,fill,
                              black!2]
                            (axis cs:-180.0, 65.0)
                            rectangle (axis cs:-175.0, 70.0);
                        \path[,fill,
                              black!3]
                            (axis cs:-180.0, 70.0)
                            rectangle (axis cs:-175.0, 75.0);
                        \path[,fill,
                              black!1]
                            (axis cs:-180.0, 75.0)
                            rectangle (axis cs:-175.0, 80.0);
                        \path[,fill,
                              black!1]
                            (axis cs:-180.0, 90.0)
                            rectangle (axis cs:-175.0, 95.0);
                        \path[,fill,
                              black!2]
                            (axis cs:-180.0, 95.0)
                            rectangle (axis cs:-175.0, 100.0);
                        \path[,fill,
                              black!1]
                            (axis cs:-180.0, 100.0)
                            rectangle (axis cs:-175.0, 105.0);
                        \path[,fill,
                              black!3]
                            (axis cs:-180.0, 105.0)
                            rectangle (axis cs:-175.0, 110.0);
                        \path[,fill,
                              black!4]
                            (axis cs:-180.0, 110.0)
                            rectangle (axis cs:-175.0, 115.0);
                        \path[,fill,
                              black!1]
                            (axis cs:-180.0, 115.0)
                            rectangle (axis cs:-175.0, 120.0);
                        \path[,fill,
                              black!3]
                            (axis cs:-180.0, 120.0)
                            rectangle (axis cs:-175.0, 125.0);
                        \path[,fill,
                              black!3]
                            (axis cs:-180.0, 125.0)
                            rectangle (axis cs:-175.0, 130.0);
                        \path[,fill,
                              black!5]
                            (axis cs:-180.0, 130.0)
                            rectangle (axis cs:-175.0, 135.0);
                        \path[,fill,
                              black!4]
                            (axis cs:-180.0, 135.0)
                            rectangle (axis cs:-175.0, 140.0);
                        \path[,fill,
                              black!7]
                            (axis cs:-180.0, 140.0)
                            rectangle (axis cs:-175.0, 145.0);
                        \path[,fill,
                              black!7]
                            (axis cs:-180.0, 145.0)
                            rectangle (axis cs:-175.0, 150.0);
                        \path[,fill,
                              black!12]
                            (axis cs:-180.0, 150.0)
                            rectangle (axis cs:-175.0, 155.0);
                        \path[,fill,
                              black!20]
                            (axis cs:-180.0, 155.0)
                            rectangle (axis cs:-175.0, 160.0);
                        \path[,fill,
                              black!23]
                            (axis cs:-180.0, 160.0)
                            rectangle (axis cs:-175.0, 165.0);
                        \path[,fill,
                              black!33]
                            (axis cs:-180.0, 165.0)
                            rectangle (axis cs:-175.0, 170.0);
                        \path[,fill,
                              black!52]
                            (axis cs:-180.0, 170.0)
                            rectangle (axis cs:-175.0, 175.0);
                        \path[,fill,
                              black!70]
                            (axis cs:-180.0, 175.0)
                            rectangle (axis cs:-175.0, 180.0);
                        \path[,fill,
                              black!65]
                            (axis cs:-175.0, -180.0)
                            rectangle (axis cs:-170.0, -175.0);
                        \path[,fill,
                              black!73]
                            (axis cs:-175.0, -175.0)
                            rectangle (axis cs:-170.0, -170.0);
                        \path[,fill,
                              black!55]
                            (axis cs:-175.0, -170.0)
                            rectangle (axis cs:-170.0, -165.0);
                        \path[,fill,
                              black!38]
                            (axis cs:-175.0, -165.0)
                            rectangle (axis cs:-170.0, -160.0);
                        \path[,fill,
                              black!27]
                            (axis cs:-175.0, -160.0)
                            rectangle (axis cs:-170.0, -155.0);
                        \path[,fill,
                              black!19]
                            (axis cs:-175.0, -155.0)
                            rectangle (axis cs:-170.0, -150.0);
                        \path[,fill,
                              black!18]
                            (axis cs:-175.0, -150.0)
                            rectangle (axis cs:-170.0, -145.0);
                        \path[,fill,
                              black!13]
                            (axis cs:-175.0, -145.0)
                            rectangle (axis cs:-170.0, -140.0);
                        \path[,fill,
                              black!11]
                            (axis cs:-175.0, -140.0)
                            rectangle (axis cs:-170.0, -135.0);
                        \path[,fill,
                              black!6]
                            (axis cs:-175.0, -135.0)
                            rectangle (axis cs:-170.0, -130.0);
                        \path[,fill,
                              black!7]
                            (axis cs:-175.0, -130.0)
                            rectangle (axis cs:-170.0, -125.0);
                        \path[,fill,
                              black!5]
                            (axis cs:-175.0, -125.0)
                            rectangle (axis cs:-170.0, -120.0);
                        \path[,fill,
                              black!7]
                            (axis cs:-175.0, -120.0)
                            rectangle (axis cs:-170.0, -115.0);
                        \path[,fill,
                              black!2]
                            (axis cs:-175.0, -115.0)
                            rectangle (axis cs:-170.0, -110.0);
                        \path[,fill,
                              black!3]
                            (axis cs:-175.0, -110.0)
                            rectangle (axis cs:-170.0, -105.0);
                        \path[,fill,
                              black!1]
                            (axis cs:-175.0, -105.0)
                            rectangle (axis cs:-170.0, -100.0);
                        \path[,fill,
                              black!1]
                            (axis cs:-175.0, -95.0)
                            rectangle (axis cs:-170.0, -90.0);
                        \path[,fill,
                              black!1]
                            (axis cs:-175.0, -85.0)
                            rectangle (axis cs:-170.0, -80.0);
                        \path[,fill,
                              black!1]
                            (axis cs:-175.0, -80.0)
                            rectangle (axis cs:-170.0, -75.0);
                        \path[,fill,
                              black!1]
                            (axis cs:-175.0, -30.0)
                            rectangle (axis cs:-170.0, -25.0);
                        \path[,fill,
                              black!1]
                            (axis cs:-175.0, -25.0)
                            rectangle (axis cs:-170.0, -20.0);
                        \path[,fill,
                              black!1]
                            (axis cs:-175.0, -10.0)
                            rectangle (axis cs:-170.0, -5.0);
                        \path[,fill,
                              black!1]
                            (axis cs:-175.0, 10.0)
                            rectangle (axis cs:-170.0, 15.0);
                        \path[,fill,
                              black!1]
                            (axis cs:-175.0, 15.0)
                            rectangle (axis cs:-170.0, 20.0);
                        \path[,fill,
                              black!1]
                            (axis cs:-175.0, 20.0)
                            rectangle (axis cs:-170.0, 25.0);
                        \path[,fill,
                              black!1]
                            (axis cs:-175.0, 25.0)
                            rectangle (axis cs:-170.0, 30.0);
                        \path[,fill,
                              black!1]
                            (axis cs:-175.0, 30.0)
                            rectangle (axis cs:-170.0, 35.0);
                        \path[,fill,
                              black!2]
                            (axis cs:-175.0, 35.0)
                            rectangle (axis cs:-170.0, 40.0);
                        \path[,fill,
                              black!2]
                            (axis cs:-175.0, 40.0)
                            rectangle (axis cs:-170.0, 45.0);
                        \path[,fill,
                              black!2]
                            (axis cs:-175.0, 45.0)
                            rectangle (axis cs:-170.0, 50.0);
                        \path[,fill,
                              black!4]
                            (axis cs:-175.0, 50.0)
                            rectangle (axis cs:-170.0, 55.0);
                        \path[,fill,
                              black!2]
                            (axis cs:-175.0, 55.0)
                            rectangle (axis cs:-170.0, 60.0);
                        \path[,fill,
                              black!3]
                            (axis cs:-175.0, 60.0)
                            rectangle (axis cs:-170.0, 65.0);
                        \path[,fill,
                              black!2]
                            (axis cs:-175.0, 65.0)
                            rectangle (axis cs:-170.0, 70.0);
                        \path[,fill,
                              black!2]
                            (axis cs:-175.0, 70.0)
                            rectangle (axis cs:-170.0, 75.0);
                        \path[,fill,
                              black!3]
                            (axis cs:-175.0, 75.0)
                            rectangle (axis cs:-170.0, 80.0);
                        \path[,fill,
                              black!2]
                            (axis cs:-175.0, 80.0)
                            rectangle (axis cs:-170.0, 85.0);
                        \path[,fill,
                              black!1]
                            (axis cs:-175.0, 85.0)
                            rectangle (axis cs:-170.0, 90.0);
                        \path[,fill,
                              black!1]
                            (axis cs:-175.0, 90.0)
                            rectangle (axis cs:-170.0, 95.0);
                        \path[,fill,
                              black!3]
                            (axis cs:-175.0, 95.0)
                            rectangle (axis cs:-170.0, 100.0);
                        \path[,fill,
                              black!1]
                            (axis cs:-175.0, 100.0)
                            rectangle (axis cs:-170.0, 105.0);
                        \path[,fill,
                              black!1]
                            (axis cs:-175.0, 105.0)
                            rectangle (axis cs:-170.0, 110.0);
                        \path[,fill,
                              black!1]
                            (axis cs:-175.0, 115.0)
                            rectangle (axis cs:-170.0, 120.0);
                        \path[,fill,
                              black!1]
                            (axis cs:-175.0, 120.0)
                            rectangle (axis cs:-170.0, 125.0);
                        \path[,fill,
                              black!2]
                            (axis cs:-175.0, 125.0)
                            rectangle (axis cs:-170.0, 130.0);
                        \path[,fill,
                              black!2]
                            (axis cs:-175.0, 130.0)
                            rectangle (axis cs:-170.0, 135.0);
                        \path[,fill,
                              black!4]
                            (axis cs:-175.0, 135.0)
                            rectangle (axis cs:-170.0, 140.0);
                        \path[,fill,
                              black!6]
                            (axis cs:-175.0, 140.0)
                            rectangle (axis cs:-170.0, 145.0);
                        \path[,fill,
                              black!5]
                            (axis cs:-175.0, 145.0)
                            rectangle (axis cs:-170.0, 150.0);
                        \path[,fill,
                              black!4]
                            (axis cs:-175.0, 150.0)
                            rectangle (axis cs:-170.0, 155.0);
                        \path[,fill,
                              black!11]
                            (axis cs:-175.0, 155.0)
                            rectangle (axis cs:-170.0, 160.0);
                        \path[,fill,
                              black!18]
                            (axis cs:-175.0, 160.0)
                            rectangle (axis cs:-170.0, 165.0);
                        \path[,fill,
                              black!21]
                            (axis cs:-175.0, 165.0)
                            rectangle (axis cs:-170.0, 170.0);
                        \path[,fill,
                              black!35]
                            (axis cs:-175.0, 170.0)
                            rectangle (axis cs:-170.0, 175.0);
                        \path[,fill,
                              black!43]
                            (axis cs:-175.0, 175.0)
                            rectangle (axis cs:-170.0, 180.0);
                        \path[,fill,
                              black!54]
                            (axis cs:-170.0, -180.0)
                            rectangle (axis cs:-165.0, -175.0);
                        \path[,fill,
                              black!69]
                            (axis cs:-170.0, -175.0)
                            rectangle (axis cs:-165.0, -170.0);
                        \path[,fill,
                              black!71]
                            (axis cs:-170.0, -170.0)
                            rectangle (axis cs:-165.0, -165.0);
                        \path[,fill,
                              black!54]
                            (axis cs:-170.0, -165.0)
                            rectangle (axis cs:-165.0, -160.0);
                        \path[,fill,
                              black!38]
                            (axis cs:-170.0, -160.0)
                            rectangle (axis cs:-165.0, -155.0);
                        \path[,fill,
                              black!27]
                            (axis cs:-170.0, -155.0)
                            rectangle (axis cs:-165.0, -150.0);
                        \path[,fill,
                              black!16]
                            (axis cs:-170.0, -150.0)
                            rectangle (axis cs:-165.0, -145.0);
                        \path[,fill,
                              black!13]
                            (axis cs:-170.0, -145.0)
                            rectangle (axis cs:-165.0, -140.0);
                        \path[,fill,
                              black!11]
                            (axis cs:-170.0, -140.0)
                            rectangle (axis cs:-165.0, -135.0);
                        \path[,fill,
                              black!7]
                            (axis cs:-170.0, -135.0)
                            rectangle (axis cs:-165.0, -130.0);
                        \path[,fill,
                              black!7]
                            (axis cs:-170.0, -130.0)
                            rectangle (axis cs:-165.0, -125.0);
                        \path[,fill,
                              black!7]
                            (axis cs:-170.0, -125.0)
                            rectangle (axis cs:-165.0, -120.0);
                        \path[,fill,
                              black!7]
                            (axis cs:-170.0, -120.0)
                            rectangle (axis cs:-165.0, -115.0);
                        \path[,fill,
                              black!3]
                            (axis cs:-170.0, -115.0)
                            rectangle (axis cs:-165.0, -110.0);
                        \path[,fill,
                              black!3]
                            (axis cs:-170.0, -110.0)
                            rectangle (axis cs:-165.0, -105.0);
                        \path[,fill,
                              black!2]
                            (axis cs:-170.0, -105.0)
                            rectangle (axis cs:-165.0, -100.0);
                        \path[,fill,
                              black!2]
                            (axis cs:-170.0, -100.0)
                            rectangle (axis cs:-165.0, -95.0);
                        \path[,fill,
                              black!1]
                            (axis cs:-170.0, -85.0)
                            rectangle (axis cs:-165.0, -80.0);
                        \path[,fill,
                              black!1]
                            (axis cs:-170.0, -80.0)
                            rectangle (axis cs:-165.0, -75.0);
                        \path[,fill,
                              black!1]
                            (axis cs:-170.0, -70.0)
                            rectangle (axis cs:-165.0, -65.0);
                        \path[,fill,
                              black!1]
                            (axis cs:-170.0, -60.0)
                            rectangle (axis cs:-165.0, -55.0);
                        \path[,fill,
                              black!1]
                            (axis cs:-170.0, -45.0)
                            rectangle (axis cs:-165.0, -40.0);
                        \path[,fill,
                              black!1]
                            (axis cs:-170.0, -30.0)
                            rectangle (axis cs:-165.0, -25.0);
                        \path[,fill,
                              black!1]
                            (axis cs:-170.0, 0.0)
                            rectangle (axis cs:-165.0, 5.0);
                        \path[,fill,
                              black!1]
                            (axis cs:-170.0, 15.0)
                            rectangle (axis cs:-165.0, 20.0);
                        \path[,fill,
                              black!2]
                            (axis cs:-170.0, 25.0)
                            rectangle (axis cs:-165.0, 30.0);
                        \path[,fill,
                              black!1]
                            (axis cs:-170.0, 30.0)
                            rectangle (axis cs:-165.0, 35.0);
                        \path[,fill,
                              black!1]
                            (axis cs:-170.0, 35.0)
                            rectangle (axis cs:-165.0, 40.0);
                        \path[,fill,
                              black!2]
                            (axis cs:-170.0, 40.0)
                            rectangle (axis cs:-165.0, 45.0);
                        \path[,fill,
                              black!1]
                            (axis cs:-170.0, 45.0)
                            rectangle (axis cs:-165.0, 50.0);
                        \path[,fill,
                              black!2]
                            (axis cs:-170.0, 50.0)
                            rectangle (axis cs:-165.0, 55.0);
                        \path[,fill,
                              black!3]
                            (axis cs:-170.0, 55.0)
                            rectangle (axis cs:-165.0, 60.0);
                        \path[,fill,
                              black!1]
                            (axis cs:-170.0, 60.0)
                            rectangle (axis cs:-165.0, 65.0);
                        \path[,fill,
                              black!1]
                            (axis cs:-170.0, 65.0)
                            rectangle (axis cs:-165.0, 70.0);
                        \path[,fill,
                              black!2]
                            (axis cs:-170.0, 75.0)
                            rectangle (axis cs:-165.0, 80.0);
                        \path[,fill,
                              black!1]
                            (axis cs:-170.0, 80.0)
                            rectangle (axis cs:-165.0, 85.0);
                        \path[,fill,
                              black!1]
                            (axis cs:-170.0, 85.0)
                            rectangle (axis cs:-165.0, 90.0);
                        \path[,fill,
                              black!1]
                            (axis cs:-170.0, 90.0)
                            rectangle (axis cs:-165.0, 95.0);
                        \path[,fill,
                              black!2]
                            (axis cs:-170.0, 95.0)
                            rectangle (axis cs:-165.0, 100.0);
                        \path[,fill,
                              black!1]
                            (axis cs:-170.0, 105.0)
                            rectangle (axis cs:-165.0, 110.0);
                        \path[,fill,
                              black!3]
                            (axis cs:-170.0, 110.0)
                            rectangle (axis cs:-165.0, 115.0);
                        \path[,fill,
                              black!2]
                            (axis cs:-170.0, 120.0)
                            rectangle (axis cs:-165.0, 125.0);
                        \path[,fill,
                              black!1]
                            (axis cs:-170.0, 125.0)
                            rectangle (axis cs:-165.0, 130.0);
                        \path[,fill,
                              black!5]
                            (axis cs:-170.0, 130.0)
                            rectangle (axis cs:-165.0, 135.0);
                        \path[,fill,
                              black!2]
                            (axis cs:-170.0, 135.0)
                            rectangle (axis cs:-165.0, 140.0);
                        \path[,fill,
                              black!1]
                            (axis cs:-170.0, 140.0)
                            rectangle (axis cs:-165.0, 145.0);
                        \path[,fill,
                              black!4]
                            (axis cs:-170.0, 145.0)
                            rectangle (axis cs:-165.0, 150.0);
                        \path[,fill,
                              black!4]
                            (axis cs:-170.0, 150.0)
                            rectangle (axis cs:-165.0, 155.0);
                        \path[,fill,
                              black!4]
                            (axis cs:-170.0, 155.0)
                            rectangle (axis cs:-165.0, 160.0);
                        \path[,fill,
                              black!9]
                            (axis cs:-170.0, 160.0)
                            rectangle (axis cs:-165.0, 165.0);
                        \path[,fill,
                              black!17]
                            (axis cs:-170.0, 165.0)
                            rectangle (axis cs:-165.0, 170.0);
                        \path[,fill,
                              black!29]
                            (axis cs:-170.0, 170.0)
                            rectangle (axis cs:-165.0, 175.0);
                        \path[,fill,
                              black!26]
                            (axis cs:-170.0, 175.0)
                            rectangle (axis cs:-165.0, 180.0);
                        \path[,fill,
                              black!30]
                            (axis cs:-165.0, -180.0)
                            rectangle (axis cs:-160.0, -175.0);
                        \path[,fill,
                              black!45]
                            (axis cs:-165.0, -175.0)
                            rectangle (axis cs:-160.0, -170.0);
                        \path[,fill,
                              black!66]
                            (axis cs:-165.0, -170.0)
                            rectangle (axis cs:-160.0, -165.0);
                        \path[,fill,
                              black!64]
                            (axis cs:-165.0, -165.0)
                            rectangle (axis cs:-160.0, -160.0);
                        \path[,fill,
                              black!62]
                            (axis cs:-165.0, -160.0)
                            rectangle (axis cs:-160.0, -155.0);
                        \path[,fill,
                              black!36]
                            (axis cs:-165.0, -155.0)
                            rectangle (axis cs:-160.0, -150.0);
                        \path[,fill,
                              black!25]
                            (axis cs:-165.0, -150.0)
                            rectangle (axis cs:-160.0, -145.0);
                        \path[,fill,
                              black!16]
                            (axis cs:-165.0, -145.0)
                            rectangle (axis cs:-160.0, -140.0);
                        \path[,fill,
                              black!17]
                            (axis cs:-165.0, -140.0)
                            rectangle (axis cs:-160.0, -135.0);
                        \path[,fill,
                              black!12]
                            (axis cs:-165.0, -135.0)
                            rectangle (axis cs:-160.0, -130.0);
                        \path[,fill,
                              black!14]
                            (axis cs:-165.0, -130.0)
                            rectangle (axis cs:-160.0, -125.0);
                        \path[,fill,
                              black!10]
                            (axis cs:-165.0, -125.0)
                            rectangle (axis cs:-160.0, -120.0);
                        \path[,fill,
                              black!8]
                            (axis cs:-165.0, -120.0)
                            rectangle (axis cs:-160.0, -115.0);
                        \path[,fill,
                              black!5]
                            (axis cs:-165.0, -115.0)
                            rectangle (axis cs:-160.0, -110.0);
                        \path[,fill,
                              black!6]
                            (axis cs:-165.0, -110.0)
                            rectangle (axis cs:-160.0, -105.0);
                        \path[,fill,
                              black!2]
                            (axis cs:-165.0, -105.0)
                            rectangle (axis cs:-160.0, -100.0);
                        \path[,fill,
                              black!2]
                            (axis cs:-165.0, -100.0)
                            rectangle (axis cs:-160.0, -95.0);
                        \path[,fill,
                              black!1]
                            (axis cs:-165.0, -95.0)
                            rectangle (axis cs:-160.0, -90.0);
                        \path[,fill,
                              black!1]
                            (axis cs:-165.0, -80.0)
                            rectangle (axis cs:-160.0, -75.0);
                        \path[,fill,
                              black!1]
                            (axis cs:-165.0, -65.0)
                            rectangle (axis cs:-160.0, -60.0);
                        \path[,fill,
                              black!1]
                            (axis cs:-165.0, -40.0)
                            rectangle (axis cs:-160.0, -35.0);
                        \path[,fill,
                              black!1]
                            (axis cs:-165.0, 20.0)
                            rectangle (axis cs:-160.0, 25.0);
                        \path[,fill,
                              black!2]
                            (axis cs:-165.0, 25.0)
                            rectangle (axis cs:-160.0, 30.0);
                        \path[,fill,
                              black!1]
                            (axis cs:-165.0, 30.0)
                            rectangle (axis cs:-160.0, 35.0);
                        \path[,fill,
                              black!1]
                            (axis cs:-165.0, 35.0)
                            rectangle (axis cs:-160.0, 40.0);
                        \path[,fill,
                              black!3]
                            (axis cs:-165.0, 40.0)
                            rectangle (axis cs:-160.0, 45.0);
                        \path[,fill,
                              black!1]
                            (axis cs:-165.0, 50.0)
                            rectangle (axis cs:-160.0, 55.0);
                        \path[,fill,
                              black!2]
                            (axis cs:-165.0, 55.0)
                            rectangle (axis cs:-160.0, 60.0);
                        \path[,fill,
                              black!1]
                            (axis cs:-165.0, 70.0)
                            rectangle (axis cs:-160.0, 75.0);
                        \path[,fill,
                              black!1]
                            (axis cs:-165.0, 80.0)
                            rectangle (axis cs:-160.0, 85.0);
                        \path[,fill,
                              black!2]
                            (axis cs:-165.0, 85.0)
                            rectangle (axis cs:-160.0, 90.0);
                        \path[,fill,
                              black!1]
                            (axis cs:-165.0, 90.0)
                            rectangle (axis cs:-160.0, 95.0);
                        \path[,fill,
                              black!1]
                            (axis cs:-165.0, 95.0)
                            rectangle (axis cs:-160.0, 100.0);
                        \path[,fill,
                              black!1]
                            (axis cs:-165.0, 100.0)
                            rectangle (axis cs:-160.0, 105.0);
                        \path[,fill,
                              black!1]
                            (axis cs:-165.0, 110.0)
                            rectangle (axis cs:-160.0, 115.0);
                        \path[,fill,
                              black!1]
                            (axis cs:-165.0, 115.0)
                            rectangle (axis cs:-160.0, 120.0);
                        \path[,fill,
                              black!1]
                            (axis cs:-165.0, 120.0)
                            rectangle (axis cs:-160.0, 125.0);
                        \path[,fill,
                              black!1]
                            (axis cs:-165.0, 130.0)
                            rectangle (axis cs:-160.0, 135.0);
                        \path[,fill,
                              black!2]
                            (axis cs:-165.0, 135.0)
                            rectangle (axis cs:-160.0, 140.0);
                        \path[,fill,
                              black!1]
                            (axis cs:-165.0, 140.0)
                            rectangle (axis cs:-160.0, 145.0);
                        \path[,fill,
                              black!4]
                            (axis cs:-165.0, 145.0)
                            rectangle (axis cs:-160.0, 150.0);
                        \path[,fill,
                              black!2]
                            (axis cs:-165.0, 150.0)
                            rectangle (axis cs:-160.0, 155.0);
                        \path[,fill,
                              black!5]
                            (axis cs:-165.0, 155.0)
                            rectangle (axis cs:-160.0, 160.0);
                        \path[,fill,
                              black!9]
                            (axis cs:-165.0, 160.0)
                            rectangle (axis cs:-160.0, 165.0);
                        \path[,fill,
                              black!16]
                            (axis cs:-165.0, 165.0)
                            rectangle (axis cs:-160.0, 170.0);
                        \path[,fill,
                              black!17]
                            (axis cs:-165.0, 170.0)
                            rectangle (axis cs:-160.0, 175.0);
                        \path[,fill,
                              black!22]
                            (axis cs:-165.0, 175.0)
                            rectangle (axis cs:-160.0, 180.0);
                        \path[,fill,
                              black!24]
                            (axis cs:-160.0, -180.0)
                            rectangle (axis cs:-155.0, -175.0);
                        \path[,fill,
                              black!31]
                            (axis cs:-160.0, -175.0)
                            rectangle (axis cs:-155.0, -170.0);
                        \path[,fill,
                              black!49]
                            (axis cs:-160.0, -170.0)
                            rectangle (axis cs:-155.0, -165.0);
                        \path[,fill,
                              black!58]
                            (axis cs:-160.0, -165.0)
                            rectangle (axis cs:-155.0, -160.0);
                        \path[,fill,
                              black!73]
                            (axis cs:-160.0, -160.0)
                            rectangle (axis cs:-155.0, -155.0);
                        \path[,fill,
                              black!60]
                            (axis cs:-160.0, -155.0)
                            rectangle (axis cs:-155.0, -150.0);
                        \path[,fill,
                              black!44]
                            (axis cs:-160.0, -150.0)
                            rectangle (axis cs:-155.0, -145.0);
                        \path[,fill,
                              black!28]
                            (axis cs:-160.0, -145.0)
                            rectangle (axis cs:-155.0, -140.0);
                        \path[,fill,
                              black!18]
                            (axis cs:-160.0, -140.0)
                            rectangle (axis cs:-155.0, -135.0);
                        \path[,fill,
                              black!16]
                            (axis cs:-160.0, -135.0)
                            rectangle (axis cs:-155.0, -130.0);
                        \path[,fill,
                              black!15]
                            (axis cs:-160.0, -130.0)
                            rectangle (axis cs:-155.0, -125.0);
                        \path[,fill,
                              black!14]
                            (axis cs:-160.0, -125.0)
                            rectangle (axis cs:-155.0, -120.0);
                        \path[,fill,
                              black!12]
                            (axis cs:-160.0, -120.0)
                            rectangle (axis cs:-155.0, -115.0);
                        \path[,fill,
                              black!5]
                            (axis cs:-160.0, -115.0)
                            rectangle (axis cs:-155.0, -110.0);
                        \path[,fill,
                              black!3]
                            (axis cs:-160.0, -110.0)
                            rectangle (axis cs:-155.0, -105.0);
                        \path[,fill,
                              black!4]
                            (axis cs:-160.0, -105.0)
                            rectangle (axis cs:-155.0, -100.0);
                        \path[,fill,
                              black!2]
                            (axis cs:-160.0, -100.0)
                            rectangle (axis cs:-155.0, -95.0);
                        \path[,fill,
                              black!2]
                            (axis cs:-160.0, -95.0)
                            rectangle (axis cs:-155.0, -90.0);
                        \path[,fill,
                              black!2]
                            (axis cs:-160.0, -90.0)
                            rectangle (axis cs:-155.0, -85.0);
                        \path[,fill,
                              black!1]
                            (axis cs:-160.0, -85.0)
                            rectangle (axis cs:-155.0, -80.0);
                        \path[,fill,
                              black!1]
                            (axis cs:-160.0, -65.0)
                            rectangle (axis cs:-155.0, -60.0);
                        \path[,fill,
                              black!1]
                            (axis cs:-160.0, -50.0)
                            rectangle (axis cs:-155.0, -45.0);
                        \path[,fill,
                              black!1]
                            (axis cs:-160.0, -30.0)
                            rectangle (axis cs:-155.0, -25.0);
                        \path[,fill,
                              black!1]
                            (axis cs:-160.0, -25.0)
                            rectangle (axis cs:-155.0, -20.0);
                        \path[,fill,
                              black!1]
                            (axis cs:-160.0, -10.0)
                            rectangle (axis cs:-155.0, -5.0);
                        \path[,fill,
                              black!1]
                            (axis cs:-160.0, 5.0)
                            rectangle (axis cs:-155.0, 10.0);
                        \path[,fill,
                              black!2]
                            (axis cs:-160.0, 15.0)
                            rectangle (axis cs:-155.0, 20.0);
                        \path[,fill,
                              black!1]
                            (axis cs:-160.0, 20.0)
                            rectangle (axis cs:-155.0, 25.0);
                        \path[,fill,
                              black!3]
                            (axis cs:-160.0, 25.0)
                            rectangle (axis cs:-155.0, 30.0);
                        \path[,fill,
                              black!4]
                            (axis cs:-160.0, 30.0)
                            rectangle (axis cs:-155.0, 35.0);
                        \path[,fill,
                              black!4]
                            (axis cs:-160.0, 35.0)
                            rectangle (axis cs:-155.0, 40.0);
                        \path[,fill,
                              black!3]
                            (axis cs:-160.0, 40.0)
                            rectangle (axis cs:-155.0, 45.0);
                        \path[,fill,
                              black!1]
                            (axis cs:-160.0, 45.0)
                            rectangle (axis cs:-155.0, 50.0);
                        \path[,fill,
                              black!2]
                            (axis cs:-160.0, 55.0)
                            rectangle (axis cs:-155.0, 60.0);
                        \path[,fill,
                              black!1]
                            (axis cs:-160.0, 60.0)
                            rectangle (axis cs:-155.0, 65.0);
                        \path[,fill,
                              black!2]
                            (axis cs:-160.0, 65.0)
                            rectangle (axis cs:-155.0, 70.0);
                        \path[,fill,
                              black!1]
                            (axis cs:-160.0, 70.0)
                            rectangle (axis cs:-155.0, 75.0);
                        \path[,fill,
                              black!1]
                            (axis cs:-160.0, 75.0)
                            rectangle (axis cs:-155.0, 80.0);
                        \path[,fill,
                              black!1]
                            (axis cs:-160.0, 85.0)
                            rectangle (axis cs:-155.0, 90.0);
                        \path[,fill,
                              black!1]
                            (axis cs:-160.0, 115.0)
                            rectangle (axis cs:-155.0, 120.0);
                        \path[,fill,
                              black!2]
                            (axis cs:-160.0, 120.0)
                            rectangle (axis cs:-155.0, 125.0);
                        \path[,fill,
                              black!1]
                            (axis cs:-160.0, 125.0)
                            rectangle (axis cs:-155.0, 130.0);
                        \path[,fill,
                              black!3]
                            (axis cs:-160.0, 130.0)
                            rectangle (axis cs:-155.0, 135.0);
                        \path[,fill,
                              black!2]
                            (axis cs:-160.0, 135.0)
                            rectangle (axis cs:-155.0, 140.0);
                        \path[,fill,
                              black!1]
                            (axis cs:-160.0, 140.0)
                            rectangle (axis cs:-155.0, 145.0);
                        \path[,fill,
                              black!4]
                            (axis cs:-160.0, 145.0)
                            rectangle (axis cs:-155.0, 150.0);
                        \path[,fill,
                              black!4]
                            (axis cs:-160.0, 150.0)
                            rectangle (axis cs:-155.0, 155.0);
                        \path[,fill,
                              black!2]
                            (axis cs:-160.0, 155.0)
                            rectangle (axis cs:-155.0, 160.0);
                        \path[,fill,
                              black!6]
                            (axis cs:-160.0, 160.0)
                            rectangle (axis cs:-155.0, 165.0);
                        \path[,fill,
                              black!11]
                            (axis cs:-160.0, 165.0)
                            rectangle (axis cs:-155.0, 170.0);
                        \path[,fill,
                              black!11]
                            (axis cs:-160.0, 170.0)
                            rectangle (axis cs:-155.0, 175.0);
                        \path[,fill,
                              black!18]
                            (axis cs:-160.0, 175.0)
                            rectangle (axis cs:-155.0, 180.0);
                        \path[,fill,
                              black!16]
                            (axis cs:-155.0, -180.0)
                            rectangle (axis cs:-150.0, -175.0);
                        \path[,fill,
                              black!26]
                            (axis cs:-155.0, -175.0)
                            rectangle (axis cs:-150.0, -170.0);
                        \path[,fill,
                              black!28]
                            (axis cs:-155.0, -170.0)
                            rectangle (axis cs:-150.0, -165.0);
                        \path[,fill,
                              black!44]
                            (axis cs:-155.0, -165.0)
                            rectangle (axis cs:-150.0, -160.0);
                        \path[,fill,
                              black!69]
                            (axis cs:-155.0, -160.0)
                            rectangle (axis cs:-150.0, -155.0);
                        \path[,fill,
                              black!76]
                            (axis cs:-155.0, -155.0)
                            rectangle (axis cs:-150.0, -150.0);
                        \path[,fill,
                              black!59]
                            (axis cs:-155.0, -150.0)
                            rectangle (axis cs:-150.0, -145.0);
                        \path[,fill,
                              black!38]
                            (axis cs:-155.0, -145.0)
                            rectangle (axis cs:-150.0, -140.0);
                        \path[,fill,
                              black!29]
                            (axis cs:-155.0, -140.0)
                            rectangle (axis cs:-150.0, -135.0);
                        \path[,fill,
                              black!23]
                            (axis cs:-155.0, -135.0)
                            rectangle (axis cs:-150.0, -130.0);
                        \path[,fill,
                              black!13]
                            (axis cs:-155.0, -130.0)
                            rectangle (axis cs:-150.0, -125.0);
                        \path[,fill,
                              black!14]
                            (axis cs:-155.0, -125.0)
                            rectangle (axis cs:-150.0, -120.0);
                        \path[,fill,
                              black!15]
                            (axis cs:-155.0, -120.0)
                            rectangle (axis cs:-150.0, -115.0);
                        \path[,fill,
                              black!8]
                            (axis cs:-155.0, -115.0)
                            rectangle (axis cs:-150.0, -110.0);
                        \path[,fill,
                              black!7]
                            (axis cs:-155.0, -110.0)
                            rectangle (axis cs:-150.0, -105.0);
                        \path[,fill,
                              black!4]
                            (axis cs:-155.0, -105.0)
                            rectangle (axis cs:-150.0, -100.0);
                        \path[,fill,
                              black!2]
                            (axis cs:-155.0, -100.0)
                            rectangle (axis cs:-150.0, -95.0);
                        \path[,fill,
                              black!1]
                            (axis cs:-155.0, -95.0)
                            rectangle (axis cs:-150.0, -90.0);
                        \path[,fill,
                              black!2]
                            (axis cs:-155.0, -90.0)
                            rectangle (axis cs:-150.0, -85.0);
                        \path[,fill,
                              black!1]
                            (axis cs:-155.0, -50.0)
                            rectangle (axis cs:-150.0, -45.0);
                        \path[,fill,
                              black!1]
                            (axis cs:-155.0, -30.0)
                            rectangle (axis cs:-150.0, -25.0);
                        \path[,fill,
                              black!1]
                            (axis cs:-155.0, -25.0)
                            rectangle (axis cs:-150.0, -20.0);
                        \path[,fill,
                              black!2]
                            (axis cs:-155.0, -5.0)
                            rectangle (axis cs:-150.0, 0.0);
                        \path[,fill,
                              black!1]
                            (axis cs:-155.0, 5.0)
                            rectangle (axis cs:-150.0, 10.0);
                        \path[,fill,
                              black!1]
                            (axis cs:-155.0, 10.0)
                            rectangle (axis cs:-150.0, 15.0);
                        \path[,fill,
                              black!1]
                            (axis cs:-155.0, 15.0)
                            rectangle (axis cs:-150.0, 20.0);
                        \path[,fill,
                              black!1]
                            (axis cs:-155.0, 20.0)
                            rectangle (axis cs:-150.0, 25.0);
                        \path[,fill,
                              black!5]
                            (axis cs:-155.0, 25.0)
                            rectangle (axis cs:-150.0, 30.0);
                        \path[,fill,
                              black!4]
                            (axis cs:-155.0, 30.0)
                            rectangle (axis cs:-150.0, 35.0);
                        \path[,fill,
                              black!2]
                            (axis cs:-155.0, 35.0)
                            rectangle (axis cs:-150.0, 40.0);
                        \path[,fill,
                              black!1]
                            (axis cs:-155.0, 40.0)
                            rectangle (axis cs:-150.0, 45.0);
                        \path[,fill,
                              black!1]
                            (axis cs:-155.0, 45.0)
                            rectangle (axis cs:-150.0, 50.0);
                        \path[,fill,
                              black!1]
                            (axis cs:-155.0, 50.0)
                            rectangle (axis cs:-150.0, 55.0);
                        \path[,fill,
                              black!1]
                            (axis cs:-155.0, 55.0)
                            rectangle (axis cs:-150.0, 60.0);
                        \path[,fill,
                              black!1]
                            (axis cs:-155.0, 60.0)
                            rectangle (axis cs:-150.0, 65.0);
                        \path[,fill,
                              black!1]
                            (axis cs:-155.0, 65.0)
                            rectangle (axis cs:-150.0, 70.0);
                        \path[,fill,
                              black!1]
                            (axis cs:-155.0, 100.0)
                            rectangle (axis cs:-150.0, 105.0);
                        \path[,fill,
                              black!1]
                            (axis cs:-155.0, 105.0)
                            rectangle (axis cs:-150.0, 110.0);
                        \path[,fill,
                              black!1]
                            (axis cs:-155.0, 115.0)
                            rectangle (axis cs:-150.0, 120.0);
                        \path[,fill,
                              black!1]
                            (axis cs:-155.0, 120.0)
                            rectangle (axis cs:-150.0, 125.0);
                        \path[,fill,
                              black!1]
                            (axis cs:-155.0, 135.0)
                            rectangle (axis cs:-150.0, 140.0);
                        \path[,fill,
                              black!1]
                            (axis cs:-155.0, 140.0)
                            rectangle (axis cs:-150.0, 145.0);
                        \path[,fill,
                              black!2]
                            (axis cs:-155.0, 145.0)
                            rectangle (axis cs:-150.0, 150.0);
                        \path[,fill,
                              black!2]
                            (axis cs:-155.0, 150.0)
                            rectangle (axis cs:-150.0, 155.0);
                        \path[,fill,
                              black!3]
                            (axis cs:-155.0, 155.0)
                            rectangle (axis cs:-150.0, 160.0);
                        \path[,fill,
                              black!3]
                            (axis cs:-155.0, 160.0)
                            rectangle (axis cs:-150.0, 165.0);
                        \path[,fill,
                              black!7]
                            (axis cs:-155.0, 165.0)
                            rectangle (axis cs:-150.0, 170.0);
                        \path[,fill,
                              black!10]
                            (axis cs:-155.0, 170.0)
                            rectangle (axis cs:-150.0, 175.0);
                        \path[,fill,
                              black!14]
                            (axis cs:-155.0, 175.0)
                            rectangle (axis cs:-150.0, 180.0);
                        \path[,fill,
                              black!12]
                            (axis cs:-150.0, -180.0)
                            rectangle (axis cs:-145.0, -175.0);
                        \path[,fill,
                              black!15]
                            (axis cs:-150.0, -175.0)
                            rectangle (axis cs:-145.0, -170.0);
                        \path[,fill,
                              black!20]
                            (axis cs:-150.0, -170.0)
                            rectangle (axis cs:-145.0, -165.0);
                        \path[,fill,
                              black!28]
                            (axis cs:-150.0, -165.0)
                            rectangle (axis cs:-145.0, -160.0);
                        \path[,fill,
                              black!46]
                            (axis cs:-150.0, -160.0)
                            rectangle (axis cs:-145.0, -155.0);
                        \path[,fill,
                              black!59]
                            (axis cs:-150.0, -155.0)
                            rectangle (axis cs:-145.0, -150.0);
                        \path[,fill,
                              black!76]
                            (axis cs:-150.0, -150.0)
                            rectangle (axis cs:-145.0, -145.0);
                        \path[,fill,
                              black!69]
                            (axis cs:-150.0, -145.0)
                            rectangle (axis cs:-145.0, -140.0);
                        \path[,fill,
                              black!46]
                            (axis cs:-150.0, -140.0)
                            rectangle (axis cs:-145.0, -135.0);
                        \path[,fill,
                              black!23]
                            (axis cs:-150.0, -135.0)
                            rectangle (axis cs:-145.0, -130.0);
                        \path[,fill,
                              black!27]
                            (axis cs:-150.0, -130.0)
                            rectangle (axis cs:-145.0, -125.0);
                        \path[,fill,
                              black!15]
                            (axis cs:-150.0, -125.0)
                            rectangle (axis cs:-145.0, -120.0);
                        \path[,fill,
                              black!14]
                            (axis cs:-150.0, -120.0)
                            rectangle (axis cs:-145.0, -115.0);
                        \path[,fill,
                              black!12]
                            (axis cs:-150.0, -115.0)
                            rectangle (axis cs:-145.0, -110.0);
                        \path[,fill,
                              black!7]
                            (axis cs:-150.0, -110.0)
                            rectangle (axis cs:-145.0, -105.0);
                        \path[,fill,
                              black!5]
                            (axis cs:-150.0, -105.0)
                            rectangle (axis cs:-145.0, -100.0);
                        \path[,fill,
                              black!2]
                            (axis cs:-150.0, -100.0)
                            rectangle (axis cs:-145.0, -95.0);
                        \path[,fill,
                              black!3]
                            (axis cs:-150.0, -95.0)
                            rectangle (axis cs:-145.0, -90.0);
                        \path[,fill,
                              black!2]
                            (axis cs:-150.0, -90.0)
                            rectangle (axis cs:-145.0, -85.0);
                        \path[,fill,
                              black!1]
                            (axis cs:-150.0, -85.0)
                            rectangle (axis cs:-145.0, -80.0);
                        \path[,fill,
                              black!1]
                            (axis cs:-150.0, -80.0)
                            rectangle (axis cs:-145.0, -75.0);
                        \path[,fill,
                              black!1]
                            (axis cs:-150.0, -75.0)
                            rectangle (axis cs:-145.0, -70.0);
                        \path[,fill,
                              black!2]
                            (axis cs:-150.0, -70.0)
                            rectangle (axis cs:-145.0, -65.0);
                        \path[,fill,
                              black!1]
                            (axis cs:-150.0, -35.0)
                            rectangle (axis cs:-145.0, -30.0);
                        \path[,fill,
                              black!1]
                            (axis cs:-150.0, -20.0)
                            rectangle (axis cs:-145.0, -15.0);
                        \path[,fill,
                              black!1]
                            (axis cs:-150.0, -15.0)
                            rectangle (axis cs:-145.0, -10.0);
                        \path[,fill,
                              black!1]
                            (axis cs:-150.0, 0.0)
                            rectangle (axis cs:-145.0, 5.0);
                        \path[,fill,
                              black!1]
                            (axis cs:-150.0, 5.0)
                            rectangle (axis cs:-145.0, 10.0);
                        \path[,fill,
                              black!1]
                            (axis cs:-150.0, 10.0)
                            rectangle (axis cs:-145.0, 15.0);
                        \path[,fill,
                              black!2]
                            (axis cs:-150.0, 15.0)
                            rectangle (axis cs:-145.0, 20.0);
                        \path[,fill,
                              black!3]
                            (axis cs:-150.0, 20.0)
                            rectangle (axis cs:-145.0, 25.0);
                        \path[,fill,
                              black!3]
                            (axis cs:-150.0, 25.0)
                            rectangle (axis cs:-145.0, 30.0);
                        \path[,fill,
                              black!3]
                            (axis cs:-150.0, 30.0)
                            rectangle (axis cs:-145.0, 35.0);
                        \path[,fill,
                              black!2]
                            (axis cs:-150.0, 35.0)
                            rectangle (axis cs:-145.0, 40.0);
                        \path[,fill,
                              black!1]
                            (axis cs:-150.0, 40.0)
                            rectangle (axis cs:-145.0, 45.0);
                        \path[,fill,
                              black!1]
                            (axis cs:-150.0, 45.0)
                            rectangle (axis cs:-145.0, 50.0);
                        \path[,fill,
                              black!1]
                            (axis cs:-150.0, 55.0)
                            rectangle (axis cs:-145.0, 60.0);
                        \path[,fill,
                              black!1]
                            (axis cs:-150.0, 75.0)
                            rectangle (axis cs:-145.0, 80.0);
                        \path[,fill,
                              black!1]
                            (axis cs:-150.0, 95.0)
                            rectangle (axis cs:-145.0, 100.0);
                        \path[,fill,
                              black!1]
                            (axis cs:-150.0, 140.0)
                            rectangle (axis cs:-145.0, 145.0);
                        \path[,fill,
                              black!2]
                            (axis cs:-150.0, 145.0)
                            rectangle (axis cs:-145.0, 150.0);
                        \path[,fill,
                              black!1]
                            (axis cs:-150.0, 150.0)
                            rectangle (axis cs:-145.0, 155.0);
                        \path[,fill,
                              black!2]
                            (axis cs:-150.0, 155.0)
                            rectangle (axis cs:-145.0, 160.0);
                        \path[,fill,
                              black!4]
                            (axis cs:-150.0, 160.0)
                            rectangle (axis cs:-145.0, 165.0);
                        \path[,fill,
                              black!8]
                            (axis cs:-150.0, 165.0)
                            rectangle (axis cs:-145.0, 170.0);
                        \path[,fill,
                              black!10]
                            (axis cs:-150.0, 170.0)
                            rectangle (axis cs:-145.0, 175.0);
                        \path[,fill,
                              black!14]
                            (axis cs:-150.0, 175.0)
                            rectangle (axis cs:-145.0, 180.0);
                        \path[,fill,
                              black!9]
                            (axis cs:-145.0, -180.0)
                            rectangle (axis cs:-140.0, -175.0);
                        \path[,fill,
                              black!14]
                            (axis cs:-145.0, -175.0)
                            rectangle (axis cs:-140.0, -170.0);
                        \path[,fill,
                              black!17]
                            (axis cs:-145.0, -170.0)
                            rectangle (axis cs:-140.0, -165.0);
                        \path[,fill,
                              black!22]
                            (axis cs:-145.0, -165.0)
                            rectangle (axis cs:-140.0, -160.0);
                        \path[,fill,
                              black!26]
                            (axis cs:-145.0, -160.0)
                            rectangle (axis cs:-140.0, -155.0);
                        \path[,fill,
                              black!35]
                            (axis cs:-145.0, -155.0)
                            rectangle (axis cs:-140.0, -150.0);
                        \path[,fill,
                              black!59]
                            (axis cs:-145.0, -150.0)
                            rectangle (axis cs:-140.0, -145.0);
                        \path[,fill,
                              black!73]
                            (axis cs:-145.0, -145.0)
                            rectangle (axis cs:-140.0, -140.0);
                        \path[,fill,
                              black!69]
                            (axis cs:-145.0, -140.0)
                            rectangle (axis cs:-140.0, -135.0);
                        \path[,fill,
                              black!43]
                            (axis cs:-145.0, -135.0)
                            rectangle (axis cs:-140.0, -130.0);
                        \path[,fill,
                              black!32]
                            (axis cs:-145.0, -130.0)
                            rectangle (axis cs:-140.0, -125.0);
                        \path[,fill,
                              black!26]
                            (axis cs:-145.0, -125.0)
                            rectangle (axis cs:-140.0, -120.0);
                        \path[,fill,
                              black!18]
                            (axis cs:-145.0, -120.0)
                            rectangle (axis cs:-140.0, -115.0);
                        \path[,fill,
                              black!12]
                            (axis cs:-145.0, -115.0)
                            rectangle (axis cs:-140.0, -110.0);
                        \path[,fill,
                              black!11]
                            (axis cs:-145.0, -110.0)
                            rectangle (axis cs:-140.0, -105.0);
                        \path[,fill,
                              black!4]
                            (axis cs:-145.0, -105.0)
                            rectangle (axis cs:-140.0, -100.0);
                        \path[,fill,
                              black!4]
                            (axis cs:-145.0, -100.0)
                            rectangle (axis cs:-140.0, -95.0);
                        \path[,fill,
                              black!4]
                            (axis cs:-145.0, -95.0)
                            rectangle (axis cs:-140.0, -90.0);
                        \path[,fill,
                              black!1]
                            (axis cs:-145.0, -90.0)
                            rectangle (axis cs:-140.0, -85.0);
                        \path[,fill,
                              black!1]
                            (axis cs:-145.0, -85.0)
                            rectangle (axis cs:-140.0, -80.0);
                        \path[,fill,
                              black!2]
                            (axis cs:-145.0, -80.0)
                            rectangle (axis cs:-140.0, -75.0);
                        \path[,fill,
                              black!2]
                            (axis cs:-145.0, -75.0)
                            rectangle (axis cs:-140.0, -70.0);
                        \path[,fill,
                              black!2]
                            (axis cs:-145.0, -70.0)
                            rectangle (axis cs:-140.0, -65.0);
                        \path[,fill,
                              black!1]
                            (axis cs:-145.0, -65.0)
                            rectangle (axis cs:-140.0, -60.0);
                        \path[,fill,
                              black!1]
                            (axis cs:-145.0, -60.0)
                            rectangle (axis cs:-140.0, -55.0);
                        \path[,fill,
                              black!1]
                            (axis cs:-145.0, -55.0)
                            rectangle (axis cs:-140.0, -50.0);
                        \path[,fill,
                              black!1]
                            (axis cs:-145.0, -45.0)
                            rectangle (axis cs:-140.0, -40.0);
                        \path[,fill,
                              black!2]
                            (axis cs:-145.0, -40.0)
                            rectangle (axis cs:-140.0, -35.0);
                        \path[,fill,
                              black!2]
                            (axis cs:-145.0, -25.0)
                            rectangle (axis cs:-140.0, -20.0);
                        \path[,fill,
                              black!1]
                            (axis cs:-145.0, -20.0)
                            rectangle (axis cs:-140.0, -15.0);
                        \path[,fill,
                              black!1]
                            (axis cs:-145.0, -15.0)
                            rectangle (axis cs:-140.0, -10.0);
                        \path[,fill,
                              black!1]
                            (axis cs:-145.0, -10.0)
                            rectangle (axis cs:-140.0, -5.0);
                        \path[,fill,
                              black!1]
                            (axis cs:-145.0, -5.0)
                            rectangle (axis cs:-140.0, 0.0);
                        \path[,fill,
                              black!2]
                            (axis cs:-145.0, 0.0)
                            rectangle (axis cs:-140.0, 5.0);
                        \path[,fill,
                              black!1]
                            (axis cs:-145.0, 10.0)
                            rectangle (axis cs:-140.0, 15.0);
                        \path[,fill,
                              black!1]
                            (axis cs:-145.0, 15.0)
                            rectangle (axis cs:-140.0, 20.0);
                        \path[,fill,
                              black!4]
                            (axis cs:-145.0, 20.0)
                            rectangle (axis cs:-140.0, 25.0);
                        \path[,fill,
                              black!2]
                            (axis cs:-145.0, 25.0)
                            rectangle (axis cs:-140.0, 30.0);
                        \path[,fill,
                              black!1]
                            (axis cs:-145.0, 30.0)
                            rectangle (axis cs:-140.0, 35.0);
                        \path[,fill,
                              black!2]
                            (axis cs:-145.0, 40.0)
                            rectangle (axis cs:-140.0, 45.0);
                        \path[,fill,
                              black!1]
                            (axis cs:-145.0, 45.0)
                            rectangle (axis cs:-140.0, 50.0);
                        \path[,fill,
                              black!1]
                            (axis cs:-145.0, 60.0)
                            rectangle (axis cs:-140.0, 65.0);
                        \path[,fill,
                              black!1]
                            (axis cs:-145.0, 65.0)
                            rectangle (axis cs:-140.0, 70.0);
                        \path[,fill,
                              black!1]
                            (axis cs:-145.0, 80.0)
                            rectangle (axis cs:-140.0, 85.0);
                        \path[,fill,
                              black!1]
                            (axis cs:-145.0, 125.0)
                            rectangle (axis cs:-140.0, 130.0);
                        \path[,fill,
                              black!1]
                            (axis cs:-145.0, 130.0)
                            rectangle (axis cs:-140.0, 135.0);
                        \path[,fill,
                              black!1]
                            (axis cs:-145.0, 135.0)
                            rectangle (axis cs:-140.0, 140.0);
                        \path[,fill,
                              black!1]
                            (axis cs:-145.0, 140.0)
                            rectangle (axis cs:-140.0, 145.0);
                        \path[,fill,
                              black!1]
                            (axis cs:-145.0, 145.0)
                            rectangle (axis cs:-140.0, 150.0);
                        \path[,fill,
                              black!2]
                            (axis cs:-145.0, 150.0)
                            rectangle (axis cs:-140.0, 155.0);
                        \path[,fill,
                              black!3]
                            (axis cs:-145.0, 155.0)
                            rectangle (axis cs:-140.0, 160.0);
                        \path[,fill,
                              black!2]
                            (axis cs:-145.0, 160.0)
                            rectangle (axis cs:-140.0, 165.0);
                        \path[,fill,
                              black!2]
                            (axis cs:-145.0, 165.0)
                            rectangle (axis cs:-140.0, 170.0);
                        \path[,fill,
                              black!8]
                            (axis cs:-145.0, 170.0)
                            rectangle (axis cs:-140.0, 175.0);
                        \path[,fill,
                              black!8]
                            (axis cs:-145.0, 175.0)
                            rectangle (axis cs:-140.0, 180.0);
                        \path[,fill,
                              black!10]
                            (axis cs:-140.0, -180.0)
                            rectangle (axis cs:-135.0, -175.0);
                        \path[,fill,
                              black!8]
                            (axis cs:-140.0, -175.0)
                            rectangle (axis cs:-135.0, -170.0);
                        \path[,fill,
                              black!11]
                            (axis cs:-140.0, -170.0)
                            rectangle (axis cs:-135.0, -165.0);
                        \path[,fill,
                              black!16]
                            (axis cs:-140.0, -165.0)
                            rectangle (axis cs:-135.0, -160.0);
                        \path[,fill,
                              black!22]
                            (axis cs:-140.0, -160.0)
                            rectangle (axis cs:-135.0, -155.0);
                        \path[,fill,
                              black!29]
                            (axis cs:-140.0, -155.0)
                            rectangle (axis cs:-135.0, -150.0);
                        \path[,fill,
                              black!36]
                            (axis cs:-140.0, -150.0)
                            rectangle (axis cs:-135.0, -145.0);
                        \path[,fill,
                              black!68]
                            (axis cs:-140.0, -145.0)
                            rectangle (axis cs:-135.0, -140.0);
                        \path[,fill,
                              black!70]
                            (axis cs:-140.0, -140.0)
                            rectangle (axis cs:-135.0, -135.0);
                        \path[,fill,
                              black!61]
                            (axis cs:-140.0, -135.0)
                            rectangle (axis cs:-135.0, -130.0);
                        \path[,fill,
                              black!48]
                            (axis cs:-140.0, -130.0)
                            rectangle (axis cs:-135.0, -125.0);
                        \path[,fill,
                              black!32]
                            (axis cs:-140.0, -125.0)
                            rectangle (axis cs:-135.0, -120.0);
                        \path[,fill,
                              black!28]
                            (axis cs:-140.0, -120.0)
                            rectangle (axis cs:-135.0, -115.0);
                        \path[,fill,
                              black!22]
                            (axis cs:-140.0, -115.0)
                            rectangle (axis cs:-135.0, -110.0);
                        \path[,fill,
                              black!11]
                            (axis cs:-140.0, -110.0)
                            rectangle (axis cs:-135.0, -105.0);
                        \path[,fill,
                              black!8]
                            (axis cs:-140.0, -105.0)
                            rectangle (axis cs:-135.0, -100.0);
                        \path[,fill,
                              black!7]
                            (axis cs:-140.0, -100.0)
                            rectangle (axis cs:-135.0, -95.0);
                        \path[,fill,
                              black!2]
                            (axis cs:-140.0, -95.0)
                            rectangle (axis cs:-135.0, -90.0);
                        \path[,fill,
                              black!4]
                            (axis cs:-140.0, -90.0)
                            rectangle (axis cs:-135.0, -85.0);
                        \path[,fill,
                              black!2]
                            (axis cs:-140.0, -85.0)
                            rectangle (axis cs:-135.0, -80.0);
                        \path[,fill,
                              black!1]
                            (axis cs:-140.0, -80.0)
                            rectangle (axis cs:-135.0, -75.0);
                        \path[,fill,
                              black!1]
                            (axis cs:-140.0, -75.0)
                            rectangle (axis cs:-135.0, -70.0);
                        \path[,fill,
                              black!1]
                            (axis cs:-140.0, -70.0)
                            rectangle (axis cs:-135.0, -65.0);
                        \path[,fill,
                              black!1]
                            (axis cs:-140.0, -60.0)
                            rectangle (axis cs:-135.0, -55.0);
                        \path[,fill,
                              black!1]
                            (axis cs:-140.0, -50.0)
                            rectangle (axis cs:-135.0, -45.0);
                        \path[,fill,
                              black!1]
                            (axis cs:-140.0, -40.0)
                            rectangle (axis cs:-135.0, -35.0);
                        \path[,fill,
                              black!1]
                            (axis cs:-140.0, -35.0)
                            rectangle (axis cs:-135.0, -30.0);
                        \path[,fill,
                              black!1]
                            (axis cs:-140.0, -30.0)
                            rectangle (axis cs:-135.0, -25.0);
                        \path[,fill,
                              black!1]
                            (axis cs:-140.0, -20.0)
                            rectangle (axis cs:-135.0, -15.0);
                        \path[,fill,
                              black!1]
                            (axis cs:-140.0, -15.0)
                            rectangle (axis cs:-135.0, -10.0);
                        \path[,fill,
                              black!2]
                            (axis cs:-140.0, -10.0)
                            rectangle (axis cs:-135.0, -5.0);
                        \path[,fill,
                              black!1]
                            (axis cs:-140.0, -5.0)
                            rectangle (axis cs:-135.0, 0.0);
                        \path[,fill,
                              black!2]
                            (axis cs:-140.0, 0.0)
                            rectangle (axis cs:-135.0, 5.0);
                        \path[,fill,
                              black!2]
                            (axis cs:-140.0, 5.0)
                            rectangle (axis cs:-135.0, 10.0);
                        \path[,fill,
                              black!1]
                            (axis cs:-140.0, 10.0)
                            rectangle (axis cs:-135.0, 15.0);
                        \path[,fill,
                              black!4]
                            (axis cs:-140.0, 15.0)
                            rectangle (axis cs:-135.0, 20.0);
                        \path[,fill,
                              black!3]
                            (axis cs:-140.0, 20.0)
                            rectangle (axis cs:-135.0, 25.0);
                        \path[,fill,
                              black!2]
                            (axis cs:-140.0, 25.0)
                            rectangle (axis cs:-135.0, 30.0);
                        \path[,fill,
                              black!1]
                            (axis cs:-140.0, 30.0)
                            rectangle (axis cs:-135.0, 35.0);
                        \path[,fill,
                              black!1]
                            (axis cs:-140.0, 35.0)
                            rectangle (axis cs:-135.0, 40.0);
                        \path[,fill,
                              black!1]
                            (axis cs:-140.0, 40.0)
                            rectangle (axis cs:-135.0, 45.0);
                        \path[,fill,
                              black!1]
                            (axis cs:-140.0, 115.0)
                            rectangle (axis cs:-135.0, 120.0);
                        \path[,fill,
                              black!1]
                            (axis cs:-140.0, 125.0)
                            rectangle (axis cs:-135.0, 130.0);
                        \path[,fill,
                              black!2]
                            (axis cs:-140.0, 150.0)
                            rectangle (axis cs:-135.0, 155.0);
                        \path[,fill,
                              black!1]
                            (axis cs:-140.0, 155.0)
                            rectangle (axis cs:-135.0, 160.0);
                        \path[,fill,
                              black!2]
                            (axis cs:-140.0, 160.0)
                            rectangle (axis cs:-135.0, 165.0);
                        \path[,fill,
                              black!5]
                            (axis cs:-140.0, 165.0)
                            rectangle (axis cs:-135.0, 170.0);
                        \path[,fill,
                              black!5]
                            (axis cs:-140.0, 170.0)
                            rectangle (axis cs:-135.0, 175.0);
                        \path[,fill,
                              black!7]
                            (axis cs:-140.0, 175.0)
                            rectangle (axis cs:-135.0, 180.0);
                        \path[,fill,
                              black!6]
                            (axis cs:-135.0, -180.0)
                            rectangle (axis cs:-130.0, -175.0);
                        \path[,fill,
                              black!8]
                            (axis cs:-135.0, -175.0)
                            rectangle (axis cs:-130.0, -170.0);
                        \path[,fill,
                              black!11]
                            (axis cs:-135.0, -170.0)
                            rectangle (axis cs:-130.0, -165.0);
                        \path[,fill,
                              black!10]
                            (axis cs:-135.0, -165.0)
                            rectangle (axis cs:-130.0, -160.0);
                        \path[,fill,
                              black!16]
                            (axis cs:-135.0, -160.0)
                            rectangle (axis cs:-130.0, -155.0);
                        \path[,fill,
                              black!19]
                            (axis cs:-135.0, -155.0)
                            rectangle (axis cs:-130.0, -150.0);
                        \path[,fill,
                              black!30]
                            (axis cs:-135.0, -150.0)
                            rectangle (axis cs:-130.0, -145.0);
                        \path[,fill,
                              black!35]
                            (axis cs:-135.0, -145.0)
                            rectangle (axis cs:-130.0, -140.0);
                        \path[,fill,
                              black!49]
                            (axis cs:-135.0, -140.0)
                            rectangle (axis cs:-130.0, -135.0);
                        \path[,fill,
                              black!78]
                            (axis cs:-135.0, -135.0)
                            rectangle (axis cs:-130.0, -130.0);
                        \path[,fill,
                              black!64]
                            (axis cs:-135.0, -130.0)
                            rectangle (axis cs:-130.0, -125.0);
                        \path[,fill,
                              black!41]
                            (axis cs:-135.0, -125.0)
                            rectangle (axis cs:-130.0, -120.0);
                        \path[,fill,
                              black!33]
                            (axis cs:-135.0, -120.0)
                            rectangle (axis cs:-130.0, -115.0);
                        \path[,fill,
                              black!25]
                            (axis cs:-135.0, -115.0)
                            rectangle (axis cs:-130.0, -110.0);
                        \path[,fill,
                              black!17]
                            (axis cs:-135.0, -110.0)
                            rectangle (axis cs:-130.0, -105.0);
                        \path[,fill,
                              black!9]
                            (axis cs:-135.0, -105.0)
                            rectangle (axis cs:-130.0, -100.0);
                        \path[,fill,
                              black!4]
                            (axis cs:-135.0, -100.0)
                            rectangle (axis cs:-130.0, -95.0);
                        \path[,fill,
                              black!7]
                            (axis cs:-135.0, -95.0)
                            rectangle (axis cs:-130.0, -90.0);
                        \path[,fill,
                              black!5]
                            (axis cs:-135.0, -90.0)
                            rectangle (axis cs:-130.0, -85.0);
                        \path[,fill,
                              black!2]
                            (axis cs:-135.0, -85.0)
                            rectangle (axis cs:-130.0, -80.0);
                        \path[,fill,
                              black!4]
                            (axis cs:-135.0, -80.0)
                            rectangle (axis cs:-130.0, -75.0);
                        \path[,fill,
                              black!2]
                            (axis cs:-135.0, -75.0)
                            rectangle (axis cs:-130.0, -70.0);
                        \path[,fill,
                              black!2]
                            (axis cs:-135.0, -70.0)
                            rectangle (axis cs:-130.0, -65.0);
                        \path[,fill,
                              black!2]
                            (axis cs:-135.0, -65.0)
                            rectangle (axis cs:-130.0, -60.0);
                        \path[,fill,
                              black!1]
                            (axis cs:-135.0, -60.0)
                            rectangle (axis cs:-130.0, -55.0);
                        \path[,fill,
                              black!1]
                            (axis cs:-135.0, -55.0)
                            rectangle (axis cs:-130.0, -50.0);
                        \path[,fill,
                              black!1]
                            (axis cs:-135.0, -50.0)
                            rectangle (axis cs:-130.0, -45.0);
                        \path[,fill,
                              black!2]
                            (axis cs:-135.0, -45.0)
                            rectangle (axis cs:-130.0, -40.0);
                        \path[,fill,
                              black!1]
                            (axis cs:-135.0, -40.0)
                            rectangle (axis cs:-130.0, -35.0);
                        \path[,fill,
                              black!2]
                            (axis cs:-135.0, -35.0)
                            rectangle (axis cs:-130.0, -30.0);
                        \path[,fill,
                              black!2]
                            (axis cs:-135.0, -30.0)
                            rectangle (axis cs:-130.0, -25.0);
                        \path[,fill,
                              black!1]
                            (axis cs:-135.0, -25.0)
                            rectangle (axis cs:-130.0, -20.0);
                        \path[,fill,
                              black!1]
                            (axis cs:-135.0, -20.0)
                            rectangle (axis cs:-130.0, -15.0);
                        \path[,fill,
                              black!1]
                            (axis cs:-135.0, -15.0)
                            rectangle (axis cs:-130.0, -10.0);
                        \path[,fill,
                              black!2]
                            (axis cs:-135.0, -5.0)
                            rectangle (axis cs:-130.0, 0.0);
                        \path[,fill,
                              black!1]
                            (axis cs:-135.0, 0.0)
                            rectangle (axis cs:-130.0, 5.0);
                        \path[,fill,
                              black!1]
                            (axis cs:-135.0, 5.0)
                            rectangle (axis cs:-130.0, 10.0);
                        \path[,fill,
                              black!3]
                            (axis cs:-135.0, 10.0)
                            rectangle (axis cs:-130.0, 15.0);
                        \path[,fill,
                              black!5]
                            (axis cs:-135.0, 15.0)
                            rectangle (axis cs:-130.0, 20.0);
                        \path[,fill,
                              black!3]
                            (axis cs:-135.0, 20.0)
                            rectangle (axis cs:-130.0, 25.0);
                        \path[,fill,
                              black!2]
                            (axis cs:-135.0, 30.0)
                            rectangle (axis cs:-130.0, 35.0);
                        \path[,fill,
                              black!2]
                            (axis cs:-135.0, 35.0)
                            rectangle (axis cs:-130.0, 40.0);
                        \path[,fill,
                              black!1]
                            (axis cs:-135.0, 45.0)
                            rectangle (axis cs:-130.0, 50.0);
                        \path[,fill,
                              black!1]
                            (axis cs:-135.0, 65.0)
                            rectangle (axis cs:-130.0, 70.0);
                        \path[,fill,
                              black!1]
                            (axis cs:-135.0, 130.0)
                            rectangle (axis cs:-130.0, 135.0);
                        \path[,fill,
                              black!1]
                            (axis cs:-135.0, 135.0)
                            rectangle (axis cs:-130.0, 140.0);
                        \path[,fill,
                              black!2]
                            (axis cs:-135.0, 140.0)
                            rectangle (axis cs:-130.0, 145.0);
                        \path[,fill,
                              black!1]
                            (axis cs:-135.0, 150.0)
                            rectangle (axis cs:-130.0, 155.0);
                        \path[,fill,
                              black!1]
                            (axis cs:-135.0, 155.0)
                            rectangle (axis cs:-130.0, 160.0);
                        \path[,fill,
                              black!1]
                            (axis cs:-135.0, 160.0)
                            rectangle (axis cs:-130.0, 165.0);
                        \path[,fill,
                              black!3]
                            (axis cs:-135.0, 165.0)
                            rectangle (axis cs:-130.0, 170.0);
                        \path[,fill,
                              black!5]
                            (axis cs:-135.0, 170.0)
                            rectangle (axis cs:-130.0, 175.0);
                        \path[,fill,
                              black!4]
                            (axis cs:-135.0, 175.0)
                            rectangle (axis cs:-130.0, 180.0);
                        \path[,fill,
                              black!7]
                            (axis cs:-130.0, -180.0)
                            rectangle (axis cs:-125.0, -175.0);
                        \path[,fill,
                              black!5]
                            (axis cs:-130.0, -175.0)
                            rectangle (axis cs:-125.0, -170.0);
                        \path[,fill,
                              black!8]
                            (axis cs:-130.0, -170.0)
                            rectangle (axis cs:-125.0, -165.0);
                        \path[,fill,
                              black!9]
                            (axis cs:-130.0, -165.0)
                            rectangle (axis cs:-125.0, -160.0);
                        \path[,fill,
                              black!13]
                            (axis cs:-130.0, -160.0)
                            rectangle (axis cs:-125.0, -155.0);
                        \path[,fill,
                              black!19]
                            (axis cs:-130.0, -155.0)
                            rectangle (axis cs:-125.0, -150.0);
                        \path[,fill,
                              black!22]
                            (axis cs:-130.0, -150.0)
                            rectangle (axis cs:-125.0, -145.0);
                        \path[,fill,
                              black!25]
                            (axis cs:-130.0, -145.0)
                            rectangle (axis cs:-125.0, -140.0);
                        \path[,fill,
                              black!35]
                            (axis cs:-130.0, -140.0)
                            rectangle (axis cs:-125.0, -135.0);
                        \path[,fill,
                              black!52]
                            (axis cs:-130.0, -135.0)
                            rectangle (axis cs:-125.0, -130.0);
                        \path[,fill,
                              black!70]
                            (axis cs:-130.0, -130.0)
                            rectangle (axis cs:-125.0, -125.0);
                        \path[,fill,
                              black!67]
                            (axis cs:-130.0, -125.0)
                            rectangle (axis cs:-125.0, -120.0);
                        \path[,fill,
                              black!43]
                            (axis cs:-130.0, -120.0)
                            rectangle (axis cs:-125.0, -115.0);
                        \path[,fill,
                              black!28]
                            (axis cs:-130.0, -115.0)
                            rectangle (axis cs:-125.0, -110.0);
                        \path[,fill,
                              black!24]
                            (axis cs:-130.0, -110.0)
                            rectangle (axis cs:-125.0, -105.0);
                        \path[,fill,
                              black!19]
                            (axis cs:-130.0, -105.0)
                            rectangle (axis cs:-125.0, -100.0);
                        \path[,fill,
                              black!14]
                            (axis cs:-130.0, -100.0)
                            rectangle (axis cs:-125.0, -95.0);
                        \path[,fill,
                              black!8]
                            (axis cs:-130.0, -95.0)
                            rectangle (axis cs:-125.0, -90.0);
                        \path[,fill,
                              black!7]
                            (axis cs:-130.0, -90.0)
                            rectangle (axis cs:-125.0, -85.0);
                        \path[,fill,
                              black!6]
                            (axis cs:-130.0, -85.0)
                            rectangle (axis cs:-125.0, -80.0);
                        \path[,fill,
                              black!4]
                            (axis cs:-130.0, -80.0)
                            rectangle (axis cs:-125.0, -75.0);
                        \path[,fill,
                              black!2]
                            (axis cs:-130.0, -75.0)
                            rectangle (axis cs:-125.0, -70.0);
                        \path[,fill,
                              black!3]
                            (axis cs:-130.0, -70.0)
                            rectangle (axis cs:-125.0, -65.0);
                        \path[,fill,
                              black!3]
                            (axis cs:-130.0, -60.0)
                            rectangle (axis cs:-125.0, -55.0);
                        \path[,fill,
                              black!1]
                            (axis cs:-130.0, -55.0)
                            rectangle (axis cs:-125.0, -50.0);
                        \path[,fill,
                              black!2]
                            (axis cs:-130.0, -50.0)
                            rectangle (axis cs:-125.0, -45.0);
                        \path[,fill,
                              black!1]
                            (axis cs:-130.0, -40.0)
                            rectangle (axis cs:-125.0, -35.0);
                        \path[,fill,
                              black!1]
                            (axis cs:-130.0, -35.0)
                            rectangle (axis cs:-125.0, -30.0);
                        \path[,fill,
                              black!1]
                            (axis cs:-130.0, -30.0)
                            rectangle (axis cs:-125.0, -25.0);
                        \path[,fill,
                              black!1]
                            (axis cs:-130.0, -25.0)
                            rectangle (axis cs:-125.0, -20.0);
                        \path[,fill,
                              black!2]
                            (axis cs:-130.0, -20.0)
                            rectangle (axis cs:-125.0, -15.0);
                        \path[,fill,
                              black!1]
                            (axis cs:-130.0, -15.0)
                            rectangle (axis cs:-125.0, -10.0);
                        \path[,fill,
                              black!2]
                            (axis cs:-130.0, -10.0)
                            rectangle (axis cs:-125.0, -5.0);
                        \path[,fill,
                              black!3]
                            (axis cs:-130.0, -5.0)
                            rectangle (axis cs:-125.0, 0.0);
                        \path[,fill,
                              black!2]
                            (axis cs:-130.0, 0.0)
                            rectangle (axis cs:-125.0, 5.0);
                        \path[,fill,
                              black!3]
                            (axis cs:-130.0, 5.0)
                            rectangle (axis cs:-125.0, 10.0);
                        \path[,fill,
                              black!1]
                            (axis cs:-130.0, 10.0)
                            rectangle (axis cs:-125.0, 15.0);
                        \path[,fill,
                              black!1]
                            (axis cs:-130.0, 15.0)
                            rectangle (axis cs:-125.0, 20.0);
                        \path[,fill,
                              black!2]
                            (axis cs:-130.0, 20.0)
                            rectangle (axis cs:-125.0, 25.0);
                        \path[,fill,
                              black!1]
                            (axis cs:-130.0, 30.0)
                            rectangle (axis cs:-125.0, 35.0);
                        \path[,fill,
                              black!2]
                            (axis cs:-130.0, 40.0)
                            rectangle (axis cs:-125.0, 45.0);
                        \path[,fill,
                              black!1]
                            (axis cs:-130.0, 55.0)
                            rectangle (axis cs:-125.0, 60.0);
                        \path[,fill,
                              black!1]
                            (axis cs:-130.0, 150.0)
                            rectangle (axis cs:-125.0, 155.0);
                        \path[,fill,
                              black!1]
                            (axis cs:-130.0, 155.0)
                            rectangle (axis cs:-125.0, 160.0);
                        \path[,fill,
                              black!2]
                            (axis cs:-130.0, 160.0)
                            rectangle (axis cs:-125.0, 165.0);
                        \path[,fill,
                              black!2]
                            (axis cs:-130.0, 165.0)
                            rectangle (axis cs:-125.0, 170.0);
                        \path[,fill,
                              black!4]
                            (axis cs:-130.0, 170.0)
                            rectangle (axis cs:-125.0, 175.0);
                        \path[,fill,
                              black!8]
                            (axis cs:-130.0, 175.0)
                            rectangle (axis cs:-125.0, 180.0);
                        \path[,fill,
                              black!6]
                            (axis cs:-125.0, -180.0)
                            rectangle (axis cs:-120.0, -175.0);
                        \path[,fill,
                              black!10]
                            (axis cs:-125.0, -175.0)
                            rectangle (axis cs:-120.0, -170.0);
                        \path[,fill,
                              black!5]
                            (axis cs:-125.0, -170.0)
                            rectangle (axis cs:-120.0, -165.0);
                        \path[,fill,
                              black!8]
                            (axis cs:-125.0, -165.0)
                            rectangle (axis cs:-120.0, -160.0);
                        \path[,fill,
                              black!8]
                            (axis cs:-125.0, -160.0)
                            rectangle (axis cs:-120.0, -155.0);
                        \path[,fill,
                              black!16]
                            (axis cs:-125.0, -155.0)
                            rectangle (axis cs:-120.0, -150.0);
                        \path[,fill,
                              black!13]
                            (axis cs:-125.0, -150.0)
                            rectangle (axis cs:-120.0, -145.0);
                        \path[,fill,
                              black!17]
                            (axis cs:-125.0, -145.0)
                            rectangle (axis cs:-120.0, -140.0);
                        \path[,fill,
                              black!19]
                            (axis cs:-125.0, -140.0)
                            rectangle (axis cs:-120.0, -135.0);
                        \path[,fill,
                              black!36]
                            (axis cs:-125.0, -135.0)
                            rectangle (axis cs:-120.0, -130.0);
                        \path[,fill,
                              black!55]
                            (axis cs:-125.0, -130.0)
                            rectangle (axis cs:-120.0, -125.0);
                        \path[,fill,
                              black!86]
                            (axis cs:-125.0, -125.0)
                            rectangle (axis cs:-120.0, -120.0);
                        \path[,fill,
                              black!74]
                            (axis cs:-125.0, -120.0)
                            rectangle (axis cs:-120.0, -115.0);
                        \path[,fill,
                              black!48]
                            (axis cs:-125.0, -115.0)
                            rectangle (axis cs:-120.0, -110.0);
                        \path[,fill,
                              black!38]
                            (axis cs:-125.0, -110.0)
                            rectangle (axis cs:-120.0, -105.0);
                        \path[,fill,
                              black!20]
                            (axis cs:-125.0, -105.0)
                            rectangle (axis cs:-120.0, -100.0);
                        \path[,fill,
                              black!16]
                            (axis cs:-125.0, -100.0)
                            rectangle (axis cs:-120.0, -95.0);
                        \path[,fill,
                              black!8]
                            (axis cs:-125.0, -95.0)
                            rectangle (axis cs:-120.0, -90.0);
                        \path[,fill,
                              black!5]
                            (axis cs:-125.0, -90.0)
                            rectangle (axis cs:-120.0, -85.0);
                        \path[,fill,
                              black!4]
                            (axis cs:-125.0, -85.0)
                            rectangle (axis cs:-120.0, -80.0);
                        \path[,fill,
                              black!4]
                            (axis cs:-125.0, -80.0)
                            rectangle (axis cs:-120.0, -75.0);
                        \path[,fill,
                              black!4]
                            (axis cs:-125.0, -75.0)
                            rectangle (axis cs:-120.0, -70.0);
                        \path[,fill,
                              black!4]
                            (axis cs:-125.0, -70.0)
                            rectangle (axis cs:-120.0, -65.0);
                        \path[,fill,
                              black!3]
                            (axis cs:-125.0, -65.0)
                            rectangle (axis cs:-120.0, -60.0);
                        \path[,fill,
                              black!3]
                            (axis cs:-125.0, -60.0)
                            rectangle (axis cs:-120.0, -55.0);
                        \path[,fill,
                              black!1]
                            (axis cs:-125.0, -55.0)
                            rectangle (axis cs:-120.0, -50.0);
                        \path[,fill,
                              black!2]
                            (axis cs:-125.0, -50.0)
                            rectangle (axis cs:-120.0, -45.0);
                        \path[,fill,
                              black!2]
                            (axis cs:-125.0, -45.0)
                            rectangle (axis cs:-120.0, -40.0);
                        \path[,fill,
                              black!2]
                            (axis cs:-125.0, -40.0)
                            rectangle (axis cs:-120.0, -35.0);
                        \path[,fill,
                              black!2]
                            (axis cs:-125.0, -35.0)
                            rectangle (axis cs:-120.0, -30.0);
                        \path[,fill,
                              black!1]
                            (axis cs:-125.0, -30.0)
                            rectangle (axis cs:-120.0, -25.0);
                        \path[,fill,
                              black!1]
                            (axis cs:-125.0, -25.0)
                            rectangle (axis cs:-120.0, -20.0);
                        \path[,fill,
                              black!2]
                            (axis cs:-125.0, -20.0)
                            rectangle (axis cs:-120.0, -15.0);
                        \path[,fill,
                              black!2]
                            (axis cs:-125.0, -15.0)
                            rectangle (axis cs:-120.0, -10.0);
                        \path[,fill,
                              black!2]
                            (axis cs:-125.0, -10.0)
                            rectangle (axis cs:-120.0, -5.0);
                        \path[,fill,
                              black!2]
                            (axis cs:-125.0, -5.0)
                            rectangle (axis cs:-120.0, 0.0);
                        \path[,fill,
                              black!3]
                            (axis cs:-125.0, 0.0)
                            rectangle (axis cs:-120.0, 5.0);
                        \path[,fill,
                              black!2]
                            (axis cs:-125.0, 5.0)
                            rectangle (axis cs:-120.0, 10.0);
                        \path[,fill,
                              black!2]
                            (axis cs:-125.0, 10.0)
                            rectangle (axis cs:-120.0, 15.0);
                        \path[,fill,
                              black!2]
                            (axis cs:-125.0, 15.0)
                            rectangle (axis cs:-120.0, 20.0);
                        \path[,fill,
                              black!1]
                            (axis cs:-125.0, 25.0)
                            rectangle (axis cs:-120.0, 30.0);
                        \path[,fill,
                              black!1]
                            (axis cs:-125.0, 65.0)
                            rectangle (axis cs:-120.0, 70.0);
                        \path[,fill,
                              black!1]
                            (axis cs:-125.0, 105.0)
                            rectangle (axis cs:-120.0, 110.0);
                        \path[,fill,
                              black!1]
                            (axis cs:-125.0, 145.0)
                            rectangle (axis cs:-120.0, 150.0);
                        \path[,fill,
                              black!1]
                            (axis cs:-125.0, 160.0)
                            rectangle (axis cs:-120.0, 165.0);
                        \path[,fill,
                              black!1]
                            (axis cs:-125.0, 165.0)
                            rectangle (axis cs:-120.0, 170.0);
                        \path[,fill,
                              black!1]
                            (axis cs:-125.0, 170.0)
                            rectangle (axis cs:-120.0, 175.0);
                        \path[,fill,
                              black!4]
                            (axis cs:-125.0, 175.0)
                            rectangle (axis cs:-120.0, 180.0);
                        \path[,fill,
                              black!5]
                            (axis cs:-120.0, -180.0)
                            rectangle (axis cs:-115.0, -175.0);
                        \path[,fill,
                              black!5]
                            (axis cs:-120.0, -175.0)
                            rectangle (axis cs:-115.0, -170.0);
                        \path[,fill,
                              black!8]
                            (axis cs:-120.0, -170.0)
                            rectangle (axis cs:-115.0, -165.0);
                        \path[,fill,
                              black!8]
                            (axis cs:-120.0, -165.0)
                            rectangle (axis cs:-115.0, -160.0);
                        \path[,fill,
                              black!8]
                            (axis cs:-120.0, -160.0)
                            rectangle (axis cs:-115.0, -155.0);
                        \path[,fill,
                              black!12]
                            (axis cs:-120.0, -155.0)
                            rectangle (axis cs:-115.0, -150.0);
                        \path[,fill,
                              black!11]
                            (axis cs:-120.0, -150.0)
                            rectangle (axis cs:-115.0, -145.0);
                        \path[,fill,
                              black!11]
                            (axis cs:-120.0, -145.0)
                            rectangle (axis cs:-115.0, -140.0);
                        \path[,fill,
                              black!21]
                            (axis cs:-120.0, -140.0)
                            rectangle (axis cs:-115.0, -135.0);
                        \path[,fill,
                              black!29]
                            (axis cs:-120.0, -135.0)
                            rectangle (axis cs:-115.0, -130.0);
                        \path[,fill,
                              black!36]
                            (axis cs:-120.0, -130.0)
                            rectangle (axis cs:-115.0, -125.0);
                        \path[,fill,
                              black!46]
                            (axis cs:-120.0, -125.0)
                            rectangle (axis cs:-115.0, -120.0);
                        \path[,fill,
                              black!78]
                            (axis cs:-120.0, -120.0)
                            rectangle (axis cs:-115.0, -115.0);
                        \path[,fill,
                              black!77]
                            (axis cs:-120.0, -115.0)
                            rectangle (axis cs:-115.0, -110.0);
                        \path[,fill,
                              black!50]
                            (axis cs:-120.0, -110.0)
                            rectangle (axis cs:-115.0, -105.0);
                        \path[,fill,
                              black!32]
                            (axis cs:-120.0, -105.0)
                            rectangle (axis cs:-115.0, -100.0);
                        \path[,fill,
                              black!23]
                            (axis cs:-120.0, -100.0)
                            rectangle (axis cs:-115.0, -95.0);
                        \path[,fill,
                              black!9]
                            (axis cs:-120.0, -95.0)
                            rectangle (axis cs:-115.0, -90.0);
                        \path[,fill,
                              black!8]
                            (axis cs:-120.0, -90.0)
                            rectangle (axis cs:-115.0, -85.0);
                        \path[,fill,
                              black!7]
                            (axis cs:-120.0, -85.0)
                            rectangle (axis cs:-115.0, -80.0);
                        \path[,fill,
                              black!4]
                            (axis cs:-120.0, -80.0)
                            rectangle (axis cs:-115.0, -75.0);
                        \path[,fill,
                              black!6]
                            (axis cs:-120.0, -75.0)
                            rectangle (axis cs:-115.0, -70.0);
                        \path[,fill,
                              black!4]
                            (axis cs:-120.0, -70.0)
                            rectangle (axis cs:-115.0, -65.0);
                        \path[,fill,
                              black!3]
                            (axis cs:-120.0, -65.0)
                            rectangle (axis cs:-115.0, -60.0);
                        \path[,fill,
                              black!2]
                            (axis cs:-120.0, -60.0)
                            rectangle (axis cs:-115.0, -55.0);
                        \path[,fill,
                              black!2]
                            (axis cs:-120.0, -55.0)
                            rectangle (axis cs:-115.0, -50.0);
                        \path[,fill,
                              black!3]
                            (axis cs:-120.0, -50.0)
                            rectangle (axis cs:-115.0, -45.0);
                        \path[,fill,
                              black!2]
                            (axis cs:-120.0, -45.0)
                            rectangle (axis cs:-115.0, -40.0);
                        \path[,fill,
                              black!3]
                            (axis cs:-120.0, -40.0)
                            rectangle (axis cs:-115.0, -35.0);
                        \path[,fill,
                              black!1]
                            (axis cs:-120.0, -35.0)
                            rectangle (axis cs:-115.0, -30.0);
                        \path[,fill,
                              black!1]
                            (axis cs:-120.0, -25.0)
                            rectangle (axis cs:-115.0, -20.0);
                        \path[,fill,
                              black!1]
                            (axis cs:-120.0, -20.0)
                            rectangle (axis cs:-115.0, -15.0);
                        \path[,fill,
                              black!2]
                            (axis cs:-120.0, -15.0)
                            rectangle (axis cs:-115.0, -10.0);
                        \path[,fill,
                              black!2]
                            (axis cs:-120.0, -10.0)
                            rectangle (axis cs:-115.0, -5.0);
                        \path[,fill,
                              black!3]
                            (axis cs:-120.0, -5.0)
                            rectangle (axis cs:-115.0, 0.0);
                        \path[,fill,
                              black!4]
                            (axis cs:-120.0, 0.0)
                            rectangle (axis cs:-115.0, 5.0);
                        \path[,fill,
                              black!4]
                            (axis cs:-120.0, 5.0)
                            rectangle (axis cs:-115.0, 10.0);
                        \path[,fill,
                              black!3]
                            (axis cs:-120.0, 10.0)
                            rectangle (axis cs:-115.0, 15.0);
                        \path[,fill,
                              black!4]
                            (axis cs:-120.0, 15.0)
                            rectangle (axis cs:-115.0, 20.0);
                        \path[,fill,
                              black!3]
                            (axis cs:-120.0, 20.0)
                            rectangle (axis cs:-115.0, 25.0);
                        \path[,fill,
                              black!1]
                            (axis cs:-120.0, 25.0)
                            rectangle (axis cs:-115.0, 30.0);
                        \path[,fill,
                              black!1]
                            (axis cs:-120.0, 40.0)
                            rectangle (axis cs:-115.0, 45.0);
                        \path[,fill,
                              black!1]
                            (axis cs:-120.0, 115.0)
                            rectangle (axis cs:-115.0, 120.0);
                        \path[,fill,
                              black!1]
                            (axis cs:-120.0, 155.0)
                            rectangle (axis cs:-115.0, 160.0);
                        \path[,fill,
                              black!2]
                            (axis cs:-120.0, 160.0)
                            rectangle (axis cs:-115.0, 165.0);
                        \path[,fill,
                              black!1]
                            (axis cs:-120.0, 165.0)
                            rectangle (axis cs:-115.0, 170.0);
                        \path[,fill,
                              black!4]
                            (axis cs:-120.0, 170.0)
                            rectangle (axis cs:-115.0, 175.0);
                        \path[,fill,
                              black!5]
                            (axis cs:-120.0, 175.0)
                            rectangle (axis cs:-115.0, 180.0);
                        \path[,fill,
                              black!5]
                            (axis cs:-115.0, -180.0)
                            rectangle (axis cs:-110.0, -175.0);
                        \path[,fill,
                              black!4]
                            (axis cs:-115.0, -175.0)
                            rectangle (axis cs:-110.0, -170.0);
                        \path[,fill,
                              black!3]
                            (axis cs:-115.0, -170.0)
                            rectangle (axis cs:-110.0, -165.0);
                        \path[,fill,
                              black!8]
                            (axis cs:-115.0, -165.0)
                            rectangle (axis cs:-110.0, -160.0);
                        \path[,fill,
                              black!7]
                            (axis cs:-115.0, -160.0)
                            rectangle (axis cs:-110.0, -155.0);
                        \path[,fill,
                              black!8]
                            (axis cs:-115.0, -155.0)
                            rectangle (axis cs:-110.0, -150.0);
                        \path[,fill,
                              black!10]
                            (axis cs:-115.0, -150.0)
                            rectangle (axis cs:-110.0, -145.0);
                        \path[,fill,
                              black!17]
                            (axis cs:-115.0, -145.0)
                            rectangle (axis cs:-110.0, -140.0);
                        \path[,fill,
                              black!12]
                            (axis cs:-115.0, -140.0)
                            rectangle (axis cs:-110.0, -135.0);
                        \path[,fill,
                              black!19]
                            (axis cs:-115.0, -135.0)
                            rectangle (axis cs:-110.0, -130.0);
                        \path[,fill,
                              black!25]
                            (axis cs:-115.0, -130.0)
                            rectangle (axis cs:-110.0, -125.0);
                        \path[,fill,
                              black!35]
                            (axis cs:-115.0, -125.0)
                            rectangle (axis cs:-110.0, -120.0);
                        \path[,fill,
                              black!58]
                            (axis cs:-115.0, -120.0)
                            rectangle (axis cs:-110.0, -115.0);
                        \path[,fill,
                              black!82]
                            (axis cs:-115.0, -115.0)
                            rectangle (axis cs:-110.0, -110.0);
                        \path[,fill,
                              black!69]
                            (axis cs:-115.0, -110.0)
                            rectangle (axis cs:-110.0, -105.0);
                        \path[,fill,
                              black!49]
                            (axis cs:-115.0, -105.0)
                            rectangle (axis cs:-110.0, -100.0);
                        \path[,fill,
                              black!31]
                            (axis cs:-115.0, -100.0)
                            rectangle (axis cs:-110.0, -95.0);
                        \path[,fill,
                              black!24]
                            (axis cs:-115.0, -95.0)
                            rectangle (axis cs:-110.0, -90.0);
                        \path[,fill,
                              black!13]
                            (axis cs:-115.0, -90.0)
                            rectangle (axis cs:-110.0, -85.0);
                        \path[,fill,
                              black!11]
                            (axis cs:-115.0, -85.0)
                            rectangle (axis cs:-110.0, -80.0);
                        \path[,fill,
                              black!9]
                            (axis cs:-115.0, -80.0)
                            rectangle (axis cs:-110.0, -75.0);
                        \path[,fill,
                              black!6]
                            (axis cs:-115.0, -75.0)
                            rectangle (axis cs:-110.0, -70.0);
                        \path[,fill,
                              black!5]
                            (axis cs:-115.0, -70.0)
                            rectangle (axis cs:-110.0, -65.0);
                        \path[,fill,
                              black!3]
                            (axis cs:-115.0, -65.0)
                            rectangle (axis cs:-110.0, -60.0);
                        \path[,fill,
                              black!2]
                            (axis cs:-115.0, -60.0)
                            rectangle (axis cs:-110.0, -55.0);
                        \path[,fill,
                              black!2]
                            (axis cs:-115.0, -55.0)
                            rectangle (axis cs:-110.0, -50.0);
                        \path[,fill,
                              black!3]
                            (axis cs:-115.0, -50.0)
                            rectangle (axis cs:-110.0, -45.0);
                        \path[,fill,
                              black!1]
                            (axis cs:-115.0, -45.0)
                            rectangle (axis cs:-110.0, -40.0);
                        \path[,fill,
                              black!2]
                            (axis cs:-115.0, -40.0)
                            rectangle (axis cs:-110.0, -35.0);
                        \path[,fill,
                              black!2]
                            (axis cs:-115.0, -35.0)
                            rectangle (axis cs:-110.0, -30.0);
                        \path[,fill,
                              black!3]
                            (axis cs:-115.0, -30.0)
                            rectangle (axis cs:-110.0, -25.0);
                        \path[,fill,
                              black!1]
                            (axis cs:-115.0, -25.0)
                            rectangle (axis cs:-110.0, -20.0);
                        \path[,fill,
                              black!2]
                            (axis cs:-115.0, -20.0)
                            rectangle (axis cs:-110.0, -15.0);
                        \path[,fill,
                              black!2]
                            (axis cs:-115.0, -15.0)
                            rectangle (axis cs:-110.0, -10.0);
                        \path[,fill,
                              black!1]
                            (axis cs:-115.0, -10.0)
                            rectangle (axis cs:-110.0, -5.0);
                        \path[,fill,
                              black!1]
                            (axis cs:-115.0, -5.0)
                            rectangle (axis cs:-110.0, 0.0);
                        \path[,fill,
                              black!4]
                            (axis cs:-115.0, 0.0)
                            rectangle (axis cs:-110.0, 5.0);
                        \path[,fill,
                              black!4]
                            (axis cs:-115.0, 5.0)
                            rectangle (axis cs:-110.0, 10.0);
                        \path[,fill,
                              black!1]
                            (axis cs:-115.0, 10.0)
                            rectangle (axis cs:-110.0, 15.0);
                        \path[,fill,
                              black!1]
                            (axis cs:-115.0, 15.0)
                            rectangle (axis cs:-110.0, 20.0);
                        \path[,fill,
                              black!2]
                            (axis cs:-115.0, 20.0)
                            rectangle (axis cs:-110.0, 25.0);
                        \path[,fill,
                              black!1]
                            (axis cs:-115.0, 125.0)
                            rectangle (axis cs:-110.0, 130.0);
                        \path[,fill,
                              black!1]
                            (axis cs:-115.0, 130.0)
                            rectangle (axis cs:-110.0, 135.0);
                        \path[,fill,
                              black!1]
                            (axis cs:-115.0, 155.0)
                            rectangle (axis cs:-110.0, 160.0);
                        \path[,fill,
                              black!1]
                            (axis cs:-115.0, 160.0)
                            rectangle (axis cs:-110.0, 165.0);
                        \path[,fill,
                              black!2]
                            (axis cs:-115.0, 170.0)
                            rectangle (axis cs:-110.0, 175.0);
                        \path[,fill,
                              black!3]
                            (axis cs:-115.0, 175.0)
                            rectangle (axis cs:-110.0, 180.0);
                        \path[,fill,
                              black!5]
                            (axis cs:-110.0, -180.0)
                            rectangle (axis cs:-105.0, -175.0);
                        \path[,fill,
                              black!5]
                            (axis cs:-110.0, -175.0)
                            rectangle (axis cs:-105.0, -170.0);
                        \path[,fill,
                              black!3]
                            (axis cs:-110.0, -170.0)
                            rectangle (axis cs:-105.0, -165.0);
                        \path[,fill,
                              black!5]
                            (axis cs:-110.0, -165.0)
                            rectangle (axis cs:-105.0, -160.0);
                        \path[,fill,
                              black!6]
                            (axis cs:-110.0, -160.0)
                            rectangle (axis cs:-105.0, -155.0);
                        \path[,fill,
                              black!6]
                            (axis cs:-110.0, -155.0)
                            rectangle (axis cs:-105.0, -150.0);
                        \path[,fill,
                              black!7]
                            (axis cs:-110.0, -150.0)
                            rectangle (axis cs:-105.0, -145.0);
                        \path[,fill,
                              black!10]
                            (axis cs:-110.0, -145.0)
                            rectangle (axis cs:-105.0, -140.0);
                        \path[,fill,
                              black!11]
                            (axis cs:-110.0, -140.0)
                            rectangle (axis cs:-105.0, -135.0);
                        \path[,fill,
                              black!16]
                            (axis cs:-110.0, -135.0)
                            rectangle (axis cs:-105.0, -130.0);
                        \path[,fill,
                              black!17]
                            (axis cs:-110.0, -130.0)
                            rectangle (axis cs:-105.0, -125.0);
                        \path[,fill,
                              black!22]
                            (axis cs:-110.0, -125.0)
                            rectangle (axis cs:-105.0, -120.0);
                        \path[,fill,
                              black!32]
                            (axis cs:-110.0, -120.0)
                            rectangle (axis cs:-105.0, -115.0);
                        \path[,fill,
                              black!61]
                            (axis cs:-110.0, -115.0)
                            rectangle (axis cs:-105.0, -110.0);
                        \path[,fill,
                              black!83]
                            (axis cs:-110.0, -110.0)
                            rectangle (axis cs:-105.0, -105.0);
                        \path[,fill,
                              black!71]
                            (axis cs:-110.0, -105.0)
                            rectangle (axis cs:-105.0, -100.0);
                        \path[,fill,
                              black!45]
                            (axis cs:-110.0, -100.0)
                            rectangle (axis cs:-105.0, -95.0);
                        \path[,fill,
                              black!31]
                            (axis cs:-110.0, -95.0)
                            rectangle (axis cs:-105.0, -90.0);
                        \path[,fill,
                              black!17]
                            (axis cs:-110.0, -90.0)
                            rectangle (axis cs:-105.0, -85.0);
                        \path[,fill,
                              black!13]
                            (axis cs:-110.0, -85.0)
                            rectangle (axis cs:-105.0, -80.0);
                        \path[,fill,
                              black!10]
                            (axis cs:-110.0, -80.0)
                            rectangle (axis cs:-105.0, -75.0);
                        \path[,fill,
                              black!8]
                            (axis cs:-110.0, -75.0)
                            rectangle (axis cs:-105.0, -70.0);
                        \path[,fill,
                              black!4]
                            (axis cs:-110.0, -70.0)
                            rectangle (axis cs:-105.0, -65.0);
                        \path[,fill,
                              black!5]
                            (axis cs:-110.0, -65.0)
                            rectangle (axis cs:-105.0, -60.0);
                        \path[,fill,
                              black!4]
                            (axis cs:-110.0, -60.0)
                            rectangle (axis cs:-105.0, -55.0);
                        \path[,fill,
                              black!3]
                            (axis cs:-110.0, -55.0)
                            rectangle (axis cs:-105.0, -50.0);
                        \path[,fill,
                              black!4]
                            (axis cs:-110.0, -50.0)
                            rectangle (axis cs:-105.0, -45.0);
                        \path[,fill,
                              black!1]
                            (axis cs:-110.0, -45.0)
                            rectangle (axis cs:-105.0, -40.0);
                        \path[,fill,
                              black!3]
                            (axis cs:-110.0, -40.0)
                            rectangle (axis cs:-105.0, -35.0);
                        \path[,fill,
                              black!4]
                            (axis cs:-110.0, -35.0)
                            rectangle (axis cs:-105.0, -30.0);
                        \path[,fill,
                              black!3]
                            (axis cs:-110.0, -30.0)
                            rectangle (axis cs:-105.0, -25.0);
                        \path[,fill,
                              black!1]
                            (axis cs:-110.0, -25.0)
                            rectangle (axis cs:-105.0, -20.0);
                        \path[,fill,
                              black!1]
                            (axis cs:-110.0, -20.0)
                            rectangle (axis cs:-105.0, -15.0);
                        \path[,fill,
                              black!2]
                            (axis cs:-110.0, -15.0)
                            rectangle (axis cs:-105.0, -10.0);
                        \path[,fill,
                              black!1]
                            (axis cs:-110.0, -10.0)
                            rectangle (axis cs:-105.0, -5.0);
                        \path[,fill,
                              black!2]
                            (axis cs:-110.0, -5.0)
                            rectangle (axis cs:-105.0, 0.0);
                        \path[,fill,
                              black!2]
                            (axis cs:-110.0, 0.0)
                            rectangle (axis cs:-105.0, 5.0);
                        \path[,fill,
                              black!3]
                            (axis cs:-110.0, 5.0)
                            rectangle (axis cs:-105.0, 10.0);
                        \path[,fill,
                              black!2]
                            (axis cs:-110.0, 10.0)
                            rectangle (axis cs:-105.0, 15.0);
                        \path[,fill,
                              black!2]
                            (axis cs:-110.0, 15.0)
                            rectangle (axis cs:-105.0, 20.0);
                        \path[,fill,
                              black!2]
                            (axis cs:-110.0, 25.0)
                            rectangle (axis cs:-105.0, 30.0);
                        \path[,fill,
                              black!1]
                            (axis cs:-110.0, 30.0)
                            rectangle (axis cs:-105.0, 35.0);
                        \path[,fill,
                              black!1]
                            (axis cs:-110.0, 35.0)
                            rectangle (axis cs:-105.0, 40.0);
                        \path[,fill,
                              black!1]
                            (axis cs:-110.0, 140.0)
                            rectangle (axis cs:-105.0, 145.0);
                        \path[,fill,
                              black!1]
                            (axis cs:-110.0, 160.0)
                            rectangle (axis cs:-105.0, 165.0);
                        \path[,fill,
                              black!1]
                            (axis cs:-110.0, 165.0)
                            rectangle (axis cs:-105.0, 170.0);
                        \path[,fill,
                              black!2]
                            (axis cs:-110.0, 170.0)
                            rectangle (axis cs:-105.0, 175.0);
                        \path[,fill,
                              black!2]
                            (axis cs:-110.0, 175.0)
                            rectangle (axis cs:-105.0, 180.0);
                        \path[,fill,
                              black!4]
                            (axis cs:-105.0, -180.0)
                            rectangle (axis cs:-100.0, -175.0);
                        \path[,fill,
                              black!4]
                            (axis cs:-105.0, -175.0)
                            rectangle (axis cs:-100.0, -170.0);
                        \path[,fill,
                              black!7]
                            (axis cs:-105.0, -170.0)
                            rectangle (axis cs:-100.0, -165.0);
                        \path[,fill,
                              black!5]
                            (axis cs:-105.0, -165.0)
                            rectangle (axis cs:-100.0, -160.0);
                        \path[,fill,
                              black!2]
                            (axis cs:-105.0, -160.0)
                            rectangle (axis cs:-100.0, -155.0);
                        \path[,fill,
                              black!5]
                            (axis cs:-105.0, -155.0)
                            rectangle (axis cs:-100.0, -150.0);
                        \path[,fill,
                              black!7]
                            (axis cs:-105.0, -150.0)
                            rectangle (axis cs:-100.0, -145.0);
                        \path[,fill,
                              black!12]
                            (axis cs:-105.0, -145.0)
                            rectangle (axis cs:-100.0, -140.0);
                        \path[,fill,
                              black!10]
                            (axis cs:-105.0, -140.0)
                            rectangle (axis cs:-100.0, -135.0);
                        \path[,fill,
                              black!8]
                            (axis cs:-105.0, -135.0)
                            rectangle (axis cs:-100.0, -130.0);
                        \path[,fill,
                              black!13]
                            (axis cs:-105.0, -130.0)
                            rectangle (axis cs:-100.0, -125.0);
                        \path[,fill,
                              black!14]
                            (axis cs:-105.0, -125.0)
                            rectangle (axis cs:-100.0, -120.0);
                        \path[,fill,
                              black!24]
                            (axis cs:-105.0, -120.0)
                            rectangle (axis cs:-100.0, -115.0);
                        \path[,fill,
                              black!41]
                            (axis cs:-105.0, -115.0)
                            rectangle (axis cs:-100.0, -110.0);
                        \path[,fill,
                              black!55]
                            (axis cs:-105.0, -110.0)
                            rectangle (axis cs:-100.0, -105.0);
                        \path[,fill,
                              black!81]
                            (axis cs:-105.0, -105.0)
                            rectangle (axis cs:-100.0, -100.0);
                        \path[,fill,
                              black!77]
                            (axis cs:-105.0, -100.0)
                            rectangle (axis cs:-100.0, -95.0);
                        \path[,fill,
                              black!47]
                            (axis cs:-105.0, -95.0)
                            rectangle (axis cs:-100.0, -90.0);
                        \path[,fill,
                              black!27]
                            (axis cs:-105.0, -90.0)
                            rectangle (axis cs:-100.0, -85.0);
                        \path[,fill,
                              black!14]
                            (axis cs:-105.0, -85.0)
                            rectangle (axis cs:-100.0, -80.0);
                        \path[,fill,
                              black!8]
                            (axis cs:-105.0, -80.0)
                            rectangle (axis cs:-100.0, -75.0);
                        \path[,fill,
                              black!9]
                            (axis cs:-105.0, -75.0)
                            rectangle (axis cs:-100.0, -70.0);
                        \path[,fill,
                              black!6]
                            (axis cs:-105.0, -70.0)
                            rectangle (axis cs:-100.0, -65.0);
                        \path[,fill,
                              black!9]
                            (axis cs:-105.0, -65.0)
                            rectangle (axis cs:-100.0, -60.0);
                        \path[,fill,
                              black!4]
                            (axis cs:-105.0, -60.0)
                            rectangle (axis cs:-100.0, -55.0);
                        \path[,fill,
                              black!7]
                            (axis cs:-105.0, -55.0)
                            rectangle (axis cs:-100.0, -50.0);
                        \path[,fill,
                              black!3]
                            (axis cs:-105.0, -50.0)
                            rectangle (axis cs:-100.0, -45.0);
                        \path[,fill,
                              black!4]
                            (axis cs:-105.0, -45.0)
                            rectangle (axis cs:-100.0, -40.0);
                        \path[,fill,
                              black!3]
                            (axis cs:-105.0, -40.0)
                            rectangle (axis cs:-100.0, -35.0);
                        \path[,fill,
                              black!4]
                            (axis cs:-105.0, -35.0)
                            rectangle (axis cs:-100.0, -30.0);
                        \path[,fill,
                              black!2]
                            (axis cs:-105.0, -30.0)
                            rectangle (axis cs:-100.0, -25.0);
                        \path[,fill,
                              black!4]
                            (axis cs:-105.0, -25.0)
                            rectangle (axis cs:-100.0, -20.0);
                        \path[,fill,
                              black!1]
                            (axis cs:-105.0, -20.0)
                            rectangle (axis cs:-100.0, -15.0);
                        \path[,fill,
                              black!1]
                            (axis cs:-105.0, -15.0)
                            rectangle (axis cs:-100.0, -10.0);
                        \path[,fill,
                              black!4]
                            (axis cs:-105.0, -10.0)
                            rectangle (axis cs:-100.0, -5.0);
                        \path[,fill,
                              black!4]
                            (axis cs:-105.0, -5.0)
                            rectangle (axis cs:-100.0, 0.0);
                        \path[,fill,
                              black!1]
                            (axis cs:-105.0, 0.0)
                            rectangle (axis cs:-100.0, 5.0);
                        \path[,fill,
                              black!2]
                            (axis cs:-105.0, 5.0)
                            rectangle (axis cs:-100.0, 10.0);
                        \path[,fill,
                              black!2]
                            (axis cs:-105.0, 10.0)
                            rectangle (axis cs:-100.0, 15.0);
                        \path[,fill,
                              black!1]
                            (axis cs:-105.0, 15.0)
                            rectangle (axis cs:-100.0, 20.0);
                        \path[,fill,
                              black!1]
                            (axis cs:-105.0, 30.0)
                            rectangle (axis cs:-100.0, 35.0);
                        \path[,fill,
                              black!1]
                            (axis cs:-105.0, 140.0)
                            rectangle (axis cs:-100.0, 145.0);
                        \path[,fill,
                              black!1]
                            (axis cs:-105.0, 150.0)
                            rectangle (axis cs:-100.0, 155.0);
                        \path[,fill,
                              black!1]
                            (axis cs:-105.0, 160.0)
                            rectangle (axis cs:-100.0, 165.0);
                        \path[,fill,
                              black!2]
                            (axis cs:-105.0, 165.0)
                            rectangle (axis cs:-100.0, 170.0);
                        \path[,fill,
                              black!3]
                            (axis cs:-105.0, 170.0)
                            rectangle (axis cs:-100.0, 175.0);
                        \path[,fill,
                              black!4]
                            (axis cs:-105.0, 175.0)
                            rectangle (axis cs:-100.0, 180.0);
                        \path[,fill,
                              black!2]
                            (axis cs:-100.0, -180.0)
                            rectangle (axis cs:-95.0, -175.0);
                        \path[,fill,
                              black!4]
                            (axis cs:-100.0, -175.0)
                            rectangle (axis cs:-95.0, -170.0);
                        \path[,fill,
                              black!5]
                            (axis cs:-100.0, -170.0)
                            rectangle (axis cs:-95.0, -165.0);
                        \path[,fill,
                              black!4]
                            (axis cs:-100.0, -165.0)
                            rectangle (axis cs:-95.0, -160.0);
                        \path[,fill,
                              black!3]
                            (axis cs:-100.0, -160.0)
                            rectangle (axis cs:-95.0, -155.0);
                        \path[,fill,
                              black!4]
                            (axis cs:-100.0, -155.0)
                            rectangle (axis cs:-95.0, -150.0);
                        \path[,fill,
                              black!3]
                            (axis cs:-100.0, -150.0)
                            rectangle (axis cs:-95.0, -145.0);
                        \path[,fill,
                              black!7]
                            (axis cs:-100.0, -145.0)
                            rectangle (axis cs:-95.0, -140.0);
                        \path[,fill,
                              black!7]
                            (axis cs:-100.0, -140.0)
                            rectangle (axis cs:-95.0, -135.0);
                        \path[,fill,
                              black!10]
                            (axis cs:-100.0, -135.0)
                            rectangle (axis cs:-95.0, -130.0);
                        \path[,fill,
                              black!8]
                            (axis cs:-100.0, -130.0)
                            rectangle (axis cs:-95.0, -125.0);
                        \path[,fill,
                              black!11]
                            (axis cs:-100.0, -125.0)
                            rectangle (axis cs:-95.0, -120.0);
                        \path[,fill,
                              black!16]
                            (axis cs:-100.0, -120.0)
                            rectangle (axis cs:-95.0, -115.0);
                        \path[,fill,
                              black!21]
                            (axis cs:-100.0, -115.0)
                            rectangle (axis cs:-95.0, -110.0);
                        \path[,fill,
                              black!40]
                            (axis cs:-100.0, -110.0)
                            rectangle (axis cs:-95.0, -105.0);
                        \path[,fill,
                              black!62]
                            (axis cs:-100.0, -105.0)
                            rectangle (axis cs:-95.0, -100.0);
                        \path[,fill,
                              black!87]
                            (axis cs:-100.0, -100.0)
                            rectangle (axis cs:-95.0, -95.0);
                        \path[,fill,
                              black!76]
                            (axis cs:-100.0, -95.0)
                            rectangle (axis cs:-95.0, -90.0);
                        \path[,fill,
                              black!43]
                            (axis cs:-100.0, -90.0)
                            rectangle (axis cs:-95.0, -85.0);
                        \path[,fill,
                              black!23]
                            (axis cs:-100.0, -85.0)
                            rectangle (axis cs:-95.0, -80.0);
                        \path[,fill,
                              black!19]
                            (axis cs:-100.0, -80.0)
                            rectangle (axis cs:-95.0, -75.0);
                        \path[,fill,
                              black!11]
                            (axis cs:-100.0, -75.0)
                            rectangle (axis cs:-95.0, -70.0);
                        \path[,fill,
                              black!7]
                            (axis cs:-100.0, -70.0)
                            rectangle (axis cs:-95.0, -65.0);
                        \path[,fill,
                              black!9]
                            (axis cs:-100.0, -65.0)
                            rectangle (axis cs:-95.0, -60.0);
                        \path[,fill,
                              black!2]
                            (axis cs:-100.0, -60.0)
                            rectangle (axis cs:-95.0, -55.0);
                        \path[,fill,
                              black!5]
                            (axis cs:-100.0, -55.0)
                            rectangle (axis cs:-95.0, -50.0);
                        \path[,fill,
                              black!3]
                            (axis cs:-100.0, -50.0)
                            rectangle (axis cs:-95.0, -45.0);
                        \path[,fill,
                              black!4]
                            (axis cs:-100.0, -45.0)
                            rectangle (axis cs:-95.0, -40.0);
                        \path[,fill,
                              black!2]
                            (axis cs:-100.0, -40.0)
                            rectangle (axis cs:-95.0, -35.0);
                        \path[,fill,
                              black!3]
                            (axis cs:-100.0, -35.0)
                            rectangle (axis cs:-95.0, -30.0);
                        \path[,fill,
                              black!3]
                            (axis cs:-100.0, -30.0)
                            rectangle (axis cs:-95.0, -25.0);
                        \path[,fill,
                              black!2]
                            (axis cs:-100.0, -25.0)
                            rectangle (axis cs:-95.0, -20.0);
                        \path[,fill,
                              black!4]
                            (axis cs:-100.0, -20.0)
                            rectangle (axis cs:-95.0, -15.0);
                        \path[,fill,
                              black!4]
                            (axis cs:-100.0, -15.0)
                            rectangle (axis cs:-95.0, -10.0);
                        \path[,fill,
                              black!4]
                            (axis cs:-100.0, -10.0)
                            rectangle (axis cs:-95.0, -5.0);
                        \path[,fill,
                              black!2]
                            (axis cs:-100.0, -5.0)
                            rectangle (axis cs:-95.0, 0.0);
                        \path[,fill,
                              black!2]
                            (axis cs:-100.0, 0.0)
                            rectangle (axis cs:-95.0, 5.0);
                        \path[,fill,
                              black!4]
                            (axis cs:-100.0, 5.0)
                            rectangle (axis cs:-95.0, 10.0);
                        \path[,fill,
                              black!3]
                            (axis cs:-100.0, 10.0)
                            rectangle (axis cs:-95.0, 15.0);
                        \path[,fill,
                              black!1]
                            (axis cs:-100.0, 15.0)
                            rectangle (axis cs:-95.0, 20.0);
                        \path[,fill,
                              black!1]
                            (axis cs:-100.0, 30.0)
                            rectangle (axis cs:-95.0, 35.0);
                        \path[,fill,
                              black!1]
                            (axis cs:-100.0, 150.0)
                            rectangle (axis cs:-95.0, 155.0);
                        \path[,fill,
                              black!1]
                            (axis cs:-100.0, 155.0)
                            rectangle (axis cs:-95.0, 160.0);
                        \path[,fill,
                              black!2]
                            (axis cs:-100.0, 160.0)
                            rectangle (axis cs:-95.0, 165.0);
                        \path[,fill,
                              black!1]
                            (axis cs:-100.0, 170.0)
                            rectangle (axis cs:-95.0, 175.0);
                        \path[,fill,
                              black!5]
                            (axis cs:-100.0, 175.0)
                            rectangle (axis cs:-95.0, 180.0);
                        \path[,fill,
                              black!2]
                            (axis cs:-95.0, -180.0)
                            rectangle (axis cs:-90.0, -175.0);
                        \path[,fill,
                              black!3]
                            (axis cs:-95.0, -175.0)
                            rectangle (axis cs:-90.0, -170.0);
                        \path[,fill,
                              black!4]
                            (axis cs:-95.0, -170.0)
                            rectangle (axis cs:-90.0, -165.0);
                        \path[,fill,
                              black!3]
                            (axis cs:-95.0, -165.0)
                            rectangle (axis cs:-90.0, -160.0);
                        \path[,fill,
                              black!2]
                            (axis cs:-95.0, -160.0)
                            rectangle (axis cs:-90.0, -155.0);
                        \path[,fill,
                              black!2]
                            (axis cs:-95.0, -155.0)
                            rectangle (axis cs:-90.0, -150.0);
                        \path[,fill,
                              black!2]
                            (axis cs:-95.0, -150.0)
                            rectangle (axis cs:-90.0, -145.0);
                        \path[,fill,
                              black!6]
                            (axis cs:-95.0, -145.0)
                            rectangle (axis cs:-90.0, -140.0);
                        \path[,fill,
                              black!7]
                            (axis cs:-95.0, -140.0)
                            rectangle (axis cs:-90.0, -135.0);
                        \path[,fill,
                              black!6]
                            (axis cs:-95.0, -135.0)
                            rectangle (axis cs:-90.0, -130.0);
                        \path[,fill,
                              black!6]
                            (axis cs:-95.0, -130.0)
                            rectangle (axis cs:-90.0, -125.0);
                        \path[,fill,
                              black!12]
                            (axis cs:-95.0, -125.0)
                            rectangle (axis cs:-90.0, -120.0);
                        \path[,fill,
                              black!13]
                            (axis cs:-95.0, -120.0)
                            rectangle (axis cs:-90.0, -115.0);
                        \path[,fill,
                              black!14]
                            (axis cs:-95.0, -115.0)
                            rectangle (axis cs:-90.0, -110.0);
                        \path[,fill,
                              black!27]
                            (axis cs:-95.0, -110.0)
                            rectangle (axis cs:-90.0, -105.0);
                        \path[,fill,
                              black!34]
                            (axis cs:-95.0, -105.0)
                            rectangle (axis cs:-90.0, -100.0);
                        \path[,fill,
                              black!62]
                            (axis cs:-95.0, -100.0)
                            rectangle (axis cs:-90.0, -95.0);
                        \path[,fill,
                              black!93]
                            (axis cs:-95.0, -95.0)
                            rectangle (axis cs:-90.0, -90.0);
                        \path[,fill,
                              black!67]
                            (axis cs:-95.0, -90.0)
                            rectangle (axis cs:-90.0, -85.0);
                        \path[,fill,
                              black!38]
                            (axis cs:-95.0, -85.0)
                            rectangle (axis cs:-90.0, -80.0);
                        \path[,fill,
                              black!23]
                            (axis cs:-95.0, -80.0)
                            rectangle (axis cs:-90.0, -75.0);
                        \path[,fill,
                              black!14]
                            (axis cs:-95.0, -75.0)
                            rectangle (axis cs:-90.0, -70.0);
                        \path[,fill,
                              black!13]
                            (axis cs:-95.0, -70.0)
                            rectangle (axis cs:-90.0, -65.0);
                        \path[,fill,
                              black!7]
                            (axis cs:-95.0, -65.0)
                            rectangle (axis cs:-90.0, -60.0);
                        \path[,fill,
                              black!8]
                            (axis cs:-95.0, -60.0)
                            rectangle (axis cs:-90.0, -55.0);
                        \path[,fill,
                              black!5]
                            (axis cs:-95.0, -55.0)
                            rectangle (axis cs:-90.0, -50.0);
                        \path[,fill,
                              black!7]
                            (axis cs:-95.0, -50.0)
                            rectangle (axis cs:-90.0, -45.0);
                        \path[,fill,
                              black!4]
                            (axis cs:-95.0, -45.0)
                            rectangle (axis cs:-90.0, -40.0);
                        \path[,fill,
                              black!5]
                            (axis cs:-95.0, -40.0)
                            rectangle (axis cs:-90.0, -35.0);
                        \path[,fill,
                              black!5]
                            (axis cs:-95.0, -35.0)
                            rectangle (axis cs:-90.0, -30.0);
                        \path[,fill,
                              black!2]
                            (axis cs:-95.0, -30.0)
                            rectangle (axis cs:-90.0, -25.0);
                        \path[,fill,
                              black!4]
                            (axis cs:-95.0, -25.0)
                            rectangle (axis cs:-90.0, -20.0);
                        \path[,fill,
                              black!2]
                            (axis cs:-95.0, -20.0)
                            rectangle (axis cs:-90.0, -15.0);
                        \path[,fill,
                              black!2]
                            (axis cs:-95.0, -15.0)
                            rectangle (axis cs:-90.0, -10.0);
                        \path[,fill,
                              black!4]
                            (axis cs:-95.0, -10.0)
                            rectangle (axis cs:-90.0, -5.0);
                        \path[,fill,
                              black!1]
                            (axis cs:-95.0, -5.0)
                            rectangle (axis cs:-90.0, 0.0);
                        \path[,fill,
                              black!3]
                            (axis cs:-95.0, 0.0)
                            rectangle (axis cs:-90.0, 5.0);
                        \path[,fill,
                              black!4]
                            (axis cs:-95.0, 5.0)
                            rectangle (axis cs:-90.0, 10.0);
                        \path[,fill,
                              black!2]
                            (axis cs:-95.0, 10.0)
                            rectangle (axis cs:-90.0, 15.0);
                        \path[,fill,
                              black!1]
                            (axis cs:-95.0, 20.0)
                            rectangle (axis cs:-90.0, 25.0);
                        \path[,fill,
                              black!2]
                            (axis cs:-95.0, 25.0)
                            rectangle (axis cs:-90.0, 30.0);
                        \path[,fill,
                              black!1]
                            (axis cs:-95.0, 155.0)
                            rectangle (axis cs:-90.0, 160.0);
                        \path[,fill,
                              black!1]
                            (axis cs:-95.0, 160.0)
                            rectangle (axis cs:-90.0, 165.0);
                        \path[,fill,
                              black!1]
                            (axis cs:-95.0, 165.0)
                            rectangle (axis cs:-90.0, 170.0);
                        \path[,fill,
                              black!4]
                            (axis cs:-95.0, 170.0)
                            rectangle (axis cs:-90.0, 175.0);
                        \path[,fill,
                              black!2]
                            (axis cs:-95.0, 175.0)
                            rectangle (axis cs:-90.0, 180.0);
                        \path[,fill,
                              black!1]
                            (axis cs:-90.0, -180.0)
                            rectangle (axis cs:-85.0, -175.0);
                        \path[,fill,
                              black!5]
                            (axis cs:-90.0, -175.0)
                            rectangle (axis cs:-85.0, -170.0);
                        \path[,fill,
                              black!3]
                            (axis cs:-90.0, -170.0)
                            rectangle (axis cs:-85.0, -165.0);
                        \path[,fill,
                              black!3]
                            (axis cs:-90.0, -165.0)
                            rectangle (axis cs:-85.0, -160.0);
                        \path[,fill,
                              black!2]
                            (axis cs:-90.0, -160.0)
                            rectangle (axis cs:-85.0, -155.0);
                        \path[,fill,
                              black!1]
                            (axis cs:-90.0, -155.0)
                            rectangle (axis cs:-85.0, -150.0);
                        \path[,fill,
                              black!2]
                            (axis cs:-90.0, -150.0)
                            rectangle (axis cs:-85.0, -145.0);
                        \path[,fill,
                              black!4]
                            (axis cs:-90.0, -145.0)
                            rectangle (axis cs:-85.0, -140.0);
                        \path[,fill,
                              black!6]
                            (axis cs:-90.0, -140.0)
                            rectangle (axis cs:-85.0, -135.0);
                        \path[,fill,
                              black!5]
                            (axis cs:-90.0, -135.0)
                            rectangle (axis cs:-85.0, -130.0);
                        \path[,fill,
                              black!8]
                            (axis cs:-90.0, -130.0)
                            rectangle (axis cs:-85.0, -125.0);
                        \path[,fill,
                              black!7]
                            (axis cs:-90.0, -125.0)
                            rectangle (axis cs:-85.0, -120.0);
                        \path[,fill,
                              black!12]
                            (axis cs:-90.0, -120.0)
                            rectangle (axis cs:-85.0, -115.0);
                        \path[,fill,
                              black!11]
                            (axis cs:-90.0, -115.0)
                            rectangle (axis cs:-85.0, -110.0);
                        \path[,fill,
                              black!14]
                            (axis cs:-90.0, -110.0)
                            rectangle (axis cs:-85.0, -105.0);
                        \path[,fill,
                              black!29]
                            (axis cs:-90.0, -105.0)
                            rectangle (axis cs:-85.0, -100.0);
                        \path[,fill,
                              black!38]
                            (axis cs:-90.0, -100.0)
                            rectangle (axis cs:-85.0, -95.0);
                        \path[,fill,
                              black!65]
                            (axis cs:-90.0, -95.0)
                            rectangle (axis cs:-85.0, -90.0);
                        \path[,fill,
                              black!83]
                            (axis cs:-90.0, -90.0)
                            rectangle (axis cs:-85.0, -85.0);
                        \path[,fill,
                              black!66]
                            (axis cs:-90.0, -85.0)
                            rectangle (axis cs:-85.0, -80.0);
                        \path[,fill,
                              black!34]
                            (axis cs:-90.0, -80.0)
                            rectangle (axis cs:-85.0, -75.0);
                        \path[,fill,
                              black!28]
                            (axis cs:-90.0, -75.0)
                            rectangle (axis cs:-85.0, -70.0);
                        \path[,fill,
                              black!18]
                            (axis cs:-90.0, -70.0)
                            rectangle (axis cs:-85.0, -65.0);
                        \path[,fill,
                              black!14]
                            (axis cs:-90.0, -65.0)
                            rectangle (axis cs:-85.0, -60.0);
                        \path[,fill,
                              black!7]
                            (axis cs:-90.0, -60.0)
                            rectangle (axis cs:-85.0, -55.0);
                        \path[,fill,
                              black!13]
                            (axis cs:-90.0, -55.0)
                            rectangle (axis cs:-85.0, -50.0);
                        \path[,fill,
                              black!8]
                            (axis cs:-90.0, -50.0)
                            rectangle (axis cs:-85.0, -45.0);
                        \path[,fill,
                              black!8]
                            (axis cs:-90.0, -45.0)
                            rectangle (axis cs:-85.0, -40.0);
                        \path[,fill,
                              black!5]
                            (axis cs:-90.0, -40.0)
                            rectangle (axis cs:-85.0, -35.0);
                        \path[,fill,
                              black!5]
                            (axis cs:-90.0, -35.0)
                            rectangle (axis cs:-85.0, -30.0);
                        \path[,fill,
                              black!4]
                            (axis cs:-90.0, -30.0)
                            rectangle (axis cs:-85.0, -25.0);
                        \path[,fill,
                              black!4]
                            (axis cs:-90.0, -25.0)
                            rectangle (axis cs:-85.0, -20.0);
                        \path[,fill,
                              black!4]
                            (axis cs:-90.0, -20.0)
                            rectangle (axis cs:-85.0, -15.0);
                        \path[,fill,
                              black!2]
                            (axis cs:-90.0, -15.0)
                            rectangle (axis cs:-85.0, -10.0);
                        \path[,fill,
                              black!6]
                            (axis cs:-90.0, -10.0)
                            rectangle (axis cs:-85.0, -5.0);
                        \path[,fill,
                              black!5]
                            (axis cs:-90.0, -5.0)
                            rectangle (axis cs:-85.0, 0.0);
                        \path[,fill,
                              black!2]
                            (axis cs:-90.0, 0.0)
                            rectangle (axis cs:-85.0, 5.0);
                        \path[,fill,
                              black!3]
                            (axis cs:-90.0, 5.0)
                            rectangle (axis cs:-85.0, 10.0);
                        \path[,fill,
                              black!3]
                            (axis cs:-90.0, 10.0)
                            rectangle (axis cs:-85.0, 15.0);
                        \path[,fill,
                              black!1]
                            (axis cs:-90.0, 15.0)
                            rectangle (axis cs:-85.0, 20.0);
                        \path[,fill,
                              black!1]
                            (axis cs:-90.0, 20.0)
                            rectangle (axis cs:-85.0, 25.0);
                        \path[,fill,
                              black!1]
                            (axis cs:-90.0, 155.0)
                            rectangle (axis cs:-85.0, 160.0);
                        \path[,fill,
                              black!1]
                            (axis cs:-90.0, 160.0)
                            rectangle (axis cs:-85.0, 165.0);
                        \path[,fill,
                              black!1]
                            (axis cs:-90.0, 165.0)
                            rectangle (axis cs:-85.0, 170.0);
                        \path[,fill,
                              black!3]
                            (axis cs:-90.0, 170.0)
                            rectangle (axis cs:-85.0, 175.0);
                        \path[,fill,
                              black!4]
                            (axis cs:-90.0, 175.0)
                            rectangle (axis cs:-85.0, 180.0);
                        \path[,fill,
                              black!2]
                            (axis cs:-85.0, -180.0)
                            rectangle (axis cs:-80.0, -175.0);
                        \path[,fill,
                              black!5]
                            (axis cs:-85.0, -175.0)
                            rectangle (axis cs:-80.0, -170.0);
                        \path[,fill,
                              black!5]
                            (axis cs:-85.0, -170.0)
                            rectangle (axis cs:-80.0, -165.0);
                        \path[,fill,
                              black!5]
                            (axis cs:-85.0, -165.0)
                            rectangle (axis cs:-80.0, -160.0);
                        \path[,fill,
                              black!2]
                            (axis cs:-85.0, -160.0)
                            rectangle (axis cs:-80.0, -155.0);
                        \path[,fill,
                              black!6]
                            (axis cs:-85.0, -155.0)
                            rectangle (axis cs:-80.0, -150.0);
                        \path[,fill,
                              black!4]
                            (axis cs:-85.0, -150.0)
                            rectangle (axis cs:-80.0, -145.0);
                        \path[,fill,
                              black!5]
                            (axis cs:-85.0, -145.0)
                            rectangle (axis cs:-80.0, -140.0);
                        \path[,fill,
                              black!4]
                            (axis cs:-85.0, -140.0)
                            rectangle (axis cs:-80.0, -135.0);
                        \path[,fill,
                              black!6]
                            (axis cs:-85.0, -135.0)
                            rectangle (axis cs:-80.0, -130.0);
                        \path[,fill,
                              black!4]
                            (axis cs:-85.0, -130.0)
                            rectangle (axis cs:-80.0, -125.0);
                        \path[,fill,
                              black!5]
                            (axis cs:-85.0, -125.0)
                            rectangle (axis cs:-80.0, -120.0);
                        \path[,fill,
                              black!6]
                            (axis cs:-85.0, -120.0)
                            rectangle (axis cs:-80.0, -115.0);
                        \path[,fill,
                              black!7]
                            (axis cs:-85.0, -115.0)
                            rectangle (axis cs:-80.0, -110.0);
                        \path[,fill,
                              black!11]
                            (axis cs:-85.0, -110.0)
                            rectangle (axis cs:-80.0, -105.0);
                        \path[,fill,
                              black!14]
                            (axis cs:-85.0, -105.0)
                            rectangle (axis cs:-80.0, -100.0);
                        \path[,fill,
                              black!25]
                            (axis cs:-85.0, -100.0)
                            rectangle (axis cs:-80.0, -95.0);
                        \path[,fill,
                              black!43]
                            (axis cs:-85.0, -95.0)
                            rectangle (axis cs:-80.0, -90.0);
                        \path[,fill,
                              black!66]
                            (axis cs:-85.0, -90.0)
                            rectangle (axis cs:-80.0, -85.0);
                        \path[,fill,
                              black!93]
                            (axis cs:-85.0, -85.0)
                            rectangle (axis cs:-80.0, -80.0);
                        \path[,fill,
                              black!64]
                            (axis cs:-85.0, -80.0)
                            rectangle (axis cs:-80.0, -75.0);
                        \path[,fill,
                              black!35]
                            (axis cs:-85.0, -75.0)
                            rectangle (axis cs:-80.0, -70.0);
                        \path[,fill,
                              black!21]
                            (axis cs:-85.0, -70.0)
                            rectangle (axis cs:-80.0, -65.0);
                        \path[,fill,
                              black!21]
                            (axis cs:-85.0, -65.0)
                            rectangle (axis cs:-80.0, -60.0);
                        \path[,fill,
                              black!13]
                            (axis cs:-85.0, -60.0)
                            rectangle (axis cs:-80.0, -55.0);
                        \path[,fill,
                              black!11]
                            (axis cs:-85.0, -55.0)
                            rectangle (axis cs:-80.0, -50.0);
                        \path[,fill,
                              black!9]
                            (axis cs:-85.0, -50.0)
                            rectangle (axis cs:-80.0, -45.0);
                        \path[,fill,
                              black!6]
                            (axis cs:-85.0, -45.0)
                            rectangle (axis cs:-80.0, -40.0);
                        \path[,fill,
                              black!8]
                            (axis cs:-85.0, -40.0)
                            rectangle (axis cs:-80.0, -35.0);
                        \path[,fill,
                              black!7]
                            (axis cs:-85.0, -35.0)
                            rectangle (axis cs:-80.0, -30.0);
                        \path[,fill,
                              black!5]
                            (axis cs:-85.0, -30.0)
                            rectangle (axis cs:-80.0, -25.0);
                        \path[,fill,
                              black!4]
                            (axis cs:-85.0, -25.0)
                            rectangle (axis cs:-80.0, -20.0);
                        \path[,fill,
                              black!4]
                            (axis cs:-85.0, -20.0)
                            rectangle (axis cs:-80.0, -15.0);
                        \path[,fill,
                              black!4]
                            (axis cs:-85.0, -15.0)
                            rectangle (axis cs:-80.0, -10.0);
                        \path[,fill,
                              black!3]
                            (axis cs:-85.0, -10.0)
                            rectangle (axis cs:-80.0, -5.0);
                        \path[,fill,
                              black!3]
                            (axis cs:-85.0, -5.0)
                            rectangle (axis cs:-80.0, 0.0);
                        \path[,fill,
                              black!6]
                            (axis cs:-85.0, 0.0)
                            rectangle (axis cs:-80.0, 5.0);
                        \path[,fill,
                              black!4]
                            (axis cs:-85.0, 5.0)
                            rectangle (axis cs:-80.0, 10.0);
                        \path[,fill,
                              black!2]
                            (axis cs:-85.0, 10.0)
                            rectangle (axis cs:-80.0, 15.0);
                        \path[,fill,
                              black!1]
                            (axis cs:-85.0, 15.0)
                            rectangle (axis cs:-80.0, 20.0);
                        \path[,fill,
                              black!1]
                            (axis cs:-85.0, 20.0)
                            rectangle (axis cs:-80.0, 25.0);
                        \path[,fill,
                              black!1]
                            (axis cs:-85.0, 150.0)
                            rectangle (axis cs:-80.0, 155.0);
                        \path[,fill,
                              black!1]
                            (axis cs:-85.0, 155.0)
                            rectangle (axis cs:-80.0, 160.0);
                        \path[,fill,
                              black!2]
                            (axis cs:-85.0, 160.0)
                            rectangle (axis cs:-80.0, 165.0);
                        \path[,fill,
                              black!1]
                            (axis cs:-85.0, 165.0)
                            rectangle (axis cs:-80.0, 170.0);
                        \path[,fill,
                              black!2]
                            (axis cs:-85.0, 170.0)
                            rectangle (axis cs:-80.0, 175.0);
                        \path[,fill,
                              black!2]
                            (axis cs:-85.0, 175.0)
                            rectangle (axis cs:-80.0, 180.0);
                        \path[,fill,
                              black!2]
                            (axis cs:-80.0, -180.0)
                            rectangle (axis cs:-75.0, -175.0);
                        \path[,fill,
                              black!1]
                            (axis cs:-80.0, -175.0)
                            rectangle (axis cs:-75.0, -170.0);
                        \path[,fill,
                              black!4]
                            (axis cs:-80.0, -170.0)
                            rectangle (axis cs:-75.0, -165.0);
                        \path[,fill,
                              black!2]
                            (axis cs:-80.0, -165.0)
                            rectangle (axis cs:-75.0, -160.0);
                        \path[,fill,
                              black!2]
                            (axis cs:-80.0, -160.0)
                            rectangle (axis cs:-75.0, -155.0);
                        \path[,fill,
                              black!2]
                            (axis cs:-80.0, -155.0)
                            rectangle (axis cs:-75.0, -150.0);
                        \path[,fill,
                              black!4]
                            (axis cs:-80.0, -150.0)
                            rectangle (axis cs:-75.0, -145.0);
                        \path[,fill,
                              black!2]
                            (axis cs:-80.0, -145.0)
                            rectangle (axis cs:-75.0, -140.0);
                        \path[,fill,
                              black!2]
                            (axis cs:-80.0, -140.0)
                            rectangle (axis cs:-75.0, -135.0);
                        \path[,fill,
                              black!4]
                            (axis cs:-80.0, -135.0)
                            rectangle (axis cs:-75.0, -130.0);
                        \path[,fill,
                              black!2]
                            (axis cs:-80.0, -130.0)
                            rectangle (axis cs:-75.0, -125.0);
                        \path[,fill,
                              black!5]
                            (axis cs:-80.0, -125.0)
                            rectangle (axis cs:-75.0, -120.0);
                        \path[,fill,
                              black!4]
                            (axis cs:-80.0, -120.0)
                            rectangle (axis cs:-75.0, -115.0);
                        \path[,fill,
                              black!7]
                            (axis cs:-80.0, -115.0)
                            rectangle (axis cs:-75.0, -110.0);
                        \path[,fill,
                              black!9]
                            (axis cs:-80.0, -110.0)
                            rectangle (axis cs:-75.0, -105.0);
                        \path[,fill,
                              black!13]
                            (axis cs:-80.0, -105.0)
                            rectangle (axis cs:-75.0, -100.0);
                        \path[,fill,
                              black!21]
                            (axis cs:-80.0, -100.0)
                            rectangle (axis cs:-75.0, -95.0);
                        \path[,fill,
                              black!25]
                            (axis cs:-80.0, -95.0)
                            rectangle (axis cs:-75.0, -90.0);
                        \path[,fill,
                              black!45]
                            (axis cs:-80.0, -90.0)
                            rectangle (axis cs:-75.0, -85.0);
                        \path[,fill,
                              black!69]
                            (axis cs:-80.0, -85.0)
                            rectangle (axis cs:-75.0, -80.0);
                        \path[,fill,
                              black!91]
                            (axis cs:-80.0, -80.0)
                            rectangle (axis cs:-75.0, -75.0);
                        \path[,fill,
                              black!53]
                            (axis cs:-80.0, -75.0)
                            rectangle (axis cs:-75.0, -70.0);
                        \path[,fill,
                              black!43]
                            (axis cs:-80.0, -70.0)
                            rectangle (axis cs:-75.0, -65.0);
                        \path[,fill,
                              black!22]
                            (axis cs:-80.0, -65.0)
                            rectangle (axis cs:-75.0, -60.0);
                        \path[,fill,
                              black!17]
                            (axis cs:-80.0, -60.0)
                            rectangle (axis cs:-75.0, -55.0);
                        \path[,fill,
                              black!15]
                            (axis cs:-80.0, -55.0)
                            rectangle (axis cs:-75.0, -50.0);
                        \path[,fill,
                              black!11]
                            (axis cs:-80.0, -50.0)
                            rectangle (axis cs:-75.0, -45.0);
                        \path[,fill,
                              black!7]
                            (axis cs:-80.0, -45.0)
                            rectangle (axis cs:-75.0, -40.0);
                        \path[,fill,
                              black!7]
                            (axis cs:-80.0, -40.0)
                            rectangle (axis cs:-75.0, -35.0);
                        \path[,fill,
                              black!5]
                            (axis cs:-80.0, -35.0)
                            rectangle (axis cs:-75.0, -30.0);
                        \path[,fill,
                              black!7]
                            (axis cs:-80.0, -30.0)
                            rectangle (axis cs:-75.0, -25.0);
                        \path[,fill,
                              black!4]
                            (axis cs:-80.0, -25.0)
                            rectangle (axis cs:-75.0, -20.0);
                        \path[,fill,
                              black!6]
                            (axis cs:-80.0, -20.0)
                            rectangle (axis cs:-75.0, -15.0);
                        \path[,fill,
                              black!5]
                            (axis cs:-80.0, -15.0)
                            rectangle (axis cs:-75.0, -10.0);
                        \path[,fill,
                              black!5]
                            (axis cs:-80.0, -10.0)
                            rectangle (axis cs:-75.0, -5.0);
                        \path[,fill,
                              black!6]
                            (axis cs:-80.0, -5.0)
                            rectangle (axis cs:-75.0, 0.0);
                        \path[,fill,
                              black!3]
                            (axis cs:-80.0, 0.0)
                            rectangle (axis cs:-75.0, 5.0);
                        \path[,fill,
                              black!2]
                            (axis cs:-80.0, 5.0)
                            rectangle (axis cs:-75.0, 10.0);
                        \path[,fill,
                              black!2]
                            (axis cs:-80.0, 10.0)
                            rectangle (axis cs:-75.0, 15.0);
                        \path[,fill,
                              black!1]
                            (axis cs:-80.0, 25.0)
                            rectangle (axis cs:-75.0, 30.0);
                        \path[,fill,
                              black!1]
                            (axis cs:-80.0, 45.0)
                            rectangle (axis cs:-75.0, 50.0);
                        \path[,fill,
                              black!1]
                            (axis cs:-80.0, 135.0)
                            rectangle (axis cs:-75.0, 140.0);
                        \path[,fill,
                              black!1]
                            (axis cs:-80.0, 165.0)
                            rectangle (axis cs:-75.0, 170.0);
                        \path[,fill,
                              black!2]
                            (axis cs:-80.0, 170.0)
                            rectangle (axis cs:-75.0, 175.0);
                        \path[,fill,
                              black!4]
                            (axis cs:-80.0, 175.0)
                            rectangle (axis cs:-75.0, 180.0);
                        \path[,fill,
                              black!2]
                            (axis cs:-75.0, -180.0)
                            rectangle (axis cs:-70.0, -175.0);
                        \path[,fill,
                              black!1]
                            (axis cs:-75.0, -175.0)
                            rectangle (axis cs:-70.0, -170.0);
                        \path[,fill,
                              black!1]
                            (axis cs:-75.0, -170.0)
                            rectangle (axis cs:-70.0, -165.0);
                        \path[,fill,
                              black!3]
                            (axis cs:-75.0, -165.0)
                            rectangle (axis cs:-70.0, -160.0);
                        \path[,fill,
                              black!2]
                            (axis cs:-75.0, -160.0)
                            rectangle (axis cs:-70.0, -155.0);
                        \path[,fill,
                              black!2]
                            (axis cs:-75.0, -155.0)
                            rectangle (axis cs:-70.0, -150.0);
                        \path[,fill,
                              black!4]
                            (axis cs:-75.0, -150.0)
                            rectangle (axis cs:-70.0, -145.0);
                        \path[,fill,
                              black!2]
                            (axis cs:-75.0, -145.0)
                            rectangle (axis cs:-70.0, -140.0);
                        \path[,fill,
                              black!3]
                            (axis cs:-75.0, -140.0)
                            rectangle (axis cs:-70.0, -135.0);
                        \path[,fill,
                              black!5]
                            (axis cs:-75.0, -135.0)
                            rectangle (axis cs:-70.0, -130.0);
                        \path[,fill,
                              black!4]
                            (axis cs:-75.0, -130.0)
                            rectangle (axis cs:-70.0, -125.0);
                        \path[,fill,
                              black!4]
                            (axis cs:-75.0, -125.0)
                            rectangle (axis cs:-70.0, -120.0);
                        \path[,fill,
                              black!6]
                            (axis cs:-75.0, -120.0)
                            rectangle (axis cs:-70.0, -115.0);
                        \path[,fill,
                              black!5]
                            (axis cs:-75.0, -115.0)
                            rectangle (axis cs:-70.0, -110.0);
                        \path[,fill,
                              black!7]
                            (axis cs:-75.0, -110.0)
                            rectangle (axis cs:-70.0, -105.0);
                        \path[,fill,
                              black!9]
                            (axis cs:-75.0, -105.0)
                            rectangle (axis cs:-70.0, -100.0);
                        \path[,fill,
                              black!14]
                            (axis cs:-75.0, -100.0)
                            rectangle (axis cs:-70.0, -95.0);
                        \path[,fill,
                              black!13]
                            (axis cs:-75.0, -95.0)
                            rectangle (axis cs:-70.0, -90.0);
                        \path[,fill,
                              black!31]
                            (axis cs:-75.0, -90.0)
                            rectangle (axis cs:-70.0, -85.0);
                        \path[,fill,
                              black!52]
                            (axis cs:-75.0, -85.0)
                            rectangle (axis cs:-70.0, -80.0);
                        \path[,fill,
                              black!70]
                            (axis cs:-75.0, -80.0)
                            rectangle (axis cs:-70.0, -75.0);
                        \path[,fill,
                              black!79]
                            (axis cs:-75.0, -75.0)
                            rectangle (axis cs:-70.0, -70.0);
                        \path[,fill,
                              black!53]
                            (axis cs:-75.0, -70.0)
                            rectangle (axis cs:-70.0, -65.0);
                        \path[,fill,
                              black!37]
                            (axis cs:-75.0, -65.0)
                            rectangle (axis cs:-70.0, -60.0);
                        \path[,fill,
                              black!29]
                            (axis cs:-75.0, -60.0)
                            rectangle (axis cs:-70.0, -55.0);
                        \path[,fill,
                              black!11]
                            (axis cs:-75.0, -55.0)
                            rectangle (axis cs:-70.0, -50.0);
                        \path[,fill,
                              black!13]
                            (axis cs:-75.0, -50.0)
                            rectangle (axis cs:-70.0, -45.0);
                        \path[,fill,
                              black!11]
                            (axis cs:-75.0, -45.0)
                            rectangle (axis cs:-70.0, -40.0);
                        \path[,fill,
                              black!6]
                            (axis cs:-75.0, -40.0)
                            rectangle (axis cs:-70.0, -35.0);
                        \path[,fill,
                              black!3]
                            (axis cs:-75.0, -35.0)
                            rectangle (axis cs:-70.0, -30.0);
                        \path[,fill,
                              black!5]
                            (axis cs:-75.0, -30.0)
                            rectangle (axis cs:-70.0, -25.0);
                        \path[,fill,
                              black!6]
                            (axis cs:-75.0, -25.0)
                            rectangle (axis cs:-70.0, -20.0);
                        \path[,fill,
                              black!4]
                            (axis cs:-75.0, -20.0)
                            rectangle (axis cs:-70.0, -15.0);
                        \path[,fill,
                              black!2]
                            (axis cs:-75.0, -15.0)
                            rectangle (axis cs:-70.0, -10.0);
                        \path[,fill,
                              black!3]
                            (axis cs:-75.0, -10.0)
                            rectangle (axis cs:-70.0, -5.0);
                        \path[,fill,
                              black!6]
                            (axis cs:-75.0, -5.0)
                            rectangle (axis cs:-70.0, 0.0);
                        \path[,fill,
                              black!4]
                            (axis cs:-75.0, 0.0)
                            rectangle (axis cs:-70.0, 5.0);
                        \path[,fill,
                              black!4]
                            (axis cs:-75.0, 5.0)
                            rectangle (axis cs:-70.0, 10.0);
                        \path[,fill,
                              black!2]
                            (axis cs:-75.0, 10.0)
                            rectangle (axis cs:-70.0, 15.0);
                        \path[,fill,
                              black!1]
                            (axis cs:-75.0, 15.0)
                            rectangle (axis cs:-70.0, 20.0);
                        \path[,fill,
                              black!1]
                            (axis cs:-75.0, 20.0)
                            rectangle (axis cs:-70.0, 25.0);
                        \path[,fill,
                              black!1]
                            (axis cs:-75.0, 30.0)
                            rectangle (axis cs:-70.0, 35.0);
                        \path[,fill,
                              black!1]
                            (axis cs:-75.0, 40.0)
                            rectangle (axis cs:-70.0, 45.0);
                        \path[,fill,
                              black!1]
                            (axis cs:-75.0, 90.0)
                            rectangle (axis cs:-70.0, 95.0);
                        \path[,fill,
                              black!1]
                            (axis cs:-75.0, 145.0)
                            rectangle (axis cs:-70.0, 150.0);
                        \path[,fill,
                              black!1]
                            (axis cs:-75.0, 150.0)
                            rectangle (axis cs:-70.0, 155.0);
                        \path[,fill,
                              black!2]
                            (axis cs:-75.0, 155.0)
                            rectangle (axis cs:-70.0, 160.0);
                        \path[,fill,
                              black!1]
                            (axis cs:-75.0, 160.0)
                            rectangle (axis cs:-70.0, 165.0);
                        \path[,fill,
                              black!2]
                            (axis cs:-75.0, 165.0)
                            rectangle (axis cs:-70.0, 170.0);
                        \path[,fill,
                              black!3]
                            (axis cs:-75.0, 170.0)
                            rectangle (axis cs:-70.0, 175.0);
                        \path[,fill,
                              black!2]
                            (axis cs:-75.0, 175.0)
                            rectangle (axis cs:-70.0, 180.0);
                        \path[,fill,
                              black!2]
                            (axis cs:-70.0, -180.0)
                            rectangle (axis cs:-65.0, -175.0);
                        \path[,fill,
                              black!2]
                            (axis cs:-70.0, -175.0)
                            rectangle (axis cs:-65.0, -170.0);
                        \path[,fill,
                              black!2]
                            (axis cs:-70.0, -170.0)
                            rectangle (axis cs:-65.0, -165.0);
                        \path[,fill,
                              black!2]
                            (axis cs:-70.0, -165.0)
                            rectangle (axis cs:-65.0, -160.0);
                        \path[,fill,
                              black!1]
                            (axis cs:-70.0, -160.0)
                            rectangle (axis cs:-65.0, -155.0);
                        \path[,fill,
                              black!3]
                            (axis cs:-70.0, -155.0)
                            rectangle (axis cs:-65.0, -150.0);
                        \path[,fill,
                              black!2]
                            (axis cs:-70.0, -150.0)
                            rectangle (axis cs:-65.0, -145.0);
                        \path[,fill,
                              black!3]
                            (axis cs:-70.0, -145.0)
                            rectangle (axis cs:-65.0, -140.0);
                        \path[,fill,
                              black!2]
                            (axis cs:-70.0, -140.0)
                            rectangle (axis cs:-65.0, -135.0);
                        \path[,fill,
                              black!2]
                            (axis cs:-70.0, -135.0)
                            rectangle (axis cs:-65.0, -130.0);
                        \path[,fill,
                              black!1]
                            (axis cs:-70.0, -130.0)
                            rectangle (axis cs:-65.0, -125.0);
                        \path[,fill,
                              black!2]
                            (axis cs:-70.0, -125.0)
                            rectangle (axis cs:-65.0, -120.0);
                        \path[,fill,
                              black!6]
                            (axis cs:-70.0, -120.0)
                            rectangle (axis cs:-65.0, -115.0);
                        \path[,fill,
                              black!5]
                            (axis cs:-70.0, -115.0)
                            rectangle (axis cs:-65.0, -110.0);
                        \path[,fill,
                              black!6]
                            (axis cs:-70.0, -110.0)
                            rectangle (axis cs:-65.0, -105.0);
                        \path[,fill,
                              black!7]
                            (axis cs:-70.0, -105.0)
                            rectangle (axis cs:-65.0, -100.0);
                        \path[,fill,
                              black!9]
                            (axis cs:-70.0, -100.0)
                            rectangle (axis cs:-65.0, -95.0);
                        \path[,fill,
                              black!12]
                            (axis cs:-70.0, -95.0)
                            rectangle (axis cs:-65.0, -90.0);
                        \path[,fill,
                              black!21]
                            (axis cs:-70.0, -90.0)
                            rectangle (axis cs:-65.0, -85.0);
                        \path[,fill,
                              black!34]
                            (axis cs:-70.0, -85.0)
                            rectangle (axis cs:-65.0, -80.0);
                        \path[,fill,
                              black!48]
                            (axis cs:-70.0, -80.0)
                            rectangle (axis cs:-65.0, -75.0);
                        \path[,fill,
                              black!68]
                            (axis cs:-70.0, -75.0)
                            rectangle (axis cs:-65.0, -70.0);
                        \path[,fill,
                              black!73]
                            (axis cs:-70.0, -70.0)
                            rectangle (axis cs:-65.0, -65.0);
                        \path[,fill,
                              black!58]
                            (axis cs:-70.0, -65.0)
                            rectangle (axis cs:-65.0, -60.0);
                        \path[,fill,
                              black!36]
                            (axis cs:-70.0, -60.0)
                            rectangle (axis cs:-65.0, -55.0);
                        \path[,fill,
                              black!26]
                            (axis cs:-70.0, -55.0)
                            rectangle (axis cs:-65.0, -50.0);
                        \path[,fill,
                              black!17]
                            (axis cs:-70.0, -50.0)
                            rectangle (axis cs:-65.0, -45.0);
                        \path[,fill,
                              black!14]
                            (axis cs:-70.0, -45.0)
                            rectangle (axis cs:-65.0, -40.0);
                        \path[,fill,
                              black!9]
                            (axis cs:-70.0, -40.0)
                            rectangle (axis cs:-65.0, -35.0);
                        \path[,fill,
                              black!13]
                            (axis cs:-70.0, -35.0)
                            rectangle (axis cs:-65.0, -30.0);
                        \path[,fill,
                              black!7]
                            (axis cs:-70.0, -30.0)
                            rectangle (axis cs:-65.0, -25.0);
                        \path[,fill,
                              black!9]
                            (axis cs:-70.0, -25.0)
                            rectangle (axis cs:-65.0, -20.0);
                        \path[,fill,
                              black!8]
                            (axis cs:-70.0, -20.0)
                            rectangle (axis cs:-65.0, -15.0);
                        \path[,fill,
                              black!6]
                            (axis cs:-70.0, -15.0)
                            rectangle (axis cs:-65.0, -10.0);
                        \path[,fill,
                              black!4]
                            (axis cs:-70.0, -10.0)
                            rectangle (axis cs:-65.0, -5.0);
                        \path[,fill,
                              black!2]
                            (axis cs:-70.0, -5.0)
                            rectangle (axis cs:-65.0, 0.0);
                        \path[,fill,
                              black!5]
                            (axis cs:-70.0, 0.0)
                            rectangle (axis cs:-65.0, 5.0);
                        \path[,fill,
                              black!1]
                            (axis cs:-70.0, 5.0)
                            rectangle (axis cs:-65.0, 10.0);
                        \path[,fill,
                              black!2]
                            (axis cs:-70.0, 10.0)
                            rectangle (axis cs:-65.0, 15.0);
                        \path[,fill,
                              black!2]
                            (axis cs:-70.0, 15.0)
                            rectangle (axis cs:-65.0, 20.0);
                        \path[,fill,
                              black!1]
                            (axis cs:-70.0, 20.0)
                            rectangle (axis cs:-65.0, 25.0);
                        \path[,fill,
                              black!1]
                            (axis cs:-70.0, 25.0)
                            rectangle (axis cs:-65.0, 30.0);
                        \path[,fill,
                              black!1]
                            (axis cs:-70.0, 115.0)
                            rectangle (axis cs:-65.0, 120.0);
                        \path[,fill,
                              black!1]
                            (axis cs:-70.0, 130.0)
                            rectangle (axis cs:-65.0, 135.0);
                        \path[,fill,
                              black!1]
                            (axis cs:-70.0, 140.0)
                            rectangle (axis cs:-65.0, 145.0);
                        \path[,fill,
                              black!1]
                            (axis cs:-70.0, 150.0)
                            rectangle (axis cs:-65.0, 155.0);
                        \path[,fill,
                              black!2]
                            (axis cs:-70.0, 155.0)
                            rectangle (axis cs:-65.0, 160.0);
                        \path[,fill,
                              black!1]
                            (axis cs:-70.0, 160.0)
                            rectangle (axis cs:-65.0, 165.0);
                        \path[,fill,
                              black!2]
                            (axis cs:-70.0, 165.0)
                            rectangle (axis cs:-65.0, 170.0);
                        \path[,fill,
                              black!4]
                            (axis cs:-70.0, 170.0)
                            rectangle (axis cs:-65.0, 175.0);
                        \path[,fill,
                              black!2]
                            (axis cs:-70.0, 175.0)
                            rectangle (axis cs:-65.0, 180.0);
                        \path[,fill,
                              black!1]
                            (axis cs:-65.0, -180.0)
                            rectangle (axis cs:-60.0, -175.0);
                        \path[,fill,
                              black!1]
                            (axis cs:-65.0, -170.0)
                            rectangle (axis cs:-60.0, -165.0);
                        \path[,fill,
                              black!1]
                            (axis cs:-65.0, -165.0)
                            rectangle (axis cs:-60.0, -160.0);
                        \path[,fill,
                              black!2]
                            (axis cs:-65.0, -160.0)
                            rectangle (axis cs:-60.0, -155.0);
                        \path[,fill,
                              black!2]
                            (axis cs:-65.0, -155.0)
                            rectangle (axis cs:-60.0, -150.0);
                        \path[,fill,
                              black!2]
                            (axis cs:-65.0, -150.0)
                            rectangle (axis cs:-60.0, -145.0);
                        \path[,fill,
                              black!4]
                            (axis cs:-65.0, -145.0)
                            rectangle (axis cs:-60.0, -140.0);
                        \path[,fill,
                              black!4]
                            (axis cs:-65.0, -135.0)
                            rectangle (axis cs:-60.0, -130.0);
                        \path[,fill,
                              black!3]
                            (axis cs:-65.0, -130.0)
                            rectangle (axis cs:-60.0, -125.0);
                        \path[,fill,
                              black!1]
                            (axis cs:-65.0, -125.0)
                            rectangle (axis cs:-60.0, -120.0);
                        \path[,fill,
                              black!3]
                            (axis cs:-65.0, -120.0)
                            rectangle (axis cs:-60.0, -115.0);
                        \path[,fill,
                              black!4]
                            (axis cs:-65.0, -115.0)
                            rectangle (axis cs:-60.0, -110.0);
                        \path[,fill,
                              black!2]
                            (axis cs:-65.0, -110.0)
                            rectangle (axis cs:-60.0, -105.0);
                        \path[,fill,
                              black!7]
                            (axis cs:-65.0, -105.0)
                            rectangle (axis cs:-60.0, -100.0);
                        \path[,fill,
                              black!6]
                            (axis cs:-65.0, -100.0)
                            rectangle (axis cs:-60.0, -95.0);
                        \path[,fill,
                              black!12]
                            (axis cs:-65.0, -95.0)
                            rectangle (axis cs:-60.0, -90.0);
                        \path[,fill,
                              black!10]
                            (axis cs:-65.0, -90.0)
                            rectangle (axis cs:-60.0, -85.0);
                        \path[,fill,
                              black!18]
                            (axis cs:-65.0, -85.0)
                            rectangle (axis cs:-60.0, -80.0);
                        \path[,fill,
                              black!43]
                            (axis cs:-65.0, -80.0)
                            rectangle (axis cs:-60.0, -75.0);
                        \path[,fill,
                              black!55]
                            (axis cs:-65.0, -75.0)
                            rectangle (axis cs:-60.0, -70.0);
                        \path[,fill,
                              black!67]
                            (axis cs:-65.0, -70.0)
                            rectangle (axis cs:-60.0, -65.0);
                        \path[,fill,
                              black!79]
                            (axis cs:-65.0, -65.0)
                            rectangle (axis cs:-60.0, -60.0);
                        \path[,fill,
                              black!51]
                            (axis cs:-65.0, -60.0)
                            rectangle (axis cs:-60.0, -55.0);
                        \path[,fill,
                              black!42]
                            (axis cs:-65.0, -55.0)
                            rectangle (axis cs:-60.0, -50.0);
                        \path[,fill,
                              black!21]
                            (axis cs:-65.0, -50.0)
                            rectangle (axis cs:-60.0, -45.0);
                        \path[,fill,
                              black!16]
                            (axis cs:-65.0, -45.0)
                            rectangle (axis cs:-60.0, -40.0);
                        \path[,fill,
                              black!11]
                            (axis cs:-65.0, -40.0)
                            rectangle (axis cs:-60.0, -35.0);
                        \path[,fill,
                              black!11]
                            (axis cs:-65.0, -35.0)
                            rectangle (axis cs:-60.0, -30.0);
                        \path[,fill,
                              black!6]
                            (axis cs:-65.0, -30.0)
                            rectangle (axis cs:-60.0, -25.0);
                        \path[,fill,
                              black!5]
                            (axis cs:-65.0, -25.0)
                            rectangle (axis cs:-60.0, -20.0);
                        \path[,fill,
                              black!9]
                            (axis cs:-65.0, -20.0)
                            rectangle (axis cs:-60.0, -15.0);
                        \path[,fill,
                              black!6]
                            (axis cs:-65.0, -15.0)
                            rectangle (axis cs:-60.0, -10.0);
                        \path[,fill,
                              black!5]
                            (axis cs:-65.0, -10.0)
                            rectangle (axis cs:-60.0, -5.0);
                        \path[,fill,
                              black!5]
                            (axis cs:-65.0, -5.0)
                            rectangle (axis cs:-60.0, 0.0);
                        \path[,fill,
                              black!5]
                            (axis cs:-65.0, 0.0)
                            rectangle (axis cs:-60.0, 5.0);
                        \path[,fill,
                              black!5]
                            (axis cs:-65.0, 5.0)
                            rectangle (axis cs:-60.0, 10.0);
                        \path[,fill,
                              black!1]
                            (axis cs:-65.0, 10.0)
                            rectangle (axis cs:-60.0, 15.0);
                        \path[,fill,
                              black!2]
                            (axis cs:-65.0, 15.0)
                            rectangle (axis cs:-60.0, 20.0);
                        \path[,fill,
                              black!1]
                            (axis cs:-65.0, 20.0)
                            rectangle (axis cs:-60.0, 25.0);
                        \path[,fill,
                              black!1]
                            (axis cs:-65.0, 25.0)
                            rectangle (axis cs:-60.0, 30.0);
                        \path[,fill,
                              black!1]
                            (axis cs:-65.0, 35.0)
                            rectangle (axis cs:-60.0, 40.0);
                        \path[,fill,
                              black!1]
                            (axis cs:-65.0, 50.0)
                            rectangle (axis cs:-60.0, 55.0);
                        \path[,fill,
                              black!1]
                            (axis cs:-65.0, 140.0)
                            rectangle (axis cs:-60.0, 145.0);
                        \path[,fill,
                              black!1]
                            (axis cs:-65.0, 145.0)
                            rectangle (axis cs:-60.0, 150.0);
                        \path[,fill,
                              black!1]
                            (axis cs:-65.0, 155.0)
                            rectangle (axis cs:-60.0, 160.0);
                        \path[,fill,
                              black!4]
                            (axis cs:-65.0, 160.0)
                            rectangle (axis cs:-60.0, 165.0);
                        \path[,fill,
                              black!5]
                            (axis cs:-65.0, 165.0)
                            rectangle (axis cs:-60.0, 170.0);
                        \path[,fill,
                              black!4]
                            (axis cs:-65.0, 170.0)
                            rectangle (axis cs:-60.0, 175.0);
                        \path[,fill,
                              black!3]
                            (axis cs:-65.0, 175.0)
                            rectangle (axis cs:-60.0, 180.0);
                        \path[,fill,
                              black!2]
                            (axis cs:-60.0, -180.0)
                            rectangle (axis cs:-55.0, -175.0);
                        \path[,fill,
                              black!1]
                            (axis cs:-60.0, -175.0)
                            rectangle (axis cs:-55.0, -170.0);
                        \path[,fill,
                              black!2]
                            (axis cs:-60.0, -170.0)
                            rectangle (axis cs:-55.0, -165.0);
                        \path[,fill,
                              black!2]
                            (axis cs:-60.0, -165.0)
                            rectangle (axis cs:-55.0, -160.0);
                        \path[,fill,
                              black!1]
                            (axis cs:-60.0, -160.0)
                            rectangle (axis cs:-55.0, -155.0);
                        \path[,fill,
                              black!1]
                            (axis cs:-60.0, -155.0)
                            rectangle (axis cs:-55.0, -150.0);
                        \path[,fill,
                              black!1]
                            (axis cs:-60.0, -150.0)
                            rectangle (axis cs:-55.0, -145.0);
                        \path[,fill,
                              black!1]
                            (axis cs:-60.0, -145.0)
                            rectangle (axis cs:-55.0, -140.0);
                        \path[,fill,
                              black!3]
                            (axis cs:-60.0, -140.0)
                            rectangle (axis cs:-55.0, -135.0);
                        \path[,fill,
                              black!2]
                            (axis cs:-60.0, -135.0)
                            rectangle (axis cs:-55.0, -130.0);
                        \path[,fill,
                              black!2]
                            (axis cs:-60.0, -130.0)
                            rectangle (axis cs:-55.0, -125.0);
                        \path[,fill,
                              black!2]
                            (axis cs:-60.0, -125.0)
                            rectangle (axis cs:-55.0, -120.0);
                        \path[,fill,
                              black!4]
                            (axis cs:-60.0, -120.0)
                            rectangle (axis cs:-55.0, -115.0);
                        \path[,fill,
                              black!1]
                            (axis cs:-60.0, -115.0)
                            rectangle (axis cs:-55.0, -110.0);
                        \path[,fill,
                              black!4]
                            (axis cs:-60.0, -110.0)
                            rectangle (axis cs:-55.0, -105.0);
                        \path[,fill,
                              black!5]
                            (axis cs:-60.0, -105.0)
                            rectangle (axis cs:-55.0, -100.0);
                        \path[,fill,
                              black!6]
                            (axis cs:-60.0, -100.0)
                            rectangle (axis cs:-55.0, -95.0);
                        \path[,fill,
                              black!5]
                            (axis cs:-60.0, -95.0)
                            rectangle (axis cs:-55.0, -90.0);
                        \path[,fill,
                              black!9]
                            (axis cs:-60.0, -90.0)
                            rectangle (axis cs:-55.0, -85.0);
                        \path[,fill,
                              black!13]
                            (axis cs:-60.0, -85.0)
                            rectangle (axis cs:-55.0, -80.0);
                        \path[,fill,
                              black!22]
                            (axis cs:-60.0, -80.0)
                            rectangle (axis cs:-55.0, -75.0);
                        \path[,fill,
                              black!35]
                            (axis cs:-60.0, -75.0)
                            rectangle (axis cs:-55.0, -70.0);
                        \path[,fill,
                              black!51]
                            (axis cs:-60.0, -70.0)
                            rectangle (axis cs:-55.0, -65.0);
                        \path[,fill,
                              black!75]
                            (axis cs:-60.0, -65.0)
                            rectangle (axis cs:-55.0, -60.0);
                        \path[,fill,
                              black!76]
                            (axis cs:-60.0, -60.0)
                            rectangle (axis cs:-55.0, -55.0);
                        \path[,fill,
                              black!52]
                            (axis cs:-60.0, -55.0)
                            rectangle (axis cs:-55.0, -50.0);
                        \path[,fill,
                              black!37]
                            (axis cs:-60.0, -50.0)
                            rectangle (axis cs:-55.0, -45.0);
                        \path[,fill,
                              black!27]
                            (axis cs:-60.0, -45.0)
                            rectangle (axis cs:-55.0, -40.0);
                        \path[,fill,
                              black!19]
                            (axis cs:-60.0, -40.0)
                            rectangle (axis cs:-55.0, -35.0);
                        \path[,fill,
                              black!17]
                            (axis cs:-60.0, -35.0)
                            rectangle (axis cs:-55.0, -30.0);
                        \path[,fill,
                              black!12]
                            (axis cs:-60.0, -30.0)
                            rectangle (axis cs:-55.0, -25.0);
                        \path[,fill,
                              black!12]
                            (axis cs:-60.0, -25.0)
                            rectangle (axis cs:-55.0, -20.0);
                        \path[,fill,
                              black!8]
                            (axis cs:-60.0, -20.0)
                            rectangle (axis cs:-55.0, -15.0);
                        \path[,fill,
                              black!8]
                            (axis cs:-60.0, -15.0)
                            rectangle (axis cs:-55.0, -10.0);
                        \path[,fill,
                              black!5]
                            (axis cs:-60.0, -10.0)
                            rectangle (axis cs:-55.0, -5.0);
                        \path[,fill,
                              black!8]
                            (axis cs:-60.0, -5.0)
                            rectangle (axis cs:-55.0, 0.0);
                        \path[,fill,
                              black!5]
                            (axis cs:-60.0, 0.0)
                            rectangle (axis cs:-55.0, 5.0);
                        \path[,fill,
                              black!4]
                            (axis cs:-60.0, 5.0)
                            rectangle (axis cs:-55.0, 10.0);
                        \path[,fill,
                              black!2]
                            (axis cs:-60.0, 10.0)
                            rectangle (axis cs:-55.0, 15.0);
                        \path[,fill,
                              black!2]
                            (axis cs:-60.0, 15.0)
                            rectangle (axis cs:-55.0, 20.0);
                        \path[,fill,
                              black!1]
                            (axis cs:-60.0, 20.0)
                            rectangle (axis cs:-55.0, 25.0);
                        \path[,fill,
                              black!1]
                            (axis cs:-60.0, 30.0)
                            rectangle (axis cs:-55.0, 35.0);
                        \path[,fill,
                              black!1]
                            (axis cs:-60.0, 35.0)
                            rectangle (axis cs:-55.0, 40.0);
                        \path[,fill,
                              black!2]
                            (axis cs:-60.0, 50.0)
                            rectangle (axis cs:-55.0, 55.0);
                        \path[,fill,
                              black!1]
                            (axis cs:-60.0, 75.0)
                            rectangle (axis cs:-55.0, 80.0);
                        \path[,fill,
                              black!1]
                            (axis cs:-60.0, 155.0)
                            rectangle (axis cs:-55.0, 160.0);
                        \path[,fill,
                              black!2]
                            (axis cs:-60.0, 160.0)
                            rectangle (axis cs:-55.0, 165.0);
                        \path[,fill,
                              black!1]
                            (axis cs:-60.0, 165.0)
                            rectangle (axis cs:-55.0, 170.0);
                        \path[,fill,
                              black!4]
                            (axis cs:-60.0, 170.0)
                            rectangle (axis cs:-55.0, 175.0);
                        \path[,fill,
                              black!1]
                            (axis cs:-60.0, 175.0)
                            rectangle (axis cs:-55.0, 180.0);
                        \path[,fill,
                              black!2]
                            (axis cs:-55.0, -180.0)
                            rectangle (axis cs:-50.0, -175.0);
                        \path[,fill,
                              black!2]
                            (axis cs:-55.0, -175.0)
                            rectangle (axis cs:-50.0, -170.0);
                        \path[,fill,
                              black!1]
                            (axis cs:-55.0, -170.0)
                            rectangle (axis cs:-50.0, -165.0);
                        \path[,fill,
                              black!1]
                            (axis cs:-55.0, -165.0)
                            rectangle (axis cs:-50.0, -160.0);
                        \path[,fill,
                              black!2]
                            (axis cs:-55.0, -160.0)
                            rectangle (axis cs:-50.0, -155.0);
                        \path[,fill,
                              black!1]
                            (axis cs:-55.0, -155.0)
                            rectangle (axis cs:-50.0, -150.0);
                        \path[,fill,
                              black!1]
                            (axis cs:-55.0, -150.0)
                            rectangle (axis cs:-50.0, -145.0);
                        \path[,fill,
                              black!2]
                            (axis cs:-55.0, -145.0)
                            rectangle (axis cs:-50.0, -140.0);
                        \path[,fill,
                              black!1]
                            (axis cs:-55.0, -140.0)
                            rectangle (axis cs:-50.0, -135.0);
                        \path[,fill,
                              black!5]
                            (axis cs:-55.0, -135.0)
                            rectangle (axis cs:-50.0, -130.0);
                        \path[,fill,
                              black!1]
                            (axis cs:-55.0, -130.0)
                            rectangle (axis cs:-50.0, -125.0);
                        \path[,fill,
                              black!1]
                            (axis cs:-55.0, -125.0)
                            rectangle (axis cs:-50.0, -120.0);
                        \path[,fill,
                              black!2]
                            (axis cs:-55.0, -120.0)
                            rectangle (axis cs:-50.0, -115.0);
                        \path[,fill,
                              black!3]
                            (axis cs:-55.0, -115.0)
                            rectangle (axis cs:-50.0, -110.0);
                        \path[,fill,
                              black!4]
                            (axis cs:-55.0, -110.0)
                            rectangle (axis cs:-50.0, -105.0);
                        \path[,fill,
                              black!5]
                            (axis cs:-55.0, -105.0)
                            rectangle (axis cs:-50.0, -100.0);
                        \path[,fill,
                              black!5]
                            (axis cs:-55.0, -100.0)
                            rectangle (axis cs:-50.0, -95.0);
                        \path[,fill,
                              black!4]
                            (axis cs:-55.0, -95.0)
                            rectangle (axis cs:-50.0, -90.0);
                        \path[,fill,
                              black!5]
                            (axis cs:-55.0, -90.0)
                            rectangle (axis cs:-50.0, -85.0);
                        \path[,fill,
                              black!10]
                            (axis cs:-55.0, -85.0)
                            rectangle (axis cs:-50.0, -80.0);
                        \path[,fill,
                              black!14]
                            (axis cs:-55.0, -80.0)
                            rectangle (axis cs:-50.0, -75.0);
                        \path[,fill,
                              black!19]
                            (axis cs:-55.0, -75.0)
                            rectangle (axis cs:-50.0, -70.0);
                        \path[,fill,
                              black!35]
                            (axis cs:-55.0, -70.0)
                            rectangle (axis cs:-50.0, -65.0);
                        \path[,fill,
                              black!54]
                            (axis cs:-55.0, -65.0)
                            rectangle (axis cs:-50.0, -60.0);
                        \path[,fill,
                              black!68]
                            (axis cs:-55.0, -60.0)
                            rectangle (axis cs:-50.0, -55.0);
                        \path[,fill,
                              black!73]
                            (axis cs:-55.0, -55.0)
                            rectangle (axis cs:-50.0, -50.0);
                        \path[,fill,
                              black!62]
                            (axis cs:-55.0, -50.0)
                            rectangle (axis cs:-50.0, -45.0);
                        \path[,fill,
                              black!42]
                            (axis cs:-55.0, -45.0)
                            rectangle (axis cs:-50.0, -40.0);
                        \path[,fill,
                              black!33]
                            (axis cs:-55.0, -40.0)
                            rectangle (axis cs:-50.0, -35.0);
                        \path[,fill,
                              black!17]
                            (axis cs:-55.0, -35.0)
                            rectangle (axis cs:-50.0, -30.0);
                        \path[,fill,
                              black!14]
                            (axis cs:-55.0, -30.0)
                            rectangle (axis cs:-50.0, -25.0);
                        \path[,fill,
                              black!13]
                            (axis cs:-55.0, -25.0)
                            rectangle (axis cs:-50.0, -20.0);
                        \path[,fill,
                              black!8]
                            (axis cs:-55.0, -20.0)
                            rectangle (axis cs:-50.0, -15.0);
                        \path[,fill,
                              black!10]
                            (axis cs:-55.0, -15.0)
                            rectangle (axis cs:-50.0, -10.0);
                        \path[,fill,
                              black!10]
                            (axis cs:-55.0, -10.0)
                            rectangle (axis cs:-50.0, -5.0);
                        \path[,fill,
                              black!4]
                            (axis cs:-55.0, -5.0)
                            rectangle (axis cs:-50.0, 0.0);
                        \path[,fill,
                              black!4]
                            (axis cs:-55.0, 0.0)
                            rectangle (axis cs:-50.0, 5.0);
                        \path[,fill,
                              black!3]
                            (axis cs:-55.0, 5.0)
                            rectangle (axis cs:-50.0, 10.0);
                        \path[,fill,
                              black!3]
                            (axis cs:-55.0, 10.0)
                            rectangle (axis cs:-50.0, 15.0);
                        \path[,fill,
                              black!1]
                            (axis cs:-55.0, 15.0)
                            rectangle (axis cs:-50.0, 20.0);
                        \path[,fill,
                              black!2]
                            (axis cs:-55.0, 20.0)
                            rectangle (axis cs:-50.0, 25.0);
                        \path[,fill,
                              black!1]
                            (axis cs:-55.0, 50.0)
                            rectangle (axis cs:-50.0, 55.0);
                        \path[,fill,
                              black!1]
                            (axis cs:-55.0, 60.0)
                            rectangle (axis cs:-50.0, 65.0);
                        \path[,fill,
                              black!1]
                            (axis cs:-55.0, 135.0)
                            rectangle (axis cs:-50.0, 140.0);
                        \path[,fill,
                              black!1]
                            (axis cs:-55.0, 145.0)
                            rectangle (axis cs:-50.0, 150.0);
                        \path[,fill,
                              black!2]
                            (axis cs:-55.0, 150.0)
                            rectangle (axis cs:-50.0, 155.0);
                        \path[,fill,
                              black!3]
                            (axis cs:-55.0, 155.0)
                            rectangle (axis cs:-50.0, 160.0);
                        \path[,fill,
                              black!1]
                            (axis cs:-55.0, 160.0)
                            rectangle (axis cs:-50.0, 165.0);
                        \path[,fill,
                              black!3]
                            (axis cs:-55.0, 165.0)
                            rectangle (axis cs:-50.0, 170.0);
                        \path[,fill,
                              black!1]
                            (axis cs:-55.0, 170.0)
                            rectangle (axis cs:-50.0, 175.0);
                        \path[,fill,
                              black!4]
                            (axis cs:-55.0, 175.0)
                            rectangle (axis cs:-50.0, 180.0);
                        \path[,fill,
                              black!1]
                            (axis cs:-50.0, -180.0)
                            rectangle (axis cs:-45.0, -175.0);
                        \path[,fill,
                              black!2]
                            (axis cs:-50.0, -175.0)
                            rectangle (axis cs:-45.0, -170.0);
                        \path[,fill,
                              black!2]
                            (axis cs:-50.0, -170.0)
                            rectangle (axis cs:-45.0, -165.0);
                        \path[,fill,
                              black!1]
                            (axis cs:-50.0, -165.0)
                            rectangle (axis cs:-45.0, -160.0);
                        \path[,fill,
                              black!1]
                            (axis cs:-50.0, -160.0)
                            rectangle (axis cs:-45.0, -155.0);
                        \path[,fill,
                              black!1]
                            (axis cs:-50.0, -155.0)
                            rectangle (axis cs:-45.0, -150.0);
                        \path[,fill,
                              black!2]
                            (axis cs:-50.0, -150.0)
                            rectangle (axis cs:-45.0, -145.0);
                        \path[,fill,
                              black!1]
                            (axis cs:-50.0, -140.0)
                            rectangle (axis cs:-45.0, -135.0);
                        \path[,fill,
                              black!1]
                            (axis cs:-50.0, -135.0)
                            rectangle (axis cs:-45.0, -130.0);
                        \path[,fill,
                              black!3]
                            (axis cs:-50.0, -130.0)
                            rectangle (axis cs:-45.0, -125.0);
                        \path[,fill,
                              black!2]
                            (axis cs:-50.0, -125.0)
                            rectangle (axis cs:-45.0, -120.0);
                        \path[,fill,
                              black!1]
                            (axis cs:-50.0, -120.0)
                            rectangle (axis cs:-45.0, -115.0);
                        \path[,fill,
                              black!2]
                            (axis cs:-50.0, -115.0)
                            rectangle (axis cs:-45.0, -110.0);
                        \path[,fill,
                              black!2]
                            (axis cs:-50.0, -110.0)
                            rectangle (axis cs:-45.0, -105.0);
                        \path[,fill,
                              black!1]
                            (axis cs:-50.0, -105.0)
                            rectangle (axis cs:-45.0, -100.0);
                        \path[,fill,
                              black!4]
                            (axis cs:-50.0, -100.0)
                            rectangle (axis cs:-45.0, -95.0);
                        \path[,fill,
                              black!7]
                            (axis cs:-50.0, -95.0)
                            rectangle (axis cs:-45.0, -90.0);
                        \path[,fill,
                              black!4]
                            (axis cs:-50.0, -90.0)
                            rectangle (axis cs:-45.0, -85.0);
                        \path[,fill,
                              black!7]
                            (axis cs:-50.0, -85.0)
                            rectangle (axis cs:-45.0, -80.0);
                        \path[,fill,
                              black!9]
                            (axis cs:-50.0, -80.0)
                            rectangle (axis cs:-45.0, -75.0);
                        \path[,fill,
                              black!16]
                            (axis cs:-50.0, -75.0)
                            rectangle (axis cs:-45.0, -70.0);
                        \path[,fill,
                              black!27]
                            (axis cs:-50.0, -70.0)
                            rectangle (axis cs:-45.0, -65.0);
                        \path[,fill,
                              black!28]
                            (axis cs:-50.0, -65.0)
                            rectangle (axis cs:-45.0, -60.0);
                        \path[,fill,
                              black!51]
                            (axis cs:-50.0, -60.0)
                            rectangle (axis cs:-45.0, -55.0);
                        \path[,fill,
                              black!73]
                            (axis cs:-50.0, -55.0)
                            rectangle (axis cs:-45.0, -50.0);
                        \path[,fill,
                              black!76]
                            (axis cs:-50.0, -50.0)
                            rectangle (axis cs:-45.0, -45.0);
                        \path[,fill,
                              black!47]
                            (axis cs:-50.0, -45.0)
                            rectangle (axis cs:-45.0, -40.0);
                        \path[,fill,
                              black!36]
                            (axis cs:-50.0, -40.0)
                            rectangle (axis cs:-45.0, -35.0);
                        \path[,fill,
                              black!25]
                            (axis cs:-50.0, -35.0)
                            rectangle (axis cs:-45.0, -30.0);
                        \path[,fill,
                              black!21]
                            (axis cs:-50.0, -30.0)
                            rectangle (axis cs:-45.0, -25.0);
                        \path[,fill,
                              black!16]
                            (axis cs:-50.0, -25.0)
                            rectangle (axis cs:-45.0, -20.0);
                        \path[,fill,
                              black!11]
                            (axis cs:-50.0, -20.0)
                            rectangle (axis cs:-45.0, -15.0);
                        \path[,fill,
                              black!11]
                            (axis cs:-50.0, -15.0)
                            rectangle (axis cs:-45.0, -10.0);
                        \path[,fill,
                              black!8]
                            (axis cs:-50.0, -10.0)
                            rectangle (axis cs:-45.0, -5.0);
                        \path[,fill,
                              black!7]
                            (axis cs:-50.0, -5.0)
                            rectangle (axis cs:-45.0, 0.0);
                        \path[,fill,
                              black!4]
                            (axis cs:-50.0, 0.0)
                            rectangle (axis cs:-45.0, 5.0);
                        \path[,fill,
                              black!4]
                            (axis cs:-50.0, 5.0)
                            rectangle (axis cs:-45.0, 10.0);
                        \path[,fill,
                              black!2]
                            (axis cs:-50.0, 10.0)
                            rectangle (axis cs:-45.0, 15.0);
                        \path[,fill,
                              black!1]
                            (axis cs:-50.0, 15.0)
                            rectangle (axis cs:-45.0, 20.0);
                        \path[,fill,
                              black!1]
                            (axis cs:-50.0, 20.0)
                            rectangle (axis cs:-45.0, 25.0);
                        \path[,fill,
                              black!2]
                            (axis cs:-50.0, 25.0)
                            rectangle (axis cs:-45.0, 30.0);
                        \path[,fill,
                              black!1]
                            (axis cs:-50.0, 30.0)
                            rectangle (axis cs:-45.0, 35.0);
                        \path[,fill,
                              black!2]
                            (axis cs:-50.0, 35.0)
                            rectangle (axis cs:-45.0, 40.0);
                        \path[,fill,
                              black!1]
                            (axis cs:-50.0, 55.0)
                            rectangle (axis cs:-45.0, 60.0);
                        \path[,fill,
                              black!1]
                            (axis cs:-50.0, 70.0)
                            rectangle (axis cs:-45.0, 75.0);
                        \path[,fill,
                              black!1]
                            (axis cs:-50.0, 75.0)
                            rectangle (axis cs:-45.0, 80.0);
                        \path[,fill,
                              black!1]
                            (axis cs:-50.0, 130.0)
                            rectangle (axis cs:-45.0, 135.0);
                        \path[,fill,
                              black!1]
                            (axis cs:-50.0, 140.0)
                            rectangle (axis cs:-45.0, 145.0);
                        \path[,fill,
                              black!2]
                            (axis cs:-50.0, 145.0)
                            rectangle (axis cs:-45.0, 150.0);
                        \path[,fill,
                              black!1]
                            (axis cs:-50.0, 150.0)
                            rectangle (axis cs:-45.0, 155.0);
                        \path[,fill,
                              black!4]
                            (axis cs:-50.0, 155.0)
                            rectangle (axis cs:-45.0, 160.0);
                        \path[,fill,
                              black!4]
                            (axis cs:-50.0, 160.0)
                            rectangle (axis cs:-45.0, 165.0);
                        \path[,fill,
                              black!3]
                            (axis cs:-50.0, 165.0)
                            rectangle (axis cs:-45.0, 170.0);
                        \path[,fill,
                              black!4]
                            (axis cs:-50.0, 170.0)
                            rectangle (axis cs:-45.0, 175.0);
                        \path[,fill,
                              black!1]
                            (axis cs:-45.0, -180.0)
                            rectangle (axis cs:-40.0, -175.0);
                        \path[,fill,
                              black!1]
                            (axis cs:-45.0, -175.0)
                            rectangle (axis cs:-40.0, -170.0);
                        \path[,fill,
                              black!1]
                            (axis cs:-45.0, -170.0)
                            rectangle (axis cs:-40.0, -165.0);
                        \path[,fill,
                              black!1]
                            (axis cs:-45.0, -155.0)
                            rectangle (axis cs:-40.0, -150.0);
                        \path[,fill,
                              black!1]
                            (axis cs:-45.0, -145.0)
                            rectangle (axis cs:-40.0, -140.0);
                        \path[,fill,
                              black!1]
                            (axis cs:-45.0, -135.0)
                            rectangle (axis cs:-40.0, -130.0);
                        \path[,fill,
                              black!2]
                            (axis cs:-45.0, -130.0)
                            rectangle (axis cs:-40.0, -125.0);
                        \path[,fill,
                              black!2]
                            (axis cs:-45.0, -120.0)
                            rectangle (axis cs:-40.0, -115.0);
                        \path[,fill,
                              black!1]
                            (axis cs:-45.0, -110.0)
                            rectangle (axis cs:-40.0, -105.0);
                        \path[,fill,
                              black!1]
                            (axis cs:-45.0, -105.0)
                            rectangle (axis cs:-40.0, -100.0);
                        \path[,fill,
                              black!1]
                            (axis cs:-45.0, -100.0)
                            rectangle (axis cs:-40.0, -95.0);
                        \path[,fill,
                              black!3]
                            (axis cs:-45.0, -95.0)
                            rectangle (axis cs:-40.0, -90.0);
                        \path[,fill,
                              black!1]
                            (axis cs:-45.0, -90.0)
                            rectangle (axis cs:-40.0, -85.0);
                        \path[,fill,
                              black!6]
                            (axis cs:-45.0, -85.0)
                            rectangle (axis cs:-40.0, -80.0);
                        \path[,fill,
                              black!7]
                            (axis cs:-45.0, -80.0)
                            rectangle (axis cs:-40.0, -75.0);
                        \path[,fill,
                              black!13]
                            (axis cs:-45.0, -75.0)
                            rectangle (axis cs:-40.0, -70.0);
                        \path[,fill,
                              black!20]
                            (axis cs:-45.0, -70.0)
                            rectangle (axis cs:-40.0, -65.0);
                        \path[,fill,
                              black!31]
                            (axis cs:-45.0, -65.0)
                            rectangle (axis cs:-40.0, -60.0);
                        \path[,fill,
                              black!34]
                            (axis cs:-45.0, -60.0)
                            rectangle (axis cs:-40.0, -55.0);
                        \path[,fill,
                              black!49]
                            (axis cs:-45.0, -55.0)
                            rectangle (axis cs:-40.0, -50.0);
                        \path[,fill,
                              black!64]
                            (axis cs:-45.0, -50.0)
                            rectangle (axis cs:-40.0, -45.0);
                        \path[,fill,
                              black!83]
                            (axis cs:-45.0, -45.0)
                            rectangle (axis cs:-40.0, -40.0);
                        \path[,fill,
                              black!56]
                            (axis cs:-45.0, -40.0)
                            rectangle (axis cs:-40.0, -35.0);
                        \path[,fill,
                              black!40]
                            (axis cs:-45.0, -35.0)
                            rectangle (axis cs:-40.0, -30.0);
                        \path[,fill,
                              black!22]
                            (axis cs:-45.0, -30.0)
                            rectangle (axis cs:-40.0, -25.0);
                        \path[,fill,
                              black!22]
                            (axis cs:-45.0, -25.0)
                            rectangle (axis cs:-40.0, -20.0);
                        \path[,fill,
                              black!13]
                            (axis cs:-45.0, -20.0)
                            rectangle (axis cs:-40.0, -15.0);
                        \path[,fill,
                              black!11]
                            (axis cs:-45.0, -15.0)
                            rectangle (axis cs:-40.0, -10.0);
                        \path[,fill,
                              black!9]
                            (axis cs:-45.0, -10.0)
                            rectangle (axis cs:-40.0, -5.0);
                        \path[,fill,
                              black!11]
                            (axis cs:-45.0, -5.0)
                            rectangle (axis cs:-40.0, 0.0);
                        \path[,fill,
                              black!5]
                            (axis cs:-45.0, 0.0)
                            rectangle (axis cs:-40.0, 5.0);
                        \path[,fill,
                              black!6]
                            (axis cs:-45.0, 5.0)
                            rectangle (axis cs:-40.0, 10.0);
                        \path[,fill,
                              black!3]
                            (axis cs:-45.0, 10.0)
                            rectangle (axis cs:-40.0, 15.0);
                        \path[,fill,
                              black!4]
                            (axis cs:-45.0, 20.0)
                            rectangle (axis cs:-40.0, 25.0);
                        \path[,fill,
                              black!1]
                            (axis cs:-45.0, 25.0)
                            rectangle (axis cs:-40.0, 30.0);
                        \path[,fill,
                              black!1]
                            (axis cs:-45.0, 30.0)
                            rectangle (axis cs:-40.0, 35.0);
                        \path[,fill,
                              black!1]
                            (axis cs:-45.0, 55.0)
                            rectangle (axis cs:-40.0, 60.0);
                        \path[,fill,
                              black!1]
                            (axis cs:-45.0, 60.0)
                            rectangle (axis cs:-40.0, 65.0);
                        \path[,fill,
                              black!1]
                            (axis cs:-45.0, 65.0)
                            rectangle (axis cs:-40.0, 70.0);
                        \path[,fill,
                              black!1]
                            (axis cs:-45.0, 80.0)
                            rectangle (axis cs:-40.0, 85.0);
                        \path[,fill,
                              black!1]
                            (axis cs:-45.0, 135.0)
                            rectangle (axis cs:-40.0, 140.0);
                        \path[,fill,
                              black!1]
                            (axis cs:-45.0, 140.0)
                            rectangle (axis cs:-40.0, 145.0);
                        \path[,fill,
                              black!1]
                            (axis cs:-45.0, 145.0)
                            rectangle (axis cs:-40.0, 150.0);
                        \path[,fill,
                              black!2]
                            (axis cs:-45.0, 150.0)
                            rectangle (axis cs:-40.0, 155.0);
                        \path[,fill,
                              black!2]
                            (axis cs:-45.0, 155.0)
                            rectangle (axis cs:-40.0, 160.0);
                        \path[,fill,
                              black!1]
                            (axis cs:-45.0, 160.0)
                            rectangle (axis cs:-40.0, 165.0);
                        \path[,fill,
                              black!3]
                            (axis cs:-45.0, 165.0)
                            rectangle (axis cs:-40.0, 170.0);
                        \path[,fill,
                              black!1]
                            (axis cs:-45.0, 170.0)
                            rectangle (axis cs:-40.0, 175.0);
                        \path[,fill,
                              black!1]
                            (axis cs:-45.0, 175.0)
                            rectangle (axis cs:-40.0, 180.0);
                        \path[,fill,
                              black!1]
                            (axis cs:-40.0, -180.0)
                            rectangle (axis cs:-35.0, -175.0);
                        \path[,fill,
                              black!1]
                            (axis cs:-40.0, -170.0)
                            rectangle (axis cs:-35.0, -165.0);
                        \path[,fill,
                              black!1]
                            (axis cs:-40.0, -165.0)
                            rectangle (axis cs:-35.0, -160.0);
                        \path[,fill,
                              black!1]
                            (axis cs:-40.0, -160.0)
                            rectangle (axis cs:-35.0, -155.0);
                        \path[,fill,
                              black!1]
                            (axis cs:-40.0, -150.0)
                            rectangle (axis cs:-35.0, -145.0);
                        \path[,fill,
                              black!1]
                            (axis cs:-40.0, -145.0)
                            rectangle (axis cs:-35.0, -140.0);
                        \path[,fill,
                              black!1]
                            (axis cs:-40.0, -130.0)
                            rectangle (axis cs:-35.0, -125.0);
                        \path[,fill,
                              black!2]
                            (axis cs:-40.0, -125.0)
                            rectangle (axis cs:-35.0, -120.0);
                        \path[,fill,
                              black!1]
                            (axis cs:-40.0, -120.0)
                            rectangle (axis cs:-35.0, -115.0);
                        \path[,fill,
                              black!2]
                            (axis cs:-40.0, -115.0)
                            rectangle (axis cs:-35.0, -110.0);
                        \path[,fill,
                              black!3]
                            (axis cs:-40.0, -105.0)
                            rectangle (axis cs:-35.0, -100.0);
                        \path[,fill,
                              black!1]
                            (axis cs:-40.0, -100.0)
                            rectangle (axis cs:-35.0, -95.0);
                        \path[,fill,
                              black!2]
                            (axis cs:-40.0, -95.0)
                            rectangle (axis cs:-35.0, -90.0);
                        \path[,fill,
                              black!3]
                            (axis cs:-40.0, -90.0)
                            rectangle (axis cs:-35.0, -85.0);
                        \path[,fill,
                              black!6]
                            (axis cs:-40.0, -85.0)
                            rectangle (axis cs:-35.0, -80.0);
                        \path[,fill,
                              black!6]
                            (axis cs:-40.0, -80.0)
                            rectangle (axis cs:-35.0, -75.0);
                        \path[,fill,
                              black!11]
                            (axis cs:-40.0, -75.0)
                            rectangle (axis cs:-35.0, -70.0);
                        \path[,fill,
                              black!13]
                            (axis cs:-40.0, -70.0)
                            rectangle (axis cs:-35.0, -65.0);
                        \path[,fill,
                              black!17]
                            (axis cs:-40.0, -65.0)
                            rectangle (axis cs:-35.0, -60.0);
                        \path[,fill,
                              black!23]
                            (axis cs:-40.0, -60.0)
                            rectangle (axis cs:-35.0, -55.0);
                        \path[,fill,
                              black!31]
                            (axis cs:-40.0, -55.0)
                            rectangle (axis cs:-35.0, -50.0);
                        \path[,fill,
                              black!43]
                            (axis cs:-40.0, -50.0)
                            rectangle (axis cs:-35.0, -45.0);
                        \path[,fill,
                              black!62]
                            (axis cs:-40.0, -45.0)
                            rectangle (axis cs:-35.0, -40.0);
                        \path[,fill,
                              black!67]
                            (axis cs:-40.0, -40.0)
                            rectangle (axis cs:-35.0, -35.0);
                        \path[,fill,
                              black!57]
                            (axis cs:-40.0, -35.0)
                            rectangle (axis cs:-35.0, -30.0);
                        \path[,fill,
                              black!48]
                            (axis cs:-40.0, -30.0)
                            rectangle (axis cs:-35.0, -25.0);
                        \path[,fill,
                              black!29]
                            (axis cs:-40.0, -25.0)
                            rectangle (axis cs:-35.0, -20.0);
                        \path[,fill,
                              black!19]
                            (axis cs:-40.0, -20.0)
                            rectangle (axis cs:-35.0, -15.0);
                        \path[,fill,
                              black!13]
                            (axis cs:-40.0, -15.0)
                            rectangle (axis cs:-35.0, -10.0);
                        \path[,fill,
                              black!13]
                            (axis cs:-40.0, -10.0)
                            rectangle (axis cs:-35.0, -5.0);
                        \path[,fill,
                              black!13]
                            (axis cs:-40.0, -5.0)
                            rectangle (axis cs:-35.0, 0.0);
                        \path[,fill,
                              black!8]
                            (axis cs:-40.0, 0.0)
                            rectangle (axis cs:-35.0, 5.0);
                        \path[,fill,
                              black!7]
                            (axis cs:-40.0, 5.0)
                            rectangle (axis cs:-35.0, 10.0);
                        \path[,fill,
                              black!7]
                            (axis cs:-40.0, 10.0)
                            rectangle (axis cs:-35.0, 15.0);
                        \path[,fill,
                              black!1]
                            (axis cs:-40.0, 15.0)
                            rectangle (axis cs:-35.0, 20.0);
                        \path[,fill,
                              black!1]
                            (axis cs:-40.0, 20.0)
                            rectangle (axis cs:-35.0, 25.0);
                        \path[,fill,
                              black!1]
                            (axis cs:-40.0, 25.0)
                            rectangle (axis cs:-35.0, 30.0);
                        \path[,fill,
                              black!1]
                            (axis cs:-40.0, 30.0)
                            rectangle (axis cs:-35.0, 35.0);
                        \path[,fill,
                              black!1]
                            (axis cs:-40.0, 45.0)
                            rectangle (axis cs:-35.0, 50.0);
                        \path[,fill,
                              black!1]
                            (axis cs:-40.0, 80.0)
                            rectangle (axis cs:-35.0, 85.0);
                        \path[,fill,
                              black!1]
                            (axis cs:-40.0, 95.0)
                            rectangle (axis cs:-35.0, 100.0);
                        \path[,fill,
                              black!1]
                            (axis cs:-40.0, 120.0)
                            rectangle (axis cs:-35.0, 125.0);
                        \path[,fill,
                              black!1]
                            (axis cs:-40.0, 125.0)
                            rectangle (axis cs:-35.0, 130.0);
                        \path[,fill,
                              black!1]
                            (axis cs:-40.0, 130.0)
                            rectangle (axis cs:-35.0, 135.0);
                        \path[,fill,
                              black!1]
                            (axis cs:-40.0, 135.0)
                            rectangle (axis cs:-35.0, 140.0);
                        \path[,fill,
                              black!2]
                            (axis cs:-40.0, 140.0)
                            rectangle (axis cs:-35.0, 145.0);
                        \path[,fill,
                              black!2]
                            (axis cs:-40.0, 145.0)
                            rectangle (axis cs:-35.0, 150.0);
                        \path[,fill,
                              black!2]
                            (axis cs:-40.0, 150.0)
                            rectangle (axis cs:-35.0, 155.0);
                        \path[,fill,
                              black!2]
                            (axis cs:-40.0, 155.0)
                            rectangle (axis cs:-35.0, 160.0);
                        \path[,fill,
                              black!1]
                            (axis cs:-40.0, 160.0)
                            rectangle (axis cs:-35.0, 165.0);
                        \path[,fill,
                              black!2]
                            (axis cs:-40.0, 165.0)
                            rectangle (axis cs:-35.0, 170.0);
                        \path[,fill,
                              black!1]
                            (axis cs:-40.0, 170.0)
                            rectangle (axis cs:-35.0, 175.0);
                        \path[,fill,
                              black!1]
                            (axis cs:-40.0, 175.0)
                            rectangle (axis cs:-35.0, 180.0);
                        \path[,fill,
                              black!1]
                            (axis cs:-35.0, -180.0)
                            rectangle (axis cs:-30.0, -175.0);
                        \path[,fill,
                              black!1]
                            (axis cs:-35.0, -170.0)
                            rectangle (axis cs:-30.0, -165.0);
                        \path[,fill,
                              black!1]
                            (axis cs:-35.0, -160.0)
                            rectangle (axis cs:-30.0, -155.0);
                        \path[,fill,
                              black!1]
                            (axis cs:-35.0, -155.0)
                            rectangle (axis cs:-30.0, -150.0);
                        \path[,fill,
                              black!1]
                            (axis cs:-35.0, -135.0)
                            rectangle (axis cs:-30.0, -130.0);
                        \path[,fill,
                              black!1]
                            (axis cs:-35.0, -130.0)
                            rectangle (axis cs:-30.0, -125.0);
                        \path[,fill,
                              black!1]
                            (axis cs:-35.0, -120.0)
                            rectangle (axis cs:-30.0, -115.0);
                        \path[,fill,
                              black!1]
                            (axis cs:-35.0, -115.0)
                            rectangle (axis cs:-30.0, -110.0);
                        \path[,fill,
                              black!1]
                            (axis cs:-35.0, -100.0)
                            rectangle (axis cs:-30.0, -95.0);
                        \path[,fill,
                              black!1]
                            (axis cs:-35.0, -95.0)
                            rectangle (axis cs:-30.0, -90.0);
                        \path[,fill,
                              black!1]
                            (axis cs:-35.0, -85.0)
                            rectangle (axis cs:-30.0, -80.0);
                        \path[,fill,
                              black!4]
                            (axis cs:-35.0, -80.0)
                            rectangle (axis cs:-30.0, -75.0);
                        \path[,fill,
                              black!9]
                            (axis cs:-35.0, -75.0)
                            rectangle (axis cs:-30.0, -70.0);
                        \path[,fill,
                              black!7]
                            (axis cs:-35.0, -70.0)
                            rectangle (axis cs:-30.0, -65.0);
                        \path[,fill,
                              black!16]
                            (axis cs:-35.0, -65.0)
                            rectangle (axis cs:-30.0, -60.0);
                        \path[,fill,
                              black!18]
                            (axis cs:-35.0, -60.0)
                            rectangle (axis cs:-30.0, -55.0);
                        \path[,fill,
                              black!25]
                            (axis cs:-35.0, -55.0)
                            rectangle (axis cs:-30.0, -50.0);
                        \path[,fill,
                              black!29]
                            (axis cs:-35.0, -50.0)
                            rectangle (axis cs:-30.0, -45.0);
                        \path[,fill,
                              black!48]
                            (axis cs:-35.0, -45.0)
                            rectangle (axis cs:-30.0, -40.0);
                        \path[,fill,
                              black!67]
                            (axis cs:-35.0, -40.0)
                            rectangle (axis cs:-30.0, -35.0);
                        \path[,fill,
                              black!80]
                            (axis cs:-35.0, -35.0)
                            rectangle (axis cs:-30.0, -30.0);
                        \path[,fill,
                              black!58]
                            (axis cs:-35.0, -30.0)
                            rectangle (axis cs:-30.0, -25.0);
                        \path[,fill,
                              black!44]
                            (axis cs:-35.0, -25.0)
                            rectangle (axis cs:-30.0, -20.0);
                        \path[,fill,
                              black!34]
                            (axis cs:-35.0, -20.0)
                            rectangle (axis cs:-30.0, -15.0);
                        \path[,fill,
                              black!16]
                            (axis cs:-35.0, -15.0)
                            rectangle (axis cs:-30.0, -10.0);
                        \path[,fill,
                              black!17]
                            (axis cs:-35.0, -10.0)
                            rectangle (axis cs:-30.0, -5.0);
                        \path[,fill,
                              black!13]
                            (axis cs:-35.0, -5.0)
                            rectangle (axis cs:-30.0, 0.0);
                        \path[,fill,
                              black!12]
                            (axis cs:-35.0, 0.0)
                            rectangle (axis cs:-30.0, 5.0);
                        \path[,fill,
                              black!9]
                            (axis cs:-35.0, 5.0)
                            rectangle (axis cs:-30.0, 10.0);
                        \path[,fill,
                              black!6]
                            (axis cs:-35.0, 10.0)
                            rectangle (axis cs:-30.0, 15.0);
                        \path[,fill,
                              black!4]
                            (axis cs:-35.0, 15.0)
                            rectangle (axis cs:-30.0, 20.0);
                        \path[,fill,
                              black!2]
                            (axis cs:-35.0, 20.0)
                            rectangle (axis cs:-30.0, 25.0);
                        \path[,fill,
                              black!2]
                            (axis cs:-35.0, 25.0)
                            rectangle (axis cs:-30.0, 30.0);
                        \path[,fill,
                              black!1]
                            (axis cs:-35.0, 30.0)
                            rectangle (axis cs:-30.0, 35.0);
                        \path[,fill,
                              black!2]
                            (axis cs:-35.0, 35.0)
                            rectangle (axis cs:-30.0, 40.0);
                        \path[,fill,
                              black!1]
                            (axis cs:-35.0, 40.0)
                            rectangle (axis cs:-30.0, 45.0);
                        \path[,fill,
                              black!2]
                            (axis cs:-35.0, 50.0)
                            rectangle (axis cs:-30.0, 55.0);
                        \path[,fill,
                              black!2]
                            (axis cs:-35.0, 70.0)
                            rectangle (axis cs:-30.0, 75.0);
                        \path[,fill,
                              black!1]
                            (axis cs:-35.0, 80.0)
                            rectangle (axis cs:-30.0, 85.0);
                        \path[,fill,
                              black!1]
                            (axis cs:-35.0, 90.0)
                            rectangle (axis cs:-30.0, 95.0);
                        \path[,fill,
                              black!1]
                            (axis cs:-35.0, 105.0)
                            rectangle (axis cs:-30.0, 110.0);
                        \path[,fill,
                              black!1]
                            (axis cs:-35.0, 110.0)
                            rectangle (axis cs:-30.0, 115.0);
                        \path[,fill,
                              black!1]
                            (axis cs:-35.0, 115.0)
                            rectangle (axis cs:-30.0, 120.0);
                        \path[,fill,
                              black!1]
                            (axis cs:-35.0, 120.0)
                            rectangle (axis cs:-30.0, 125.0);
                        \path[,fill,
                              black!1]
                            (axis cs:-35.0, 130.0)
                            rectangle (axis cs:-30.0, 135.0);
                        \path[,fill,
                              black!2]
                            (axis cs:-35.0, 140.0)
                            rectangle (axis cs:-30.0, 145.0);
                        \path[,fill,
                              black!1]
                            (axis cs:-35.0, 145.0)
                            rectangle (axis cs:-30.0, 150.0);
                        \path[,fill,
                              black!2]
                            (axis cs:-35.0, 150.0)
                            rectangle (axis cs:-30.0, 155.0);
                        \path[,fill,
                              black!2]
                            (axis cs:-35.0, 155.0)
                            rectangle (axis cs:-30.0, 160.0);
                        \path[,fill,
                              black!1]
                            (axis cs:-35.0, 160.0)
                            rectangle (axis cs:-30.0, 165.0);
                        \path[,fill,
                              black!2]
                            (axis cs:-35.0, 165.0)
                            rectangle (axis cs:-30.0, 170.0);
                        \path[,fill,
                              black!1]
                            (axis cs:-30.0, -180.0)
                            rectangle (axis cs:-25.0, -175.0);
                        \path[,fill,
                              black!1]
                            (axis cs:-30.0, -175.0)
                            rectangle (axis cs:-25.0, -170.0);
                        \path[,fill,
                              black!1]
                            (axis cs:-30.0, -170.0)
                            rectangle (axis cs:-25.0, -165.0);
                        \path[,fill,
                              black!1]
                            (axis cs:-30.0, -155.0)
                            rectangle (axis cs:-25.0, -150.0);
                        \path[,fill,
                              black!1]
                            (axis cs:-30.0, -145.0)
                            rectangle (axis cs:-25.0, -140.0);
                        \path[,fill,
                              black!1]
                            (axis cs:-30.0, -125.0)
                            rectangle (axis cs:-25.0, -120.0);
                        \path[,fill,
                              black!1]
                            (axis cs:-30.0, -105.0)
                            rectangle (axis cs:-25.0, -100.0);
                        \path[,fill,
                              black!1]
                            (axis cs:-30.0, -100.0)
                            rectangle (axis cs:-25.0, -95.0);
                        \path[,fill,
                              black!2]
                            (axis cs:-30.0, -95.0)
                            rectangle (axis cs:-25.0, -90.0);
                        \path[,fill,
                              black!2]
                            (axis cs:-30.0, -90.0)
                            rectangle (axis cs:-25.0, -85.0);
                        \path[,fill,
                              black!4]
                            (axis cs:-30.0, -85.0)
                            rectangle (axis cs:-25.0, -80.0);
                        \path[,fill,
                              black!1]
                            (axis cs:-30.0, -80.0)
                            rectangle (axis cs:-25.0, -75.0);
                        \path[,fill,
                              black!6]
                            (axis cs:-30.0, -75.0)
                            rectangle (axis cs:-25.0, -70.0);
                        \path[,fill,
                              black!5]
                            (axis cs:-30.0, -70.0)
                            rectangle (axis cs:-25.0, -65.0);
                        \path[,fill,
                              black!9]
                            (axis cs:-30.0, -65.0)
                            rectangle (axis cs:-25.0, -60.0);
                        \path[,fill,
                              black!18]
                            (axis cs:-30.0, -60.0)
                            rectangle (axis cs:-25.0, -55.0);
                        \path[,fill,
                              black!14]
                            (axis cs:-30.0, -55.0)
                            rectangle (axis cs:-25.0, -50.0);
                        \path[,fill,
                              black!18]
                            (axis cs:-30.0, -50.0)
                            rectangle (axis cs:-25.0, -45.0);
                        \path[,fill,
                              black!28]
                            (axis cs:-30.0, -45.0)
                            rectangle (axis cs:-25.0, -40.0);
                        \path[,fill,
                              black!39]
                            (axis cs:-30.0, -40.0)
                            rectangle (axis cs:-25.0, -35.0);
                        \path[,fill,
                              black!71]
                            (axis cs:-30.0, -35.0)
                            rectangle (axis cs:-25.0, -30.0);
                        \path[,fill,
                              black!75]
                            (axis cs:-30.0, -30.0)
                            rectangle (axis cs:-25.0, -25.0);
                        \path[,fill,
                              black!58]
                            (axis cs:-30.0, -25.0)
                            rectangle (axis cs:-25.0, -20.0);
                        \path[,fill,
                              black!38]
                            (axis cs:-30.0, -20.0)
                            rectangle (axis cs:-25.0, -15.0);
                        \path[,fill,
                              black!32]
                            (axis cs:-30.0, -15.0)
                            rectangle (axis cs:-25.0, -10.0);
                        \path[,fill,
                              black!21]
                            (axis cs:-30.0, -10.0)
                            rectangle (axis cs:-25.0, -5.0);
                        \path[,fill,
                              black!20]
                            (axis cs:-30.0, -5.0)
                            rectangle (axis cs:-25.0, 0.0);
                        \path[,fill,
                              black!16]
                            (axis cs:-30.0, 0.0)
                            rectangle (axis cs:-25.0, 5.0);
                        \path[,fill,
                              black!9]
                            (axis cs:-30.0, 5.0)
                            rectangle (axis cs:-25.0, 10.0);
                        \path[,fill,
                              black!6]
                            (axis cs:-30.0, 10.0)
                            rectangle (axis cs:-25.0, 15.0);
                        \path[,fill,
                              black!4]
                            (axis cs:-30.0, 15.0)
                            rectangle (axis cs:-25.0, 20.0);
                        \path[,fill,
                              black!2]
                            (axis cs:-30.0, 20.0)
                            rectangle (axis cs:-25.0, 25.0);
                        \path[,fill,
                              black!2]
                            (axis cs:-30.0, 25.0)
                            rectangle (axis cs:-25.0, 30.0);
                        \path[,fill,
                              black!2]
                            (axis cs:-30.0, 30.0)
                            rectangle (axis cs:-25.0, 35.0);
                        \path[,fill,
                              black!2]
                            (axis cs:-30.0, 35.0)
                            rectangle (axis cs:-25.0, 40.0);
                        \path[,fill,
                              black!1]
                            (axis cs:-30.0, 40.0)
                            rectangle (axis cs:-25.0, 45.0);
                        \path[,fill,
                              black!2]
                            (axis cs:-30.0, 45.0)
                            rectangle (axis cs:-25.0, 50.0);
                        \path[,fill,
                              black!1]
                            (axis cs:-30.0, 50.0)
                            rectangle (axis cs:-25.0, 55.0);
                        \path[,fill,
                              black!1]
                            (axis cs:-30.0, 65.0)
                            rectangle (axis cs:-25.0, 70.0);
                        \path[,fill,
                              black!1]
                            (axis cs:-30.0, 80.0)
                            rectangle (axis cs:-25.0, 85.0);
                        \path[,fill,
                              black!1]
                            (axis cs:-30.0, 105.0)
                            rectangle (axis cs:-25.0, 110.0);
                        \path[,fill,
                              black!1]
                            (axis cs:-30.0, 120.0)
                            rectangle (axis cs:-25.0, 125.0);
                        \path[,fill,
                              black!1]
                            (axis cs:-30.0, 125.0)
                            rectangle (axis cs:-25.0, 130.0);
                        \path[,fill,
                              black!2]
                            (axis cs:-30.0, 130.0)
                            rectangle (axis cs:-25.0, 135.0);
                        \path[,fill,
                              black!1]
                            (axis cs:-30.0, 135.0)
                            rectangle (axis cs:-25.0, 140.0);
                        \path[,fill,
                              black!2]
                            (axis cs:-30.0, 140.0)
                            rectangle (axis cs:-25.0, 145.0);
                        \path[,fill,
                              black!5]
                            (axis cs:-30.0, 145.0)
                            rectangle (axis cs:-25.0, 150.0);
                        \path[,fill,
                              black!3]
                            (axis cs:-30.0, 150.0)
                            rectangle (axis cs:-25.0, 155.0);
                        \path[,fill,
                              black!2]
                            (axis cs:-30.0, 155.0)
                            rectangle (axis cs:-25.0, 160.0);
                        \path[,fill,
                              black!2]
                            (axis cs:-30.0, 160.0)
                            rectangle (axis cs:-25.0, 165.0);
                        \path[,fill,
                              black!3]
                            (axis cs:-30.0, 170.0)
                            rectangle (axis cs:-25.0, 175.0);
                        \path[,fill,
                              black!1]
                            (axis cs:-25.0, -115.0)
                            rectangle (axis cs:-20.0, -110.0);
                        \path[,fill,
                              black!1]
                            (axis cs:-25.0, -95.0)
                            rectangle (axis cs:-20.0, -90.0);
                        \path[,fill,
                              black!2]
                            (axis cs:-25.0, -90.0)
                            rectangle (axis cs:-20.0, -85.0);
                        \path[,fill,
                              black!1]
                            (axis cs:-25.0, -85.0)
                            rectangle (axis cs:-20.0, -80.0);
                        \path[,fill,
                              black!4]
                            (axis cs:-25.0, -80.0)
                            rectangle (axis cs:-20.0, -75.0);
                        \path[,fill,
                              black!4]
                            (axis cs:-25.0, -75.0)
                            rectangle (axis cs:-20.0, -70.0);
                        \path[,fill,
                              black!7]
                            (axis cs:-25.0, -70.0)
                            rectangle (axis cs:-20.0, -65.0);
                        \path[,fill,
                              black!12]
                            (axis cs:-25.0, -65.0)
                            rectangle (axis cs:-20.0, -60.0);
                        \path[,fill,
                              black!11]
                            (axis cs:-25.0, -60.0)
                            rectangle (axis cs:-20.0, -55.0);
                        \path[,fill,
                              black!13]
                            (axis cs:-25.0, -55.0)
                            rectangle (axis cs:-20.0, -50.0);
                        \path[,fill,
                              black!14]
                            (axis cs:-25.0, -50.0)
                            rectangle (axis cs:-20.0, -45.0);
                        \path[,fill,
                              black!26]
                            (axis cs:-25.0, -45.0)
                            rectangle (axis cs:-20.0, -40.0);
                        \path[,fill,
                              black!29]
                            (axis cs:-25.0, -40.0)
                            rectangle (axis cs:-20.0, -35.0);
                        \path[,fill,
                              black!46]
                            (axis cs:-25.0, -35.0)
                            rectangle (axis cs:-20.0, -30.0);
                        \path[,fill,
                              black!70]
                            (axis cs:-25.0, -30.0)
                            rectangle (axis cs:-20.0, -25.0);
                        \path[,fill,
                              black!78]
                            (axis cs:-25.0, -25.0)
                            rectangle (axis cs:-20.0, -20.0);
                        \path[,fill,
                              black!63]
                            (axis cs:-25.0, -20.0)
                            rectangle (axis cs:-20.0, -15.0);
                        \path[,fill,
                              black!45]
                            (axis cs:-25.0, -15.0)
                            rectangle (axis cs:-20.0, -10.0);
                        \path[,fill,
                              black!31]
                            (axis cs:-25.0, -10.0)
                            rectangle (axis cs:-20.0, -5.0);
                        \path[,fill,
                              black!26]
                            (axis cs:-25.0, -5.0)
                            rectangle (axis cs:-20.0, 0.0);
                        \path[,fill,
                              black!18]
                            (axis cs:-25.0, 0.0)
                            rectangle (axis cs:-20.0, 5.0);
                        \path[,fill,
                              black!13]
                            (axis cs:-25.0, 5.0)
                            rectangle (axis cs:-20.0, 10.0);
                        \path[,fill,
                              black!9]
                            (axis cs:-25.0, 10.0)
                            rectangle (axis cs:-20.0, 15.0);
                        \path[,fill,
                              black!6]
                            (axis cs:-25.0, 15.0)
                            rectangle (axis cs:-20.0, 20.0);
                        \path[,fill,
                              black!2]
                            (axis cs:-25.0, 20.0)
                            rectangle (axis cs:-20.0, 25.0);
                        \path[,fill,
                              black!3]
                            (axis cs:-25.0, 25.0)
                            rectangle (axis cs:-20.0, 30.0);
                        \path[,fill,
                              black!2]
                            (axis cs:-25.0, 30.0)
                            rectangle (axis cs:-20.0, 35.0);
                        \path[,fill,
                              black!3]
                            (axis cs:-25.0, 35.0)
                            rectangle (axis cs:-20.0, 40.0);
                        \path[,fill,
                              black!2]
                            (axis cs:-25.0, 40.0)
                            rectangle (axis cs:-20.0, 45.0);
                        \path[,fill,
                              black!4]
                            (axis cs:-25.0, 45.0)
                            rectangle (axis cs:-20.0, 50.0);
                        \path[,fill,
                              black!1]
                            (axis cs:-25.0, 50.0)
                            rectangle (axis cs:-20.0, 55.0);
                        \path[,fill,
                              black!1]
                            (axis cs:-25.0, 55.0)
                            rectangle (axis cs:-20.0, 60.0);
                        \path[,fill,
                              black!2]
                            (axis cs:-25.0, 60.0)
                            rectangle (axis cs:-20.0, 65.0);
                        \path[,fill,
                              black!1]
                            (axis cs:-25.0, 70.0)
                            rectangle (axis cs:-20.0, 75.0);
                        \path[,fill,
                              black!1]
                            (axis cs:-25.0, 80.0)
                            rectangle (axis cs:-20.0, 85.0);
                        \path[,fill,
                              black!1]
                            (axis cs:-25.0, 85.0)
                            rectangle (axis cs:-20.0, 90.0);
                        \path[,fill,
                              black!1]
                            (axis cs:-25.0, 90.0)
                            rectangle (axis cs:-20.0, 95.0);
                        \path[,fill,
                              black!1]
                            (axis cs:-25.0, 100.0)
                            rectangle (axis cs:-20.0, 105.0);
                        \path[,fill,
                              black!1]
                            (axis cs:-25.0, 105.0)
                            rectangle (axis cs:-20.0, 110.0);
                        \path[,fill,
                              black!1]
                            (axis cs:-25.0, 115.0)
                            rectangle (axis cs:-20.0, 120.0);
                        \path[,fill,
                              black!1]
                            (axis cs:-25.0, 120.0)
                            rectangle (axis cs:-20.0, 125.0);
                        \path[,fill,
                              black!1]
                            (axis cs:-25.0, 130.0)
                            rectangle (axis cs:-20.0, 135.0);
                        \path[,fill,
                              black!1]
                            (axis cs:-25.0, 135.0)
                            rectangle (axis cs:-20.0, 140.0);
                        \path[,fill,
                              black!2]
                            (axis cs:-25.0, 140.0)
                            rectangle (axis cs:-20.0, 145.0);
                        \path[,fill,
                              black!2]
                            (axis cs:-25.0, 145.0)
                            rectangle (axis cs:-20.0, 150.0);
                        \path[,fill,
                              black!2]
                            (axis cs:-25.0, 150.0)
                            rectangle (axis cs:-20.0, 155.0);
                        \path[,fill,
                              black!1]
                            (axis cs:-25.0, 160.0)
                            rectangle (axis cs:-20.0, 165.0);
                        \path[,fill,
                              black!2]
                            (axis cs:-25.0, 165.0)
                            rectangle (axis cs:-20.0, 170.0);
                        \path[,fill,
                              black!1]
                            (axis cs:-20.0, -105.0)
                            rectangle (axis cs:-15.0, -100.0);
                        \path[,fill,
                              black!2]
                            (axis cs:-20.0, -95.0)
                            rectangle (axis cs:-15.0, -90.0);
                        \path[,fill,
                              black!1]
                            (axis cs:-20.0, -90.0)
                            rectangle (axis cs:-15.0, -85.0);
                        \path[,fill,
                              black!1]
                            (axis cs:-20.0, -85.0)
                            rectangle (axis cs:-15.0, -80.0);
                        \path[,fill,
                              black!2]
                            (axis cs:-20.0, -80.0)
                            rectangle (axis cs:-15.0, -75.0);
                        \path[,fill,
                              black!5]
                            (axis cs:-20.0, -75.0)
                            rectangle (axis cs:-15.0, -70.0);
                        \path[,fill,
                              black!5]
                            (axis cs:-20.0, -70.0)
                            rectangle (axis cs:-15.0, -65.0);
                        \path[,fill,
                              black!7]
                            (axis cs:-20.0, -65.0)
                            rectangle (axis cs:-15.0, -60.0);
                        \path[,fill,
                              black!13]
                            (axis cs:-20.0, -60.0)
                            rectangle (axis cs:-15.0, -55.0);
                        \path[,fill,
                              black!10]
                            (axis cs:-20.0, -55.0)
                            rectangle (axis cs:-15.0, -50.0);
                        \path[,fill,
                              black!15]
                            (axis cs:-20.0, -50.0)
                            rectangle (axis cs:-15.0, -45.0);
                        \path[,fill,
                              black!17]
                            (axis cs:-20.0, -45.0)
                            rectangle (axis cs:-15.0, -40.0);
                        \path[,fill,
                              black!19]
                            (axis cs:-20.0, -40.0)
                            rectangle (axis cs:-15.0, -35.0);
                        \path[,fill,
                              black!22]
                            (axis cs:-20.0, -35.0)
                            rectangle (axis cs:-15.0, -30.0);
                        \path[,fill,
                              black!37]
                            (axis cs:-20.0, -30.0)
                            rectangle (axis cs:-15.0, -25.0);
                        \path[,fill,
                              black!59]
                            (axis cs:-20.0, -25.0)
                            rectangle (axis cs:-15.0, -20.0);
                        \path[,fill,
                              black!81]
                            (axis cs:-20.0, -20.0)
                            rectangle (axis cs:-15.0, -15.0);
                        \path[,fill,
                              black!62]
                            (axis cs:-20.0, -15.0)
                            rectangle (axis cs:-15.0, -10.0);
                        \path[,fill,
                              black!44]
                            (axis cs:-20.0, -10.0)
                            rectangle (axis cs:-15.0, -5.0);
                        \path[,fill,
                              black!32]
                            (axis cs:-20.0, -5.0)
                            rectangle (axis cs:-15.0, 0.0);
                        \path[,fill,
                              black!21]
                            (axis cs:-20.0, 0.0)
                            rectangle (axis cs:-15.0, 5.0);
                        \path[,fill,
                              black!19]
                            (axis cs:-20.0, 5.0)
                            rectangle (axis cs:-15.0, 10.0);
                        \path[,fill,
                              black!13]
                            (axis cs:-20.0, 10.0)
                            rectangle (axis cs:-15.0, 15.0);
                        \path[,fill,
                              black!10]
                            (axis cs:-20.0, 15.0)
                            rectangle (axis cs:-15.0, 20.0);
                        \path[,fill,
                              black!3]
                            (axis cs:-20.0, 20.0)
                            rectangle (axis cs:-15.0, 25.0);
                        \path[,fill,
                              black!4]
                            (axis cs:-20.0, 25.0)
                            rectangle (axis cs:-15.0, 30.0);
                        \path[,fill,
                              black!3]
                            (axis cs:-20.0, 30.0)
                            rectangle (axis cs:-15.0, 35.0);
                        \path[,fill,
                              black!3]
                            (axis cs:-20.0, 35.0)
                            rectangle (axis cs:-15.0, 40.0);
                        \path[,fill,
                              black!1]
                            (axis cs:-20.0, 40.0)
                            rectangle (axis cs:-15.0, 45.0);
                        \path[,fill,
                              black!2]
                            (axis cs:-20.0, 45.0)
                            rectangle (axis cs:-15.0, 50.0);
                        \path[,fill,
                              black!1]
                            (axis cs:-20.0, 50.0)
                            rectangle (axis cs:-15.0, 55.0);
                        \path[,fill,
                              black!1]
                            (axis cs:-20.0, 55.0)
                            rectangle (axis cs:-15.0, 60.0);
                        \path[,fill,
                              black!1]
                            (axis cs:-20.0, 60.0)
                            rectangle (axis cs:-15.0, 65.0);
                        \path[,fill,
                              black!2]
                            (axis cs:-20.0, 70.0)
                            rectangle (axis cs:-15.0, 75.0);
                        \path[,fill,
                              black!1]
                            (axis cs:-20.0, 75.0)
                            rectangle (axis cs:-15.0, 80.0);
                        \path[,fill,
                              black!1]
                            (axis cs:-20.0, 80.0)
                            rectangle (axis cs:-15.0, 85.0);
                        \path[,fill,
                              black!1]
                            (axis cs:-20.0, 85.0)
                            rectangle (axis cs:-15.0, 90.0);
                        \path[,fill,
                              black!1]
                            (axis cs:-20.0, 90.0)
                            rectangle (axis cs:-15.0, 95.0);
                        \path[,fill,
                              black!1]
                            (axis cs:-20.0, 95.0)
                            rectangle (axis cs:-15.0, 100.0);
                        \path[,fill,
                              black!1]
                            (axis cs:-20.0, 110.0)
                            rectangle (axis cs:-15.0, 115.0);
                        \path[,fill,
                              black!1]
                            (axis cs:-20.0, 115.0)
                            rectangle (axis cs:-15.0, 120.0);
                        \path[,fill,
                              black!1]
                            (axis cs:-20.0, 120.0)
                            rectangle (axis cs:-15.0, 125.0);
                        \path[,fill,
                              black!2]
                            (axis cs:-20.0, 125.0)
                            rectangle (axis cs:-15.0, 130.0);
                        \path[,fill,
                              black!2]
                            (axis cs:-20.0, 130.0)
                            rectangle (axis cs:-15.0, 135.0);
                        \path[,fill,
                              black!2]
                            (axis cs:-20.0, 135.0)
                            rectangle (axis cs:-15.0, 140.0);
                        \path[,fill,
                              black!3]
                            (axis cs:-20.0, 140.0)
                            rectangle (axis cs:-15.0, 145.0);
                        \path[,fill,
                              black!2]
                            (axis cs:-20.0, 145.0)
                            rectangle (axis cs:-15.0, 150.0);
                        \path[,fill,
                              black!1]
                            (axis cs:-20.0, 155.0)
                            rectangle (axis cs:-15.0, 160.0);
                        \path[,fill,
                              black!1]
                            (axis cs:-20.0, 160.0)
                            rectangle (axis cs:-15.0, 165.0);
                        \path[,fill,
                              black!1]
                            (axis cs:-15.0, -155.0)
                            rectangle (axis cs:-10.0, -150.0);
                        \path[,fill,
                              black!1]
                            (axis cs:-15.0, -150.0)
                            rectangle (axis cs:-10.0, -145.0);
                        \path[,fill,
                              black!1]
                            (axis cs:-15.0, -130.0)
                            rectangle (axis cs:-10.0, -125.0);
                        \path[,fill,
                              black!1]
                            (axis cs:-15.0, -115.0)
                            rectangle (axis cs:-10.0, -110.0);
                        \path[,fill,
                              black!1]
                            (axis cs:-15.0, -110.0)
                            rectangle (axis cs:-10.0, -105.0);
                        \path[,fill,
                              black!1]
                            (axis cs:-15.0, -105.0)
                            rectangle (axis cs:-10.0, -100.0);
                        \path[,fill,
                              black!2]
                            (axis cs:-15.0, -85.0)
                            rectangle (axis cs:-10.0, -80.0);
                        \path[,fill,
                              black!2]
                            (axis cs:-15.0, -80.0)
                            rectangle (axis cs:-10.0, -75.0);
                        \path[,fill,
                              black!2]
                            (axis cs:-15.0, -75.0)
                            rectangle (axis cs:-10.0, -70.0);
                        \path[,fill,
                              black!3]
                            (axis cs:-15.0, -70.0)
                            rectangle (axis cs:-10.0, -65.0);
                        \path[,fill,
                              black!8]
                            (axis cs:-15.0, -65.0)
                            rectangle (axis cs:-10.0, -60.0);
                        \path[,fill,
                              black!8]
                            (axis cs:-15.0, -60.0)
                            rectangle (axis cs:-10.0, -55.0);
                        \path[,fill,
                              black!10]
                            (axis cs:-15.0, -55.0)
                            rectangle (axis cs:-10.0, -50.0);
                        \path[,fill,
                              black!9]
                            (axis cs:-15.0, -50.0)
                            rectangle (axis cs:-10.0, -45.0);
                        \path[,fill,
                              black!12]
                            (axis cs:-15.0, -45.0)
                            rectangle (axis cs:-10.0, -40.0);
                        \path[,fill,
                              black!11]
                            (axis cs:-15.0, -40.0)
                            rectangle (axis cs:-10.0, -35.0);
                        \path[,fill,
                              black!20]
                            (axis cs:-15.0, -35.0)
                            rectangle (axis cs:-10.0, -30.0);
                        \path[,fill,
                              black!23]
                            (axis cs:-15.0, -30.0)
                            rectangle (axis cs:-10.0, -25.0);
                        \path[,fill,
                              black!38]
                            (axis cs:-15.0, -25.0)
                            rectangle (axis cs:-10.0, -20.0);
                        \path[,fill,
                              black!66]
                            (axis cs:-15.0, -20.0)
                            rectangle (axis cs:-10.0, -15.0);
                        \path[,fill,
                              black!79]
                            (axis cs:-15.0, -15.0)
                            rectangle (axis cs:-10.0, -10.0);
                        \path[,fill,
                              black!80]
                            (axis cs:-15.0, -10.0)
                            rectangle (axis cs:-10.0, -5.0);
                        \path[,fill,
                              black!47]
                            (axis cs:-15.0, -5.0)
                            rectangle (axis cs:-10.0, 0.0);
                        \path[,fill,
                              black!36]
                            (axis cs:-15.0, 0.0)
                            rectangle (axis cs:-10.0, 5.0);
                        \path[,fill,
                              black!26]
                            (axis cs:-15.0, 5.0)
                            rectangle (axis cs:-10.0, 10.0);
                        \path[,fill,
                              black!16]
                            (axis cs:-15.0, 10.0)
                            rectangle (axis cs:-10.0, 15.0);
                        \path[,fill,
                              black!11]
                            (axis cs:-15.0, 15.0)
                            rectangle (axis cs:-10.0, 20.0);
                        \path[,fill,
                              black!5]
                            (axis cs:-15.0, 20.0)
                            rectangle (axis cs:-10.0, 25.0);
                        \path[,fill,
                              black!4]
                            (axis cs:-15.0, 25.0)
                            rectangle (axis cs:-10.0, 30.0);
                        \path[,fill,
                              black!4]
                            (axis cs:-15.0, 30.0)
                            rectangle (axis cs:-10.0, 35.0);
                        \path[,fill,
                              black!3]
                            (axis cs:-15.0, 35.0)
                            rectangle (axis cs:-10.0, 40.0);
                        \path[,fill,
                              black!3]
                            (axis cs:-15.0, 40.0)
                            rectangle (axis cs:-10.0, 45.0);
                        \path[,fill,
                              black!4]
                            (axis cs:-15.0, 45.0)
                            rectangle (axis cs:-10.0, 50.0);
                        \path[,fill,
                              black!4]
                            (axis cs:-15.0, 50.0)
                            rectangle (axis cs:-10.0, 55.0);
                        \path[,fill,
                              black!3]
                            (axis cs:-15.0, 55.0)
                            rectangle (axis cs:-10.0, 60.0);
                        \path[,fill,
                              black!2]
                            (axis cs:-15.0, 60.0)
                            rectangle (axis cs:-10.0, 65.0);
                        \path[,fill,
                              black!1]
                            (axis cs:-15.0, 65.0)
                            rectangle (axis cs:-10.0, 70.0);
                        \path[,fill,
                              black!1]
                            (axis cs:-15.0, 70.0)
                            rectangle (axis cs:-10.0, 75.0);
                        \path[,fill,
                              black!2]
                            (axis cs:-15.0, 80.0)
                            rectangle (axis cs:-10.0, 85.0);
                        \path[,fill,
                              black!2]
                            (axis cs:-15.0, 85.0)
                            rectangle (axis cs:-10.0, 90.0);
                        \path[,fill,
                              black!1]
                            (axis cs:-15.0, 90.0)
                            rectangle (axis cs:-10.0, 95.0);
                        \path[,fill,
                              black!1]
                            (axis cs:-15.0, 95.0)
                            rectangle (axis cs:-10.0, 100.0);
                        \path[,fill,
                              black!3]
                            (axis cs:-15.0, 100.0)
                            rectangle (axis cs:-10.0, 105.0);
                        \path[,fill,
                              black!1]
                            (axis cs:-15.0, 105.0)
                            rectangle (axis cs:-10.0, 110.0);
                        \path[,fill,
                              black!1]
                            (axis cs:-15.0, 110.0)
                            rectangle (axis cs:-10.0, 115.0);
                        \path[,fill,
                              black!2]
                            (axis cs:-15.0, 115.0)
                            rectangle (axis cs:-10.0, 120.0);
                        \path[,fill,
                              black!1]
                            (axis cs:-15.0, 120.0)
                            rectangle (axis cs:-10.0, 125.0);
                        \path[,fill,
                              black!1]
                            (axis cs:-15.0, 125.0)
                            rectangle (axis cs:-10.0, 130.0);
                        \path[,fill,
                              black!4]
                            (axis cs:-15.0, 130.0)
                            rectangle (axis cs:-10.0, 135.0);
                        \path[,fill,
                              black!3]
                            (axis cs:-15.0, 135.0)
                            rectangle (axis cs:-10.0, 140.0);
                        \path[,fill,
                              black!2]
                            (axis cs:-15.0, 140.0)
                            rectangle (axis cs:-10.0, 145.0);
                        \path[,fill,
                              black!1]
                            (axis cs:-15.0, 145.0)
                            rectangle (axis cs:-10.0, 150.0);
                        \path[,fill,
                              black!1]
                            (axis cs:-15.0, 150.0)
                            rectangle (axis cs:-10.0, 155.0);
                        \path[,fill,
                              black!1]
                            (axis cs:-15.0, 160.0)
                            rectangle (axis cs:-10.0, 165.0);
                        \path[,fill,
                              black!1]
                            (axis cs:-10.0, -100.0)
                            rectangle (axis cs:-5.0, -95.0);
                        \path[,fill,
                              black!1]
                            (axis cs:-10.0, -95.0)
                            rectangle (axis cs:-5.0, -90.0);
                        \path[,fill,
                              black!1]
                            (axis cs:-10.0, -90.0)
                            rectangle (axis cs:-5.0, -85.0);
                        \path[,fill,
                              black!2]
                            (axis cs:-10.0, -85.0)
                            rectangle (axis cs:-5.0, -80.0);
                        \path[,fill,
                              black!1]
                            (axis cs:-10.0, -80.0)
                            rectangle (axis cs:-5.0, -75.0);
                        \path[,fill,
                              black!2]
                            (axis cs:-10.0, -75.0)
                            rectangle (axis cs:-5.0, -70.0);
                        \path[,fill,
                              black!5]
                            (axis cs:-10.0, -70.0)
                            rectangle (axis cs:-5.0, -65.0);
                        \path[,fill,
                              black!9]
                            (axis cs:-10.0, -65.0)
                            rectangle (axis cs:-5.0, -60.0);
                        \path[,fill,
                              black!5]
                            (axis cs:-10.0, -60.0)
                            rectangle (axis cs:-5.0, -55.0);
                        \path[,fill,
                              black!9]
                            (axis cs:-10.0, -55.0)
                            rectangle (axis cs:-5.0, -50.0);
                        \path[,fill,
                              black!10]
                            (axis cs:-10.0, -50.0)
                            rectangle (axis cs:-5.0, -45.0);
                        \path[,fill,
                              black!7]
                            (axis cs:-10.0, -45.0)
                            rectangle (axis cs:-5.0, -40.0);
                        \path[,fill,
                              black!10]
                            (axis cs:-10.0, -40.0)
                            rectangle (axis cs:-5.0, -35.0);
                        \path[,fill,
                              black!15]
                            (axis cs:-10.0, -35.0)
                            rectangle (axis cs:-5.0, -30.0);
                        \path[,fill,
                              black!17]
                            (axis cs:-10.0, -30.0)
                            rectangle (axis cs:-5.0, -25.0);
                        \path[,fill,
                              black!25]
                            (axis cs:-10.0, -25.0)
                            rectangle (axis cs:-5.0, -20.0);
                        \path[,fill,
                              black!29]
                            (axis cs:-10.0, -20.0)
                            rectangle (axis cs:-5.0, -15.0);
                        \path[,fill,
                              black!56]
                            (axis cs:-10.0, -15.0)
                            rectangle (axis cs:-5.0, -10.0);
                        \path[,fill,
                              black!77]
                            (axis cs:-10.0, -10.0)
                            rectangle (axis cs:-5.0, -5.0);
                        \path[,fill,
                              black!63]
                            (axis cs:-10.0, -5.0)
                            rectangle (axis cs:-5.0, 0.0);
                        \path[,fill,
                              black!48]
                            (axis cs:-10.0, 0.0)
                            rectangle (axis cs:-5.0, 5.0);
                        \path[,fill,
                              black!31]
                            (axis cs:-10.0, 5.0)
                            rectangle (axis cs:-5.0, 10.0);
                        \path[,fill,
                              black!25]
                            (axis cs:-10.0, 10.0)
                            rectangle (axis cs:-5.0, 15.0);
                        \path[,fill,
                              black!15]
                            (axis cs:-10.0, 15.0)
                            rectangle (axis cs:-5.0, 20.0);
                        \path[,fill,
                              black!11]
                            (axis cs:-10.0, 20.0)
                            rectangle (axis cs:-5.0, 25.0);
                        \path[,fill,
                              black!8]
                            (axis cs:-10.0, 25.0)
                            rectangle (axis cs:-5.0, 30.0);
                        \path[,fill,
                              black!6]
                            (axis cs:-10.0, 30.0)
                            rectangle (axis cs:-5.0, 35.0);
                        \path[,fill,
                              black!7]
                            (axis cs:-10.0, 35.0)
                            rectangle (axis cs:-5.0, 40.0);
                        \path[,fill,
                              black!1]
                            (axis cs:-10.0, 40.0)
                            rectangle (axis cs:-5.0, 45.0);
                        \path[,fill,
                              black!1]
                            (axis cs:-10.0, 50.0)
                            rectangle (axis cs:-5.0, 55.0);
                        \path[,fill,
                              black!2]
                            (axis cs:-10.0, 55.0)
                            rectangle (axis cs:-5.0, 60.0);
                        \path[,fill,
                              black!1]
                            (axis cs:-10.0, 60.0)
                            rectangle (axis cs:-5.0, 65.0);
                        \path[,fill,
                              black!2]
                            (axis cs:-10.0, 65.0)
                            rectangle (axis cs:-5.0, 70.0);
                        \path[,fill,
                              black!2]
                            (axis cs:-10.0, 70.0)
                            rectangle (axis cs:-5.0, 75.0);
                        \path[,fill,
                              black!2]
                            (axis cs:-10.0, 75.0)
                            rectangle (axis cs:-5.0, 80.0);
                        \path[,fill,
                              black!3]
                            (axis cs:-10.0, 80.0)
                            rectangle (axis cs:-5.0, 85.0);
                        \path[,fill,
                              black!1]
                            (axis cs:-10.0, 85.0)
                            rectangle (axis cs:-5.0, 90.0);
                        \path[,fill,
                              black!2]
                            (axis cs:-10.0, 90.0)
                            rectangle (axis cs:-5.0, 95.0);
                        \path[,fill,
                              black!1]
                            (axis cs:-10.0, 95.0)
                            rectangle (axis cs:-5.0, 100.0);
                        \path[,fill,
                              black!1]
                            (axis cs:-10.0, 100.0)
                            rectangle (axis cs:-5.0, 105.0);
                        \path[,fill,
                              black!1]
                            (axis cs:-10.0, 105.0)
                            rectangle (axis cs:-5.0, 110.0);
                        \path[,fill,
                              black!3]
                            (axis cs:-10.0, 110.0)
                            rectangle (axis cs:-5.0, 115.0);
                        \path[,fill,
                              black!2]
                            (axis cs:-10.0, 115.0)
                            rectangle (axis cs:-5.0, 120.0);
                        \path[,fill,
                              black!1]
                            (axis cs:-10.0, 120.0)
                            rectangle (axis cs:-5.0, 125.0);
                        \path[,fill,
                              black!3]
                            (axis cs:-10.0, 125.0)
                            rectangle (axis cs:-5.0, 130.0);
                        \path[,fill,
                              black!2]
                            (axis cs:-10.0, 130.0)
                            rectangle (axis cs:-5.0, 135.0);
                        \path[,fill,
                              black!3]
                            (axis cs:-10.0, 135.0)
                            rectangle (axis cs:-5.0, 140.0);
                        \path[,fill,
                              black!4]
                            (axis cs:-10.0, 140.0)
                            rectangle (axis cs:-5.0, 145.0);
                        \path[,fill,
                              black!1]
                            (axis cs:-10.0, 145.0)
                            rectangle (axis cs:-5.0, 150.0);
                        \path[,fill,
                              black!1]
                            (axis cs:-10.0, 150.0)
                            rectangle (axis cs:-5.0, 155.0);
                        \path[,fill,
                              black!1]
                            (axis cs:-10.0, 155.0)
                            rectangle (axis cs:-5.0, 160.0);
                        \path[,fill,
                              black!1]
                            (axis cs:-10.0, 165.0)
                            rectangle (axis cs:-5.0, 170.0);
                        \path[,fill,
                              black!1]
                            (axis cs:-5.0, -175.0)
                            rectangle (axis cs:0.0, -170.0);
                        \path[,fill,
                              black!1]
                            (axis cs:-5.0, -95.0)
                            rectangle (axis cs:0.0, -90.0);
                        \path[,fill,
                              black!1]
                            (axis cs:-5.0, -80.0)
                            rectangle (axis cs:0.0, -75.0);
                        \path[,fill,
                              black!4]
                            (axis cs:-5.0, -75.0)
                            rectangle (axis cs:0.0, -70.0);
                        \path[,fill,
                              black!4]
                            (axis cs:-5.0, -70.0)
                            rectangle (axis cs:0.0, -65.0);
                        \path[,fill,
                              black!5]
                            (axis cs:-5.0, -65.0)
                            rectangle (axis cs:0.0, -60.0);
                        \path[,fill,
                              black!10]
                            (axis cs:-5.0, -60.0)
                            rectangle (axis cs:0.0, -55.0);
                        \path[,fill,
                              black!6]
                            (axis cs:-5.0, -55.0)
                            rectangle (axis cs:0.0, -50.0);
                        \path[,fill,
                              black!10]
                            (axis cs:-5.0, -50.0)
                            rectangle (axis cs:0.0, -45.0);
                        \path[,fill,
                              black!10]
                            (axis cs:-5.0, -45.0)
                            rectangle (axis cs:0.0, -40.0);
                        \path[,fill,
                              black!10]
                            (axis cs:-5.0, -40.0)
                            rectangle (axis cs:0.0, -35.0);
                        \path[,fill,
                              black!13]
                            (axis cs:-5.0, -35.0)
                            rectangle (axis cs:0.0, -30.0);
                        \path[,fill,
                              black!10]
                            (axis cs:-5.0, -30.0)
                            rectangle (axis cs:0.0, -25.0);
                        \path[,fill,
                              black!22]
                            (axis cs:-5.0, -25.0)
                            rectangle (axis cs:0.0, -20.0);
                        \path[,fill,
                              black!29]
                            (axis cs:-5.0, -20.0)
                            rectangle (axis cs:0.0, -15.0);
                        \path[,fill,
                              black!27]
                            (axis cs:-5.0, -15.0)
                            rectangle (axis cs:0.0, -10.0);
                        \path[,fill,
                              black!57]
                            (axis cs:-5.0, -10.0)
                            rectangle (axis cs:0.0, -5.0);
                        \path[,fill,
                              black!78]
                            (axis cs:-5.0, -5.0)
                            rectangle (axis cs:0.0, 0.0);
                        \path[,fill,
                              black!64]
                            (axis cs:-5.0, 0.0)
                            rectangle (axis cs:0.0, 5.0);
                        \path[,fill,
                              black!39]
                            (axis cs:-5.0, 5.0)
                            rectangle (axis cs:0.0, 10.0);
                        \path[,fill,
                              black!35]
                            (axis cs:-5.0, 10.0)
                            rectangle (axis cs:0.0, 15.0);
                        \path[,fill,
                              black!22]
                            (axis cs:-5.0, 15.0)
                            rectangle (axis cs:0.0, 20.0);
                        \path[,fill,
                              black!15]
                            (axis cs:-5.0, 20.0)
                            rectangle (axis cs:0.0, 25.0);
                        \path[,fill,
                              black!11]
                            (axis cs:-5.0, 25.0)
                            rectangle (axis cs:0.0, 30.0);
                        \path[,fill,
                              black!8]
                            (axis cs:-5.0, 30.0)
                            rectangle (axis cs:0.0, 35.0);
                        \path[,fill,
                              black!6]
                            (axis cs:-5.0, 35.0)
                            rectangle (axis cs:0.0, 40.0);
                        \path[,fill,
                              black!6]
                            (axis cs:-5.0, 40.0)
                            rectangle (axis cs:0.0, 45.0);
                        \path[,fill,
                              black!5]
                            (axis cs:-5.0, 45.0)
                            rectangle (axis cs:0.0, 50.0);
                        \path[,fill,
                              black!4]
                            (axis cs:-5.0, 50.0)
                            rectangle (axis cs:0.0, 55.0);
                        \path[,fill,
                              black!4]
                            (axis cs:-5.0, 55.0)
                            rectangle (axis cs:0.0, 60.0);
                        \path[,fill,
                              black!1]
                            (axis cs:-5.0, 60.0)
                            rectangle (axis cs:0.0, 65.0);
                        \path[,fill,
                              black!2]
                            (axis cs:-5.0, 65.0)
                            rectangle (axis cs:0.0, 70.0);
                        \path[,fill,
                              black!1]
                            (axis cs:-5.0, 70.0)
                            rectangle (axis cs:0.0, 75.0);
                        \path[,fill,
                              black!2]
                            (axis cs:-5.0, 75.0)
                            rectangle (axis cs:0.0, 80.0);
                        \path[,fill,
                              black!1]
                            (axis cs:-5.0, 80.0)
                            rectangle (axis cs:0.0, 85.0);
                        \path[,fill,
                              black!1]
                            (axis cs:-5.0, 85.0)
                            rectangle (axis cs:0.0, 90.0);
                        \path[,fill,
                              black!3]
                            (axis cs:-5.0, 90.0)
                            rectangle (axis cs:0.0, 95.0);
                        \path[,fill,
                              black!1]
                            (axis cs:-5.0, 95.0)
                            rectangle (axis cs:0.0, 100.0);
                        \path[,fill,
                              black!2]
                            (axis cs:-5.0, 100.0)
                            rectangle (axis cs:0.0, 105.0);
                        \path[,fill,
                              black!1]
                            (axis cs:-5.0, 110.0)
                            rectangle (axis cs:0.0, 115.0);
                        \path[,fill,
                              black!3]
                            (axis cs:-5.0, 115.0)
                            rectangle (axis cs:0.0, 120.0);
                        \path[,fill,
                              black!1]
                            (axis cs:-5.0, 120.0)
                            rectangle (axis cs:0.0, 125.0);
                        \path[,fill,
                              black!3]
                            (axis cs:-5.0, 125.0)
                            rectangle (axis cs:0.0, 130.0);
                        \path[,fill,
                              black!1]
                            (axis cs:-5.0, 130.0)
                            rectangle (axis cs:0.0, 135.0);
                        \path[,fill,
                              black!2]
                            (axis cs:-5.0, 135.0)
                            rectangle (axis cs:0.0, 140.0);
                        \path[,fill,
                              black!1]
                            (axis cs:-5.0, 140.0)
                            rectangle (axis cs:0.0, 145.0);
                        \path[,fill,
                              black!2]
                            (axis cs:-5.0, 145.0)
                            rectangle (axis cs:0.0, 150.0);
                        \path[,fill,
                              black!2]
                            (axis cs:-5.0, 155.0)
                            rectangle (axis cs:0.0, 160.0);
                        \path[,fill,
                              black!1]
                            (axis cs:0.0, -100.0)
                            rectangle (axis cs:5.0, -95.0);
                        \path[,fill,
                              black!1]
                            (axis cs:0.0, -90.0)
                            rectangle (axis cs:5.0, -85.0);
                        \path[,fill,
                              black!1]
                            (axis cs:0.0, -85.0)
                            rectangle (axis cs:5.0, -80.0);
                        \path[,fill,
                              black!1]
                            (axis cs:0.0, -80.0)
                            rectangle (axis cs:5.0, -75.0);
                        \path[,fill,
                              black!1]
                            (axis cs:0.0, -75.0)
                            rectangle (axis cs:5.0, -70.0);
                        \path[,fill,
                              black!5]
                            (axis cs:0.0, -70.0)
                            rectangle (axis cs:5.0, -65.0);
                        \path[,fill,
                              black!5]
                            (axis cs:0.0, -65.0)
                            rectangle (axis cs:5.0, -60.0);
                        \path[,fill,
                              black!3]
                            (axis cs:0.0, -60.0)
                            rectangle (axis cs:5.0, -55.0);
                        \path[,fill,
                              black!4]
                            (axis cs:0.0, -55.0)
                            rectangle (axis cs:5.0, -50.0);
                        \path[,fill,
                              black!8]
                            (axis cs:0.0, -50.0)
                            rectangle (axis cs:5.0, -45.0);
                        \path[,fill,
                              black!7]
                            (axis cs:0.0, -45.0)
                            rectangle (axis cs:5.0, -40.0);
                        \path[,fill,
                              black!4]
                            (axis cs:0.0, -40.0)
                            rectangle (axis cs:5.0, -35.0);
                        \path[,fill,
                              black!8]
                            (axis cs:0.0, -35.0)
                            rectangle (axis cs:5.0, -30.0);
                        \path[,fill,
                              black!10]
                            (axis cs:0.0, -30.0)
                            rectangle (axis cs:5.0, -25.0);
                        \path[,fill,
                              black!17]
                            (axis cs:0.0, -25.0)
                            rectangle (axis cs:5.0, -20.0);
                        \path[,fill,
                              black!12]
                            (axis cs:0.0, -20.0)
                            rectangle (axis cs:5.0, -15.0);
                        \path[,fill,
                              black!27]
                            (axis cs:0.0, -15.0)
                            rectangle (axis cs:5.0, -10.0);
                        \path[,fill,
                              black!39]
                            (axis cs:0.0, -10.0)
                            rectangle (axis cs:5.0, -5.0);
                        \path[,fill,
                              black!62]
                            (axis cs:0.0, -5.0)
                            rectangle (axis cs:5.0, 0.0);
                        \path[,fill,
                              black!79]
                            (axis cs:0.0, 0.0)
                            rectangle (axis cs:5.0, 5.0);
                        \path[,fill,
                              black!70]
                            (axis cs:0.0, 5.0)
                            rectangle (axis cs:5.0, 10.0);
                        \path[,fill,
                              black!50]
                            (axis cs:0.0, 10.0)
                            rectangle (axis cs:5.0, 15.0);
                        \path[,fill,
                              black!41]
                            (axis cs:0.0, 15.0)
                            rectangle (axis cs:5.0, 20.0);
                        \path[,fill,
                              black!26]
                            (axis cs:0.0, 20.0)
                            rectangle (axis cs:5.0, 25.0);
                        \path[,fill,
                              black!16]
                            (axis cs:0.0, 25.0)
                            rectangle (axis cs:5.0, 30.0);
                        \path[,fill,
                              black!14]
                            (axis cs:0.0, 30.0)
                            rectangle (axis cs:5.0, 35.0);
                        \path[,fill,
                              black!7]
                            (axis cs:0.0, 35.0)
                            rectangle (axis cs:5.0, 40.0);
                        \path[,fill,
                              black!7]
                            (axis cs:0.0, 40.0)
                            rectangle (axis cs:5.0, 45.0);
                        \path[,fill,
                              black!5]
                            (axis cs:0.0, 45.0)
                            rectangle (axis cs:5.0, 50.0);
                        \path[,fill,
                              black!4]
                            (axis cs:0.0, 50.0)
                            rectangle (axis cs:5.0, 55.0);
                        \path[,fill,
                              black!2]
                            (axis cs:0.0, 55.0)
                            rectangle (axis cs:5.0, 60.0);
                        \path[,fill,
                              black!5]
                            (axis cs:0.0, 60.0)
                            rectangle (axis cs:5.0, 65.0);
                        \path[,fill,
                              black!1]
                            (axis cs:0.0, 65.0)
                            rectangle (axis cs:5.0, 70.0);
                        \path[,fill,
                              black!2]
                            (axis cs:0.0, 70.0)
                            rectangle (axis cs:5.0, 75.0);
                        \path[,fill,
                              black!4]
                            (axis cs:0.0, 75.0)
                            rectangle (axis cs:5.0, 80.0);
                        \path[,fill,
                              black!2]
                            (axis cs:0.0, 80.0)
                            rectangle (axis cs:5.0, 85.0);
                        \path[,fill,
                              black!1]
                            (axis cs:0.0, 85.0)
                            rectangle (axis cs:5.0, 90.0);
                        \path[,fill,
                              black!2]
                            (axis cs:0.0, 90.0)
                            rectangle (axis cs:5.0, 95.0);
                        \path[,fill,
                              black!2]
                            (axis cs:0.0, 100.0)
                            rectangle (axis cs:5.0, 105.0);
                        \path[,fill,
                              black!1]
                            (axis cs:0.0, 105.0)
                            rectangle (axis cs:5.0, 110.0);
                        \path[,fill,
                              black!2]
                            (axis cs:0.0, 110.0)
                            rectangle (axis cs:5.0, 115.0);
                        \path[,fill,
                              black!3]
                            (axis cs:0.0, 115.0)
                            rectangle (axis cs:5.0, 120.0);
                        \path[,fill,
                              black!1]
                            (axis cs:0.0, 120.0)
                            rectangle (axis cs:5.0, 125.0);
                        \path[,fill,
                              black!2]
                            (axis cs:0.0, 125.0)
                            rectangle (axis cs:5.0, 130.0);
                        \path[,fill,
                              black!4]
                            (axis cs:0.0, 130.0)
                            rectangle (axis cs:5.0, 135.0);
                        \path[,fill,
                              black!2]
                            (axis cs:0.0, 135.0)
                            rectangle (axis cs:5.0, 140.0);
                        \path[,fill,
                              black!1]
                            (axis cs:0.0, 140.0)
                            rectangle (axis cs:5.0, 145.0);
                        \path[,fill,
                              black!1]
                            (axis cs:0.0, 145.0)
                            rectangle (axis cs:5.0, 150.0);
                        \path[,fill,
                              black!2]
                            (axis cs:0.0, 150.0)
                            rectangle (axis cs:5.0, 155.0);
                        \path[,fill,
                              black!1]
                            (axis cs:0.0, 160.0)
                            rectangle (axis cs:5.0, 165.0);
                        \path[,fill,
                              black!1]
                            (axis cs:0.0, 175.0)
                            rectangle (axis cs:5.0, 180.0);
                        \path[,fill,
                              black!1]
                            (axis cs:5.0, -100.0)
                            rectangle (axis cs:10.0, -95.0);
                        \path[,fill,
                              black!1]
                            (axis cs:5.0, -90.0)
                            rectangle (axis cs:10.0, -85.0);
                        \path[,fill,
                              black!2]
                            (axis cs:5.0, -85.0)
                            rectangle (axis cs:10.0, -80.0);
                        \path[,fill,
                              black!2]
                            (axis cs:5.0, -80.0)
                            rectangle (axis cs:10.0, -75.0);
                        \path[,fill,
                              black!3]
                            (axis cs:5.0, -75.0)
                            rectangle (axis cs:10.0, -70.0);
                        \path[,fill,
                              black!2]
                            (axis cs:5.0, -70.0)
                            rectangle (axis cs:10.0, -65.0);
                        \path[,fill,
                              black!3]
                            (axis cs:5.0, -65.0)
                            rectangle (axis cs:10.0, -60.0);
                        \path[,fill,
                              black!8]
                            (axis cs:5.0, -60.0)
                            rectangle (axis cs:10.0, -55.0);
                        \path[,fill,
                              black!4]
                            (axis cs:5.0, -55.0)
                            rectangle (axis cs:10.0, -50.0);
                        \path[,fill,
                              black!5]
                            (axis cs:5.0, -50.0)
                            rectangle (axis cs:10.0, -45.0);
                        \path[,fill,
                              black!6]
                            (axis cs:5.0, -45.0)
                            rectangle (axis cs:10.0, -40.0);
                        \path[,fill,
                              black!4]
                            (axis cs:5.0, -40.0)
                            rectangle (axis cs:10.0, -35.0);
                        \path[,fill,
                              black!6]
                            (axis cs:5.0, -35.0)
                            rectangle (axis cs:10.0, -30.0);
                        \path[,fill,
                              black!7]
                            (axis cs:5.0, -30.0)
                            rectangle (axis cs:10.0, -25.0);
                        \path[,fill,
                              black!14]
                            (axis cs:5.0, -25.0)
                            rectangle (axis cs:10.0, -20.0);
                        \path[,fill,
                              black!14]
                            (axis cs:5.0, -20.0)
                            rectangle (axis cs:10.0, -15.0);
                        \path[,fill,
                              black!19]
                            (axis cs:5.0, -15.0)
                            rectangle (axis cs:10.0, -10.0);
                        \path[,fill,
                              black!23]
                            (axis cs:5.0, -10.0)
                            rectangle (axis cs:10.0, -5.0);
                        \path[,fill,
                              black!38]
                            (axis cs:5.0, -5.0)
                            rectangle (axis cs:10.0, 0.0);
                        \path[,fill,
                              black!62]
                            (axis cs:5.0, 0.0)
                            rectangle (axis cs:10.0, 5.0);
                        \path[,fill,
                              black!82]
                            (axis cs:5.0, 5.0)
                            rectangle (axis cs:10.0, 10.0);
                        \path[,fill,
                              black!71]
                            (axis cs:5.0, 10.0)
                            rectangle (axis cs:10.0, 15.0);
                        \path[,fill,
                              black!45]
                            (axis cs:5.0, 15.0)
                            rectangle (axis cs:10.0, 20.0);
                        \path[,fill,
                              black!32]
                            (axis cs:5.0, 20.0)
                            rectangle (axis cs:10.0, 25.0);
                        \path[,fill,
                              black!17]
                            (axis cs:5.0, 25.0)
                            rectangle (axis cs:10.0, 30.0);
                        \path[,fill,
                              black!15]
                            (axis cs:5.0, 30.0)
                            rectangle (axis cs:10.0, 35.0);
                        \path[,fill,
                              black!9]
                            (axis cs:5.0, 35.0)
                            rectangle (axis cs:10.0, 40.0);
                        \path[,fill,
                              black!7]
                            (axis cs:5.0, 40.0)
                            rectangle (axis cs:10.0, 45.0);
                        \path[,fill,
                              black!7]
                            (axis cs:5.0, 45.0)
                            rectangle (axis cs:10.0, 50.0);
                        \path[,fill,
                              black!6]
                            (axis cs:5.0, 50.0)
                            rectangle (axis cs:10.0, 55.0);
                        \path[,fill,
                              black!4]
                            (axis cs:5.0, 55.0)
                            rectangle (axis cs:10.0, 60.0);
                        \path[,fill,
                              black!4]
                            (axis cs:5.0, 60.0)
                            rectangle (axis cs:10.0, 65.0);
                        \path[,fill,
                              black!2]
                            (axis cs:5.0, 65.0)
                            rectangle (axis cs:10.0, 70.0);
                        \path[,fill,
                              black!2]
                            (axis cs:5.0, 70.0)
                            rectangle (axis cs:10.0, 75.0);
                        \path[,fill,
                              black!1]
                            (axis cs:5.0, 75.0)
                            rectangle (axis cs:10.0, 80.0);
                        \path[,fill,
                              black!2]
                            (axis cs:5.0, 80.0)
                            rectangle (axis cs:10.0, 85.0);
                        \path[,fill,
                              black!3]
                            (axis cs:5.0, 85.0)
                            rectangle (axis cs:10.0, 90.0);
                        \path[,fill,
                              black!4]
                            (axis cs:5.0, 90.0)
                            rectangle (axis cs:10.0, 95.0);
                        \path[,fill,
                              black!1]
                            (axis cs:5.0, 95.0)
                            rectangle (axis cs:10.0, 100.0);
                        \path[,fill,
                              black!1]
                            (axis cs:5.0, 100.0)
                            rectangle (axis cs:10.0, 105.0);
                        \path[,fill,
                              black!2]
                            (axis cs:5.0, 105.0)
                            rectangle (axis cs:10.0, 110.0);
                        \path[,fill,
                              black!3]
                            (axis cs:5.0, 110.0)
                            rectangle (axis cs:10.0, 115.0);
                        \path[,fill,
                              black!3]
                            (axis cs:5.0, 115.0)
                            rectangle (axis cs:10.0, 120.0);
                        \path[,fill,
                              black!3]
                            (axis cs:5.0, 120.0)
                            rectangle (axis cs:10.0, 125.0);
                        \path[,fill,
                              black!1]
                            (axis cs:5.0, 125.0)
                            rectangle (axis cs:10.0, 130.0);
                        \path[,fill,
                              black!2]
                            (axis cs:5.0, 130.0)
                            rectangle (axis cs:10.0, 135.0);
                        \path[,fill,
                              black!2]
                            (axis cs:5.0, 135.0)
                            rectangle (axis cs:10.0, 140.0);
                        \path[,fill,
                              black!1]
                            (axis cs:5.0, 140.0)
                            rectangle (axis cs:10.0, 145.0);
                        \path[,fill,
                              black!1]
                            (axis cs:10.0, -120.0)
                            rectangle (axis cs:15.0, -115.0);
                        \path[,fill,
                              black!1]
                            (axis cs:10.0, -90.0)
                            rectangle (axis cs:15.0, -85.0);
                        \path[,fill,
                              black!1]
                            (axis cs:10.0, -85.0)
                            rectangle (axis cs:15.0, -80.0);
                        \path[,fill,
                              black!1]
                            (axis cs:10.0, -80.0)
                            rectangle (axis cs:15.0, -75.0);
                        \path[,fill,
                              black!1]
                            (axis cs:10.0, -75.0)
                            rectangle (axis cs:15.0, -70.0);
                        \path[,fill,
                              black!1]
                            (axis cs:10.0, -70.0)
                            rectangle (axis cs:15.0, -65.0);
                        \path[,fill,
                              black!5]
                            (axis cs:10.0, -65.0)
                            rectangle (axis cs:15.0, -60.0);
                        \path[,fill,
                              black!4]
                            (axis cs:10.0, -60.0)
                            rectangle (axis cs:15.0, -55.0);
                        \path[,fill,
                              black!5]
                            (axis cs:10.0, -55.0)
                            rectangle (axis cs:15.0, -50.0);
                        \path[,fill,
                              black!2]
                            (axis cs:10.0, -50.0)
                            rectangle (axis cs:15.0, -45.0);
                        \path[,fill,
                              black!3]
                            (axis cs:10.0, -45.0)
                            rectangle (axis cs:15.0, -40.0);
                        \path[,fill,
                              black!7]
                            (axis cs:10.0, -40.0)
                            rectangle (axis cs:15.0, -35.0);
                        \path[,fill,
                              black!5]
                            (axis cs:10.0, -35.0)
                            rectangle (axis cs:15.0, -30.0);
                        \path[,fill,
                              black!7]
                            (axis cs:10.0, -30.0)
                            rectangle (axis cs:15.0, -25.0);
                        \path[,fill,
                              black!6]
                            (axis cs:10.0, -25.0)
                            rectangle (axis cs:15.0, -20.0);
                        \path[,fill,
                              black!10]
                            (axis cs:10.0, -20.0)
                            rectangle (axis cs:15.0, -15.0);
                        \path[,fill,
                              black!13]
                            (axis cs:10.0, -15.0)
                            rectangle (axis cs:15.0, -10.0);
                        \path[,fill,
                              black!18]
                            (axis cs:10.0, -10.0)
                            rectangle (axis cs:15.0, -5.0);
                        \path[,fill,
                              black!24]
                            (axis cs:10.0, -5.0)
                            rectangle (axis cs:15.0, 0.0);
                        \path[,fill,
                              black!40]
                            (axis cs:10.0, 0.0)
                            rectangle (axis cs:15.0, 5.0);
                        \path[,fill,
                              black!61]
                            (axis cs:10.0, 5.0)
                            rectangle (axis cs:15.0, 10.0);
                        \path[,fill,
                              black!85]
                            (axis cs:10.0, 10.0)
                            rectangle (axis cs:15.0, 15.0);
                        \path[,fill,
                              black!72]
                            (axis cs:10.0, 15.0)
                            rectangle (axis cs:15.0, 20.0);
                        \path[,fill,
                              black!47]
                            (axis cs:10.0, 20.0)
                            rectangle (axis cs:15.0, 25.0);
                        \path[,fill,
                              black!26]
                            (axis cs:10.0, 25.0)
                            rectangle (axis cs:15.0, 30.0);
                        \path[,fill,
                              black!19]
                            (axis cs:10.0, 30.0)
                            rectangle (axis cs:15.0, 35.0);
                        \path[,fill,
                              black!13]
                            (axis cs:10.0, 35.0)
                            rectangle (axis cs:15.0, 40.0);
                        \path[,fill,
                              black!9]
                            (axis cs:10.0, 40.0)
                            rectangle (axis cs:15.0, 45.0);
                        \path[,fill,
                              black!5]
                            (axis cs:10.0, 45.0)
                            rectangle (axis cs:15.0, 50.0);
                        \path[,fill,
                              black!4]
                            (axis cs:10.0, 50.0)
                            rectangle (axis cs:15.0, 55.0);
                        \path[,fill,
                              black!4]
                            (axis cs:10.0, 55.0)
                            rectangle (axis cs:15.0, 60.0);
                        \path[,fill,
                              black!2]
                            (axis cs:10.0, 60.0)
                            rectangle (axis cs:15.0, 65.0);
                        \path[,fill,
                              black!4]
                            (axis cs:10.0, 65.0)
                            rectangle (axis cs:15.0, 70.0);
                        \path[,fill,
                              black!2]
                            (axis cs:10.0, 70.0)
                            rectangle (axis cs:15.0, 75.0);
                        \path[,fill,
                              black!2]
                            (axis cs:10.0, 75.0)
                            rectangle (axis cs:15.0, 80.0);
                        \path[,fill,
                              black!4]
                            (axis cs:10.0, 80.0)
                            rectangle (axis cs:15.0, 85.0);
                        \path[,fill,
                              black!2]
                            (axis cs:10.0, 85.0)
                            rectangle (axis cs:15.0, 90.0);
                        \path[,fill,
                              black!2]
                            (axis cs:10.0, 90.0)
                            rectangle (axis cs:15.0, 95.0);
                        \path[,fill,
                              black!3]
                            (axis cs:10.0, 95.0)
                            rectangle (axis cs:15.0, 100.0);
                        \path[,fill,
                              black!2]
                            (axis cs:10.0, 100.0)
                            rectangle (axis cs:15.0, 105.0);
                        \path[,fill,
                              black!2]
                            (axis cs:10.0, 105.0)
                            rectangle (axis cs:15.0, 110.0);
                        \path[,fill,
                              black!2]
                            (axis cs:10.0, 110.0)
                            rectangle (axis cs:15.0, 115.0);
                        \path[,fill,
                              black!2]
                            (axis cs:10.0, 115.0)
                            rectangle (axis cs:15.0, 120.0);
                        \path[,fill,
                              black!3]
                            (axis cs:10.0, 120.0)
                            rectangle (axis cs:15.0, 125.0);
                        \path[,fill,
                              black!1]
                            (axis cs:10.0, 125.0)
                            rectangle (axis cs:15.0, 130.0);
                        \path[,fill,
                              black!2]
                            (axis cs:10.0, 130.0)
                            rectangle (axis cs:15.0, 135.0);
                        \path[,fill,
                              black!2]
                            (axis cs:10.0, 135.0)
                            rectangle (axis cs:15.0, 140.0);
                        \path[,fill,
                              black!1]
                            (axis cs:10.0, 140.0)
                            rectangle (axis cs:15.0, 145.0);
                        \path[,fill,
                              black!1]
                            (axis cs:10.0, 145.0)
                            rectangle (axis cs:15.0, 150.0);
                        \path[,fill,
                              black!2]
                            (axis cs:10.0, 150.0)
                            rectangle (axis cs:15.0, 155.0);
                        \path[,fill,
                              black!1]
                            (axis cs:10.0, 155.0)
                            rectangle (axis cs:15.0, 160.0);
                        \path[,fill,
                              black!1]
                            (axis cs:15.0, -90.0)
                            rectangle (axis cs:20.0, -85.0);
                        \path[,fill,
                              black!1]
                            (axis cs:15.0, -80.0)
                            rectangle (axis cs:20.0, -75.0);
                        \path[,fill,
                              black!2]
                            (axis cs:15.0, -75.0)
                            rectangle (axis cs:20.0, -70.0);
                        \path[,fill,
                              black!2]
                            (axis cs:15.0, -70.0)
                            rectangle (axis cs:20.0, -65.0);
                        \path[,fill,
                              black!1]
                            (axis cs:15.0, -65.0)
                            rectangle (axis cs:20.0, -60.0);
                        \path[,fill,
                              black!4]
                            (axis cs:15.0, -60.0)
                            rectangle (axis cs:20.0, -55.0);
                        \path[,fill,
                              black!5]
                            (axis cs:15.0, -55.0)
                            rectangle (axis cs:20.0, -50.0);
                        \path[,fill,
                              black!4]
                            (axis cs:15.0, -50.0)
                            rectangle (axis cs:20.0, -45.0);
                        \path[,fill,
                              black!3]
                            (axis cs:15.0, -45.0)
                            rectangle (axis cs:20.0, -40.0);
                        \path[,fill,
                              black!7]
                            (axis cs:15.0, -40.0)
                            rectangle (axis cs:20.0, -35.0);
                        \path[,fill,
                              black!3]
                            (axis cs:15.0, -35.0)
                            rectangle (axis cs:20.0, -30.0);
                        \path[,fill,
                              black!8]
                            (axis cs:15.0, -30.0)
                            rectangle (axis cs:20.0, -25.0);
                        \path[,fill,
                              black!5]
                            (axis cs:15.0, -25.0)
                            rectangle (axis cs:20.0, -20.0);
                        \path[,fill,
                              black!10]
                            (axis cs:15.0, -20.0)
                            rectangle (axis cs:20.0, -15.0);
                        \path[,fill,
                              black!10]
                            (axis cs:15.0, -15.0)
                            rectangle (axis cs:20.0, -10.0);
                        \path[,fill,
                              black!10]
                            (axis cs:15.0, -10.0)
                            rectangle (axis cs:20.0, -5.0);
                        \path[,fill,
                              black!16]
                            (axis cs:15.0, -5.0)
                            rectangle (axis cs:20.0, 0.0);
                        \path[,fill,
                              black!28]
                            (axis cs:15.0, 0.0)
                            rectangle (axis cs:20.0, 5.0);
                        \path[,fill,
                              black!34]
                            (axis cs:15.0, 5.0)
                            rectangle (axis cs:20.0, 10.0);
                        \path[,fill,
                              black!63]
                            (axis cs:15.0, 10.0)
                            rectangle (axis cs:20.0, 15.0);
                        \path[,fill,
                              black!85]
                            (axis cs:15.0, 15.0)
                            rectangle (axis cs:20.0, 20.0);
                        \path[,fill,
                              black!65]
                            (axis cs:15.0, 20.0)
                            rectangle (axis cs:20.0, 25.0);
                        \path[,fill,
                              black!43]
                            (axis cs:15.0, 25.0)
                            rectangle (axis cs:20.0, 30.0);
                        \path[,fill,
                              black!24]
                            (axis cs:15.0, 30.0)
                            rectangle (axis cs:20.0, 35.0);
                        \path[,fill,
                              black!22]
                            (axis cs:15.0, 35.0)
                            rectangle (axis cs:20.0, 40.0);
                        \path[,fill,
                              black!11]
                            (axis cs:15.0, 40.0)
                            rectangle (axis cs:20.0, 45.0);
                        \path[,fill,
                              black!10]
                            (axis cs:15.0, 45.0)
                            rectangle (axis cs:20.0, 50.0);
                        \path[,fill,
                              black!3]
                            (axis cs:15.0, 50.0)
                            rectangle (axis cs:20.0, 55.0);
                        \path[,fill,
                              black!4]
                            (axis cs:15.0, 55.0)
                            rectangle (axis cs:20.0, 60.0);
                        \path[,fill,
                              black!7]
                            (axis cs:15.0, 60.0)
                            rectangle (axis cs:20.0, 65.0);
                        \path[,fill,
                              black!3]
                            (axis cs:15.0, 65.0)
                            rectangle (axis cs:20.0, 70.0);
                        \path[,fill,
                              black!2]
                            (axis cs:15.0, 70.0)
                            rectangle (axis cs:20.0, 75.0);
                        \path[,fill,
                              black!5]
                            (axis cs:15.0, 75.0)
                            rectangle (axis cs:20.0, 80.0);
                        \path[,fill,
                              black!4]
                            (axis cs:15.0, 80.0)
                            rectangle (axis cs:20.0, 85.0);
                        \path[,fill,
                              black!3]
                            (axis cs:15.0, 85.0)
                            rectangle (axis cs:20.0, 90.0);
                        \path[,fill,
                              black!4]
                            (axis cs:15.0, 90.0)
                            rectangle (axis cs:20.0, 95.0);
                        \path[,fill,
                              black!2]
                            (axis cs:15.0, 95.0)
                            rectangle (axis cs:20.0, 100.0);
                        \path[,fill,
                              black!5]
                            (axis cs:15.0, 100.0)
                            rectangle (axis cs:20.0, 105.0);
                        \path[,fill,
                              black!5]
                            (axis cs:15.0, 105.0)
                            rectangle (axis cs:20.0, 110.0);
                        \path[,fill,
                              black!2]
                            (axis cs:15.0, 110.0)
                            rectangle (axis cs:20.0, 115.0);
                        \path[,fill,
                              black!2]
                            (axis cs:15.0, 115.0)
                            rectangle (axis cs:20.0, 120.0);
                        \path[,fill,
                              black!3]
                            (axis cs:15.0, 120.0)
                            rectangle (axis cs:20.0, 125.0);
                        \path[,fill,
                              black!5]
                            (axis cs:15.0, 125.0)
                            rectangle (axis cs:20.0, 130.0);
                        \path[,fill,
                              black!1]
                            (axis cs:15.0, 130.0)
                            rectangle (axis cs:20.0, 135.0);
                        \path[,fill,
                              black!2]
                            (axis cs:15.0, 135.0)
                            rectangle (axis cs:20.0, 140.0);
                        \path[,fill,
                              black!2]
                            (axis cs:15.0, 145.0)
                            rectangle (axis cs:20.0, 150.0);
                        \path[,fill,
                              black!1]
                            (axis cs:15.0, 155.0)
                            rectangle (axis cs:20.0, 160.0);
                        \path[,fill,
                              black!1]
                            (axis cs:20.0, -95.0)
                            rectangle (axis cs:25.0, -90.0);
                        \path[,fill,
                              black!1]
                            (axis cs:20.0, -80.0)
                            rectangle (axis cs:25.0, -75.0);
                        \path[,fill,
                              black!2]
                            (axis cs:20.0, -75.0)
                            rectangle (axis cs:25.0, -70.0);
                        \path[,fill,
                              black!4]
                            (axis cs:20.0, -70.0)
                            rectangle (axis cs:25.0, -65.0);
                        \path[,fill,
                              black!2]
                            (axis cs:20.0, -65.0)
                            rectangle (axis cs:25.0, -60.0);
                        \path[,fill,
                              black!5]
                            (axis cs:20.0, -60.0)
                            rectangle (axis cs:25.0, -55.0);
                        \path[,fill,
                              black!4]
                            (axis cs:20.0, -55.0)
                            rectangle (axis cs:25.0, -50.0);
                        \path[,fill,
                              black!3]
                            (axis cs:20.0, -50.0)
                            rectangle (axis cs:25.0, -45.0);
                        \path[,fill,
                              black!6]
                            (axis cs:20.0, -45.0)
                            rectangle (axis cs:25.0, -40.0);
                        \path[,fill,
                              black!5]
                            (axis cs:20.0, -40.0)
                            rectangle (axis cs:25.0, -35.0);
                        \path[,fill,
                              black!5]
                            (axis cs:20.0, -35.0)
                            rectangle (axis cs:25.0, -30.0);
                        \path[,fill,
                              black!4]
                            (axis cs:20.0, -30.0)
                            rectangle (axis cs:25.0, -25.0);
                        \path[,fill,
                              black!5]
                            (axis cs:20.0, -25.0)
                            rectangle (axis cs:25.0, -20.0);
                        \path[,fill,
                              black!8]
                            (axis cs:20.0, -20.0)
                            rectangle (axis cs:25.0, -15.0);
                        \path[,fill,
                              black!7]
                            (axis cs:20.0, -15.0)
                            rectangle (axis cs:25.0, -10.0);
                        \path[,fill,
                              black!11]
                            (axis cs:20.0, -10.0)
                            rectangle (axis cs:25.0, -5.0);
                        \path[,fill,
                              black!10]
                            (axis cs:20.0, -5.0)
                            rectangle (axis cs:25.0, 0.0);
                        \path[,fill,
                              black!13]
                            (axis cs:20.0, 0.0)
                            rectangle (axis cs:25.0, 5.0);
                        \path[,fill,
                              black!21]
                            (axis cs:20.0, 5.0)
                            rectangle (axis cs:25.0, 10.0);
                        \path[,fill,
                              black!35]
                            (axis cs:20.0, 10.0)
                            rectangle (axis cs:25.0, 15.0);
                        \path[,fill,
                              black!58]
                            (axis cs:20.0, 15.0)
                            rectangle (axis cs:25.0, 20.0);
                        \path[,fill,
                              black!85]
                            (axis cs:20.0, 20.0)
                            rectangle (axis cs:25.0, 25.0);
                        \path[,fill,
                              black!73]
                            (axis cs:20.0, 25.0)
                            rectangle (axis cs:25.0, 30.0);
                        \path[,fill,
                              black!40]
                            (axis cs:20.0, 30.0)
                            rectangle (axis cs:25.0, 35.0);
                        \path[,fill,
                              black!23]
                            (axis cs:20.0, 35.0)
                            rectangle (axis cs:25.0, 40.0);
                        \path[,fill,
                              black!16]
                            (axis cs:20.0, 40.0)
                            rectangle (axis cs:25.0, 45.0);
                        \path[,fill,
                              black!14]
                            (axis cs:20.0, 45.0)
                            rectangle (axis cs:25.0, 50.0);
                        \path[,fill,
                              black!5]
                            (axis cs:20.0, 50.0)
                            rectangle (axis cs:25.0, 55.0);
                        \path[,fill,
                              black!8]
                            (axis cs:20.0, 55.0)
                            rectangle (axis cs:25.0, 60.0);
                        \path[,fill,
                              black!7]
                            (axis cs:20.0, 60.0)
                            rectangle (axis cs:25.0, 65.0);
                        \path[,fill,
                              black!3]
                            (axis cs:20.0, 65.0)
                            rectangle (axis cs:25.0, 70.0);
                        \path[,fill,
                              black!4]
                            (axis cs:20.0, 70.0)
                            rectangle (axis cs:25.0, 75.0);
                        \path[,fill,
                              black!4]
                            (axis cs:20.0, 75.0)
                            rectangle (axis cs:25.0, 80.0);
                        \path[,fill,
                              black!3]
                            (axis cs:20.0, 80.0)
                            rectangle (axis cs:25.0, 85.0);
                        \path[,fill,
                              black!2]
                            (axis cs:20.0, 85.0)
                            rectangle (axis cs:25.0, 90.0);
                        \path[,fill,
                              black!5]
                            (axis cs:20.0, 90.0)
                            rectangle (axis cs:25.0, 95.0);
                        \path[,fill,
                              black!4]
                            (axis cs:20.0, 95.0)
                            rectangle (axis cs:25.0, 100.0);
                        \path[,fill,
                              black!1]
                            (axis cs:20.0, 100.0)
                            rectangle (axis cs:25.0, 105.0);
                        \path[,fill,
                              black!2]
                            (axis cs:20.0, 105.0)
                            rectangle (axis cs:25.0, 110.0);
                        \path[,fill,
                              black!4]
                            (axis cs:20.0, 110.0)
                            rectangle (axis cs:25.0, 115.0);
                        \path[,fill,
                              black!5]
                            (axis cs:20.0, 115.0)
                            rectangle (axis cs:25.0, 120.0);
                        \path[,fill,
                              black!4]
                            (axis cs:20.0, 120.0)
                            rectangle (axis cs:25.0, 125.0);
                        \path[,fill,
                              black!2]
                            (axis cs:20.0, 125.0)
                            rectangle (axis cs:25.0, 130.0);
                        \path[,fill,
                              black!1]
                            (axis cs:20.0, 130.0)
                            rectangle (axis cs:25.0, 135.0);
                        \path[,fill,
                              black!1]
                            (axis cs:20.0, 135.0)
                            rectangle (axis cs:25.0, 140.0);
                        \path[,fill,
                              black!1]
                            (axis cs:20.0, 140.0)
                            rectangle (axis cs:25.0, 145.0);
                        \path[,fill,
                              black!1]
                            (axis cs:20.0, 145.0)
                            rectangle (axis cs:25.0, 150.0);
                        \path[,fill,
                              black!1]
                            (axis cs:20.0, 175.0)
                            rectangle (axis cs:25.0, 180.0);
                        \path[,fill,
                              black!1]
                            (axis cs:25.0, -170.0)
                            rectangle (axis cs:30.0, -165.0);
                        \path[,fill,
                              black!1]
                            (axis cs:25.0, -80.0)
                            rectangle (axis cs:30.0, -75.0);
                        \path[,fill,
                              black!2]
                            (axis cs:25.0, -75.0)
                            rectangle (axis cs:30.0, -70.0);
                        \path[,fill,
                              black!3]
                            (axis cs:25.0, -70.0)
                            rectangle (axis cs:30.0, -65.0);
                        \path[,fill,
                              black!2]
                            (axis cs:25.0, -65.0)
                            rectangle (axis cs:30.0, -60.0);
                        \path[,fill,
                              black!3]
                            (axis cs:25.0, -60.0)
                            rectangle (axis cs:30.0, -55.0);
                        \path[,fill,
                              black!4]
                            (axis cs:25.0, -55.0)
                            rectangle (axis cs:30.0, -50.0);
                        \path[,fill,
                              black!4]
                            (axis cs:25.0, -50.0)
                            rectangle (axis cs:30.0, -45.0);
                        \path[,fill,
                              black!6]
                            (axis cs:25.0, -45.0)
                            rectangle (axis cs:30.0, -40.0);
                        \path[,fill,
                              black!2]
                            (axis cs:25.0, -40.0)
                            rectangle (axis cs:30.0, -35.0);
                        \path[,fill,
                              black!7]
                            (axis cs:25.0, -35.0)
                            rectangle (axis cs:30.0, -30.0);
                        \path[,fill,
                              black!5]
                            (axis cs:25.0, -30.0)
                            rectangle (axis cs:30.0, -25.0);
                        \path[,fill,
                              black!4]
                            (axis cs:25.0, -25.0)
                            rectangle (axis cs:30.0, -20.0);
                        \path[,fill,
                              black!6]
                            (axis cs:25.0, -20.0)
                            rectangle (axis cs:30.0, -15.0);
                        \path[,fill,
                              black!8]
                            (axis cs:25.0, -15.0)
                            rectangle (axis cs:30.0, -10.0);
                        \path[,fill,
                              black!9]
                            (axis cs:25.0, -10.0)
                            rectangle (axis cs:30.0, -5.0);
                        \path[,fill,
                              black!9]
                            (axis cs:25.0, -5.0)
                            rectangle (axis cs:30.0, 0.0);
                        \path[,fill,
                              black!10]
                            (axis cs:25.0, 0.0)
                            rectangle (axis cs:30.0, 5.0);
                        \path[,fill,
                              black!18]
                            (axis cs:25.0, 5.0)
                            rectangle (axis cs:30.0, 10.0);
                        \path[,fill,
                              black!26]
                            (axis cs:25.0, 10.0)
                            rectangle (axis cs:30.0, 15.0);
                        \path[,fill,
                              black!38]
                            (axis cs:25.0, 15.0)
                            rectangle (axis cs:30.0, 20.0);
                        \path[,fill,
                              black!60]
                            (axis cs:25.0, 20.0)
                            rectangle (axis cs:30.0, 25.0);
                        \path[,fill,
                              black!85]
                            (axis cs:25.0, 25.0)
                            rectangle (axis cs:30.0, 30.0);
                        \path[,fill,
                              black!72]
                            (axis cs:25.0, 30.0)
                            rectangle (axis cs:30.0, 35.0);
                        \path[,fill,
                              black!41]
                            (axis cs:25.0, 35.0)
                            rectangle (axis cs:30.0, 40.0);
                        \path[,fill,
                              black!27]
                            (axis cs:25.0, 40.0)
                            rectangle (axis cs:30.0, 45.0);
                        \path[,fill,
                              black!14]
                            (axis cs:25.0, 45.0)
                            rectangle (axis cs:30.0, 50.0);
                        \path[,fill,
                              black!10]
                            (axis cs:25.0, 50.0)
                            rectangle (axis cs:30.0, 55.0);
                        \path[,fill,
                              black!11]
                            (axis cs:25.0, 55.0)
                            rectangle (axis cs:30.0, 60.0);
                        \path[,fill,
                              black!7]
                            (axis cs:25.0, 60.0)
                            rectangle (axis cs:30.0, 65.0);
                        \path[,fill,
                              black!5]
                            (axis cs:25.0, 65.0)
                            rectangle (axis cs:30.0, 70.0);
                        \path[,fill,
                              black!7]
                            (axis cs:25.0, 70.0)
                            rectangle (axis cs:30.0, 75.0);
                        \path[,fill,
                              black!8]
                            (axis cs:25.0, 75.0)
                            rectangle (axis cs:30.0, 80.0);
                        \path[,fill,
                              black!5]
                            (axis cs:25.0, 80.0)
                            rectangle (axis cs:30.0, 85.0);
                        \path[,fill,
                              black!2]
                            (axis cs:25.0, 85.0)
                            rectangle (axis cs:30.0, 90.0);
                        \path[,fill,
                              black!2]
                            (axis cs:25.0, 90.0)
                            rectangle (axis cs:30.0, 95.0);
                        \path[,fill,
                              black!5]
                            (axis cs:25.0, 95.0)
                            rectangle (axis cs:30.0, 100.0);
                        \path[,fill,
                              black!4]
                            (axis cs:25.0, 100.0)
                            rectangle (axis cs:30.0, 105.0);
                        \path[,fill,
                              black!2]
                            (axis cs:25.0, 105.0)
                            rectangle (axis cs:30.0, 110.0);
                        \path[,fill,
                              black!2]
                            (axis cs:25.0, 110.0)
                            rectangle (axis cs:30.0, 115.0);
                        \path[,fill,
                              black!3]
                            (axis cs:25.0, 115.0)
                            rectangle (axis cs:30.0, 120.0);
                        \path[,fill,
                              black!4]
                            (axis cs:25.0, 120.0)
                            rectangle (axis cs:30.0, 125.0);
                        \path[,fill,
                              black!2]
                            (axis cs:25.0, 125.0)
                            rectangle (axis cs:30.0, 130.0);
                        \path[,fill,
                              black!3]
                            (axis cs:25.0, 130.0)
                            rectangle (axis cs:30.0, 135.0);
                        \path[,fill,
                              black!1]
                            (axis cs:25.0, 135.0)
                            rectangle (axis cs:30.0, 140.0);
                        \path[,fill,
                              black!1]
                            (axis cs:25.0, 140.0)
                            rectangle (axis cs:30.0, 145.0);
                        \path[,fill,
                              black!1]
                            (axis cs:25.0, 155.0)
                            rectangle (axis cs:30.0, 160.0);
                        \path[,fill,
                              black!1]
                            (axis cs:30.0, -85.0)
                            rectangle (axis cs:35.0, -80.0);
                        \path[,fill,
                              black!1]
                            (axis cs:30.0, -80.0)
                            rectangle (axis cs:35.0, -75.0);
                        \path[,fill,
                              black!2]
                            (axis cs:30.0, -75.0)
                            rectangle (axis cs:35.0, -70.0);
                        \path[,fill,
                              black!2]
                            (axis cs:30.0, -70.0)
                            rectangle (axis cs:35.0, -65.0);
                        \path[,fill,
                              black!1]
                            (axis cs:30.0, -65.0)
                            rectangle (axis cs:35.0, -60.0);
                        \path[,fill,
                              black!2]
                            (axis cs:30.0, -60.0)
                            rectangle (axis cs:35.0, -55.0);
                        \path[,fill,
                              black!2]
                            (axis cs:30.0, -55.0)
                            rectangle (axis cs:35.0, -50.0);
                        \path[,fill,
                              black!2]
                            (axis cs:30.0, -50.0)
                            rectangle (axis cs:35.0, -45.0);
                        \path[,fill,
                              black!2]
                            (axis cs:30.0, -45.0)
                            rectangle (axis cs:35.0, -40.0);
                        \path[,fill,
                              black!3]
                            (axis cs:30.0, -40.0)
                            rectangle (axis cs:35.0, -35.0);
                        \path[,fill,
                              black!6]
                            (axis cs:30.0, -35.0)
                            rectangle (axis cs:35.0, -30.0);
                        \path[,fill,
                              black!2]
                            (axis cs:30.0, -30.0)
                            rectangle (axis cs:35.0, -25.0);
                        \path[,fill,
                              black!4]
                            (axis cs:30.0, -25.0)
                            rectangle (axis cs:35.0, -20.0);
                        \path[,fill,
                              black!9]
                            (axis cs:30.0, -20.0)
                            rectangle (axis cs:35.0, -15.0);
                        \path[,fill,
                              black!7]
                            (axis cs:30.0, -15.0)
                            rectangle (axis cs:35.0, -10.0);
                        \path[,fill,
                              black!7]
                            (axis cs:30.0, -10.0)
                            rectangle (axis cs:35.0, -5.0);
                        \path[,fill,
                              black!10]
                            (axis cs:30.0, -5.0)
                            rectangle (axis cs:35.0, 0.0);
                        \path[,fill,
                              black!10]
                            (axis cs:30.0, 0.0)
                            rectangle (axis cs:35.0, 5.0);
                        \path[,fill,
                              black!15]
                            (axis cs:30.0, 5.0)
                            rectangle (axis cs:35.0, 10.0);
                        \path[,fill,
                              black!19]
                            (axis cs:30.0, 10.0)
                            rectangle (axis cs:35.0, 15.0);
                        \path[,fill,
                              black!28]
                            (axis cs:30.0, 15.0)
                            rectangle (axis cs:35.0, 20.0);
                        \path[,fill,
                              black!39]
                            (axis cs:30.0, 20.0)
                            rectangle (axis cs:35.0, 25.0);
                        \path[,fill,
                              black!75]
                            (axis cs:30.0, 25.0)
                            rectangle (axis cs:35.0, 30.0);
                        \path[,fill,
                              black!85]
                            (axis cs:30.0, 30.0)
                            rectangle (axis cs:35.0, 35.0);
                        \path[,fill,
                              black!58]
                            (axis cs:30.0, 35.0)
                            rectangle (axis cs:35.0, 40.0);
                        \path[,fill,
                              black!37]
                            (axis cs:30.0, 40.0)
                            rectangle (axis cs:35.0, 45.0);
                        \path[,fill,
                              black!26]
                            (axis cs:30.0, 45.0)
                            rectangle (axis cs:35.0, 50.0);
                        \path[,fill,
                              black!19]
                            (axis cs:30.0, 50.0)
                            rectangle (axis cs:35.0, 55.0);
                        \path[,fill,
                              black!16]
                            (axis cs:30.0, 55.0)
                            rectangle (axis cs:35.0, 60.0);
                        \path[,fill,
                              black!9]
                            (axis cs:30.0, 60.0)
                            rectangle (axis cs:35.0, 65.0);
                        \path[,fill,
                              black!7]
                            (axis cs:30.0, 65.0)
                            rectangle (axis cs:35.0, 70.0);
                        \path[,fill,
                              black!5]
                            (axis cs:30.0, 70.0)
                            rectangle (axis cs:35.0, 75.0);
                        \path[,fill,
                              black!9]
                            (axis cs:30.0, 75.0)
                            rectangle (axis cs:35.0, 80.0);
                        \path[,fill,
                              black!3]
                            (axis cs:30.0, 80.0)
                            rectangle (axis cs:35.0, 85.0);
                        \path[,fill,
                              black!8]
                            (axis cs:30.0, 85.0)
                            rectangle (axis cs:35.0, 90.0);
                        \path[,fill,
                              black!3]
                            (axis cs:30.0, 90.0)
                            rectangle (axis cs:35.0, 95.0);
                        \path[,fill,
                              black!5]
                            (axis cs:30.0, 95.0)
                            rectangle (axis cs:35.0, 100.0);
                        \path[,fill,
                              black!4]
                            (axis cs:30.0, 100.0)
                            rectangle (axis cs:35.0, 105.0);
                        \path[,fill,
                              black!3]
                            (axis cs:30.0, 105.0)
                            rectangle (axis cs:35.0, 110.0);
                        \path[,fill,
                              black!3]
                            (axis cs:30.0, 110.0)
                            rectangle (axis cs:35.0, 115.0);
                        \path[,fill,
                              black!2]
                            (axis cs:30.0, 115.0)
                            rectangle (axis cs:35.0, 120.0);
                        \path[,fill,
                              black!7]
                            (axis cs:30.0, 120.0)
                            rectangle (axis cs:35.0, 125.0);
                        \path[,fill,
                              black!2]
                            (axis cs:30.0, 125.0)
                            rectangle (axis cs:35.0, 130.0);
                        \path[,fill,
                              black!1]
                            (axis cs:30.0, 130.0)
                            rectangle (axis cs:35.0, 135.0);
                        \path[,fill,
                              black!1]
                            (axis cs:30.0, 135.0)
                            rectangle (axis cs:35.0, 140.0);
                        \path[,fill,
                              black!1]
                            (axis cs:30.0, 140.0)
                            rectangle (axis cs:35.0, 145.0);
                        \path[,fill,
                              black!1]
                            (axis cs:30.0, 150.0)
                            rectangle (axis cs:35.0, 155.0);
                        \path[,fill,
                              black!1]
                            (axis cs:30.0, 165.0)
                            rectangle (axis cs:35.0, 170.0);
                        \path[,fill,
                              black!1]
                            (axis cs:35.0, -120.0)
                            rectangle (axis cs:40.0, -115.0);
                        \path[,fill,
                              black!1]
                            (axis cs:35.0, -115.0)
                            rectangle (axis cs:40.0, -110.0);
                        \path[,fill,
                              black!1]
                            (axis cs:35.0, -85.0)
                            rectangle (axis cs:40.0, -80.0);
                        \path[,fill,
                              black!2]
                            (axis cs:35.0, -80.0)
                            rectangle (axis cs:40.0, -75.0);
                        \path[,fill,
                              black!3]
                            (axis cs:35.0, -75.0)
                            rectangle (axis cs:40.0, -70.0);
                        \path[,fill,
                              black!1]
                            (axis cs:35.0, -70.0)
                            rectangle (axis cs:40.0, -65.0);
                        \path[,fill,
                              black!3]
                            (axis cs:35.0, -65.0)
                            rectangle (axis cs:40.0, -60.0);
                        \path[,fill,
                              black!3]
                            (axis cs:35.0, -60.0)
                            rectangle (axis cs:40.0, -55.0);
                        \path[,fill,
                              black!2]
                            (axis cs:35.0, -55.0)
                            rectangle (axis cs:40.0, -50.0);
                        \path[,fill,
                              black!3]
                            (axis cs:35.0, -50.0)
                            rectangle (axis cs:40.0, -45.0);
                        \path[,fill,
                              black!2]
                            (axis cs:35.0, -45.0)
                            rectangle (axis cs:40.0, -40.0);
                        \path[,fill,
                              black!2]
                            (axis cs:35.0, -40.0)
                            rectangle (axis cs:40.0, -35.0);
                        \path[,fill,
                              black!4]
                            (axis cs:35.0, -35.0)
                            rectangle (axis cs:40.0, -30.0);
                        \path[,fill,
                              black!4]
                            (axis cs:35.0, -30.0)
                            rectangle (axis cs:40.0, -25.0);
                        \path[,fill,
                              black!5]
                            (axis cs:35.0, -25.0)
                            rectangle (axis cs:40.0, -20.0);
                        \path[,fill,
                              black!4]
                            (axis cs:35.0, -20.0)
                            rectangle (axis cs:40.0, -15.0);
                        \path[,fill,
                              black!8]
                            (axis cs:35.0, -15.0)
                            rectangle (axis cs:40.0, -10.0);
                        \path[,fill,
                              black!4]
                            (axis cs:35.0, -10.0)
                            rectangle (axis cs:40.0, -5.0);
                        \path[,fill,
                              black!6]
                            (axis cs:35.0, -5.0)
                            rectangle (axis cs:40.0, 0.0);
                        \path[,fill,
                              black!10]
                            (axis cs:35.0, 0.0)
                            rectangle (axis cs:40.0, 5.0);
                        \path[,fill,
                              black!8]
                            (axis cs:35.0, 5.0)
                            rectangle (axis cs:40.0, 10.0);
                        \path[,fill,
                              black!10]
                            (axis cs:35.0, 10.0)
                            rectangle (axis cs:40.0, 15.0);
                        \path[,fill,
                              black!22]
                            (axis cs:35.0, 15.0)
                            rectangle (axis cs:40.0, 20.0);
                        \path[,fill,
                              black!32]
                            (axis cs:35.0, 20.0)
                            rectangle (axis cs:40.0, 25.0);
                        \path[,fill,
                              black!52]
                            (axis cs:35.0, 25.0)
                            rectangle (axis cs:40.0, 30.0);
                        \path[,fill,
                              black!76]
                            (axis cs:35.0, 30.0)
                            rectangle (axis cs:40.0, 35.0);
                        \path[,fill,
                              black!86]
                            (axis cs:35.0, 35.0)
                            rectangle (axis cs:40.0, 40.0);
                        \path[,fill,
                              black!58]
                            (axis cs:35.0, 40.0)
                            rectangle (axis cs:40.0, 45.0);
                        \path[,fill,
                              black!39]
                            (axis cs:35.0, 45.0)
                            rectangle (axis cs:40.0, 50.0);
                        \path[,fill,
                              black!26]
                            (axis cs:35.0, 50.0)
                            rectangle (axis cs:40.0, 55.0);
                        \path[,fill,
                              black!18]
                            (axis cs:35.0, 55.0)
                            rectangle (axis cs:40.0, 60.0);
                        \path[,fill,
                              black!11]
                            (axis cs:35.0, 60.0)
                            rectangle (axis cs:40.0, 65.0);
                        \path[,fill,
                              black!9]
                            (axis cs:35.0, 65.0)
                            rectangle (axis cs:40.0, 70.0);
                        \path[,fill,
                              black!6]
                            (axis cs:35.0, 70.0)
                            rectangle (axis cs:40.0, 75.0);
                        \path[,fill,
                              black!9]
                            (axis cs:35.0, 75.0)
                            rectangle (axis cs:40.0, 80.0);
                        \path[,fill,
                              black!7]
                            (axis cs:35.0, 80.0)
                            rectangle (axis cs:40.0, 85.0);
                        \path[,fill,
                              black!6]
                            (axis cs:35.0, 85.0)
                            rectangle (axis cs:40.0, 90.0);
                        \path[,fill,
                              black!4]
                            (axis cs:35.0, 90.0)
                            rectangle (axis cs:40.0, 95.0);
                        \path[,fill,
                              black!4]
                            (axis cs:35.0, 95.0)
                            rectangle (axis cs:40.0, 100.0);
                        \path[,fill,
                              black!6]
                            (axis cs:35.0, 100.0)
                            rectangle (axis cs:40.0, 105.0);
                        \path[,fill,
                              black!2]
                            (axis cs:35.0, 105.0)
                            rectangle (axis cs:40.0, 110.0);
                        \path[,fill,
                              black!1]
                            (axis cs:35.0, 110.0)
                            rectangle (axis cs:40.0, 115.0);
                        \path[,fill,
                              black!2]
                            (axis cs:35.0, 115.0)
                            rectangle (axis cs:40.0, 120.0);
                        \path[,fill,
                              black!2]
                            (axis cs:35.0, 120.0)
                            rectangle (axis cs:40.0, 125.0);
                        \path[,fill,
                              black!2]
                            (axis cs:35.0, 125.0)
                            rectangle (axis cs:40.0, 130.0);
                        \path[,fill,
                              black!1]
                            (axis cs:35.0, 130.0)
                            rectangle (axis cs:40.0, 135.0);
                        \path[,fill,
                              black!2]
                            (axis cs:35.0, 135.0)
                            rectangle (axis cs:40.0, 140.0);
                        \path[,fill,
                              black!1]
                            (axis cs:35.0, 145.0)
                            rectangle (axis cs:40.0, 150.0);
                        \path[,fill,
                              black!1]
                            (axis cs:40.0, -135.0)
                            rectangle (axis cs:45.0, -130.0);
                        \path[,fill,
                              black!1]
                            (axis cs:40.0, -95.0)
                            rectangle (axis cs:45.0, -90.0);
                        \path[,fill,
                              black!1]
                            (axis cs:40.0, -85.0)
                            rectangle (axis cs:45.0, -80.0);
                        \path[,fill,
                              black!1]
                            (axis cs:40.0, -80.0)
                            rectangle (axis cs:45.0, -75.0);
                        \path[,fill,
                              black!2]
                            (axis cs:40.0, -75.0)
                            rectangle (axis cs:45.0, -70.0);
                        \path[,fill,
                              black!3]
                            (axis cs:40.0, -70.0)
                            rectangle (axis cs:45.0, -65.0);
                        \path[,fill,
                              black!2]
                            (axis cs:40.0, -65.0)
                            rectangle (axis cs:45.0, -60.0);
                        \path[,fill,
                              black!2]
                            (axis cs:40.0, -60.0)
                            rectangle (axis cs:45.0, -55.0);
                        \path[,fill,
                              black!3]
                            (axis cs:40.0, -55.0)
                            rectangle (axis cs:45.0, -50.0);
                        \path[,fill,
                              black!2]
                            (axis cs:40.0, -50.0)
                            rectangle (axis cs:45.0, -45.0);
                        \path[,fill,
                              black!3]
                            (axis cs:40.0, -45.0)
                            rectangle (axis cs:45.0, -40.0);
                        \path[,fill,
                              black!4]
                            (axis cs:40.0, -40.0)
                            rectangle (axis cs:45.0, -35.0);
                        \path[,fill,
                              black!5]
                            (axis cs:40.0, -35.0)
                            rectangle (axis cs:45.0, -30.0);
                        \path[,fill,
                              black!1]
                            (axis cs:40.0, -30.0)
                            rectangle (axis cs:45.0, -25.0);
                        \path[,fill,
                              black!2]
                            (axis cs:40.0, -25.0)
                            rectangle (axis cs:45.0, -20.0);
                        \path[,fill,
                              black!4]
                            (axis cs:40.0, -20.0)
                            rectangle (axis cs:45.0, -15.0);
                        \path[,fill,
                              black!4]
                            (axis cs:40.0, -15.0)
                            rectangle (axis cs:45.0, -10.0);
                        \path[,fill,
                              black!4]
                            (axis cs:40.0, -10.0)
                            rectangle (axis cs:45.0, -5.0);
                        \path[,fill,
                              black!5]
                            (axis cs:40.0, -5.0)
                            rectangle (axis cs:45.0, 0.0);
                        \path[,fill,
                              black!4]
                            (axis cs:40.0, 0.0)
                            rectangle (axis cs:45.0, 5.0);
                        \path[,fill,
                              black!8]
                            (axis cs:40.0, 5.0)
                            rectangle (axis cs:45.0, 10.0);
                        \path[,fill,
                              black!11]
                            (axis cs:40.0, 10.0)
                            rectangle (axis cs:45.0, 15.0);
                        \path[,fill,
                              black!8]
                            (axis cs:40.0, 15.0)
                            rectangle (axis cs:45.0, 20.0);
                        \path[,fill,
                              black!16]
                            (axis cs:40.0, 20.0)
                            rectangle (axis cs:45.0, 25.0);
                        \path[,fill,
                              black!34]
                            (axis cs:40.0, 25.0)
                            rectangle (axis cs:45.0, 30.0);
                        \path[,fill,
                              black!50]
                            (axis cs:40.0, 30.0)
                            rectangle (axis cs:45.0, 35.0);
                        \path[,fill,
                              black!87]
                            (axis cs:40.0, 35.0)
                            rectangle (axis cs:45.0, 40.0);
                        \path[,fill,
                              black!82]
                            (axis cs:40.0, 40.0)
                            rectangle (axis cs:45.0, 45.0);
                        \path[,fill,
                              black!63]
                            (axis cs:40.0, 45.0)
                            rectangle (axis cs:45.0, 50.0);
                        \path[,fill,
                              black!37]
                            (axis cs:40.0, 50.0)
                            rectangle (axis cs:45.0, 55.0);
                        \path[,fill,
                              black!27]
                            (axis cs:40.0, 55.0)
                            rectangle (axis cs:45.0, 60.0);
                        \path[,fill,
                              black!19]
                            (axis cs:40.0, 60.0)
                            rectangle (axis cs:45.0, 65.0);
                        \path[,fill,
                              black!10]
                            (axis cs:40.0, 65.0)
                            rectangle (axis cs:45.0, 70.0);
                        \path[,fill,
                              black!9]
                            (axis cs:40.0, 70.0)
                            rectangle (axis cs:45.0, 75.0);
                        \path[,fill,
                              black!10]
                            (axis cs:40.0, 75.0)
                            rectangle (axis cs:45.0, 80.0);
                        \path[,fill,
                              black!6]
                            (axis cs:40.0, 80.0)
                            rectangle (axis cs:45.0, 85.0);
                        \path[,fill,
                              black!4]
                            (axis cs:40.0, 85.0)
                            rectangle (axis cs:45.0, 90.0);
                        \path[,fill,
                              black!2]
                            (axis cs:40.0, 90.0)
                            rectangle (axis cs:45.0, 95.0);
                        \path[,fill,
                              black!7]
                            (axis cs:40.0, 95.0)
                            rectangle (axis cs:45.0, 100.0);
                        \path[,fill,
                              black!6]
                            (axis cs:40.0, 100.0)
                            rectangle (axis cs:45.0, 105.0);
                        \path[,fill,
                              black!2]
                            (axis cs:40.0, 105.0)
                            rectangle (axis cs:45.0, 110.0);
                        \path[,fill,
                              black!5]
                            (axis cs:40.0, 110.0)
                            rectangle (axis cs:45.0, 115.0);
                        \path[,fill,
                              black!5]
                            (axis cs:40.0, 115.0)
                            rectangle (axis cs:45.0, 120.0);
                        \path[,fill,
                              black!4]
                            (axis cs:40.0, 120.0)
                            rectangle (axis cs:45.0, 125.0);
                        \path[,fill,
                              black!2]
                            (axis cs:40.0, 125.0)
                            rectangle (axis cs:45.0, 130.0);
                        \path[,fill,
                              black!1]
                            (axis cs:40.0, 130.0)
                            rectangle (axis cs:45.0, 135.0);
                        \path[,fill,
                              black!2]
                            (axis cs:40.0, 135.0)
                            rectangle (axis cs:45.0, 140.0);
                        \path[,fill,
                              black!1]
                            (axis cs:40.0, 145.0)
                            rectangle (axis cs:45.0, 150.0);
                        \path[,fill,
                              black!1]
                            (axis cs:45.0, -95.0)
                            rectangle (axis cs:50.0, -90.0);
                        \path[,fill,
                              black!1]
                            (axis cs:45.0, -90.0)
                            rectangle (axis cs:50.0, -85.0);
                        \path[,fill,
                              black!2]
                            (axis cs:45.0, -75.0)
                            rectangle (axis cs:50.0, -70.0);
                        \path[,fill,
                              black!4]
                            (axis cs:45.0, -70.0)
                            rectangle (axis cs:50.0, -65.0);
                        \path[,fill,
                              black!2]
                            (axis cs:45.0, -65.0)
                            rectangle (axis cs:50.0, -60.0);
                        \path[,fill,
                              black!4]
                            (axis cs:45.0, -60.0)
                            rectangle (axis cs:50.0, -55.0);
                        \path[,fill,
                              black!2]
                            (axis cs:45.0, -55.0)
                            rectangle (axis cs:50.0, -50.0);
                        \path[,fill,
                              black!3]
                            (axis cs:45.0, -50.0)
                            rectangle (axis cs:50.0, -45.0);
                        \path[,fill,
                              black!1]
                            (axis cs:45.0, -45.0)
                            rectangle (axis cs:50.0, -40.0);
                        \path[,fill,
                              black!3]
                            (axis cs:45.0, -40.0)
                            rectangle (axis cs:50.0, -35.0);
                        \path[,fill,
                              black!4]
                            (axis cs:45.0, -35.0)
                            rectangle (axis cs:50.0, -30.0);
                        \path[,fill,
                              black!5]
                            (axis cs:45.0, -30.0)
                            rectangle (axis cs:50.0, -25.0);
                        \path[,fill,
                              black!2]
                            (axis cs:45.0, -25.0)
                            rectangle (axis cs:50.0, -20.0);
                        \path[,fill,
                              black!1]
                            (axis cs:45.0, -20.0)
                            rectangle (axis cs:50.0, -15.0);
                        \path[,fill,
                              black!4]
                            (axis cs:45.0, -15.0)
                            rectangle (axis cs:50.0, -10.0);
                        \path[,fill,
                              black!1]
                            (axis cs:45.0, -10.0)
                            rectangle (axis cs:50.0, -5.0);
                        \path[,fill,
                              black!3]
                            (axis cs:45.0, -5.0)
                            rectangle (axis cs:50.0, 0.0);
                        \path[,fill,
                              black!5]
                            (axis cs:45.0, 0.0)
                            rectangle (axis cs:50.0, 5.0);
                        \path[,fill,
                              black!6]
                            (axis cs:45.0, 5.0)
                            rectangle (axis cs:50.0, 10.0);
                        \path[,fill,
                              black!5]
                            (axis cs:45.0, 10.0)
                            rectangle (axis cs:50.0, 15.0);
                        \path[,fill,
                              black!10]
                            (axis cs:45.0, 15.0)
                            rectangle (axis cs:50.0, 20.0);
                        \path[,fill,
                              black!10]
                            (axis cs:45.0, 20.0)
                            rectangle (axis cs:50.0, 25.0);
                        \path[,fill,
                              black!15]
                            (axis cs:45.0, 25.0)
                            rectangle (axis cs:50.0, 30.0);
                        \path[,fill,
                              black!31]
                            (axis cs:45.0, 30.0)
                            rectangle (axis cs:50.0, 35.0);
                        \path[,fill,
                              black!49]
                            (axis cs:45.0, 35.0)
                            rectangle (axis cs:50.0, 40.0);
                        \path[,fill,
                              black!71]
                            (axis cs:45.0, 40.0)
                            rectangle (axis cs:50.0, 45.0);
                        \path[,fill,
                              black!81]
                            (axis cs:45.0, 45.0)
                            rectangle (axis cs:50.0, 50.0);
                        \path[,fill,
                              black!58]
                            (axis cs:45.0, 50.0)
                            rectangle (axis cs:50.0, 55.0);
                        \path[,fill,
                              black!35]
                            (axis cs:45.0, 55.0)
                            rectangle (axis cs:50.0, 60.0);
                        \path[,fill,
                              black!22]
                            (axis cs:45.0, 60.0)
                            rectangle (axis cs:50.0, 65.0);
                        \path[,fill,
                              black!14]
                            (axis cs:45.0, 65.0)
                            rectangle (axis cs:50.0, 70.0);
                        \path[,fill,
                              black!11]
                            (axis cs:45.0, 70.0)
                            rectangle (axis cs:50.0, 75.0);
                        \path[,fill,
                              black!11]
                            (axis cs:45.0, 75.0)
                            rectangle (axis cs:50.0, 80.0);
                        \path[,fill,
                              black!14]
                            (axis cs:45.0, 80.0)
                            rectangle (axis cs:50.0, 85.0);
                        \path[,fill,
                              black!8]
                            (axis cs:45.0, 85.0)
                            rectangle (axis cs:50.0, 90.0);
                        \path[,fill,
                              black!6]
                            (axis cs:45.0, 90.0)
                            rectangle (axis cs:50.0, 95.0);
                        \path[,fill,
                              black!6]
                            (axis cs:45.0, 95.0)
                            rectangle (axis cs:50.0, 100.0);
                        \path[,fill,
                              black!6]
                            (axis cs:45.0, 100.0)
                            rectangle (axis cs:50.0, 105.0);
                        \path[,fill,
                              black!6]
                            (axis cs:45.0, 105.0)
                            rectangle (axis cs:50.0, 110.0);
                        \path[,fill,
                              black!5]
                            (axis cs:45.0, 110.0)
                            rectangle (axis cs:50.0, 115.0);
                        \path[,fill,
                              black!2]
                            (axis cs:45.0, 115.0)
                            rectangle (axis cs:50.0, 120.0);
                        \path[,fill,
                              black!4]
                            (axis cs:45.0, 120.0)
                            rectangle (axis cs:50.0, 125.0);
                        \path[,fill,
                              black!2]
                            (axis cs:45.0, 125.0)
                            rectangle (axis cs:50.0, 130.0);
                        \path[,fill,
                              black!2]
                            (axis cs:45.0, 145.0)
                            rectangle (axis cs:50.0, 150.0);
                        \path[,fill,
                              black!1]
                            (axis cs:45.0, 150.0)
                            rectangle (axis cs:50.0, 155.0);
                        \path[,fill,
                              black!1]
                            (axis cs:50.0, -85.0)
                            rectangle (axis cs:55.0, -80.0);
                        \path[,fill,
                              black!1]
                            (axis cs:50.0, -80.0)
                            rectangle (axis cs:55.0, -75.0);
                        \path[,fill,
                              black!2]
                            (axis cs:50.0, -75.0)
                            rectangle (axis cs:55.0, -70.0);
                        \path[,fill,
                              black!2]
                            (axis cs:50.0, -70.0)
                            rectangle (axis cs:55.0, -65.0);
                        \path[,fill,
                              black!3]
                            (axis cs:50.0, -65.0)
                            rectangle (axis cs:55.0, -60.0);
                        \path[,fill,
                              black!3]
                            (axis cs:50.0, -60.0)
                            rectangle (axis cs:55.0, -55.0);
                        \path[,fill,
                              black!2]
                            (axis cs:50.0, -55.0)
                            rectangle (axis cs:55.0, -50.0);
                        \path[,fill,
                              black!2]
                            (axis cs:50.0, -50.0)
                            rectangle (axis cs:55.0, -45.0);
                        \path[,fill,
                              black!2]
                            (axis cs:50.0, -45.0)
                            rectangle (axis cs:55.0, -40.0);
                        \path[,fill,
                              black!2]
                            (axis cs:50.0, -40.0)
                            rectangle (axis cs:55.0, -35.0);
                        \path[,fill,
                              black!2]
                            (axis cs:50.0, -35.0)
                            rectangle (axis cs:55.0, -30.0);
                        \path[,fill,
                              black!2]
                            (axis cs:50.0, -30.0)
                            rectangle (axis cs:55.0, -25.0);
                        \path[,fill,
                              black!1]
                            (axis cs:50.0, -25.0)
                            rectangle (axis cs:55.0, -20.0);
                        \path[,fill,
                              black!1]
                            (axis cs:50.0, -20.0)
                            rectangle (axis cs:55.0, -15.0);
                        \path[,fill,
                              black!4]
                            (axis cs:50.0, -15.0)
                            rectangle (axis cs:55.0, -10.0);
                        \path[,fill,
                              black!4]
                            (axis cs:50.0, -10.0)
                            rectangle (axis cs:55.0, -5.0);
                        \path[,fill,
                              black!3]
                            (axis cs:50.0, -5.0)
                            rectangle (axis cs:55.0, 0.0);
                        \path[,fill,
                              black!6]
                            (axis cs:50.0, 0.0)
                            rectangle (axis cs:55.0, 5.0);
                        \path[,fill,
                              black!6]
                            (axis cs:50.0, 5.0)
                            rectangle (axis cs:55.0, 10.0);
                        \path[,fill,
                              black!5]
                            (axis cs:50.0, 10.0)
                            rectangle (axis cs:55.0, 15.0);
                        \path[,fill,
                              black!8]
                            (axis cs:50.0, 15.0)
                            rectangle (axis cs:55.0, 20.0);
                        \path[,fill,
                              black!11]
                            (axis cs:50.0, 20.0)
                            rectangle (axis cs:55.0, 25.0);
                        \path[,fill,
                              black!10]
                            (axis cs:50.0, 25.0)
                            rectangle (axis cs:55.0, 30.0);
                        \path[,fill,
                              black!19]
                            (axis cs:50.0, 30.0)
                            rectangle (axis cs:55.0, 35.0);
                        \path[,fill,
                              black!33]
                            (axis cs:50.0, 35.0)
                            rectangle (axis cs:55.0, 40.0);
                        \path[,fill,
                              black!48]
                            (axis cs:50.0, 40.0)
                            rectangle (axis cs:55.0, 45.0);
                        \path[,fill,
                              black!78]
                            (axis cs:50.0, 45.0)
                            rectangle (axis cs:55.0, 50.0);
                        \path[,fill,
                              black!71]
                            (axis cs:50.0, 50.0)
                            rectangle (axis cs:55.0, 55.0);
                        \path[,fill,
                              black!54]
                            (axis cs:50.0, 55.0)
                            rectangle (axis cs:55.0, 60.0);
                        \path[,fill,
                              black!34]
                            (axis cs:50.0, 60.0)
                            rectangle (axis cs:55.0, 65.0);
                        \path[,fill,
                              black!25]
                            (axis cs:50.0, 65.0)
                            rectangle (axis cs:55.0, 70.0);
                        \path[,fill,
                              black!14]
                            (axis cs:50.0, 70.0)
                            rectangle (axis cs:55.0, 75.0);
                        \path[,fill,
                              black!11]
                            (axis cs:50.0, 75.0)
                            rectangle (axis cs:55.0, 80.0);
                        \path[,fill,
                              black!8]
                            (axis cs:50.0, 80.0)
                            rectangle (axis cs:55.0, 85.0);
                        \path[,fill,
                              black!8]
                            (axis cs:50.0, 85.0)
                            rectangle (axis cs:55.0, 90.0);
                        \path[,fill,
                              black!9]
                            (axis cs:50.0, 90.0)
                            rectangle (axis cs:55.0, 95.0);
                        \path[,fill,
                              black!7]
                            (axis cs:50.0, 95.0)
                            rectangle (axis cs:55.0, 100.0);
                        \path[,fill,
                              black!6]
                            (axis cs:50.0, 100.0)
                            rectangle (axis cs:55.0, 105.0);
                        \path[,fill,
                              black!7]
                            (axis cs:50.0, 105.0)
                            rectangle (axis cs:55.0, 110.0);
                        \path[,fill,
                              black!5]
                            (axis cs:50.0, 110.0)
                            rectangle (axis cs:55.0, 115.0);
                        \path[,fill,
                              black!6]
                            (axis cs:50.0, 115.0)
                            rectangle (axis cs:55.0, 120.0);
                        \path[,fill,
                              black!6]
                            (axis cs:50.0, 120.0)
                            rectangle (axis cs:55.0, 125.0);
                        \path[,fill,
                              black!2]
                            (axis cs:50.0, 125.0)
                            rectangle (axis cs:55.0, 130.0);
                        \path[,fill,
                              black!2]
                            (axis cs:50.0, 130.0)
                            rectangle (axis cs:55.0, 135.0);
                        \path[,fill,
                              black!2]
                            (axis cs:50.0, 135.0)
                            rectangle (axis cs:55.0, 140.0);
                        \path[,fill,
                              black!1]
                            (axis cs:50.0, 140.0)
                            rectangle (axis cs:55.0, 145.0);
                        \path[,fill,
                              black!1]
                            (axis cs:50.0, 145.0)
                            rectangle (axis cs:55.0, 150.0);
                        \path[,fill,
                              black!1]
                            (axis cs:50.0, 155.0)
                            rectangle (axis cs:55.0, 160.0);
                        \path[,fill,
                              black!1]
                            (axis cs:50.0, 165.0)
                            rectangle (axis cs:55.0, 170.0);
                        \path[,fill,
                              black!1]
                            (axis cs:55.0, -175.0)
                            rectangle (axis cs:60.0, -170.0);
                        \path[,fill,
                              black!1]
                            (axis cs:55.0, -90.0)
                            rectangle (axis cs:60.0, -85.0);
                        \path[,fill,
                              black!1]
                            (axis cs:55.0, -80.0)
                            rectangle (axis cs:60.0, -75.0);
                        \path[,fill,
                              black!3]
                            (axis cs:55.0, -75.0)
                            rectangle (axis cs:60.0, -70.0);
                        \path[,fill,
                              black!2]
                            (axis cs:55.0, -70.0)
                            rectangle (axis cs:60.0, -65.0);
                        \path[,fill,
                              black!2]
                            (axis cs:55.0, -65.0)
                            rectangle (axis cs:60.0, -60.0);
                        \path[,fill,
                              black!2]
                            (axis cs:55.0, -60.0)
                            rectangle (axis cs:60.0, -55.0);
                        \path[,fill,
                              black!4]
                            (axis cs:55.0, -55.0)
                            rectangle (axis cs:60.0, -50.0);
                        \path[,fill,
                              black!1]
                            (axis cs:55.0, -50.0)
                            rectangle (axis cs:60.0, -45.0);
                        \path[,fill,
                              black!1]
                            (axis cs:55.0, -45.0)
                            rectangle (axis cs:60.0, -40.0);
                        \path[,fill,
                              black!2]
                            (axis cs:55.0, -40.0)
                            rectangle (axis cs:60.0, -35.0);
                        \path[,fill,
                              black!1]
                            (axis cs:55.0, -35.0)
                            rectangle (axis cs:60.0, -30.0);
                        \path[,fill,
                              black!1]
                            (axis cs:55.0, -30.0)
                            rectangle (axis cs:60.0, -25.0);
                        \path[,fill,
                              black!2]
                            (axis cs:55.0, -25.0)
                            rectangle (axis cs:60.0, -20.0);
                        \path[,fill,
                              black!2]
                            (axis cs:55.0, -20.0)
                            rectangle (axis cs:60.0, -15.0);
                        \path[,fill,
                              black!4]
                            (axis cs:55.0, -15.0)
                            rectangle (axis cs:60.0, -10.0);
                        \path[,fill,
                              black!2]
                            (axis cs:55.0, -10.0)
                            rectangle (axis cs:60.0, -5.0);
                        \path[,fill,
                              black!3]
                            (axis cs:55.0, -5.0)
                            rectangle (axis cs:60.0, 0.0);
                        \path[,fill,
                              black!1]
                            (axis cs:55.0, 0.0)
                            rectangle (axis cs:60.0, 5.0);
                        \path[,fill,
                              black!2]
                            (axis cs:55.0, 5.0)
                            rectangle (axis cs:60.0, 10.0);
                        \path[,fill,
                              black!3]
                            (axis cs:55.0, 10.0)
                            rectangle (axis cs:60.0, 15.0);
                        \path[,fill,
                              black!7]
                            (axis cs:55.0, 15.0)
                            rectangle (axis cs:60.0, 20.0);
                        \path[,fill,
                              black!9]
                            (axis cs:55.0, 20.0)
                            rectangle (axis cs:60.0, 25.0);
                        \path[,fill,
                              black!8]
                            (axis cs:55.0, 25.0)
                            rectangle (axis cs:60.0, 30.0);
                        \path[,fill,
                              black!11]
                            (axis cs:55.0, 30.0)
                            rectangle (axis cs:60.0, 35.0);
                        \path[,fill,
                              black!16]
                            (axis cs:55.0, 35.0)
                            rectangle (axis cs:60.0, 40.0);
                        \path[,fill,
                              black!37]
                            (axis cs:55.0, 40.0)
                            rectangle (axis cs:60.0, 45.0);
                        \path[,fill,
                              black!55]
                            (axis cs:55.0, 45.0)
                            rectangle (axis cs:60.0, 50.0);
                        \path[,fill,
                              black!66]
                            (axis cs:55.0, 50.0)
                            rectangle (axis cs:60.0, 55.0);
                        \path[,fill,
                              black!82]
                            (axis cs:55.0, 55.0)
                            rectangle (axis cs:60.0, 60.0);
                        \path[,fill,
                              black!52]
                            (axis cs:55.0, 60.0)
                            rectangle (axis cs:60.0, 65.0);
                        \path[,fill,
                              black!42]
                            (axis cs:55.0, 65.0)
                            rectangle (axis cs:60.0, 70.0);
                        \path[,fill,
                              black!22]
                            (axis cs:55.0, 70.0)
                            rectangle (axis cs:60.0, 75.0);
                        \path[,fill,
                              black!16]
                            (axis cs:55.0, 75.0)
                            rectangle (axis cs:60.0, 80.0);
                        \path[,fill,
                              black!14]
                            (axis cs:55.0, 80.0)
                            rectangle (axis cs:60.0, 85.0);
                        \path[,fill,
                              black!11]
                            (axis cs:55.0, 85.0)
                            rectangle (axis cs:60.0, 90.0);
                        \path[,fill,
                              black!6]
                            (axis cs:55.0, 90.0)
                            rectangle (axis cs:60.0, 95.0);
                        \path[,fill,
                              black!9]
                            (axis cs:55.0, 95.0)
                            rectangle (axis cs:60.0, 100.0);
                        \path[,fill,
                              black!7]
                            (axis cs:55.0, 100.0)
                            rectangle (axis cs:60.0, 105.0);
                        \path[,fill,
                              black!4]
                            (axis cs:55.0, 105.0)
                            rectangle (axis cs:60.0, 110.0);
                        \path[,fill,
                              black!3]
                            (axis cs:55.0, 110.0)
                            rectangle (axis cs:60.0, 115.0);
                        \path[,fill,
                              black!6]
                            (axis cs:55.0, 115.0)
                            rectangle (axis cs:60.0, 120.0);
                        \path[,fill,
                              black!4]
                            (axis cs:55.0, 120.0)
                            rectangle (axis cs:60.0, 125.0);
                        \path[,fill,
                              black!4]
                            (axis cs:55.0, 125.0)
                            rectangle (axis cs:60.0, 130.0);
                        \path[,fill,
                              black!1]
                            (axis cs:55.0, 130.0)
                            rectangle (axis cs:60.0, 135.0);
                        \path[,fill,
                              black!1]
                            (axis cs:55.0, 135.0)
                            rectangle (axis cs:60.0, 140.0);
                        \path[,fill,
                              black!1]
                            (axis cs:60.0, -125.0)
                            rectangle (axis cs:65.0, -120.0);
                        \path[,fill,
                              black!1]
                            (axis cs:60.0, -105.0)
                            rectangle (axis cs:65.0, -100.0);
                        \path[,fill,
                              black!1]
                            (axis cs:60.0, -95.0)
                            rectangle (axis cs:65.0, -90.0);
                        \path[,fill,
                              black!2]
                            (axis cs:60.0, -85.0)
                            rectangle (axis cs:65.0, -80.0);
                        \path[,fill,
                              black!3]
                            (axis cs:60.0, -80.0)
                            rectangle (axis cs:65.0, -75.0);
                        \path[,fill,
                              black!2]
                            (axis cs:60.0, -75.0)
                            rectangle (axis cs:65.0, -70.0);
                        \path[,fill,
                              black!2]
                            (axis cs:60.0, -65.0)
                            rectangle (axis cs:65.0, -60.0);
                        \path[,fill,
                              black!2]
                            (axis cs:60.0, -60.0)
                            rectangle (axis cs:65.0, -55.0);
                        \path[,fill,
                              black!2]
                            (axis cs:60.0, -55.0)
                            rectangle (axis cs:65.0, -50.0);
                        \path[,fill,
                              black!1]
                            (axis cs:60.0, -50.0)
                            rectangle (axis cs:65.0, -45.0);
                        \path[,fill,
                              black!2]
                            (axis cs:60.0, -45.0)
                            rectangle (axis cs:65.0, -40.0);
                        \path[,fill,
                              black!1]
                            (axis cs:60.0, -40.0)
                            rectangle (axis cs:65.0, -35.0);
                        \path[,fill,
                              black!1]
                            (axis cs:60.0, -35.0)
                            rectangle (axis cs:65.0, -30.0);
                        \path[,fill,
                              black!2]
                            (axis cs:60.0, -30.0)
                            rectangle (axis cs:65.0, -25.0);
                        \path[,fill,
                              black!2]
                            (axis cs:60.0, -20.0)
                            rectangle (axis cs:65.0, -15.0);
                        \path[,fill,
                              black!2]
                            (axis cs:60.0, -15.0)
                            rectangle (axis cs:65.0, -10.0);
                        \path[,fill,
                              black!2]
                            (axis cs:60.0, -10.0)
                            rectangle (axis cs:65.0, -5.0);
                        \path[,fill,
                              black!2]
                            (axis cs:60.0, -5.0)
                            rectangle (axis cs:65.0, 0.0);
                        \path[,fill,
                              black!4]
                            (axis cs:60.0, 0.0)
                            rectangle (axis cs:65.0, 5.0);
                        \path[,fill,
                              black!3]
                            (axis cs:60.0, 5.0)
                            rectangle (axis cs:65.0, 10.0);
                        \path[,fill,
                              black!2]
                            (axis cs:60.0, 10.0)
                            rectangle (axis cs:65.0, 15.0);
                        \path[,fill,
                              black!2]
                            (axis cs:60.0, 15.0)
                            rectangle (axis cs:65.0, 20.0);
                        \path[,fill,
                              black!7]
                            (axis cs:60.0, 20.0)
                            rectangle (axis cs:65.0, 25.0);
                        \path[,fill,
                              black!8]
                            (axis cs:60.0, 25.0)
                            rectangle (axis cs:65.0, 30.0);
                        \path[,fill,
                              black!11]
                            (axis cs:60.0, 30.0)
                            rectangle (axis cs:65.0, 35.0);
                        \path[,fill,
                              black!16]
                            (axis cs:60.0, 35.0)
                            rectangle (axis cs:65.0, 40.0);
                        \path[,fill,
                              black!26]
                            (axis cs:60.0, 40.0)
                            rectangle (axis cs:65.0, 45.0);
                        \path[,fill,
                              black!40]
                            (axis cs:60.0, 45.0)
                            rectangle (axis cs:65.0, 50.0);
                        \path[,fill,
                              black!57]
                            (axis cs:60.0, 50.0)
                            rectangle (axis cs:65.0, 55.0);
                        \path[,fill,
                              black!68]
                            (axis cs:60.0, 55.0)
                            rectangle (axis cs:65.0, 60.0);
                        \path[,fill,
                              black!66]
                            (axis cs:60.0, 60.0)
                            rectangle (axis cs:65.0, 65.0);
                        \path[,fill,
                              black!47]
                            (axis cs:60.0, 65.0)
                            rectangle (axis cs:65.0, 70.0);
                        \path[,fill,
                              black!35]
                            (axis cs:60.0, 70.0)
                            rectangle (axis cs:65.0, 75.0);
                        \path[,fill,
                              black!27]
                            (axis cs:60.0, 75.0)
                            rectangle (axis cs:65.0, 80.0);
                        \path[,fill,
                              black!20]
                            (axis cs:60.0, 80.0)
                            rectangle (axis cs:65.0, 85.0);
                        \path[,fill,
                              black!17]
                            (axis cs:60.0, 85.0)
                            rectangle (axis cs:65.0, 90.0);
                        \path[,fill,
                              black!13]
                            (axis cs:60.0, 90.0)
                            rectangle (axis cs:65.0, 95.0);
                        \path[,fill,
                              black!11]
                            (axis cs:60.0, 95.0)
                            rectangle (axis cs:65.0, 100.0);
                        \path[,fill,
                              black!8]
                            (axis cs:60.0, 100.0)
                            rectangle (axis cs:65.0, 105.0);
                        \path[,fill,
                              black!7]
                            (axis cs:60.0, 105.0)
                            rectangle (axis cs:65.0, 110.0);
                        \path[,fill,
                              black!7]
                            (axis cs:60.0, 110.0)
                            rectangle (axis cs:65.0, 115.0);
                        \path[,fill,
                              black!9]
                            (axis cs:60.0, 115.0)
                            rectangle (axis cs:65.0, 120.0);
                        \path[,fill,
                              black!4]
                            (axis cs:60.0, 120.0)
                            rectangle (axis cs:65.0, 125.0);
                        \path[,fill,
                              black!1]
                            (axis cs:60.0, 125.0)
                            rectangle (axis cs:65.0, 130.0);
                        \path[,fill,
                              black!3]
                            (axis cs:60.0, 130.0)
                            rectangle (axis cs:65.0, 135.0);
                        \path[,fill,
                              black!2]
                            (axis cs:60.0, 135.0)
                            rectangle (axis cs:65.0, 140.0);
                        \path[,fill,
                              black!1]
                            (axis cs:60.0, 140.0)
                            rectangle (axis cs:65.0, 145.0);
                        \path[,fill,
                              black!1]
                            (axis cs:65.0, -180.0)
                            rectangle (axis cs:70.0, -175.0);
                        \path[,fill,
                              black!1]
                            (axis cs:65.0, -130.0)
                            rectangle (axis cs:70.0, -125.0);
                        \path[,fill,
                              black!2]
                            (axis cs:65.0, -90.0)
                            rectangle (axis cs:70.0, -85.0);
                        \path[,fill,
                              black!2]
                            (axis cs:65.0, -85.0)
                            rectangle (axis cs:70.0, -80.0);
                        \path[,fill,
                              black!4]
                            (axis cs:65.0, -80.0)
                            rectangle (axis cs:70.0, -75.0);
                        \path[,fill,
                              black!2]
                            (axis cs:65.0, -75.0)
                            rectangle (axis cs:70.0, -70.0);
                        \path[,fill,
                              black!1]
                            (axis cs:65.0, -70.0)
                            rectangle (axis cs:70.0, -65.0);
                        \path[,fill,
                              black!2]
                            (axis cs:65.0, -65.0)
                            rectangle (axis cs:70.0, -60.0);
                        \path[,fill,
                              black!3]
                            (axis cs:65.0, -60.0)
                            rectangle (axis cs:70.0, -55.0);
                        \path[,fill,
                              black!4]
                            (axis cs:65.0, -55.0)
                            rectangle (axis cs:70.0, -50.0);
                        \path[,fill,
                              black!1]
                            (axis cs:65.0, -50.0)
                            rectangle (axis cs:70.0, -45.0);
                        \path[,fill,
                              black!2]
                            (axis cs:65.0, -45.0)
                            rectangle (axis cs:70.0, -40.0);
                        \path[,fill,
                              black!2]
                            (axis cs:65.0, -40.0)
                            rectangle (axis cs:70.0, -35.0);
                        \path[,fill,
                              black!1]
                            (axis cs:65.0, -35.0)
                            rectangle (axis cs:70.0, -30.0);
                        \path[,fill,
                              black!2]
                            (axis cs:65.0, -30.0)
                            rectangle (axis cs:70.0, -25.0);
                        \path[,fill,
                              black!1]
                            (axis cs:65.0, -25.0)
                            rectangle (axis cs:70.0, -20.0);
                        \path[,fill,
                              black!2]
                            (axis cs:65.0, -20.0)
                            rectangle (axis cs:70.0, -15.0);
                        \path[,fill,
                              black!4]
                            (axis cs:65.0, -15.0)
                            rectangle (axis cs:70.0, -10.0);
                        \path[,fill,
                              black!1]
                            (axis cs:65.0, -10.0)
                            rectangle (axis cs:70.0, -5.0);
                        \path[,fill,
                              black!3]
                            (axis cs:65.0, -5.0)
                            rectangle (axis cs:70.0, 0.0);
                        \path[,fill,
                              black!1]
                            (axis cs:65.0, 0.0)
                            rectangle (axis cs:70.0, 5.0);
                        \path[,fill,
                              black!2]
                            (axis cs:65.0, 5.0)
                            rectangle (axis cs:70.0, 10.0);
                        \path[,fill,
                              black!3]
                            (axis cs:65.0, 10.0)
                            rectangle (axis cs:70.0, 15.0);
                        \path[,fill,
                              black!4]
                            (axis cs:65.0, 15.0)
                            rectangle (axis cs:70.0, 20.0);
                        \path[,fill,
                              black!4]
                            (axis cs:65.0, 20.0)
                            rectangle (axis cs:70.0, 25.0);
                        \path[,fill,
                              black!4]
                            (axis cs:65.0, 25.0)
                            rectangle (axis cs:70.0, 30.0);
                        \path[,fill,
                              black!10]
                            (axis cs:65.0, 30.0)
                            rectangle (axis cs:70.0, 35.0);
                        \path[,fill,
                              black!11]
                            (axis cs:65.0, 35.0)
                            rectangle (axis cs:70.0, 40.0);
                        \path[,fill,
                              black!14]
                            (axis cs:65.0, 40.0)
                            rectangle (axis cs:70.0, 45.0);
                        \path[,fill,
                              black!17]
                            (axis cs:65.0, 45.0)
                            rectangle (axis cs:70.0, 50.0);
                        \path[,fill,
                              black!41]
                            (axis cs:65.0, 50.0)
                            rectangle (axis cs:70.0, 55.0);
                        \path[,fill,
                              black!60]
                            (axis cs:65.0, 55.0)
                            rectangle (axis cs:70.0, 60.0);
                        \path[,fill,
                              black!67]
                            (axis cs:65.0, 60.0)
                            rectangle (axis cs:70.0, 65.0);
                        \path[,fill,
                              black!73]
                            (axis cs:65.0, 65.0)
                            rectangle (axis cs:70.0, 70.0);
                        \path[,fill,
                              black!55]
                            (axis cs:65.0, 70.0)
                            rectangle (axis cs:70.0, 75.0);
                        \path[,fill,
                              black!36]
                            (axis cs:65.0, 75.0)
                            rectangle (axis cs:70.0, 80.0);
                        \path[,fill,
                              black!27]
                            (axis cs:65.0, 80.0)
                            rectangle (axis cs:70.0, 85.0);
                        \path[,fill,
                              black!21]
                            (axis cs:65.0, 85.0)
                            rectangle (axis cs:70.0, 90.0);
                        \path[,fill,
                              black!13]
                            (axis cs:65.0, 90.0)
                            rectangle (axis cs:70.0, 95.0);
                        \path[,fill,
                              black!13]
                            (axis cs:65.0, 95.0)
                            rectangle (axis cs:70.0, 100.0);
                        \path[,fill,
                              black!10]
                            (axis cs:65.0, 100.0)
                            rectangle (axis cs:70.0, 105.0);
                        \path[,fill,
                              black!8]
                            (axis cs:65.0, 105.0)
                            rectangle (axis cs:70.0, 110.0);
                        \path[,fill,
                              black!8]
                            (axis cs:65.0, 110.0)
                            rectangle (axis cs:70.0, 115.0);
                        \path[,fill,
                              black!6]
                            (axis cs:65.0, 115.0)
                            rectangle (axis cs:70.0, 120.0);
                        \path[,fill,
                              black!7]
                            (axis cs:65.0, 120.0)
                            rectangle (axis cs:70.0, 125.0);
                        \path[,fill,
                              black!4]
                            (axis cs:65.0, 125.0)
                            rectangle (axis cs:70.0, 130.0);
                        \path[,fill,
                              black!3]
                            (axis cs:65.0, 130.0)
                            rectangle (axis cs:70.0, 135.0);
                        \path[,fill,
                              black!2]
                            (axis cs:65.0, 135.0)
                            rectangle (axis cs:70.0, 140.0);
                        \path[,fill,
                              black!1]
                            (axis cs:65.0, 140.0)
                            rectangle (axis cs:70.0, 145.0);
                        \path[,fill,
                              black!1]
                            (axis cs:65.0, 145.0)
                            rectangle (axis cs:70.0, 150.0);
                        \path[,fill,
                              black!1]
                            (axis cs:65.0, 150.0)
                            rectangle (axis cs:70.0, 155.0);
                        \path[,fill,
                              black!1]
                            (axis cs:65.0, 155.0)
                            rectangle (axis cs:70.0, 160.0);
                        \path[,fill,
                              black!1]
                            (axis cs:65.0, 160.0)
                            rectangle (axis cs:70.0, 165.0);
                        \path[,fill,
                              black!1]
                            (axis cs:65.0, 165.0)
                            rectangle (axis cs:70.0, 170.0);
                        \path[,fill,
                              black!1]
                            (axis cs:70.0, -110.0)
                            rectangle (axis cs:75.0, -105.0);
                        \path[,fill,
                              black!1]
                            (axis cs:70.0, -95.0)
                            rectangle (axis cs:75.0, -90.0);
                        \path[,fill,
                              black!1]
                            (axis cs:70.0, -90.0)
                            rectangle (axis cs:75.0, -85.0);
                        \path[,fill,
                              black!1]
                            (axis cs:70.0, -85.0)
                            rectangle (axis cs:75.0, -80.0);
                        \path[,fill,
                              black!1]
                            (axis cs:70.0, -80.0)
                            rectangle (axis cs:75.0, -75.0);
                        \path[,fill,
                              black!2]
                            (axis cs:70.0, -75.0)
                            rectangle (axis cs:75.0, -70.0);
                        \path[,fill,
                              black!2]
                            (axis cs:70.0, -70.0)
                            rectangle (axis cs:75.0, -65.0);
                        \path[,fill,
                              black!1]
                            (axis cs:70.0, -65.0)
                            rectangle (axis cs:75.0, -60.0);
                        \path[,fill,
                              black!4]
                            (axis cs:70.0, -60.0)
                            rectangle (axis cs:75.0, -55.0);
                        \path[,fill,
                              black!1]
                            (axis cs:70.0, -55.0)
                            rectangle (axis cs:75.0, -50.0);
                        \path[,fill,
                              black!1]
                            (axis cs:70.0, -45.0)
                            rectangle (axis cs:75.0, -40.0);
                        \path[,fill,
                              black!1]
                            (axis cs:70.0, -40.0)
                            rectangle (axis cs:75.0, -35.0);
                        \path[,fill,
                              black!1]
                            (axis cs:70.0, -35.0)
                            rectangle (axis cs:75.0, -30.0);
                        \path[,fill,
                              black!1]
                            (axis cs:70.0, -30.0)
                            rectangle (axis cs:75.0, -25.0);
                        \path[,fill,
                              black!1]
                            (axis cs:70.0, -25.0)
                            rectangle (axis cs:75.0, -20.0);
                        \path[,fill,
                              black!1]
                            (axis cs:70.0, -20.0)
                            rectangle (axis cs:75.0, -15.0);
                        \path[,fill,
                              black!1]
                            (axis cs:70.0, -15.0)
                            rectangle (axis cs:75.0, -10.0);
                        \path[,fill,
                              black!1]
                            (axis cs:70.0, -10.0)
                            rectangle (axis cs:75.0, -5.0);
                        \path[,fill,
                              black!1]
                            (axis cs:70.0, -5.0)
                            rectangle (axis cs:75.0, 0.0);
                        \path[,fill,
                              black!3]
                            (axis cs:70.0, 0.0)
                            rectangle (axis cs:75.0, 5.0);
                        \path[,fill,
                              black!2]
                            (axis cs:70.0, 5.0)
                            rectangle (axis cs:75.0, 10.0);
                        \path[,fill,
                              black!3]
                            (axis cs:70.0, 10.0)
                            rectangle (axis cs:75.0, 15.0);
                        \path[,fill,
                              black!4]
                            (axis cs:70.0, 15.0)
                            rectangle (axis cs:75.0, 20.0);
                        \path[,fill,
                              black!4]
                            (axis cs:70.0, 20.0)
                            rectangle (axis cs:75.0, 25.0);
                        \path[,fill,
                              black!4]
                            (axis cs:70.0, 25.0)
                            rectangle (axis cs:75.0, 30.0);
                        \path[,fill,
                              black!4]
                            (axis cs:70.0, 30.0)
                            rectangle (axis cs:75.0, 35.0);
                        \path[,fill,
                              black!9]
                            (axis cs:70.0, 35.0)
                            rectangle (axis cs:75.0, 40.0);
                        \path[,fill,
                              black!16]
                            (axis cs:70.0, 40.0)
                            rectangle (axis cs:75.0, 45.0);
                        \path[,fill,
                              black!17]
                            (axis cs:70.0, 45.0)
                            rectangle (axis cs:75.0, 50.0);
                        \path[,fill,
                              black!22]
                            (axis cs:70.0, 50.0)
                            rectangle (axis cs:75.0, 55.0);
                        \path[,fill,
                              black!38]
                            (axis cs:70.0, 55.0)
                            rectangle (axis cs:75.0, 60.0);
                        \path[,fill,
                              black!43]
                            (axis cs:70.0, 60.0)
                            rectangle (axis cs:75.0, 65.0);
                        \path[,fill,
                              black!68]
                            (axis cs:70.0, 65.0)
                            rectangle (axis cs:75.0, 70.0);
                        \path[,fill,
                              black!76]
                            (axis cs:70.0, 70.0)
                            rectangle (axis cs:75.0, 75.0);
                        \path[,fill,
                              black!57]
                            (axis cs:70.0, 75.0)
                            rectangle (axis cs:75.0, 80.0);
                        \path[,fill,
                              black!38]
                            (axis cs:70.0, 80.0)
                            rectangle (axis cs:75.0, 85.0);
                        \path[,fill,
                              black!32]
                            (axis cs:70.0, 85.0)
                            rectangle (axis cs:75.0, 90.0);
                        \path[,fill,
                              black!19]
                            (axis cs:70.0, 90.0)
                            rectangle (axis cs:75.0, 95.0);
                        \path[,fill,
                              black!10]
                            (axis cs:70.0, 95.0)
                            rectangle (axis cs:75.0, 100.0);
                        \path[,fill,
                              black!11]
                            (axis cs:70.0, 100.0)
                            rectangle (axis cs:75.0, 105.0);
                        \path[,fill,
                              black!9]
                            (axis cs:70.0, 105.0)
                            rectangle (axis cs:75.0, 110.0);
                        \path[,fill,
                              black!8]
                            (axis cs:70.0, 110.0)
                            rectangle (axis cs:75.0, 115.0);
                        \path[,fill,
                              black!6]
                            (axis cs:70.0, 115.0)
                            rectangle (axis cs:75.0, 120.0);
                        \path[,fill,
                              black!6]
                            (axis cs:70.0, 120.0)
                            rectangle (axis cs:75.0, 125.0);
                        \path[,fill,
                              black!4]
                            (axis cs:70.0, 125.0)
                            rectangle (axis cs:75.0, 130.0);
                        \path[,fill,
                              black!2]
                            (axis cs:70.0, 130.0)
                            rectangle (axis cs:75.0, 135.0);
                        \path[,fill,
                              black!1]
                            (axis cs:70.0, 135.0)
                            rectangle (axis cs:75.0, 140.0);
                        \path[,fill,
                              black!1]
                            (axis cs:70.0, 140.0)
                            rectangle (axis cs:75.0, 145.0);
                        \path[,fill,
                              black!1]
                            (axis cs:70.0, 145.0)
                            rectangle (axis cs:75.0, 150.0);
                        \path[,fill,
                              black!1]
                            (axis cs:75.0, -175.0)
                            rectangle (axis cs:80.0, -170.0);
                        \path[,fill,
                              black!1]
                            (axis cs:75.0, -170.0)
                            rectangle (axis cs:80.0, -165.0);
                        \path[,fill,
                              black!1]
                            (axis cs:75.0, -110.0)
                            rectangle (axis cs:80.0, -105.0);
                        \path[,fill,
                              black!1]
                            (axis cs:75.0, -100.0)
                            rectangle (axis cs:80.0, -95.0);
                        \path[,fill,
                              black!3]
                            (axis cs:75.0, -95.0)
                            rectangle (axis cs:80.0, -90.0);
                        \path[,fill,
                              black!4]
                            (axis cs:75.0, -90.0)
                            rectangle (axis cs:80.0, -85.0);
                        \path[,fill,
                              black!4]
                            (axis cs:75.0, -85.0)
                            rectangle (axis cs:80.0, -80.0);
                        \path[,fill,
                              black!1]
                            (axis cs:75.0, -80.0)
                            rectangle (axis cs:80.0, -75.0);
                        \path[,fill,
                              black!2]
                            (axis cs:75.0, -75.0)
                            rectangle (axis cs:80.0, -70.0);
                        \path[,fill,
                              black!2]
                            (axis cs:75.0, -70.0)
                            rectangle (axis cs:80.0, -65.0);
                        \path[,fill,
                              black!2]
                            (axis cs:75.0, -65.0)
                            rectangle (axis cs:80.0, -60.0);
                        \path[,fill,
                              black!2]
                            (axis cs:75.0, -60.0)
                            rectangle (axis cs:80.0, -55.0);
                        \path[,fill,
                              black!1]
                            (axis cs:75.0, -50.0)
                            rectangle (axis cs:80.0, -45.0);
                        \path[,fill,
                              black!1]
                            (axis cs:75.0, -45.0)
                            rectangle (axis cs:80.0, -40.0);
                        \path[,fill,
                              black!1]
                            (axis cs:75.0, -40.0)
                            rectangle (axis cs:80.0, -35.0);
                        \path[,fill,
                              black!1]
                            (axis cs:75.0, -35.0)
                            rectangle (axis cs:80.0, -30.0);
                        \path[,fill,
                              black!2]
                            (axis cs:75.0, -30.0)
                            rectangle (axis cs:80.0, -25.0);
                        \path[,fill,
                              black!1]
                            (axis cs:75.0, -25.0)
                            rectangle (axis cs:80.0, -20.0);
                        \path[,fill,
                              black!1]
                            (axis cs:75.0, -10.0)
                            rectangle (axis cs:80.0, -5.0);
                        \path[,fill,
                              black!1]
                            (axis cs:75.0, -5.0)
                            rectangle (axis cs:80.0, 0.0);
                        \path[,fill,
                              black!1]
                            (axis cs:75.0, 5.0)
                            rectangle (axis cs:80.0, 10.0);
                        \path[,fill,
                              black!2]
                            (axis cs:75.0, 10.0)
                            rectangle (axis cs:80.0, 15.0);
                        \path[,fill,
                              black!2]
                            (axis cs:75.0, 15.0)
                            rectangle (axis cs:80.0, 20.0);
                        \path[,fill,
                              black!2]
                            (axis cs:75.0, 20.0)
                            rectangle (axis cs:80.0, 25.0);
                        \path[,fill,
                              black!1]
                            (axis cs:75.0, 25.0)
                            rectangle (axis cs:80.0, 30.0);
                        \path[,fill,
                              black!4]
                            (axis cs:75.0, 30.0)
                            rectangle (axis cs:80.0, 35.0);
                        \path[,fill,
                              black!5]
                            (axis cs:75.0, 35.0)
                            rectangle (axis cs:80.0, 40.0);
                        \path[,fill,
                              black!7]
                            (axis cs:75.0, 40.0)
                            rectangle (axis cs:80.0, 45.0);
                        \path[,fill,
                              black!14]
                            (axis cs:75.0, 45.0)
                            rectangle (axis cs:80.0, 50.0);
                        \path[,fill,
                              black!17]
                            (axis cs:75.0, 50.0)
                            rectangle (axis cs:80.0, 55.0);
                        \path[,fill,
                              black!24]
                            (axis cs:75.0, 55.0)
                            rectangle (axis cs:80.0, 60.0);
                        \path[,fill,
                              black!34]
                            (axis cs:75.0, 60.0)
                            rectangle (axis cs:80.0, 65.0);
                        \path[,fill,
                              black!44]
                            (axis cs:75.0, 65.0)
                            rectangle (axis cs:80.0, 70.0);
                        \path[,fill,
                              black!73]
                            (axis cs:75.0, 70.0)
                            rectangle (axis cs:80.0, 75.0);
                        \path[,fill,
                              black!76]
                            (axis cs:75.0, 75.0)
                            rectangle (axis cs:80.0, 80.0);
                        \path[,fill,
                              black!55]
                            (axis cs:75.0, 80.0)
                            rectangle (axis cs:80.0, 85.0);
                        \path[,fill,
                              black!41]
                            (axis cs:75.0, 85.0)
                            rectangle (axis cs:80.0, 90.0);
                        \path[,fill,
                              black!22]
                            (axis cs:75.0, 90.0)
                            rectangle (axis cs:80.0, 95.0);
                        \path[,fill,
                              black!19]
                            (axis cs:75.0, 95.0)
                            rectangle (axis cs:80.0, 100.0);
                        \path[,fill,
                              black!13]
                            (axis cs:75.0, 100.0)
                            rectangle (axis cs:80.0, 105.0);
                        \path[,fill,
                              black!15]
                            (axis cs:75.0, 105.0)
                            rectangle (axis cs:80.0, 110.0);
                        \path[,fill,
                              black!11]
                            (axis cs:75.0, 110.0)
                            rectangle (axis cs:80.0, 115.0);
                        \path[,fill,
                              black!8]
                            (axis cs:75.0, 115.0)
                            rectangle (axis cs:80.0, 120.0);
                        \path[,fill,
                              black!9]
                            (axis cs:75.0, 120.0)
                            rectangle (axis cs:80.0, 125.0);
                        \path[,fill,
                              black!4]
                            (axis cs:75.0, 125.0)
                            rectangle (axis cs:80.0, 130.0);
                        \path[,fill,
                              black!1]
                            (axis cs:75.0, 130.0)
                            rectangle (axis cs:80.0, 135.0);
                        \path[,fill,
                              black!1]
                            (axis cs:75.0, 135.0)
                            rectangle (axis cs:80.0, 140.0);
                        \path[,fill,
                              black!2]
                            (axis cs:75.0, 140.0)
                            rectangle (axis cs:80.0, 145.0);
                        \path[,fill,
                              black!2]
                            (axis cs:75.0, 145.0)
                            rectangle (axis cs:80.0, 150.0);
                        \path[,fill,
                              black!1]
                            (axis cs:75.0, 150.0)
                            rectangle (axis cs:80.0, 155.0);
                        \path[,fill,
                              black!2]
                            (axis cs:75.0, 155.0)
                            rectangle (axis cs:80.0, 160.0);
                        \path[,fill,
                              black!1]
                            (axis cs:80.0, -180.0)
                            rectangle (axis cs:85.0, -175.0);
                        \path[,fill,
                              black!1]
                            (axis cs:80.0, -170.0)
                            rectangle (axis cs:85.0, -165.0);
                        \path[,fill,
                              black!1]
                            (axis cs:80.0, -135.0)
                            rectangle (axis cs:85.0, -130.0);
                        \path[,fill,
                              black!1]
                            (axis cs:80.0, -120.0)
                            rectangle (axis cs:85.0, -115.0);
                        \path[,fill,
                              black!1]
                            (axis cs:80.0, -110.0)
                            rectangle (axis cs:85.0, -105.0);
                        \path[,fill,
                              black!1]
                            (axis cs:80.0, -105.0)
                            rectangle (axis cs:85.0, -100.0);
                        \path[,fill,
                              black!2]
                            (axis cs:80.0, -100.0)
                            rectangle (axis cs:85.0, -95.0);
                        \path[,fill,
                              black!2]
                            (axis cs:80.0, -95.0)
                            rectangle (axis cs:85.0, -90.0);
                        \path[,fill,
                              black!1]
                            (axis cs:80.0, -90.0)
                            rectangle (axis cs:85.0, -85.0);
                        \path[,fill,
                              black!4]
                            (axis cs:80.0, -85.0)
                            rectangle (axis cs:85.0, -80.0);
                        \path[,fill,
                              black!1]
                            (axis cs:80.0, -80.0)
                            rectangle (axis cs:85.0, -75.0);
                        \path[,fill,
                              black!1]
                            (axis cs:80.0, -75.0)
                            rectangle (axis cs:85.0, -70.0);
                        \path[,fill,
                              black!1]
                            (axis cs:80.0, -70.0)
                            rectangle (axis cs:85.0, -65.0);
                        \path[,fill,
                              black!2]
                            (axis cs:80.0, -65.0)
                            rectangle (axis cs:85.0, -60.0);
                        \path[,fill,
                              black!1]
                            (axis cs:80.0, -55.0)
                            rectangle (axis cs:85.0, -50.0);
                        \path[,fill,
                              black!1]
                            (axis cs:80.0, -50.0)
                            rectangle (axis cs:85.0, -45.0);
                        \path[,fill,
                              black!1]
                            (axis cs:80.0, -40.0)
                            rectangle (axis cs:85.0, -35.0);
                        \path[,fill,
                              black!1]
                            (axis cs:80.0, -35.0)
                            rectangle (axis cs:85.0, -30.0);
                        \path[,fill,
                              black!1]
                            (axis cs:80.0, -30.0)
                            rectangle (axis cs:85.0, -25.0);
                        \path[,fill,
                              black!1]
                            (axis cs:80.0, -20.0)
                            rectangle (axis cs:85.0, -15.0);
                        \path[,fill,
                              black!1]
                            (axis cs:80.0, 0.0)
                            rectangle (axis cs:85.0, 5.0);
                        \path[,fill,
                              black!1]
                            (axis cs:80.0, 5.0)
                            rectangle (axis cs:85.0, 10.0);
                        \path[,fill,
                              black!1]
                            (axis cs:80.0, 10.0)
                            rectangle (axis cs:85.0, 15.0);
                        \path[,fill,
                              black!2]
                            (axis cs:80.0, 15.0)
                            rectangle (axis cs:85.0, 20.0);
                        \path[,fill,
                              black!1]
                            (axis cs:80.0, 20.0)
                            rectangle (axis cs:85.0, 25.0);
                        \path[,fill,
                              black!2]
                            (axis cs:80.0, 25.0)
                            rectangle (axis cs:85.0, 30.0);
                        \path[,fill,
                              black!2]
                            (axis cs:80.0, 30.0)
                            rectangle (axis cs:85.0, 35.0);
                        \path[,fill,
                              black!4]
                            (axis cs:80.0, 35.0)
                            rectangle (axis cs:85.0, 40.0);
                        \path[,fill,
                              black!9]
                            (axis cs:80.0, 40.0)
                            rectangle (axis cs:85.0, 45.0);
                        \path[,fill,
                              black!9]
                            (axis cs:80.0, 45.0)
                            rectangle (axis cs:85.0, 50.0);
                        \path[,fill,
                              black!16]
                            (axis cs:80.0, 50.0)
                            rectangle (axis cs:85.0, 55.0);
                        \path[,fill,
                              black!14]
                            (axis cs:80.0, 55.0)
                            rectangle (axis cs:85.0, 60.0);
                        \path[,fill,
                              black!22]
                            (axis cs:80.0, 60.0)
                            rectangle (axis cs:85.0, 65.0);
                        \path[,fill,
                              black!30]
                            (axis cs:80.0, 65.0)
                            rectangle (axis cs:85.0, 70.0);
                        \path[,fill,
                              black!39]
                            (axis cs:80.0, 70.0)
                            rectangle (axis cs:85.0, 75.0);
                        \path[,fill,
                              black!77]
                            (axis cs:80.0, 75.0)
                            rectangle (axis cs:85.0, 80.0);
                        \path[,fill,
                              black!76]
                            (axis cs:80.0, 80.0)
                            rectangle (axis cs:85.0, 85.0);
                        \path[,fill,
                              black!58]
                            (axis cs:80.0, 85.0)
                            rectangle (axis cs:85.0, 90.0);
                        \path[,fill,
                              black!46]
                            (axis cs:80.0, 90.0)
                            rectangle (axis cs:85.0, 95.0);
                        \path[,fill,
                              black!24]
                            (axis cs:80.0, 95.0)
                            rectangle (axis cs:85.0, 100.0);
                        \path[,fill,
                              black!18]
                            (axis cs:80.0, 100.0)
                            rectangle (axis cs:85.0, 105.0);
                        \path[,fill,
                              black!15]
                            (axis cs:80.0, 105.0)
                            rectangle (axis cs:85.0, 110.0);
                        \path[,fill,
                              black!12]
                            (axis cs:80.0, 110.0)
                            rectangle (axis cs:85.0, 115.0);
                        \path[,fill,
                              black!11]
                            (axis cs:80.0, 115.0)
                            rectangle (axis cs:85.0, 120.0);
                        \path[,fill,
                              black!8]
                            (axis cs:80.0, 120.0)
                            rectangle (axis cs:85.0, 125.0);
                        \path[,fill,
                              black!6]
                            (axis cs:80.0, 125.0)
                            rectangle (axis cs:85.0, 130.0);
                        \path[,fill,
                              black!6]
                            (axis cs:80.0, 130.0)
                            rectangle (axis cs:85.0, 135.0);
                        \path[,fill,
                              black!3]
                            (axis cs:80.0, 135.0)
                            rectangle (axis cs:85.0, 140.0);
                        \path[,fill,
                              black!3]
                            (axis cs:80.0, 140.0)
                            rectangle (axis cs:85.0, 145.0);
                        \path[,fill,
                              black!2]
                            (axis cs:80.0, 145.0)
                            rectangle (axis cs:85.0, 150.0);
                        \path[,fill,
                              black!3]
                            (axis cs:80.0, 155.0)
                            rectangle (axis cs:85.0, 160.0);
                        \path[,fill,
                              black!1]
                            (axis cs:80.0, 160.0)
                            rectangle (axis cs:85.0, 165.0);
                        \path[,fill,
                              black!1]
                            (axis cs:80.0, 165.0)
                            rectangle (axis cs:85.0, 170.0);
                        \path[,fill,
                              black!1]
                            (axis cs:85.0, -175.0)
                            rectangle (axis cs:90.0, -170.0);
                        \path[,fill,
                              black!1]
                            (axis cs:85.0, -165.0)
                            rectangle (axis cs:90.0, -160.0);
                        \path[,fill,
                              black!1]
                            (axis cs:85.0, -160.0)
                            rectangle (axis cs:90.0, -155.0);
                        \path[,fill,
                              black!1]
                            (axis cs:85.0, -140.0)
                            rectangle (axis cs:90.0, -135.0);
                        \path[,fill,
                              black!1]
                            (axis cs:85.0, -110.0)
                            rectangle (axis cs:90.0, -105.0);
                        \path[,fill,
                              black!1]
                            (axis cs:85.0, -105.0)
                            rectangle (axis cs:90.0, -100.0);
                        \path[,fill,
                              black!1]
                            (axis cs:85.0, -100.0)
                            rectangle (axis cs:90.0, -95.0);
                        \path[,fill,
                              black!2]
                            (axis cs:85.0, -95.0)
                            rectangle (axis cs:90.0, -90.0);
                        \path[,fill,
                              black!2]
                            (axis cs:85.0, -90.0)
                            rectangle (axis cs:90.0, -85.0);
                        \path[,fill,
                              black!3]
                            (axis cs:85.0, -85.0)
                            rectangle (axis cs:90.0, -80.0);
                        \path[,fill,
                              black!2]
                            (axis cs:85.0, -80.0)
                            rectangle (axis cs:90.0, -75.0);
                        \path[,fill,
                              black!2]
                            (axis cs:85.0, -75.0)
                            rectangle (axis cs:90.0, -70.0);
                        \path[,fill,
                              black!1]
                            (axis cs:85.0, -70.0)
                            rectangle (axis cs:90.0, -65.0);
                        \path[,fill,
                              black!1]
                            (axis cs:85.0, -65.0)
                            rectangle (axis cs:90.0, -60.0);
                        \path[,fill,
                              black!1]
                            (axis cs:85.0, -60.0)
                            rectangle (axis cs:90.0, -55.0);
                        \path[,fill,
                              black!1]
                            (axis cs:85.0, -45.0)
                            rectangle (axis cs:90.0, -40.0);
                        \path[,fill,
                              black!1]
                            (axis cs:85.0, -35.0)
                            rectangle (axis cs:90.0, -30.0);
                        \path[,fill,
                              black!1]
                            (axis cs:85.0, -30.0)
                            rectangle (axis cs:90.0, -25.0);
                        \path[,fill,
                              black!1]
                            (axis cs:85.0, -20.0)
                            rectangle (axis cs:90.0, -15.0);
                        \path[,fill,
                              black!1]
                            (axis cs:85.0, -10.0)
                            rectangle (axis cs:90.0, -5.0);
                        \path[,fill,
                              black!1]
                            (axis cs:85.0, 0.0)
                            rectangle (axis cs:90.0, 5.0);
                        \path[,fill,
                              black!1]
                            (axis cs:85.0, 5.0)
                            rectangle (axis cs:90.0, 10.0);
                        \path[,fill,
                              black!1]
                            (axis cs:85.0, 15.0)
                            rectangle (axis cs:90.0, 20.0);
                        \path[,fill,
                              black!1]
                            (axis cs:85.0, 20.0)
                            rectangle (axis cs:90.0, 25.0);
                        \path[,fill,
                              black!1]
                            (axis cs:85.0, 25.0)
                            rectangle (axis cs:90.0, 30.0);
                        \path[,fill,
                              black!2]
                            (axis cs:85.0, 30.0)
                            rectangle (axis cs:90.0, 35.0);
                        \path[,fill,
                              black!2]
                            (axis cs:85.0, 35.0)
                            rectangle (axis cs:90.0, 40.0);
                        \path[,fill,
                              black!5]
                            (axis cs:85.0, 40.0)
                            rectangle (axis cs:90.0, 45.0);
                        \path[,fill,
                              black!8]
                            (axis cs:85.0, 45.0)
                            rectangle (axis cs:90.0, 50.0);
                        \path[,fill,
                              black!10]
                            (axis cs:85.0, 50.0)
                            rectangle (axis cs:90.0, 55.0);
                        \path[,fill,
                              black!14]
                            (axis cs:85.0, 55.0)
                            rectangle (axis cs:90.0, 60.0);
                        \path[,fill,
                              black!16]
                            (axis cs:85.0, 60.0)
                            rectangle (axis cs:90.0, 65.0);
                        \path[,fill,
                              black!26]
                            (axis cs:85.0, 65.0)
                            rectangle (axis cs:90.0, 70.0);
                        \path[,fill,
                              black!31]
                            (axis cs:85.0, 70.0)
                            rectangle (axis cs:90.0, 75.0);
                        \path[,fill,
                              black!42]
                            (axis cs:85.0, 75.0)
                            rectangle (axis cs:90.0, 80.0);
                        \path[,fill,
                              black!62]
                            (axis cs:85.0, 80.0)
                            rectangle (axis cs:90.0, 85.0);
                        \path[,fill,
                              black!79]
                            (axis cs:85.0, 85.0)
                            rectangle (axis cs:90.0, 90.0);
                        \path[,fill,
                              black!61]
                            (axis cs:85.0, 90.0)
                            rectangle (axis cs:90.0, 95.0);
                        \path[,fill,
                              black!38]
                            (axis cs:85.0, 95.0)
                            rectangle (axis cs:90.0, 100.0);
                        \path[,fill,
                              black!26]
                            (axis cs:85.0, 100.0)
                            rectangle (axis cs:90.0, 105.0);
                        \path[,fill,
                              black!20]
                            (axis cs:85.0, 105.0)
                            rectangle (axis cs:90.0, 110.0);
                        \path[,fill,
                              black!13]
                            (axis cs:85.0, 110.0)
                            rectangle (axis cs:90.0, 115.0);
                        \path[,fill,
                              black!12]
                            (axis cs:85.0, 115.0)
                            rectangle (axis cs:90.0, 120.0);
                        \path[,fill,
                              black!10]
                            (axis cs:85.0, 120.0)
                            rectangle (axis cs:90.0, 125.0);
                        \path[,fill,
                              black!6]
                            (axis cs:85.0, 125.0)
                            rectangle (axis cs:90.0, 130.0);
                        \path[,fill,
                              black!5]
                            (axis cs:85.0, 130.0)
                            rectangle (axis cs:90.0, 135.0);
                        \path[,fill,
                              black!2]
                            (axis cs:85.0, 135.0)
                            rectangle (axis cs:90.0, 140.0);
                        \path[,fill,
                              black!2]
                            (axis cs:85.0, 140.0)
                            rectangle (axis cs:90.0, 145.0);
                        \path[,fill,
                              black!1]
                            (axis cs:85.0, 145.0)
                            rectangle (axis cs:90.0, 150.0);
                        \path[,fill,
                              black!1]
                            (axis cs:85.0, 150.0)
                            rectangle (axis cs:90.0, 155.0);
                        \path[,fill,
                              black!1]
                            (axis cs:85.0, 165.0)
                            rectangle (axis cs:90.0, 170.0);
                        \path[,fill,
                              black!1]
                            (axis cs:90.0, -180.0)
                            rectangle (axis cs:95.0, -175.0);
                        \path[,fill,
                              black!1]
                            (axis cs:90.0, -175.0)
                            rectangle (axis cs:95.0, -170.0);
                        \path[,fill,
                              black!1]
                            (axis cs:90.0, -170.0)
                            rectangle (axis cs:95.0, -165.0);
                        \path[,fill,
                              black!1]
                            (axis cs:90.0, -165.0)
                            rectangle (axis cs:95.0, -160.0);
                        \path[,fill,
                              black!1]
                            (axis cs:90.0, -160.0)
                            rectangle (axis cs:95.0, -155.0);
                        \path[,fill,
                              black!1]
                            (axis cs:90.0, -155.0)
                            rectangle (axis cs:95.0, -150.0);
                        \path[,fill,
                              black!1]
                            (axis cs:90.0, -150.0)
                            rectangle (axis cs:95.0, -145.0);
                        \path[,fill,
                              black!1]
                            (axis cs:90.0, -145.0)
                            rectangle (axis cs:95.0, -140.0);
                        \path[,fill,
                              black!1]
                            (axis cs:90.0, -140.0)
                            rectangle (axis cs:95.0, -135.0);
                        \path[,fill,
                              black!1]
                            (axis cs:90.0, -135.0)
                            rectangle (axis cs:95.0, -130.0);
                        \path[,fill,
                              black!1]
                            (axis cs:90.0, -130.0)
                            rectangle (axis cs:95.0, -125.0);
                        \path[,fill,
                              black!1]
                            (axis cs:90.0, -125.0)
                            rectangle (axis cs:95.0, -120.0);
                        \path[,fill,
                              black!2]
                            (axis cs:90.0, -120.0)
                            rectangle (axis cs:95.0, -115.0);
                        \path[,fill,
                              black!1]
                            (axis cs:90.0, -115.0)
                            rectangle (axis cs:95.0, -110.0);
                        \path[,fill,
                              black!1]
                            (axis cs:90.0, -110.0)
                            rectangle (axis cs:95.0, -105.0);
                        \path[,fill,
                              black!2]
                            (axis cs:90.0, -105.0)
                            rectangle (axis cs:95.0, -100.0);
                        \path[,fill,
                              black!2]
                            (axis cs:90.0, -100.0)
                            rectangle (axis cs:95.0, -95.0);
                        \path[,fill,
                              black!3]
                            (axis cs:90.0, -95.0)
                            rectangle (axis cs:95.0, -90.0);
                        \path[,fill,
                              black!2]
                            (axis cs:90.0, -90.0)
                            rectangle (axis cs:95.0, -85.0);
                        \path[,fill,
                              black!2]
                            (axis cs:90.0, -85.0)
                            rectangle (axis cs:95.0, -80.0);
                        \path[,fill,
                              black!2]
                            (axis cs:90.0, -80.0)
                            rectangle (axis cs:95.0, -75.0);
                        \path[,fill,
                              black!1]
                            (axis cs:90.0, -65.0)
                            rectangle (axis cs:95.0, -60.0);
                        \path[,fill,
                              black!1]
                            (axis cs:90.0, -60.0)
                            rectangle (axis cs:95.0, -55.0);
                        \path[,fill,
                              black!1]
                            (axis cs:90.0, -55.0)
                            rectangle (axis cs:95.0, -50.0);
                        \path[,fill,
                              black!1]
                            (axis cs:90.0, -50.0)
                            rectangle (axis cs:95.0, -45.0);
                        \path[,fill,
                              black!1]
                            (axis cs:90.0, 15.0)
                            rectangle (axis cs:95.0, 20.0);
                        \path[,fill,
                              black!1]
                            (axis cs:90.0, 20.0)
                            rectangle (axis cs:95.0, 25.0);
                        \path[,fill,
                              black!2]
                            (axis cs:90.0, 35.0)
                            rectangle (axis cs:95.0, 40.0);
                        \path[,fill,
                              black!4]
                            (axis cs:90.0, 40.0)
                            rectangle (axis cs:95.0, 45.0);
                        \path[,fill,
                              black!5]
                            (axis cs:90.0, 45.0)
                            rectangle (axis cs:95.0, 50.0);
                        \path[,fill,
                              black!6]
                            (axis cs:90.0, 50.0)
                            rectangle (axis cs:95.0, 55.0);
                        \path[,fill,
                              black!10]
                            (axis cs:90.0, 55.0)
                            rectangle (axis cs:95.0, 60.0);
                        \path[,fill,
                              black!17]
                            (axis cs:90.0, 60.0)
                            rectangle (axis cs:95.0, 65.0);
                        \path[,fill,
                              black!15]
                            (axis cs:90.0, 65.0)
                            rectangle (axis cs:95.0, 70.0);
                        \path[,fill,
                              black!21]
                            (axis cs:90.0, 70.0)
                            rectangle (axis cs:95.0, 75.0);
                        \path[,fill,
                              black!30]
                            (axis cs:90.0, 75.0)
                            rectangle (axis cs:95.0, 80.0);
                        \path[,fill,
                              black!45]
                            (axis cs:90.0, 80.0)
                            rectangle (axis cs:95.0, 85.0);
                        \path[,fill,
                              black!59]
                            (axis cs:90.0, 85.0)
                            rectangle (axis cs:95.0, 90.0);
                        \path[,fill,
                              black!80]
                            (axis cs:90.0, 90.0)
                            rectangle (axis cs:95.0, 95.0);
                        \path[,fill,
                              black!56]
                            (axis cs:90.0, 95.0)
                            rectangle (axis cs:95.0, 100.0);
                        \path[,fill,
                              black!43]
                            (axis cs:90.0, 100.0)
                            rectangle (axis cs:95.0, 105.0);
                        \path[,fill,
                              black!32]
                            (axis cs:90.0, 105.0)
                            rectangle (axis cs:95.0, 110.0);
                        \path[,fill,
                              black!22]
                            (axis cs:90.0, 110.0)
                            rectangle (axis cs:95.0, 115.0);
                        \path[,fill,
                              black!19]
                            (axis cs:90.0, 115.0)
                            rectangle (axis cs:95.0, 120.0);
                        \path[,fill,
                              black!13]
                            (axis cs:90.0, 120.0)
                            rectangle (axis cs:95.0, 125.0);
                        \path[,fill,
                              black!8]
                            (axis cs:90.0, 125.0)
                            rectangle (axis cs:95.0, 130.0);
                        \path[,fill,
                              black!10]
                            (axis cs:90.0, 130.0)
                            rectangle (axis cs:95.0, 135.0);
                        \path[,fill,
                              black!7]
                            (axis cs:90.0, 135.0)
                            rectangle (axis cs:95.0, 140.0);
                        \path[,fill,
                              black!4]
                            (axis cs:90.0, 140.0)
                            rectangle (axis cs:95.0, 145.0);
                        \path[,fill,
                              black!1]
                            (axis cs:90.0, 145.0)
                            rectangle (axis cs:95.0, 150.0);
                        \path[,fill,
                              black!2]
                            (axis cs:90.0, 150.0)
                            rectangle (axis cs:95.0, 155.0);
                        \path[,fill,
                              black!1]
                            (axis cs:90.0, 160.0)
                            rectangle (axis cs:95.0, 165.0);
                        \path[,fill,
                              black!1]
                            (axis cs:90.0, 165.0)
                            rectangle (axis cs:95.0, 170.0);
                        \path[,fill,
                              black!1]
                            (axis cs:90.0, 175.0)
                            rectangle (axis cs:95.0, 180.0);
                        \path[,fill,
                              black!1]
                            (axis cs:95.0, -180.0)
                            rectangle (axis cs:100.0, -175.0);
                        \path[,fill,
                              black!2]
                            (axis cs:95.0, -155.0)
                            rectangle (axis cs:100.0, -150.0);
                        \path[,fill,
                              black!1]
                            (axis cs:95.0, -150.0)
                            rectangle (axis cs:100.0, -145.0);
                        \path[,fill,
                              black!1]
                            (axis cs:95.0, -145.0)
                            rectangle (axis cs:100.0, -140.0);
                        \path[,fill,
                              black!1]
                            (axis cs:95.0, -135.0)
                            rectangle (axis cs:100.0, -130.0);
                        \path[,fill,
                              black!1]
                            (axis cs:95.0, -130.0)
                            rectangle (axis cs:100.0, -125.0);
                        \path[,fill,
                              black!1]
                            (axis cs:95.0, -125.0)
                            rectangle (axis cs:100.0, -120.0);
                        \path[,fill,
                              black!1]
                            (axis cs:95.0, -120.0)
                            rectangle (axis cs:100.0, -115.0);
                        \path[,fill,
                              black!3]
                            (axis cs:95.0, -110.0)
                            rectangle (axis cs:100.0, -105.0);
                        \path[,fill,
                              black!1]
                            (axis cs:95.0, -105.0)
                            rectangle (axis cs:100.0, -100.0);
                        \path[,fill,
                              black!4]
                            (axis cs:95.0, -100.0)
                            rectangle (axis cs:100.0, -95.0);
                        \path[,fill,
                              black!2]
                            (axis cs:95.0, -95.0)
                            rectangle (axis cs:100.0, -90.0);
                        \path[,fill,
                              black!1]
                            (axis cs:95.0, -90.0)
                            rectangle (axis cs:100.0, -85.0);
                        \path[,fill,
                              black!1]
                            (axis cs:95.0, -85.0)
                            rectangle (axis cs:100.0, -80.0);
                        \path[,fill,
                              black!1]
                            (axis cs:95.0, -80.0)
                            rectangle (axis cs:100.0, -75.0);
                        \path[,fill,
                              black!2]
                            (axis cs:95.0, -70.0)
                            rectangle (axis cs:100.0, -65.0);
                        \path[,fill,
                              black!1]
                            (axis cs:95.0, -50.0)
                            rectangle (axis cs:100.0, -45.0);
                        \path[,fill,
                              black!1]
                            (axis cs:95.0, -5.0)
                            rectangle (axis cs:100.0, 0.0);
                        \path[,fill,
                              black!1]
                            (axis cs:95.0, 15.0)
                            rectangle (axis cs:100.0, 20.0);
                        \path[,fill,
                              black!2]
                            (axis cs:95.0, 25.0)
                            rectangle (axis cs:100.0, 30.0);
                        \path[,fill,
                              black!1]
                            (axis cs:95.0, 30.0)
                            rectangle (axis cs:100.0, 35.0);
                        \path[,fill,
                              black!3]
                            (axis cs:95.0, 35.0)
                            rectangle (axis cs:100.0, 40.0);
                        \path[,fill,
                              black!2]
                            (axis cs:95.0, 40.0)
                            rectangle (axis cs:100.0, 45.0);
                        \path[,fill,
                              black!5]
                            (axis cs:95.0, 45.0)
                            rectangle (axis cs:100.0, 50.0);
                        \path[,fill,
                              black!8]
                            (axis cs:95.0, 50.0)
                            rectangle (axis cs:100.0, 55.0);
                        \path[,fill,
                              black!11]
                            (axis cs:95.0, 55.0)
                            rectangle (axis cs:100.0, 60.0);
                        \path[,fill,
                              black!13]
                            (axis cs:95.0, 60.0)
                            rectangle (axis cs:100.0, 65.0);
                        \path[,fill,
                              black!11]
                            (axis cs:95.0, 65.0)
                            rectangle (axis cs:100.0, 70.0);
                        \path[,fill,
                              black!17]
                            (axis cs:95.0, 70.0)
                            rectangle (axis cs:100.0, 75.0);
                        \path[,fill,
                              black!23]
                            (axis cs:95.0, 75.0)
                            rectangle (axis cs:100.0, 80.0);
                        \path[,fill,
                              black!26]
                            (axis cs:95.0, 80.0)
                            rectangle (axis cs:100.0, 85.0);
                        \path[,fill,
                              black!45]
                            (axis cs:95.0, 85.0)
                            rectangle (axis cs:100.0, 90.0);
                        \path[,fill,
                              black!58]
                            (axis cs:95.0, 90.0)
                            rectangle (axis cs:100.0, 95.0);
                        \path[,fill,
                              black!81]
                            (axis cs:95.0, 95.0)
                            rectangle (axis cs:100.0, 100.0);
                        \path[,fill,
                              black!61]
                            (axis cs:95.0, 100.0)
                            rectangle (axis cs:100.0, 105.0);
                        \path[,fill,
                              black!41]
                            (axis cs:95.0, 105.0)
                            rectangle (axis cs:100.0, 110.0);
                        \path[,fill,
                              black!31]
                            (axis cs:95.0, 110.0)
                            rectangle (axis cs:100.0, 115.0);
                        \path[,fill,
                              black!25]
                            (axis cs:95.0, 115.0)
                            rectangle (axis cs:100.0, 120.0);
                        \path[,fill,
                              black!18]
                            (axis cs:95.0, 120.0)
                            rectangle (axis cs:100.0, 125.0);
                        \path[,fill,
                              black!11]
                            (axis cs:95.0, 125.0)
                            rectangle (axis cs:100.0, 130.0);
                        \path[,fill,
                              black!10]
                            (axis cs:95.0, 130.0)
                            rectangle (axis cs:100.0, 135.0);
                        \path[,fill,
                              black!5]
                            (axis cs:95.0, 135.0)
                            rectangle (axis cs:100.0, 140.0);
                        \path[,fill,
                              black!4]
                            (axis cs:95.0, 140.0)
                            rectangle (axis cs:100.0, 145.0);
                        \path[,fill,
                              black!2]
                            (axis cs:95.0, 145.0)
                            rectangle (axis cs:100.0, 150.0);
                        \path[,fill,
                              black!2]
                            (axis cs:95.0, 150.0)
                            rectangle (axis cs:100.0, 155.0);
                        \path[,fill,
                              black!1]
                            (axis cs:95.0, 155.0)
                            rectangle (axis cs:100.0, 160.0);
                        \path[,fill,
                              black!2]
                            (axis cs:95.0, 160.0)
                            rectangle (axis cs:100.0, 165.0);
                        \path[,fill,
                              black!1]
                            (axis cs:95.0, 165.0)
                            rectangle (axis cs:100.0, 170.0);
                        \path[,fill,
                              black!1]
                            (axis cs:95.0, 170.0)
                            rectangle (axis cs:100.0, 175.0);
                        \path[,fill,
                              black!1]
                            (axis cs:95.0, 175.0)
                            rectangle (axis cs:100.0, 180.0);
                        \path[,fill,
                              black!2]
                            (axis cs:100.0, -180.0)
                            rectangle (axis cs:105.0, -175.0);
                        \path[,fill,
                              black!1]
                            (axis cs:100.0, -170.0)
                            rectangle (axis cs:105.0, -165.0);
                        \path[,fill,
                              black!2]
                            (axis cs:100.0, -155.0)
                            rectangle (axis cs:105.0, -150.0);
                        \path[,fill,
                              black!1]
                            (axis cs:100.0, -150.0)
                            rectangle (axis cs:105.0, -145.0);
                        \path[,fill,
                              black!1]
                            (axis cs:100.0, -145.0)
                            rectangle (axis cs:105.0, -140.0);
                        \path[,fill,
                              black!1]
                            (axis cs:100.0, -140.0)
                            rectangle (axis cs:105.0, -135.0);
                        \path[,fill,
                              black!1]
                            (axis cs:100.0, -130.0)
                            rectangle (axis cs:105.0, -125.0);
                        \path[,fill,
                              black!1]
                            (axis cs:100.0, -125.0)
                            rectangle (axis cs:105.0, -120.0);
                        \path[,fill,
                              black!1]
                            (axis cs:100.0, -120.0)
                            rectangle (axis cs:105.0, -115.0);
                        \path[,fill,
                              black!1]
                            (axis cs:100.0, -115.0)
                            rectangle (axis cs:105.0, -110.0);
                        \path[,fill,
                              black!2]
                            (axis cs:100.0, -110.0)
                            rectangle (axis cs:105.0, -105.0);
                        \path[,fill,
                              black!1]
                            (axis cs:100.0, -105.0)
                            rectangle (axis cs:105.0, -100.0);
                        \path[,fill,
                              black!3]
                            (axis cs:100.0, -100.0)
                            rectangle (axis cs:105.0, -95.0);
                        \path[,fill,
                              black!1]
                            (axis cs:100.0, -95.0)
                            rectangle (axis cs:105.0, -90.0);
                        \path[,fill,
                              black!1]
                            (axis cs:100.0, -65.0)
                            rectangle (axis cs:105.0, -60.0);
                        \path[,fill,
                              black!1]
                            (axis cs:100.0, 15.0)
                            rectangle (axis cs:105.0, 20.0);
                        \path[,fill,
                              black!2]
                            (axis cs:100.0, 25.0)
                            rectangle (axis cs:105.0, 30.0);
                        \path[,fill,
                              black!2]
                            (axis cs:100.0, 35.0)
                            rectangle (axis cs:105.0, 40.0);
                        \path[,fill,
                              black!2]
                            (axis cs:100.0, 40.0)
                            rectangle (axis cs:105.0, 45.0);
                        \path[,fill,
                              black!4]
                            (axis cs:100.0, 45.0)
                            rectangle (axis cs:105.0, 50.0);
                        \path[,fill,
                              black!7]
                            (axis cs:100.0, 50.0)
                            rectangle (axis cs:105.0, 55.0);
                        \path[,fill,
                              black!7]
                            (axis cs:100.0, 55.0)
                            rectangle (axis cs:105.0, 60.0);
                        \path[,fill,
                              black!9]
                            (axis cs:100.0, 60.0)
                            rectangle (axis cs:105.0, 65.0);
                        \path[,fill,
                              black!8]
                            (axis cs:100.0, 65.0)
                            rectangle (axis cs:105.0, 70.0);
                        \path[,fill,
                              black!13]
                            (axis cs:100.0, 70.0)
                            rectangle (axis cs:105.0, 75.0);
                        \path[,fill,
                              black!15]
                            (axis cs:100.0, 75.0)
                            rectangle (axis cs:105.0, 80.0);
                        \path[,fill,
                              black!20]
                            (axis cs:100.0, 80.0)
                            rectangle (axis cs:105.0, 85.0);
                        \path[,fill,
                              black!31]
                            (axis cs:100.0, 85.0)
                            rectangle (axis cs:105.0, 90.0);
                        \path[,fill,
                              black!40]
                            (axis cs:100.0, 90.0)
                            rectangle (axis cs:105.0, 95.0);
                        \path[,fill,
                              black!53]
                            (axis cs:100.0, 95.0)
                            rectangle (axis cs:105.0, 100.0);
                        \path[,fill,
                              black!63]
                            (axis cs:100.0, 100.0)
                            rectangle (axis cs:105.0, 105.0);
                        \path[,fill,
                              black!60]
                            (axis cs:100.0, 105.0)
                            rectangle (axis cs:105.0, 110.0);
                        \path[,fill,
                              black!48]
                            (axis cs:100.0, 110.0)
                            rectangle (axis cs:105.0, 115.0);
                        \path[,fill,
                              black!28]
                            (axis cs:100.0, 115.0)
                            rectangle (axis cs:105.0, 120.0);
                        \path[,fill,
                              black!24]
                            (axis cs:100.0, 120.0)
                            rectangle (axis cs:105.0, 125.0);
                        \path[,fill,
                              black!15]
                            (axis cs:100.0, 125.0)
                            rectangle (axis cs:105.0, 130.0);
                        \path[,fill,
                              black!13]
                            (axis cs:100.0, 130.0)
                            rectangle (axis cs:105.0, 135.0);
                        \path[,fill,
                              black!13]
                            (axis cs:100.0, 135.0)
                            rectangle (axis cs:105.0, 140.0);
                        \path[,fill,
                              black!5]
                            (axis cs:100.0, 140.0)
                            rectangle (axis cs:105.0, 145.0);
                        \path[,fill,
                              black!4]
                            (axis cs:100.0, 145.0)
                            rectangle (axis cs:105.0, 150.0);
                        \path[,fill,
                              black!4]
                            (axis cs:100.0, 150.0)
                            rectangle (axis cs:105.0, 155.0);
                        \path[,fill,
                              black!1]
                            (axis cs:100.0, 155.0)
                            rectangle (axis cs:105.0, 160.0);
                        \path[,fill,
                              black!1]
                            (axis cs:100.0, 160.0)
                            rectangle (axis cs:105.0, 165.0);
                        \path[,fill,
                              black!1]
                            (axis cs:100.0, 165.0)
                            rectangle (axis cs:105.0, 170.0);
                        \path[,fill,
                              black!1]
                            (axis cs:100.0, 170.0)
                            rectangle (axis cs:105.0, 175.0);
                        \path[,fill,
                              black!1]
                            (axis cs:100.0, 175.0)
                            rectangle (axis cs:105.0, 180.0);
                        \path[,fill,
                              black!1]
                            (axis cs:105.0, -180.0)
                            rectangle (axis cs:110.0, -175.0);
                        \path[,fill,
                              black!1]
                            (axis cs:105.0, -170.0)
                            rectangle (axis cs:110.0, -165.0);
                        \path[,fill,
                              black!1]
                            (axis cs:105.0, -160.0)
                            rectangle (axis cs:110.0, -155.0);
                        \path[,fill,
                              black!1]
                            (axis cs:105.0, -150.0)
                            rectangle (axis cs:110.0, -145.0);
                        \path[,fill,
                              black!1]
                            (axis cs:105.0, -145.0)
                            rectangle (axis cs:110.0, -140.0);
                        \path[,fill,
                              black!2]
                            (axis cs:105.0, -130.0)
                            rectangle (axis cs:110.0, -125.0);
                        \path[,fill,
                              black!1]
                            (axis cs:105.0, -125.0)
                            rectangle (axis cs:110.0, -120.0);
                        \path[,fill,
                              black!3]
                            (axis cs:105.0, -120.0)
                            rectangle (axis cs:110.0, -115.0);
                        \path[,fill,
                              black!2]
                            (axis cs:105.0, -115.0)
                            rectangle (axis cs:110.0, -110.0);
                        \path[,fill,
                              black!2]
                            (axis cs:105.0, -110.0)
                            rectangle (axis cs:110.0, -105.0);
                        \path[,fill,
                              black!3]
                            (axis cs:105.0, -105.0)
                            rectangle (axis cs:110.0, -100.0);
                        \path[,fill,
                              black!3]
                            (axis cs:105.0, -100.0)
                            rectangle (axis cs:110.0, -95.0);
                        \path[,fill,
                              black!3]
                            (axis cs:105.0, -95.0)
                            rectangle (axis cs:110.0, -90.0);
                        \path[,fill,
                              black!1]
                            (axis cs:105.0, -90.0)
                            rectangle (axis cs:110.0, -85.0);
                        \path[,fill,
                              black!1]
                            (axis cs:105.0, -80.0)
                            rectangle (axis cs:110.0, -75.0);
                        \path[,fill,
                              black!1]
                            (axis cs:105.0, -75.0)
                            rectangle (axis cs:110.0, -70.0);
                        \path[,fill,
                              black!1]
                            (axis cs:105.0, 25.0)
                            rectangle (axis cs:110.0, 30.0);
                        \path[,fill,
                              black!2]
                            (axis cs:105.0, 30.0)
                            rectangle (axis cs:110.0, 35.0);
                        \path[,fill,
                              black!1]
                            (axis cs:105.0, 35.0)
                            rectangle (axis cs:110.0, 40.0);
                        \path[,fill,
                              black!3]
                            (axis cs:105.0, 40.0)
                            rectangle (axis cs:110.0, 45.0);
                        \path[,fill,
                              black!3]
                            (axis cs:105.0, 45.0)
                            rectangle (axis cs:110.0, 50.0);
                        \path[,fill,
                              black!4]
                            (axis cs:105.0, 50.0)
                            rectangle (axis cs:110.0, 55.0);
                        \path[,fill,
                              black!6]
                            (axis cs:105.0, 55.0)
                            rectangle (axis cs:110.0, 60.0);
                        \path[,fill,
                              black!9]
                            (axis cs:105.0, 60.0)
                            rectangle (axis cs:110.0, 65.0);
                        \path[,fill,
                              black!10]
                            (axis cs:105.0, 65.0)
                            rectangle (axis cs:110.0, 70.0);
                        \path[,fill,
                              black!11]
                            (axis cs:105.0, 70.0)
                            rectangle (axis cs:110.0, 75.0);
                        \path[,fill,
                              black!12]
                            (axis cs:105.0, 75.0)
                            rectangle (axis cs:110.0, 80.0);
                        \path[,fill,
                              black!16]
                            (axis cs:105.0, 80.0)
                            rectangle (axis cs:110.0, 85.0);
                        \path[,fill,
                              black!17]
                            (axis cs:105.0, 85.0)
                            rectangle (axis cs:110.0, 90.0);
                        \path[,fill,
                              black!25]
                            (axis cs:105.0, 90.0)
                            rectangle (axis cs:110.0, 95.0);
                        \path[,fill,
                              black!43]
                            (axis cs:105.0, 95.0)
                            rectangle (axis cs:110.0, 100.0);
                        \path[,fill,
                              black!59]
                            (axis cs:105.0, 100.0)
                            rectangle (axis cs:110.0, 105.0);
                        \path[,fill,
                              black!70]
                            (axis cs:105.0, 105.0)
                            rectangle (axis cs:110.0, 110.0);
                        \path[,fill,
                              black!76]
                            (axis cs:105.0, 110.0)
                            rectangle (axis cs:110.0, 115.0);
                        \path[,fill,
                              black!44]
                            (axis cs:105.0, 115.0)
                            rectangle (axis cs:110.0, 120.0);
                        \path[,fill,
                              black!36]
                            (axis cs:105.0, 120.0)
                            rectangle (axis cs:110.0, 125.0);
                        \path[,fill,
                              black!24]
                            (axis cs:105.0, 125.0)
                            rectangle (axis cs:110.0, 130.0);
                        \path[,fill,
                              black!19]
                            (axis cs:105.0, 130.0)
                            rectangle (axis cs:110.0, 135.0);
                        \path[,fill,
                              black!11]
                            (axis cs:105.0, 135.0)
                            rectangle (axis cs:110.0, 140.0);
                        \path[,fill,
                              black!8]
                            (axis cs:105.0, 140.0)
                            rectangle (axis cs:110.0, 145.0);
                        \path[,fill,
                              black!7]
                            (axis cs:105.0, 145.0)
                            rectangle (axis cs:110.0, 150.0);
                        \path[,fill,
                              black!6]
                            (axis cs:105.0, 150.0)
                            rectangle (axis cs:110.0, 155.0);
                        \path[,fill,
                              black!4]
                            (axis cs:105.0, 155.0)
                            rectangle (axis cs:110.0, 160.0);
                        \path[,fill,
                              black!1]
                            (axis cs:105.0, 160.0)
                            rectangle (axis cs:110.0, 165.0);
                        \path[,fill,
                              black!1]
                            (axis cs:105.0, 165.0)
                            rectangle (axis cs:110.0, 170.0);
                        \path[,fill,
                              black!2]
                            (axis cs:105.0, 170.0)
                            rectangle (axis cs:110.0, 175.0);
                        \path[,fill,
                              black!2]
                            (axis cs:110.0, -180.0)
                            rectangle (axis cs:115.0, -175.0);
                        \path[,fill,
                              black!1]
                            (axis cs:110.0, -175.0)
                            rectangle (axis cs:115.0, -170.0);
                        \path[,fill,
                              black!1]
                            (axis cs:110.0, -165.0)
                            rectangle (axis cs:115.0, -160.0);
                        \path[,fill,
                              black!1]
                            (axis cs:110.0, -160.0)
                            rectangle (axis cs:115.0, -155.0);
                        \path[,fill,
                              black!1]
                            (axis cs:110.0, -155.0)
                            rectangle (axis cs:115.0, -150.0);
                        \path[,fill,
                              black!1]
                            (axis cs:110.0, -150.0)
                            rectangle (axis cs:115.0, -145.0);
                        \path[,fill,
                              black!1]
                            (axis cs:110.0, -145.0)
                            rectangle (axis cs:115.0, -140.0);
                        \path[,fill,
                              black!2]
                            (axis cs:110.0, -135.0)
                            rectangle (axis cs:115.0, -130.0);
                        \path[,fill,
                              black!2]
                            (axis cs:110.0, -125.0)
                            rectangle (axis cs:115.0, -120.0);
                        \path[,fill,
                              black!3]
                            (axis cs:110.0, -120.0)
                            rectangle (axis cs:115.0, -115.0);
                        \path[,fill,
                              black!2]
                            (axis cs:110.0, -115.0)
                            rectangle (axis cs:115.0, -110.0);
                        \path[,fill,
                              black!2]
                            (axis cs:110.0, -110.0)
                            rectangle (axis cs:115.0, -105.0);
                        \path[,fill,
                              black!3]
                            (axis cs:110.0, -105.0)
                            rectangle (axis cs:115.0, -100.0);
                        \path[,fill,
                              black!2]
                            (axis cs:110.0, -100.0)
                            rectangle (axis cs:115.0, -95.0);
                        \path[,fill,
                              black!2]
                            (axis cs:110.0, -95.0)
                            rectangle (axis cs:115.0, -90.0);
                        \path[,fill,
                              black!2]
                            (axis cs:110.0, -90.0)
                            rectangle (axis cs:115.0, -85.0);
                        \path[,fill,
                              black!1]
                            (axis cs:110.0, -85.0)
                            rectangle (axis cs:115.0, -80.0);
                        \path[,fill,
                              black!1]
                            (axis cs:110.0, -75.0)
                            rectangle (axis cs:115.0, -70.0);
                        \path[,fill,
                              black!1]
                            (axis cs:110.0, -70.0)
                            rectangle (axis cs:115.0, -65.0);
                        \path[,fill,
                              black!1]
                            (axis cs:110.0, -30.0)
                            rectangle (axis cs:115.0, -25.0);
                        \path[,fill,
                              black!1]
                            (axis cs:110.0, -15.0)
                            rectangle (axis cs:115.0, -10.0);
                        \path[,fill,
                              black!1]
                            (axis cs:110.0, 20.0)
                            rectangle (axis cs:115.0, 25.0);
                        \path[,fill,
                              black!1]
                            (axis cs:110.0, 30.0)
                            rectangle (axis cs:115.0, 35.0);
                        \path[,fill,
                              black!2]
                            (axis cs:110.0, 35.0)
                            rectangle (axis cs:115.0, 40.0);
                        \path[,fill,
                              black!2]
                            (axis cs:110.0, 40.0)
                            rectangle (axis cs:115.0, 45.0);
                        \path[,fill,
                              black!3]
                            (axis cs:110.0, 45.0)
                            rectangle (axis cs:115.0, 50.0);
                        \path[,fill,
                              black!4]
                            (axis cs:110.0, 50.0)
                            rectangle (axis cs:115.0, 55.0);
                        \path[,fill,
                              black!6]
                            (axis cs:110.0, 55.0)
                            rectangle (axis cs:115.0, 60.0);
                        \path[,fill,
                              black!8]
                            (axis cs:110.0, 60.0)
                            rectangle (axis cs:115.0, 65.0);
                        \path[,fill,
                              black!5]
                            (axis cs:110.0, 65.0)
                            rectangle (axis cs:115.0, 70.0);
                        \path[,fill,
                              black!8]
                            (axis cs:110.0, 70.0)
                            rectangle (axis cs:115.0, 75.0);
                        \path[,fill,
                              black!14]
                            (axis cs:110.0, 75.0)
                            rectangle (axis cs:115.0, 80.0);
                        \path[,fill,
                              black!12]
                            (axis cs:110.0, 80.0)
                            rectangle (axis cs:115.0, 85.0);
                        \path[,fill,
                              black!15]
                            (axis cs:110.0, 85.0)
                            rectangle (axis cs:115.0, 90.0);
                        \path[,fill,
                              black!22]
                            (axis cs:110.0, 90.0)
                            rectangle (axis cs:115.0, 95.0);
                        \path[,fill,
                              black!24]
                            (axis cs:110.0, 95.0)
                            rectangle (axis cs:115.0, 100.0);
                        \path[,fill,
                              black!38]
                            (axis cs:110.0, 100.0)
                            rectangle (axis cs:115.0, 105.0);
                        \path[,fill,
                              black!55]
                            (axis cs:110.0, 105.0)
                            rectangle (axis cs:115.0, 110.0);
                        \path[,fill,
                              black!75]
                            (axis cs:110.0, 110.0)
                            rectangle (axis cs:115.0, 115.0);
                        \path[,fill,
                              black!74]
                            (axis cs:110.0, 115.0)
                            rectangle (axis cs:115.0, 120.0);
                        \path[,fill,
                              black!55]
                            (axis cs:110.0, 120.0)
                            rectangle (axis cs:115.0, 125.0);
                        \path[,fill,
                              black!43]
                            (axis cs:110.0, 125.0)
                            rectangle (axis cs:115.0, 130.0);
                        \path[,fill,
                              black!25]
                            (axis cs:110.0, 130.0)
                            rectangle (axis cs:115.0, 135.0);
                        \path[,fill,
                              black!15]
                            (axis cs:110.0, 135.0)
                            rectangle (axis cs:115.0, 140.0);
                        \path[,fill,
                              black!9]
                            (axis cs:110.0, 140.0)
                            rectangle (axis cs:115.0, 145.0);
                        \path[,fill,
                              black!5]
                            (axis cs:110.0, 145.0)
                            rectangle (axis cs:115.0, 150.0);
                        \path[,fill,
                              black!2]
                            (axis cs:110.0, 150.0)
                            rectangle (axis cs:115.0, 155.0);
                        \path[,fill,
                              black!4]
                            (axis cs:110.0, 155.0)
                            rectangle (axis cs:115.0, 160.0);
                        \path[,fill,
                              black!5]
                            (axis cs:110.0, 160.0)
                            rectangle (axis cs:115.0, 165.0);
                        \path[,fill,
                              black!2]
                            (axis cs:110.0, 165.0)
                            rectangle (axis cs:115.0, 170.0);
                        \path[,fill,
                              black!1]
                            (axis cs:110.0, 170.0)
                            rectangle (axis cs:115.0, 175.0);
                        \path[,fill,
                              black!1]
                            (axis cs:110.0, 175.0)
                            rectangle (axis cs:115.0, 180.0);
                        \path[,fill,
                              black!1]
                            (axis cs:115.0, -180.0)
                            rectangle (axis cs:120.0, -175.0);
                        \path[,fill,
                              black!1]
                            (axis cs:115.0, -175.0)
                            rectangle (axis cs:120.0, -170.0);
                        \path[,fill,
                              black!1]
                            (axis cs:115.0, -170.0)
                            rectangle (axis cs:120.0, -165.0);
                        \path[,fill,
                              black!1]
                            (axis cs:115.0, -165.0)
                            rectangle (axis cs:120.0, -160.0);
                        \path[,fill,
                              black!1]
                            (axis cs:115.0, -160.0)
                            rectangle (axis cs:120.0, -155.0);
                        \path[,fill,
                              black!2]
                            (axis cs:115.0, -150.0)
                            rectangle (axis cs:120.0, -145.0);
                        \path[,fill,
                              black!1]
                            (axis cs:115.0, -145.0)
                            rectangle (axis cs:120.0, -140.0);
                        \path[,fill,
                              black!3]
                            (axis cs:115.0, -140.0)
                            rectangle (axis cs:120.0, -135.0);
                        \path[,fill,
                              black!1]
                            (axis cs:115.0, -135.0)
                            rectangle (axis cs:120.0, -130.0);
                        \path[,fill,
                              black!1]
                            (axis cs:115.0, -130.0)
                            rectangle (axis cs:120.0, -125.0);
                        \path[,fill,
                              black!3]
                            (axis cs:115.0, -125.0)
                            rectangle (axis cs:120.0, -120.0);
                        \path[,fill,
                              black!2]
                            (axis cs:115.0, -120.0)
                            rectangle (axis cs:120.0, -115.0);
                        \path[,fill,
                              black!1]
                            (axis cs:115.0, -115.0)
                            rectangle (axis cs:120.0, -110.0);
                        \path[,fill,
                              black!4]
                            (axis cs:115.0, -110.0)
                            rectangle (axis cs:120.0, -105.0);
                        \path[,fill,
                              black!1]
                            (axis cs:115.0, -105.0)
                            rectangle (axis cs:120.0, -100.0);
                        \path[,fill,
                              black!1]
                            (axis cs:115.0, -100.0)
                            rectangle (axis cs:120.0, -95.0);
                        \path[,fill,
                              black!4]
                            (axis cs:115.0, -95.0)
                            rectangle (axis cs:120.0, -90.0);
                        \path[,fill,
                              black!1]
                            (axis cs:115.0, -90.0)
                            rectangle (axis cs:120.0, -85.0);
                        \path[,fill,
                              black!1]
                            (axis cs:115.0, -80.0)
                            rectangle (axis cs:120.0, -75.0);
                        \path[,fill,
                              black!1]
                            (axis cs:115.0, -20.0)
                            rectangle (axis cs:120.0, -15.0);
                        \path[,fill,
                              black!1]
                            (axis cs:115.0, 0.0)
                            rectangle (axis cs:120.0, 5.0);
                        \path[,fill,
                              black!3]
                            (axis cs:115.0, 35.0)
                            rectangle (axis cs:120.0, 40.0);
                        \path[,fill,
                              black!3]
                            (axis cs:115.0, 40.0)
                            rectangle (axis cs:120.0, 45.0);
                        \path[,fill,
                              black!4]
                            (axis cs:115.0, 45.0)
                            rectangle (axis cs:120.0, 50.0);
                        \path[,fill,
                              black!5]
                            (axis cs:115.0, 50.0)
                            rectangle (axis cs:120.0, 55.0);
                        \path[,fill,
                              black!6]
                            (axis cs:115.0, 55.0)
                            rectangle (axis cs:120.0, 60.0);
                        \path[,fill,
                              black!3]
                            (axis cs:115.0, 60.0)
                            rectangle (axis cs:120.0, 65.0);
                        \path[,fill,
                              black!6]
                            (axis cs:115.0, 65.0)
                            rectangle (axis cs:120.0, 70.0);
                        \path[,fill,
                              black!7]
                            (axis cs:115.0, 70.0)
                            rectangle (axis cs:120.0, 75.0);
                        \path[,fill,
                              black!10]
                            (axis cs:115.0, 75.0)
                            rectangle (axis cs:120.0, 80.0);
                        \path[,fill,
                              black!8]
                            (axis cs:115.0, 80.0)
                            rectangle (axis cs:120.0, 85.0);
                        \path[,fill,
                              black!11]
                            (axis cs:115.0, 85.0)
                            rectangle (axis cs:120.0, 90.0);
                        \path[,fill,
                              black!16]
                            (axis cs:115.0, 90.0)
                            rectangle (axis cs:120.0, 95.0);
                        \path[,fill,
                              black!17]
                            (axis cs:115.0, 95.0)
                            rectangle (axis cs:120.0, 100.0);
                        \path[,fill,
                              black!26]
                            (axis cs:115.0, 100.0)
                            rectangle (axis cs:120.0, 105.0);
                        \path[,fill,
                              black!42]
                            (axis cs:115.0, 105.0)
                            rectangle (axis cs:120.0, 110.0);
                        \path[,fill,
                              black!61]
                            (axis cs:115.0, 110.0)
                            rectangle (axis cs:120.0, 115.0);
                        \path[,fill,
                              black!85]
                            (axis cs:115.0, 115.0)
                            rectangle (axis cs:120.0, 120.0);
                        \path[,fill,
                              black!62]
                            (axis cs:115.0, 120.0)
                            rectangle (axis cs:120.0, 125.0);
                        \path[,fill,
                              black!49]
                            (axis cs:115.0, 125.0)
                            rectangle (axis cs:120.0, 130.0);
                        \path[,fill,
                              black!37]
                            (axis cs:115.0, 130.0)
                            rectangle (axis cs:120.0, 135.0);
                        \path[,fill,
                              black!24]
                            (axis cs:115.0, 135.0)
                            rectangle (axis cs:120.0, 140.0);
                        \path[,fill,
                              black!13]
                            (axis cs:115.0, 140.0)
                            rectangle (axis cs:120.0, 145.0);
                        \path[,fill,
                              black!9]
                            (axis cs:115.0, 145.0)
                            rectangle (axis cs:120.0, 150.0);
                        \path[,fill,
                              black!3]
                            (axis cs:115.0, 150.0)
                            rectangle (axis cs:120.0, 155.0);
                        \path[,fill,
                              black!5]
                            (axis cs:115.0, 155.0)
                            rectangle (axis cs:120.0, 160.0);
                        \path[,fill,
                              black!2]
                            (axis cs:115.0, 160.0)
                            rectangle (axis cs:120.0, 165.0);
                        \path[,fill,
                              black!6]
                            (axis cs:115.0, 165.0)
                            rectangle (axis cs:120.0, 170.0);
                        \path[,fill,
                              black!2]
                            (axis cs:115.0, 170.0)
                            rectangle (axis cs:120.0, 175.0);
                        \path[,fill,
                              black!2]
                            (axis cs:115.0, 175.0)
                            rectangle (axis cs:120.0, 180.0);
                        \path[,fill,
                              black!2]
                            (axis cs:120.0, -180.0)
                            rectangle (axis cs:125.0, -175.0);
                        \path[,fill,
                              black!2]
                            (axis cs:120.0, -175.0)
                            rectangle (axis cs:125.0, -170.0);
                        \path[,fill,
                              black!5]
                            (axis cs:120.0, -170.0)
                            rectangle (axis cs:125.0, -165.0);
                        \path[,fill,
                              black!1]
                            (axis cs:120.0, -165.0)
                            rectangle (axis cs:125.0, -160.0);
                        \path[,fill,
                              black!5]
                            (axis cs:120.0, -160.0)
                            rectangle (axis cs:125.0, -155.0);
                        \path[,fill,
                              black!1]
                            (axis cs:120.0, -155.0)
                            rectangle (axis cs:125.0, -150.0);
                        \path[,fill,
                              black!2]
                            (axis cs:120.0, -150.0)
                            rectangle (axis cs:125.0, -145.0);
                        \path[,fill,
                              black!1]
                            (axis cs:120.0, -145.0)
                            rectangle (axis cs:125.0, -140.0);
                        \path[,fill,
                              black!2]
                            (axis cs:120.0, -140.0)
                            rectangle (axis cs:125.0, -135.0);
                        \path[,fill,
                              black!2]
                            (axis cs:120.0, -135.0)
                            rectangle (axis cs:125.0, -130.0);
                        \path[,fill,
                              black!1]
                            (axis cs:120.0, -130.0)
                            rectangle (axis cs:125.0, -125.0);
                        \path[,fill,
                              black!3]
                            (axis cs:120.0, -125.0)
                            rectangle (axis cs:125.0, -120.0);
                        \path[,fill,
                              black!1]
                            (axis cs:120.0, -120.0)
                            rectangle (axis cs:125.0, -115.0);
                        \path[,fill,
                              black!2]
                            (axis cs:120.0, -115.0)
                            rectangle (axis cs:125.0, -110.0);
                        \path[,fill,
                              black!5]
                            (axis cs:120.0, -110.0)
                            rectangle (axis cs:125.0, -105.0);
                        \path[,fill,
                              black!2]
                            (axis cs:120.0, -105.0)
                            rectangle (axis cs:125.0, -100.0);
                        \path[,fill,
                              black!1]
                            (axis cs:120.0, -100.0)
                            rectangle (axis cs:125.0, -95.0);
                        \path[,fill,
                              black!2]
                            (axis cs:120.0, -95.0)
                            rectangle (axis cs:125.0, -90.0);
                        \path[,fill,
                              black!1]
                            (axis cs:120.0, -45.0)
                            rectangle (axis cs:125.0, -40.0);
                        \path[,fill,
                              black!1]
                            (axis cs:120.0, 30.0)
                            rectangle (axis cs:125.0, 35.0);
                        \path[,fill,
                              black!1]
                            (axis cs:120.0, 35.0)
                            rectangle (axis cs:125.0, 40.0);
                        \path[,fill,
                              black!1]
                            (axis cs:120.0, 40.0)
                            rectangle (axis cs:125.0, 45.0);
                        \path[,fill,
                              black!2]
                            (axis cs:120.0, 45.0)
                            rectangle (axis cs:125.0, 50.0);
                        \path[,fill,
                              black!5]
                            (axis cs:120.0, 50.0)
                            rectangle (axis cs:125.0, 55.0);
                        \path[,fill,
                              black!6]
                            (axis cs:120.0, 55.0)
                            rectangle (axis cs:125.0, 60.0);
                        \path[,fill,
                              black!5]
                            (axis cs:120.0, 60.0)
                            rectangle (axis cs:125.0, 65.0);
                        \path[,fill,
                              black!5]
                            (axis cs:120.0, 65.0)
                            rectangle (axis cs:125.0, 70.0);
                        \path[,fill,
                              black!5]
                            (axis cs:120.0, 70.0)
                            rectangle (axis cs:125.0, 75.0);
                        \path[,fill,
                              black!7]
                            (axis cs:120.0, 75.0)
                            rectangle (axis cs:125.0, 80.0);
                        \path[,fill,
                              black!7]
                            (axis cs:120.0, 80.0)
                            rectangle (axis cs:125.0, 85.0);
                        \path[,fill,
                              black!10]
                            (axis cs:120.0, 85.0)
                            rectangle (axis cs:125.0, 90.0);
                        \path[,fill,
                              black!11]
                            (axis cs:120.0, 90.0)
                            rectangle (axis cs:125.0, 95.0);
                        \path[,fill,
                              black!14]
                            (axis cs:120.0, 95.0)
                            rectangle (axis cs:125.0, 100.0);
                        \path[,fill,
                              black!18]
                            (axis cs:120.0, 100.0)
                            rectangle (axis cs:125.0, 105.0);
                        \path[,fill,
                              black!27]
                            (axis cs:120.0, 105.0)
                            rectangle (axis cs:125.0, 110.0);
                        \path[,fill,
                              black!38]
                            (axis cs:120.0, 110.0)
                            rectangle (axis cs:125.0, 115.0);
                        \path[,fill,
                              black!58]
                            (axis cs:120.0, 115.0)
                            rectangle (axis cs:125.0, 120.0);
                        \path[,fill,
                              black!75]
                            (axis cs:120.0, 120.0)
                            rectangle (axis cs:125.0, 125.0);
                        \path[,fill,
                              black!74]
                            (axis cs:120.0, 125.0)
                            rectangle (axis cs:125.0, 130.0);
                        \path[,fill,
                              black!50]
                            (axis cs:120.0, 130.0)
                            rectangle (axis cs:125.0, 135.0);
                        \path[,fill,
                              black!37]
                            (axis cs:120.0, 135.0)
                            rectangle (axis cs:125.0, 140.0);
                        \path[,fill,
                              black!22]
                            (axis cs:120.0, 140.0)
                            rectangle (axis cs:125.0, 145.0);
                        \path[,fill,
                              black!15]
                            (axis cs:120.0, 145.0)
                            rectangle (axis cs:125.0, 150.0);
                        \path[,fill,
                              black!7]
                            (axis cs:120.0, 150.0)
                            rectangle (axis cs:125.0, 155.0);
                        \path[,fill,
                              black!6]
                            (axis cs:120.0, 155.0)
                            rectangle (axis cs:125.0, 160.0);
                        \path[,fill,
                              black!4]
                            (axis cs:120.0, 160.0)
                            rectangle (axis cs:125.0, 165.0);
                        \path[,fill,
                              black!3]
                            (axis cs:120.0, 165.0)
                            rectangle (axis cs:125.0, 170.0);
                        \path[,fill,
                              black!2]
                            (axis cs:120.0, 170.0)
                            rectangle (axis cs:125.0, 175.0);
                        \path[,fill,
                              black!3]
                            (axis cs:120.0, 175.0)
                            rectangle (axis cs:125.0, 180.0);
                        \path[,fill,
                              black!4]
                            (axis cs:125.0, -180.0)
                            rectangle (axis cs:130.0, -175.0);
                        \path[,fill,
                              black!3]
                            (axis cs:125.0, -175.0)
                            rectangle (axis cs:130.0, -170.0);
                        \path[,fill,
                              black!2]
                            (axis cs:125.0, -170.0)
                            rectangle (axis cs:130.0, -165.0);
                        \path[,fill,
                              black!1]
                            (axis cs:125.0, -165.0)
                            rectangle (axis cs:130.0, -160.0);
                        \path[,fill,
                              black!2]
                            (axis cs:125.0, -160.0)
                            rectangle (axis cs:130.0, -155.0);
                        \path[,fill,
                              black!1]
                            (axis cs:125.0, -155.0)
                            rectangle (axis cs:130.0, -150.0);
                        \path[,fill,
                              black!1]
                            (axis cs:125.0, -150.0)
                            rectangle (axis cs:130.0, -145.0);
                        \path[,fill,
                              black!1]
                            (axis cs:125.0, -145.0)
                            rectangle (axis cs:130.0, -140.0);
                        \path[,fill,
                              black!4]
                            (axis cs:125.0, -140.0)
                            rectangle (axis cs:130.0, -135.0);
                        \path[,fill,
                              black!5]
                            (axis cs:125.0, -135.0)
                            rectangle (axis cs:130.0, -130.0);
                        \path[,fill,
                              black!2]
                            (axis cs:125.0, -130.0)
                            rectangle (axis cs:130.0, -125.0);
                        \path[,fill,
                              black!2]
                            (axis cs:125.0, -125.0)
                            rectangle (axis cs:130.0, -120.0);
                        \path[,fill,
                              black!3]
                            (axis cs:125.0, -120.0)
                            rectangle (axis cs:130.0, -115.0);
                        \path[,fill,
                              black!1]
                            (axis cs:125.0, -115.0)
                            rectangle (axis cs:130.0, -110.0);
                        \path[,fill,
                              black!2]
                            (axis cs:125.0, -110.0)
                            rectangle (axis cs:130.0, -105.0);
                        \path[,fill,
                              black!1]
                            (axis cs:125.0, -95.0)
                            rectangle (axis cs:130.0, -90.0);
                        \path[,fill,
                              black!1]
                            (axis cs:125.0, -75.0)
                            rectangle (axis cs:130.0, -70.0);
                        \path[,fill,
                              black!1]
                            (axis cs:125.0, 25.0)
                            rectangle (axis cs:130.0, 30.0);
                        \path[,fill,
                              black!1]
                            (axis cs:125.0, 35.0)
                            rectangle (axis cs:130.0, 40.0);
                        \path[,fill,
                              black!1]
                            (axis cs:125.0, 40.0)
                            rectangle (axis cs:130.0, 45.0);
                        \path[,fill,
                              black!1]
                            (axis cs:125.0, 45.0)
                            rectangle (axis cs:130.0, 50.0);
                        \path[,fill,
                              black!5]
                            (axis cs:125.0, 50.0)
                            rectangle (axis cs:130.0, 55.0);
                        \path[,fill,
                              black!5]
                            (axis cs:125.0, 55.0)
                            rectangle (axis cs:130.0, 60.0);
                        \path[,fill,
                              black!7]
                            (axis cs:125.0, 60.0)
                            rectangle (axis cs:130.0, 65.0);
                        \path[,fill,
                              black!8]
                            (axis cs:125.0, 65.0)
                            rectangle (axis cs:130.0, 70.0);
                        \path[,fill,
                              black!4]
                            (axis cs:125.0, 70.0)
                            rectangle (axis cs:130.0, 75.0);
                        \path[,fill,
                              black!8]
                            (axis cs:125.0, 75.0)
                            rectangle (axis cs:130.0, 80.0);
                        \path[,fill,
                              black!7]
                            (axis cs:125.0, 80.0)
                            rectangle (axis cs:130.0, 85.0);
                        \path[,fill,
                              black!9]
                            (axis cs:125.0, 85.0)
                            rectangle (axis cs:130.0, 90.0);
                        \path[,fill,
                              black!7]
                            (axis cs:125.0, 90.0)
                            rectangle (axis cs:130.0, 95.0);
                        \path[,fill,
                              black!10]
                            (axis cs:125.0, 95.0)
                            rectangle (axis cs:130.0, 100.0);
                        \path[,fill,
                              black!16]
                            (axis cs:125.0, 100.0)
                            rectangle (axis cs:130.0, 105.0);
                        \path[,fill,
                              black!23]
                            (axis cs:125.0, 105.0)
                            rectangle (axis cs:130.0, 110.0);
                        \path[,fill,
                              black!30]
                            (axis cs:125.0, 110.0)
                            rectangle (axis cs:130.0, 115.0);
                        \path[,fill,
                              black!38]
                            (axis cs:125.0, 115.0)
                            rectangle (axis cs:130.0, 120.0);
                        \path[,fill,
                              black!51]
                            (axis cs:125.0, 120.0)
                            rectangle (axis cs:130.0, 125.0);
                        \path[,fill,
                              black!76]
                            (axis cs:125.0, 125.0)
                            rectangle (axis cs:130.0, 130.0);
                        \path[,fill,
                              black!74]
                            (axis cs:125.0, 130.0)
                            rectangle (axis cs:130.0, 135.0);
                        \path[,fill,
                              black!57]
                            (axis cs:125.0, 135.0)
                            rectangle (axis cs:130.0, 140.0);
                        \path[,fill,
                              black!27]
                            (axis cs:125.0, 140.0)
                            rectangle (axis cs:130.0, 145.0);
                        \path[,fill,
                              black!19]
                            (axis cs:125.0, 145.0)
                            rectangle (axis cs:130.0, 150.0);
                        \path[,fill,
                              black!13]
                            (axis cs:125.0, 150.0)
                            rectangle (axis cs:130.0, 155.0);
                        \path[,fill,
                              black!8]
                            (axis cs:125.0, 155.0)
                            rectangle (axis cs:130.0, 160.0);
                        \path[,fill,
                              black!8]
                            (axis cs:125.0, 160.0)
                            rectangle (axis cs:130.0, 165.0);
                        \path[,fill,
                              black!7]
                            (axis cs:125.0, 165.0)
                            rectangle (axis cs:130.0, 170.0);
                        \path[,fill,
                              black!3]
                            (axis cs:125.0, 170.0)
                            rectangle (axis cs:130.0, 175.0);
                        \path[,fill,
                              black!6]
                            (axis cs:125.0, 175.0)
                            rectangle (axis cs:130.0, 180.0);
                        \path[,fill,
                              black!4]
                            (axis cs:130.0, -180.0)
                            rectangle (axis cs:135.0, -175.0);
                        \path[,fill,
                              black!5]
                            (axis cs:130.0, -175.0)
                            rectangle (axis cs:135.0, -170.0);
                        \path[,fill,
                              black!3]
                            (axis cs:130.0, -170.0)
                            rectangle (axis cs:135.0, -165.0);
                        \path[,fill,
                              black!2]
                            (axis cs:130.0, -165.0)
                            rectangle (axis cs:135.0, -160.0);
                        \path[,fill,
                              black!4]
                            (axis cs:130.0, -160.0)
                            rectangle (axis cs:135.0, -155.0);
                        \path[,fill,
                              black!2]
                            (axis cs:130.0, -155.0)
                            rectangle (axis cs:135.0, -150.0);
                        \path[,fill,
                              black!2]
                            (axis cs:130.0, -150.0)
                            rectangle (axis cs:135.0, -145.0);
                        \path[,fill,
                              black!4]
                            (axis cs:130.0, -145.0)
                            rectangle (axis cs:135.0, -140.0);
                        \path[,fill,
                              black!2]
                            (axis cs:130.0, -140.0)
                            rectangle (axis cs:135.0, -135.0);
                        \path[,fill,
                              black!1]
                            (axis cs:130.0, -135.0)
                            rectangle (axis cs:135.0, -130.0);
                        \path[,fill,
                              black!3]
                            (axis cs:130.0, -130.0)
                            rectangle (axis cs:135.0, -125.0);
                        \path[,fill,
                              black!2]
                            (axis cs:130.0, -125.0)
                            rectangle (axis cs:135.0, -120.0);
                        \path[,fill,
                              black!2]
                            (axis cs:130.0, -120.0)
                            rectangle (axis cs:135.0, -115.0);
                        \path[,fill,
                              black!1]
                            (axis cs:130.0, -115.0)
                            rectangle (axis cs:135.0, -110.0);
                        \path[,fill,
                              black!3]
                            (axis cs:130.0, -110.0)
                            rectangle (axis cs:135.0, -105.0);
                        \path[,fill,
                              black!1]
                            (axis cs:130.0, -105.0)
                            rectangle (axis cs:135.0, -100.0);
                        \path[,fill,
                              black!1]
                            (axis cs:130.0, -100.0)
                            rectangle (axis cs:135.0, -95.0);
                        \path[,fill,
                              black!1]
                            (axis cs:130.0, -95.0)
                            rectangle (axis cs:135.0, -90.0);
                        \path[,fill,
                              black!1]
                            (axis cs:130.0, -75.0)
                            rectangle (axis cs:135.0, -70.0);
                        \path[,fill,
                              black!1]
                            (axis cs:130.0, -25.0)
                            rectangle (axis cs:135.0, -20.0);
                        \path[,fill,
                              black!1]
                            (axis cs:130.0, 10.0)
                            rectangle (axis cs:135.0, 15.0);
                        \path[,fill,
                              black!1]
                            (axis cs:130.0, 30.0)
                            rectangle (axis cs:135.0, 35.0);
                        \path[,fill,
                              black!2]
                            (axis cs:130.0, 35.0)
                            rectangle (axis cs:135.0, 40.0);
                        \path[,fill,
                              black!1]
                            (axis cs:130.0, 40.0)
                            rectangle (axis cs:135.0, 45.0);
                        \path[,fill,
                              black!1]
                            (axis cs:130.0, 45.0)
                            rectangle (axis cs:135.0, 50.0);
                        \path[,fill,
                              black!2]
                            (axis cs:130.0, 50.0)
                            rectangle (axis cs:135.0, 55.0);
                        \path[,fill,
                              black!2]
                            (axis cs:130.0, 55.0)
                            rectangle (axis cs:135.0, 60.0);
                        \path[,fill,
                              black!3]
                            (axis cs:130.0, 60.0)
                            rectangle (axis cs:135.0, 65.0);
                        \path[,fill,
                              black!2]
                            (axis cs:130.0, 65.0)
                            rectangle (axis cs:135.0, 70.0);
                        \path[,fill,
                              black!6]
                            (axis cs:130.0, 70.0)
                            rectangle (axis cs:135.0, 75.0);
                        \path[,fill,
                              black!6]
                            (axis cs:130.0, 75.0)
                            rectangle (axis cs:135.0, 80.0);
                        \path[,fill,
                              black!7]
                            (axis cs:130.0, 80.0)
                            rectangle (axis cs:135.0, 85.0);
                        \path[,fill,
                              black!5]
                            (axis cs:130.0, 85.0)
                            rectangle (axis cs:135.0, 90.0);
                        \path[,fill,
                              black!8]
                            (axis cs:130.0, 90.0)
                            rectangle (axis cs:135.0, 95.0);
                        \path[,fill,
                              black!10]
                            (axis cs:130.0, 95.0)
                            rectangle (axis cs:135.0, 100.0);
                        \path[,fill,
                              black!11]
                            (axis cs:130.0, 100.0)
                            rectangle (axis cs:135.0, 105.0);
                        \path[,fill,
                              black!13]
                            (axis cs:130.0, 105.0)
                            rectangle (axis cs:135.0, 110.0);
                        \path[,fill,
                              black!15]
                            (axis cs:130.0, 110.0)
                            rectangle (axis cs:135.0, 115.0);
                        \path[,fill,
                              black!24]
                            (axis cs:130.0, 115.0)
                            rectangle (axis cs:135.0, 120.0);
                        \path[,fill,
                              black!35]
                            (axis cs:130.0, 120.0)
                            rectangle (axis cs:135.0, 125.0);
                        \path[,fill,
                              black!51]
                            (axis cs:130.0, 125.0)
                            rectangle (axis cs:135.0, 130.0);
                        \path[,fill,
                              black!88]
                            (axis cs:130.0, 130.0)
                            rectangle (axis cs:135.0, 135.0);
                        \path[,fill,
                              black!74]
                            (axis cs:130.0, 135.0)
                            rectangle (axis cs:135.0, 140.0);
                        \path[,fill,
                              black!44]
                            (axis cs:130.0, 140.0)
                            rectangle (axis cs:135.0, 145.0);
                        \path[,fill,
                              black!26]
                            (axis cs:130.0, 145.0)
                            rectangle (axis cs:135.0, 150.0);
                        \path[,fill,
                              black!22]
                            (axis cs:130.0, 150.0)
                            rectangle (axis cs:135.0, 155.0);
                        \path[,fill,
                              black!12]
                            (axis cs:130.0, 155.0)
                            rectangle (axis cs:135.0, 160.0);
                        \path[,fill,
                              black!10]
                            (axis cs:130.0, 160.0)
                            rectangle (axis cs:135.0, 165.0);
                        \path[,fill,
                              black!7]
                            (axis cs:130.0, 165.0)
                            rectangle (axis cs:135.0, 170.0);
                        \path[,fill,
                              black!4]
                            (axis cs:130.0, 170.0)
                            rectangle (axis cs:135.0, 175.0);
                        \path[,fill,
                              black!5]
                            (axis cs:130.0, 175.0)
                            rectangle (axis cs:135.0, 180.0);
                        \path[,fill,
                              black!6]
                            (axis cs:135.0, -180.0)
                            rectangle (axis cs:140.0, -175.0);
                        \path[,fill,
                              black!5]
                            (axis cs:135.0, -175.0)
                            rectangle (axis cs:140.0, -170.0);
                        \path[,fill,
                              black!1]
                            (axis cs:135.0, -170.0)
                            rectangle (axis cs:140.0, -165.0);
                        \path[,fill,
                              black!4]
                            (axis cs:135.0, -165.0)
                            rectangle (axis cs:140.0, -160.0);
                        \path[,fill,
                              black!1]
                            (axis cs:135.0, -160.0)
                            rectangle (axis cs:140.0, -155.0);
                        \path[,fill,
                              black!3]
                            (axis cs:135.0, -155.0)
                            rectangle (axis cs:140.0, -150.0);
                        \path[,fill,
                              black!4]
                            (axis cs:135.0, -150.0)
                            rectangle (axis cs:140.0, -145.0);
                        \path[,fill,
                              black!2]
                            (axis cs:135.0, -145.0)
                            rectangle (axis cs:140.0, -140.0);
                        \path[,fill,
                              black!2]
                            (axis cs:135.0, -140.0)
                            rectangle (axis cs:140.0, -135.0);
                        \path[,fill,
                              black!1]
                            (axis cs:135.0, -135.0)
                            rectangle (axis cs:140.0, -130.0);
                        \path[,fill,
                              black!4]
                            (axis cs:135.0, -130.0)
                            rectangle (axis cs:140.0, -125.0);
                        \path[,fill,
                              black!2]
                            (axis cs:135.0, -125.0)
                            rectangle (axis cs:140.0, -120.0);
                        \path[,fill,
                              black!2]
                            (axis cs:135.0, -120.0)
                            rectangle (axis cs:140.0, -115.0);
                        \path[,fill,
                              black!2]
                            (axis cs:135.0, -115.0)
                            rectangle (axis cs:140.0, -110.0);
                        \path[,fill,
                              black!4]
                            (axis cs:135.0, -110.0)
                            rectangle (axis cs:140.0, -105.0);
                        \path[,fill,
                              black!4]
                            (axis cs:135.0, -105.0)
                            rectangle (axis cs:140.0, -100.0);
                        \path[,fill,
                              black!1]
                            (axis cs:135.0, -90.0)
                            rectangle (axis cs:140.0, -85.0);
                        \path[,fill,
                              black!1]
                            (axis cs:135.0, -75.0)
                            rectangle (axis cs:140.0, -70.0);
                        \path[,fill,
                              black!1]
                            (axis cs:135.0, 15.0)
                            rectangle (axis cs:140.0, 20.0);
                        \path[,fill,
                              black!2]
                            (axis cs:135.0, 45.0)
                            rectangle (axis cs:140.0, 50.0);
                        \path[,fill,
                              black!3]
                            (axis cs:135.0, 50.0)
                            rectangle (axis cs:140.0, 55.0);
                        \path[,fill,
                              black!5]
                            (axis cs:135.0, 55.0)
                            rectangle (axis cs:140.0, 60.0);
                        \path[,fill,
                              black!1]
                            (axis cs:135.0, 60.0)
                            rectangle (axis cs:140.0, 65.0);
                        \path[,fill,
                              black!4]
                            (axis cs:135.0, 65.0)
                            rectangle (axis cs:140.0, 70.0);
                        \path[,fill,
                              black!4]
                            (axis cs:135.0, 70.0)
                            rectangle (axis cs:140.0, 75.0);
                        \path[,fill,
                              black!4]
                            (axis cs:135.0, 75.0)
                            rectangle (axis cs:140.0, 80.0);
                        \path[,fill,
                              black!7]
                            (axis cs:135.0, 80.0)
                            rectangle (axis cs:140.0, 85.0);
                        \path[,fill,
                              black!6]
                            (axis cs:135.0, 85.0)
                            rectangle (axis cs:140.0, 90.0);
                        \path[,fill,
                              black!7]
                            (axis cs:135.0, 90.0)
                            rectangle (axis cs:140.0, 95.0);
                        \path[,fill,
                              black!10]
                            (axis cs:135.0, 95.0)
                            rectangle (axis cs:140.0, 100.0);
                        \path[,fill,
                              black!8]
                            (axis cs:135.0, 100.0)
                            rectangle (axis cs:140.0, 105.0);
                        \path[,fill,
                              black!7]
                            (axis cs:135.0, 105.0)
                            rectangle (axis cs:140.0, 110.0);
                        \path[,fill,
                              black!11]
                            (axis cs:135.0, 110.0)
                            rectangle (axis cs:140.0, 115.0);
                        \path[,fill,
                              black!18]
                            (axis cs:135.0, 115.0)
                            rectangle (axis cs:140.0, 120.0);
                        \path[,fill,
                              black!22]
                            (axis cs:135.0, 120.0)
                            rectangle (axis cs:140.0, 125.0);
                        \path[,fill,
                              black!39]
                            (axis cs:135.0, 125.0)
                            rectangle (axis cs:140.0, 130.0);
                        \path[,fill,
                              black!59]
                            (axis cs:135.0, 130.0)
                            rectangle (axis cs:140.0, 135.0);
                        \path[,fill,
                              black!95]
                            (axis cs:135.0, 135.0)
                            rectangle (axis cs:140.0, 140.0);
                        \path[,fill,
                              black!68]
                            (axis cs:135.0, 140.0)
                            rectangle (axis cs:140.0, 145.0);
                        \path[,fill,
                              black!38]
                            (axis cs:135.0, 145.0)
                            rectangle (axis cs:140.0, 150.0);
                        \path[,fill,
                              black!26]
                            (axis cs:135.0, 150.0)
                            rectangle (axis cs:140.0, 155.0);
                        \path[,fill,
                              black!16]
                            (axis cs:135.0, 155.0)
                            rectangle (axis cs:140.0, 160.0);
                        \path[,fill,
                              black!10]
                            (axis cs:135.0, 160.0)
                            rectangle (axis cs:140.0, 165.0);
                        \path[,fill,
                              black!7]
                            (axis cs:135.0, 165.0)
                            rectangle (axis cs:140.0, 170.0);
                        \path[,fill,
                              black!4]
                            (axis cs:135.0, 170.0)
                            rectangle (axis cs:140.0, 175.0);
                        \path[,fill,
                              black!3]
                            (axis cs:135.0, 175.0)
                            rectangle (axis cs:140.0, 180.0);
                        \path[,fill,
                              black!7]
                            (axis cs:140.0, -180.0)
                            rectangle (axis cs:145.0, -175.0);
                        \path[,fill,
                              black!4]
                            (axis cs:140.0, -175.0)
                            rectangle (axis cs:145.0, -170.0);
                        \path[,fill,
                              black!4]
                            (axis cs:140.0, -170.0)
                            rectangle (axis cs:145.0, -165.0);
                        \path[,fill,
                              black!4]
                            (axis cs:140.0, -165.0)
                            rectangle (axis cs:145.0, -160.0);
                        \path[,fill,
                              black!3]
                            (axis cs:140.0, -160.0)
                            rectangle (axis cs:145.0, -155.0);
                        \path[,fill,
                              black!3]
                            (axis cs:140.0, -155.0)
                            rectangle (axis cs:145.0, -150.0);
                        \path[,fill,
                              black!4]
                            (axis cs:140.0, -150.0)
                            rectangle (axis cs:145.0, -145.0);
                        \path[,fill,
                              black!2]
                            (axis cs:140.0, -145.0)
                            rectangle (axis cs:145.0, -140.0);
                        \path[,fill,
                              black!2]
                            (axis cs:140.0, -140.0)
                            rectangle (axis cs:145.0, -135.0);
                        \path[,fill,
                              black!2]
                            (axis cs:140.0, -135.0)
                            rectangle (axis cs:145.0, -130.0);
                        \path[,fill,
                              black!2]
                            (axis cs:140.0, -130.0)
                            rectangle (axis cs:145.0, -125.0);
                        \path[,fill,
                              black!3]
                            (axis cs:140.0, -125.0)
                            rectangle (axis cs:145.0, -120.0);
                        \path[,fill,
                              black!4]
                            (axis cs:140.0, -120.0)
                            rectangle (axis cs:145.0, -115.0);
                        \path[,fill,
                              black!3]
                            (axis cs:140.0, -115.0)
                            rectangle (axis cs:145.0, -110.0);
                        \path[,fill,
                              black!1]
                            (axis cs:140.0, -105.0)
                            rectangle (axis cs:145.0, -100.0);
                        \path[,fill,
                              black!1]
                            (axis cs:140.0, -100.0)
                            rectangle (axis cs:145.0, -95.0);
                        \path[,fill,
                              black!1]
                            (axis cs:140.0, -95.0)
                            rectangle (axis cs:145.0, -90.0);
                        \path[,fill,
                              black!1]
                            (axis cs:140.0, 25.0)
                            rectangle (axis cs:145.0, 30.0);
                        \path[,fill,
                              black!1]
                            (axis cs:140.0, 45.0)
                            rectangle (axis cs:145.0, 50.0);
                        \path[,fill,
                              black!2]
                            (axis cs:140.0, 50.0)
                            rectangle (axis cs:145.0, 55.0);
                        \path[,fill,
                              black!4]
                            (axis cs:140.0, 55.0)
                            rectangle (axis cs:145.0, 60.0);
                        \path[,fill,
                              black!2]
                            (axis cs:140.0, 60.0)
                            rectangle (axis cs:145.0, 65.0);
                        \path[,fill,
                              black!4]
                            (axis cs:140.0, 65.0)
                            rectangle (axis cs:145.0, 70.0);
                        \path[,fill,
                              black!3]
                            (axis cs:140.0, 70.0)
                            rectangle (axis cs:145.0, 75.0);
                        \path[,fill,
                              black!5]
                            (axis cs:140.0, 75.0)
                            rectangle (axis cs:145.0, 80.0);
                        \path[,fill,
                              black!7]
                            (axis cs:140.0, 80.0)
                            rectangle (axis cs:145.0, 85.0);
                        \path[,fill,
                              black!8]
                            (axis cs:140.0, 85.0)
                            rectangle (axis cs:145.0, 90.0);
                        \path[,fill,
                              black!4]
                            (axis cs:140.0, 90.0)
                            rectangle (axis cs:145.0, 95.0);
                        \path[,fill,
                              black!8]
                            (axis cs:140.0, 95.0)
                            rectangle (axis cs:145.0, 100.0);
                        \path[,fill,
                              black!7]
                            (axis cs:140.0, 100.0)
                            rectangle (axis cs:145.0, 105.0);
                        \path[,fill,
                              black!6]
                            (axis cs:140.0, 105.0)
                            rectangle (axis cs:145.0, 110.0);
                        \path[,fill,
                              black!8]
                            (axis cs:140.0, 110.0)
                            rectangle (axis cs:145.0, 115.0);
                        \path[,fill,
                              black!14]
                            (axis cs:140.0, 115.0)
                            rectangle (axis cs:145.0, 120.0);
                        \path[,fill,
                              black!16]
                            (axis cs:140.0, 120.0)
                            rectangle (axis cs:145.0, 125.0);
                        \path[,fill,
                              black!30]
                            (axis cs:140.0, 125.0)
                            rectangle (axis cs:145.0, 130.0);
                        \path[,fill,
                              black!40]
                            (axis cs:140.0, 130.0)
                            rectangle (axis cs:145.0, 135.0);
                        \path[,fill,
                              black!62]
                            (axis cs:140.0, 135.0)
                            rectangle (axis cs:145.0, 140.0);
                        \path[,fill,
                              black!92]
                            (axis cs:140.0, 140.0)
                            rectangle (axis cs:145.0, 145.0);
                        \path[,fill,
                              black!77]
                            (axis cs:140.0, 145.0)
                            rectangle (axis cs:145.0, 150.0);
                        \path[,fill,
                              black!47]
                            (axis cs:140.0, 150.0)
                            rectangle (axis cs:145.0, 155.0);
                        \path[,fill,
                              black!22]
                            (axis cs:140.0, 155.0)
                            rectangle (axis cs:145.0, 160.0);
                        \path[,fill,
                              black!18]
                            (axis cs:140.0, 160.0)
                            rectangle (axis cs:145.0, 165.0);
                        \path[,fill,
                              black!10]
                            (axis cs:140.0, 165.0)
                            rectangle (axis cs:145.0, 170.0);
                        \path[,fill,
                              black!11]
                            (axis cs:140.0, 170.0)
                            rectangle (axis cs:145.0, 175.0);
                        \path[,fill,
                              black!7]
                            (axis cs:140.0, 175.0)
                            rectangle (axis cs:145.0, 180.0);
                        \path[,fill,
                              black!7]
                            (axis cs:145.0, -180.0)
                            rectangle (axis cs:150.0, -175.0);
                        \path[,fill,
                              black!6]
                            (axis cs:145.0, -175.0)
                            rectangle (axis cs:150.0, -170.0);
                        \path[,fill,
                              black!5]
                            (axis cs:145.0, -170.0)
                            rectangle (axis cs:150.0, -165.0);
                        \path[,fill,
                              black!3]
                            (axis cs:145.0, -165.0)
                            rectangle (axis cs:150.0, -160.0);
                        \path[,fill,
                              black!2]
                            (axis cs:145.0, -160.0)
                            rectangle (axis cs:150.0, -155.0);
                        \path[,fill,
                              black!4]
                            (axis cs:145.0, -155.0)
                            rectangle (axis cs:150.0, -150.0);
                        \path[,fill,
                              black!7]
                            (axis cs:145.0, -150.0)
                            rectangle (axis cs:150.0, -145.0);
                        \path[,fill,
                              black!3]
                            (axis cs:145.0, -145.0)
                            rectangle (axis cs:150.0, -140.0);
                        \path[,fill,
                              black!2]
                            (axis cs:145.0, -140.0)
                            rectangle (axis cs:150.0, -135.0);
                        \path[,fill,
                              black!4]
                            (axis cs:145.0, -135.0)
                            rectangle (axis cs:150.0, -130.0);
                        \path[,fill,
                              black!2]
                            (axis cs:145.0, -130.0)
                            rectangle (axis cs:150.0, -125.0);
                        \path[,fill,
                              black!3]
                            (axis cs:145.0, -125.0)
                            rectangle (axis cs:150.0, -120.0);
                        \path[,fill,
                              black!4]
                            (axis cs:145.0, -120.0)
                            rectangle (axis cs:150.0, -115.0);
                        \path[,fill,
                              black!5]
                            (axis cs:145.0, -115.0)
                            rectangle (axis cs:150.0, -110.0);
                        \path[,fill,
                              black!1]
                            (axis cs:145.0, -105.0)
                            rectangle (axis cs:150.0, -100.0);
                        \path[,fill,
                              black!1]
                            (axis cs:145.0, -100.0)
                            rectangle (axis cs:150.0, -95.0);
                        \path[,fill,
                              black!1]
                            (axis cs:145.0, -95.0)
                            rectangle (axis cs:150.0, -90.0);
                        \path[,fill,
                              black!1]
                            (axis cs:145.0, 35.0)
                            rectangle (axis cs:150.0, 40.0);
                        \path[,fill,
                              black!2]
                            (axis cs:145.0, 40.0)
                            rectangle (axis cs:150.0, 45.0);
                        \path[,fill,
                              black!1]
                            (axis cs:145.0, 45.0)
                            rectangle (axis cs:150.0, 50.0);
                        \path[,fill,
                              black!2]
                            (axis cs:145.0, 50.0)
                            rectangle (axis cs:150.0, 55.0);
                        \path[,fill,
                              black!4]
                            (axis cs:145.0, 55.0)
                            rectangle (axis cs:150.0, 60.0);
                        \path[,fill,
                              black!5]
                            (axis cs:145.0, 60.0)
                            rectangle (axis cs:150.0, 65.0);
                        \path[,fill,
                              black!4]
                            (axis cs:145.0, 65.0)
                            rectangle (axis cs:150.0, 70.0);
                        \path[,fill,
                              black!3]
                            (axis cs:145.0, 70.0)
                            rectangle (axis cs:150.0, 75.0);
                        \path[,fill,
                              black!3]
                            (axis cs:145.0, 75.0)
                            rectangle (axis cs:150.0, 80.0);
                        \path[,fill,
                              black!6]
                            (axis cs:145.0, 80.0)
                            rectangle (axis cs:150.0, 85.0);
                        \path[,fill,
                              black!4]
                            (axis cs:145.0, 85.0)
                            rectangle (axis cs:150.0, 90.0);
                        \path[,fill,
                              black!5]
                            (axis cs:145.0, 90.0)
                            rectangle (axis cs:150.0, 95.0);
                        \path[,fill,
                              black!2]
                            (axis cs:145.0, 95.0)
                            rectangle (axis cs:150.0, 100.0);
                        \path[,fill,
                              black!9]
                            (axis cs:145.0, 100.0)
                            rectangle (axis cs:150.0, 105.0);
                        \path[,fill,
                              black!4]
                            (axis cs:145.0, 105.0)
                            rectangle (axis cs:150.0, 110.0);
                        \path[,fill,
                              black!8]
                            (axis cs:145.0, 110.0)
                            rectangle (axis cs:150.0, 115.0);
                        \path[,fill,
                              black!9]
                            (axis cs:145.0, 115.0)
                            rectangle (axis cs:150.0, 120.0);
                        \path[,fill,
                              black!16]
                            (axis cs:145.0, 120.0)
                            rectangle (axis cs:150.0, 125.0);
                        \path[,fill,
                              black!16]
                            (axis cs:145.0, 125.0)
                            rectangle (axis cs:150.0, 130.0);
                        \path[,fill,
                              black!27]
                            (axis cs:145.0, 130.0)
                            rectangle (axis cs:150.0, 135.0);
                        \path[,fill,
                              black!38]
                            (axis cs:145.0, 135.0)
                            rectangle (axis cs:150.0, 140.0);
                        \path[,fill,
                              black!68]
                            (axis cs:145.0, 140.0)
                            rectangle (axis cs:150.0, 145.0);
                        \path[,fill,
                              black!100]
                            (axis cs:145.0, 145.0)
                            rectangle (axis cs:150.0, 150.0);
                        \path[,fill,
                              black!72]
                            (axis cs:145.0, 150.0)
                            rectangle (axis cs:150.0, 155.0);
                        \path[,fill,
                              black!40]
                            (axis cs:145.0, 155.0)
                            rectangle (axis cs:150.0, 160.0);
                        \path[,fill,
                              black!32]
                            (axis cs:145.0, 160.0)
                            rectangle (axis cs:150.0, 165.0);
                        \path[,fill,
                              black!11]
                            (axis cs:145.0, 165.0)
                            rectangle (axis cs:150.0, 170.0);
                        \path[,fill,
                              black!14]
                            (axis cs:145.0, 170.0)
                            rectangle (axis cs:150.0, 175.0);
                        \path[,fill,
                              black!11]
                            (axis cs:145.0, 175.0)
                            rectangle (axis cs:150.0, 180.0);
                        \path[,fill,
                              black!13]
                            (axis cs:150.0, -180.0)
                            rectangle (axis cs:155.0, -175.0);
                        \path[,fill,
                              black!6]
                            (axis cs:150.0, -175.0)
                            rectangle (axis cs:155.0, -170.0);
                        \path[,fill,
                              black!5]
                            (axis cs:150.0, -170.0)
                            rectangle (axis cs:155.0, -165.0);
                        \path[,fill,
                              black!4]
                            (axis cs:150.0, -165.0)
                            rectangle (axis cs:155.0, -160.0);
                        \path[,fill,
                              black!5]
                            (axis cs:150.0, -160.0)
                            rectangle (axis cs:155.0, -155.0);
                        \path[,fill,
                              black!6]
                            (axis cs:150.0, -155.0)
                            rectangle (axis cs:155.0, -150.0);
                        \path[,fill,
                              black!4]
                            (axis cs:150.0, -150.0)
                            rectangle (axis cs:155.0, -145.0);
                        \path[,fill,
                              black!4]
                            (axis cs:150.0, -145.0)
                            rectangle (axis cs:155.0, -140.0);
                        \path[,fill,
                              black!4]
                            (axis cs:150.0, -140.0)
                            rectangle (axis cs:155.0, -135.0);
                        \path[,fill,
                              black!2]
                            (axis cs:150.0, -135.0)
                            rectangle (axis cs:155.0, -130.0);
                        \path[,fill,
                              black!3]
                            (axis cs:150.0, -130.0)
                            rectangle (axis cs:155.0, -125.0);
                        \path[,fill,
                              black!4]
                            (axis cs:150.0, -125.0)
                            rectangle (axis cs:155.0, -120.0);
                        \path[,fill,
                              black!2]
                            (axis cs:150.0, -120.0)
                            rectangle (axis cs:155.0, -115.0);
                        \path[,fill,
                              black!4]
                            (axis cs:150.0, -115.0)
                            rectangle (axis cs:155.0, -110.0);
                        \path[,fill,
                              black!1]
                            (axis cs:150.0, -110.0)
                            rectangle (axis cs:155.0, -105.0);
                        \path[,fill,
                              black!1]
                            (axis cs:150.0, -105.0)
                            rectangle (axis cs:155.0, -100.0);
                        \path[,fill,
                              black!1]
                            (axis cs:150.0, -100.0)
                            rectangle (axis cs:155.0, -95.0);
                        \path[,fill,
                              black!1]
                            (axis cs:150.0, -95.0)
                            rectangle (axis cs:155.0, -90.0);
                        \path[,fill,
                              black!1]
                            (axis cs:150.0, -90.0)
                            rectangle (axis cs:155.0, -85.0);
                        \path[,fill,
                              black!2]
                            (axis cs:150.0, 35.0)
                            rectangle (axis cs:155.0, 40.0);
                        \path[,fill,
                              black!1]
                            (axis cs:150.0, 40.0)
                            rectangle (axis cs:155.0, 45.0);
                        \path[,fill,
                              black!1]
                            (axis cs:150.0, 45.0)
                            rectangle (axis cs:155.0, 50.0);
                        \path[,fill,
                              black!3]
                            (axis cs:150.0, 50.0)
                            rectangle (axis cs:155.0, 55.0);
                        \path[,fill,
                              black!4]
                            (axis cs:150.0, 55.0)
                            rectangle (axis cs:155.0, 60.0);
                        \path[,fill,
                              black!3]
                            (axis cs:150.0, 60.0)
                            rectangle (axis cs:155.0, 65.0);
                        \path[,fill,
                              black!2]
                            (axis cs:150.0, 65.0)
                            rectangle (axis cs:155.0, 70.0);
                        \path[,fill,
                              black!3]
                            (axis cs:150.0, 70.0)
                            rectangle (axis cs:155.0, 75.0);
                        \path[,fill,
                              black!4]
                            (axis cs:150.0, 75.0)
                            rectangle (axis cs:155.0, 80.0);
                        \path[,fill,
                              black!1]
                            (axis cs:150.0, 80.0)
                            rectangle (axis cs:155.0, 85.0);
                        \path[,fill,
                              black!3]
                            (axis cs:150.0, 85.0)
                            rectangle (axis cs:155.0, 90.0);
                        \path[,fill,
                              black!3]
                            (axis cs:150.0, 90.0)
                            rectangle (axis cs:155.0, 95.0);
                        \path[,fill,
                              black!4]
                            (axis cs:150.0, 95.0)
                            rectangle (axis cs:155.0, 100.0);
                        \path[,fill,
                              black!2]
                            (axis cs:150.0, 100.0)
                            rectangle (axis cs:155.0, 105.0);
                        \path[,fill,
                              black!5]
                            (axis cs:150.0, 105.0)
                            rectangle (axis cs:155.0, 110.0);
                        \path[,fill,
                              black!4]
                            (axis cs:150.0, 110.0)
                            rectangle (axis cs:155.0, 115.0);
                        \path[,fill,
                              black!9]
                            (axis cs:150.0, 115.0)
                            rectangle (axis cs:155.0, 120.0);
                        \path[,fill,
                              black!12]
                            (axis cs:150.0, 120.0)
                            rectangle (axis cs:155.0, 125.0);
                        \path[,fill,
                              black!12]
                            (axis cs:150.0, 125.0)
                            rectangle (axis cs:155.0, 130.0);
                        \path[,fill,
                              black!22]
                            (axis cs:150.0, 130.0)
                            rectangle (axis cs:155.0, 135.0);
                        \path[,fill,
                              black!30]
                            (axis cs:150.0, 135.0)
                            rectangle (axis cs:155.0, 140.0);
                        \path[,fill,
                              black!43]
                            (axis cs:150.0, 140.0)
                            rectangle (axis cs:155.0, 145.0);
                        \path[,fill,
                              black!70]
                            (axis cs:150.0, 145.0)
                            rectangle (axis cs:155.0, 150.0);
                        \path[,fill,
                              black!89]
                            (axis cs:150.0, 150.0)
                            rectangle (axis cs:155.0, 155.0);
                        \path[,fill,
                              black!56]
                            (axis cs:150.0, 155.0)
                            rectangle (axis cs:155.0, 160.0);
                        \path[,fill,
                              black!44]
                            (axis cs:150.0, 160.0)
                            rectangle (axis cs:155.0, 165.0);
                        \path[,fill,
                              black!24]
                            (axis cs:150.0, 165.0)
                            rectangle (axis cs:155.0, 170.0);
                        \path[,fill,
                              black!22]
                            (axis cs:150.0, 170.0)
                            rectangle (axis cs:155.0, 175.0);
                        \path[,fill,
                              black!11]
                            (axis cs:150.0, 175.0)
                            rectangle (axis cs:155.0, 180.0);
                        \path[,fill,
                              black!11]
                            (axis cs:155.0, -180.0)
                            rectangle (axis cs:160.0, -175.0);
                        \path[,fill,
                              black!10]
                            (axis cs:155.0, -175.0)
                            rectangle (axis cs:160.0, -170.0);
                        \path[,fill,
                              black!10]
                            (axis cs:155.0, -170.0)
                            rectangle (axis cs:160.0, -165.0);
                        \path[,fill,
                              black!7]
                            (axis cs:155.0, -165.0)
                            rectangle (axis cs:160.0, -160.0);
                        \path[,fill,
                              black!7]
                            (axis cs:155.0, -160.0)
                            rectangle (axis cs:160.0, -155.0);
                        \path[,fill,
                              black!7]
                            (axis cs:155.0, -155.0)
                            rectangle (axis cs:160.0, -150.0);
                        \path[,fill,
                              black!5]
                            (axis cs:155.0, -150.0)
                            rectangle (axis cs:160.0, -145.0);
                        \path[,fill,
                              black!4]
                            (axis cs:155.0, -145.0)
                            rectangle (axis cs:160.0, -140.0);
                        \path[,fill,
                              black!2]
                            (axis cs:155.0, -140.0)
                            rectangle (axis cs:160.0, -135.0);
                        \path[,fill,
                              black!4]
                            (axis cs:155.0, -135.0)
                            rectangle (axis cs:160.0, -130.0);
                        \path[,fill,
                              black!6]
                            (axis cs:155.0, -130.0)
                            rectangle (axis cs:160.0, -125.0);
                        \path[,fill,
                              black!4]
                            (axis cs:155.0, -125.0)
                            rectangle (axis cs:160.0, -120.0);
                        \path[,fill,
                              black!2]
                            (axis cs:155.0, -120.0)
                            rectangle (axis cs:160.0, -115.0);
                        \path[,fill,
                              black!2]
                            (axis cs:155.0, -115.0)
                            rectangle (axis cs:160.0, -110.0);
                        \path[,fill,
                              black!1]
                            (axis cs:155.0, -110.0)
                            rectangle (axis cs:160.0, -105.0);
                        \path[,fill,
                              black!1]
                            (axis cs:155.0, -105.0)
                            rectangle (axis cs:160.0, -100.0);
                        \path[,fill,
                              black!1]
                            (axis cs:155.0, -100.0)
                            rectangle (axis cs:160.0, -95.0);
                        \path[,fill,
                              black!1]
                            (axis cs:155.0, 20.0)
                            rectangle (axis cs:160.0, 25.0);
                        \path[,fill,
                              black!1]
                            (axis cs:155.0, 35.0)
                            rectangle (axis cs:160.0, 40.0);
                        \path[,fill,
                              black!2]
                            (axis cs:155.0, 40.0)
                            rectangle (axis cs:160.0, 45.0);
                        \path[,fill,
                              black!2]
                            (axis cs:155.0, 45.0)
                            rectangle (axis cs:160.0, 50.0);
                        \path[,fill,
                              black!1]
                            (axis cs:155.0, 50.0)
                            rectangle (axis cs:160.0, 55.0);
                        \path[,fill,
                              black!4]
                            (axis cs:155.0, 55.0)
                            rectangle (axis cs:160.0, 60.0);
                        \path[,fill,
                              black!5]
                            (axis cs:155.0, 60.0)
                            rectangle (axis cs:160.0, 65.0);
                        \path[,fill,
                              black!2]
                            (axis cs:155.0, 65.0)
                            rectangle (axis cs:160.0, 70.0);
                        \path[,fill,
                              black!3]
                            (axis cs:155.0, 70.0)
                            rectangle (axis cs:160.0, 75.0);
                        \path[,fill,
                              black!3]
                            (axis cs:155.0, 75.0)
                            rectangle (axis cs:160.0, 80.0);
                        \path[,fill,
                              black!3]
                            (axis cs:155.0, 80.0)
                            rectangle (axis cs:160.0, 85.0);
                        \path[,fill,
                              black!2]
                            (axis cs:155.0, 85.0)
                            rectangle (axis cs:160.0, 90.0);
                        \path[,fill,
                              black!4]
                            (axis cs:155.0, 90.0)
                            rectangle (axis cs:160.0, 95.0);
                        \path[,fill,
                              black!4]
                            (axis cs:155.0, 95.0)
                            rectangle (axis cs:160.0, 100.0);
                        \path[,fill,
                              black!2]
                            (axis cs:155.0, 100.0)
                            rectangle (axis cs:160.0, 105.0);
                        \path[,fill,
                              black!3]
                            (axis cs:155.0, 105.0)
                            rectangle (axis cs:160.0, 110.0);
                        \path[,fill,
                              black!7]
                            (axis cs:155.0, 110.0)
                            rectangle (axis cs:160.0, 115.0);
                        \path[,fill,
                              black!5]
                            (axis cs:155.0, 115.0)
                            rectangle (axis cs:160.0, 120.0);
                        \path[,fill,
                              black!10]
                            (axis cs:155.0, 120.0)
                            rectangle (axis cs:160.0, 125.0);
                        \path[,fill,
                              black!8]
                            (axis cs:155.0, 125.0)
                            rectangle (axis cs:160.0, 130.0);
                        \path[,fill,
                              black!11]
                            (axis cs:155.0, 130.0)
                            rectangle (axis cs:160.0, 135.0);
                        \path[,fill,
                              black!16]
                            (axis cs:155.0, 135.0)
                            rectangle (axis cs:160.0, 140.0);
                        \path[,fill,
                              black!24]
                            (axis cs:155.0, 140.0)
                            rectangle (axis cs:160.0, 145.0);
                        \path[,fill,
                              black!42]
                            (axis cs:155.0, 145.0)
                            rectangle (axis cs:160.0, 150.0);
                        \path[,fill,
                              black!72]
                            (axis cs:155.0, 150.0)
                            rectangle (axis cs:160.0, 155.0);
                        \path[,fill,
                              black!88]
                            (axis cs:155.0, 155.0)
                            rectangle (axis cs:160.0, 160.0);
                        \path[,fill,
                              black!67]
                            (axis cs:155.0, 160.0)
                            rectangle (axis cs:160.0, 165.0);
                        \path[,fill,
                              black!40]
                            (axis cs:155.0, 165.0)
                            rectangle (axis cs:160.0, 170.0);
                        \path[,fill,
                              black!25]
                            (axis cs:155.0, 170.0)
                            rectangle (axis cs:160.0, 175.0);
                        \path[,fill,
                              black!17]
                            (axis cs:155.0, 175.0)
                            rectangle (axis cs:160.0, 180.0);
                        \path[,fill,
                              black!16]
                            (axis cs:160.0, -180.0)
                            rectangle (axis cs:165.0, -175.0);
                        \path[,fill,
                              black!12]
                            (axis cs:160.0, -175.0)
                            rectangle (axis cs:165.0, -170.0);
                        \path[,fill,
                              black!11]
                            (axis cs:160.0, -170.0)
                            rectangle (axis cs:165.0, -165.0);
                        \path[,fill,
                              black!8]
                            (axis cs:160.0, -165.0)
                            rectangle (axis cs:165.0, -160.0);
                        \path[,fill,
                              black!7]
                            (axis cs:160.0, -160.0)
                            rectangle (axis cs:165.0, -155.0);
                        \path[,fill,
                              black!8]
                            (axis cs:160.0, -155.0)
                            rectangle (axis cs:165.0, -150.0);
                        \path[,fill,
                              black!6]
                            (axis cs:160.0, -150.0)
                            rectangle (axis cs:165.0, -145.0);
                        \path[,fill,
                              black!6]
                            (axis cs:160.0, -145.0)
                            rectangle (axis cs:165.0, -140.0);
                        \path[,fill,
                              black!5]
                            (axis cs:160.0, -140.0)
                            rectangle (axis cs:165.0, -135.0);
                        \path[,fill,
                              black!5]
                            (axis cs:160.0, -135.0)
                            rectangle (axis cs:165.0, -130.0);
                        \path[,fill,
                              black!4]
                            (axis cs:160.0, -130.0)
                            rectangle (axis cs:165.0, -125.0);
                        \path[,fill,
                              black!3]
                            (axis cs:160.0, -125.0)
                            rectangle (axis cs:165.0, -120.0);
                        \path[,fill,
                              black!5]
                            (axis cs:160.0, -120.0)
                            rectangle (axis cs:165.0, -115.0);
                        \path[,fill,
                              black!4]
                            (axis cs:160.0, -115.0)
                            rectangle (axis cs:165.0, -110.0);
                        \path[,fill,
                              black!1]
                            (axis cs:160.0, -110.0)
                            rectangle (axis cs:165.0, -105.0);
                        \path[,fill,
                              black!1]
                            (axis cs:160.0, -105.0)
                            rectangle (axis cs:165.0, -100.0);
                        \path[,fill,
                              black!1]
                            (axis cs:160.0, -80.0)
                            rectangle (axis cs:165.0, -75.0);
                        \path[,fill,
                              black!1]
                            (axis cs:160.0, -5.0)
                            rectangle (axis cs:165.0, 0.0);
                        \path[,fill,
                              black!1]
                            (axis cs:160.0, 30.0)
                            rectangle (axis cs:165.0, 35.0);
                        \path[,fill,
                              black!3]
                            (axis cs:160.0, 40.0)
                            rectangle (axis cs:165.0, 45.0);
                        \path[,fill,
                              black!1]
                            (axis cs:160.0, 45.0)
                            rectangle (axis cs:165.0, 50.0);
                        \path[,fill,
                              black!4]
                            (axis cs:160.0, 50.0)
                            rectangle (axis cs:165.0, 55.0);
                        \path[,fill,
                              black!1]
                            (axis cs:160.0, 55.0)
                            rectangle (axis cs:165.0, 60.0);
                        \path[,fill,
                              black!3]
                            (axis cs:160.0, 60.0)
                            rectangle (axis cs:165.0, 65.0);
                        \path[,fill,
                              black!3]
                            (axis cs:160.0, 65.0)
                            rectangle (axis cs:165.0, 70.0);
                        \path[,fill,
                              black!4]
                            (axis cs:160.0, 70.0)
                            rectangle (axis cs:165.0, 75.0);
                        \path[,fill,
                              black!2]
                            (axis cs:160.0, 75.0)
                            rectangle (axis cs:165.0, 80.0);
                        \path[,fill,
                              black!2]
                            (axis cs:160.0, 80.0)
                            rectangle (axis cs:165.0, 85.0);
                        \path[,fill,
                              black!4]
                            (axis cs:160.0, 85.0)
                            rectangle (axis cs:165.0, 90.0);
                        \path[,fill,
                              black!2]
                            (axis cs:160.0, 90.0)
                            rectangle (axis cs:165.0, 95.0);
                        \path[,fill,
                              black!2]
                            (axis cs:160.0, 95.0)
                            rectangle (axis cs:165.0, 100.0);
                        \path[,fill,
                              black!5]
                            (axis cs:160.0, 100.0)
                            rectangle (axis cs:165.0, 105.0);
                        \path[,fill,
                              black!4]
                            (axis cs:160.0, 105.0)
                            rectangle (axis cs:165.0, 110.0);
                        \path[,fill,
                              black!5]
                            (axis cs:160.0, 110.0)
                            rectangle (axis cs:165.0, 115.0);
                        \path[,fill,
                              black!4]
                            (axis cs:160.0, 115.0)
                            rectangle (axis cs:165.0, 120.0);
                        \path[,fill,
                              black!5]
                            (axis cs:160.0, 120.0)
                            rectangle (axis cs:165.0, 125.0);
                        \path[,fill,
                              black!6]
                            (axis cs:160.0, 125.0)
                            rectangle (axis cs:165.0, 130.0);
                        \path[,fill,
                              black!8]
                            (axis cs:160.0, 130.0)
                            rectangle (axis cs:165.0, 135.0);
                        \path[,fill,
                              black!11]
                            (axis cs:160.0, 135.0)
                            rectangle (axis cs:165.0, 140.0);
                        \path[,fill,
                              black!17]
                            (axis cs:160.0, 140.0)
                            rectangle (axis cs:165.0, 145.0);
                        \path[,fill,
                              black!31]
                            (axis cs:160.0, 145.0)
                            rectangle (axis cs:165.0, 150.0);
                        \path[,fill,
                              black!45]
                            (axis cs:160.0, 150.0)
                            rectangle (axis cs:165.0, 155.0);
                        \path[,fill,
                              black!73]
                            (axis cs:160.0, 155.0)
                            rectangle (axis cs:165.0, 160.0);
                        \path[,fill,
                              black!81]
                            (axis cs:160.0, 160.0)
                            rectangle (axis cs:165.0, 165.0);
                        \path[,fill,
                              black!56]
                            (axis cs:160.0, 165.0)
                            rectangle (axis cs:165.0, 170.0);
                        \path[,fill,
                              black!34]
                            (axis cs:160.0, 170.0)
                            rectangle (axis cs:165.0, 175.0);
                        \path[,fill,
                              black!23]
                            (axis cs:160.0, 175.0)
                            rectangle (axis cs:165.0, 180.0);
                        \path[,fill,
                              black!26]
                            (axis cs:165.0, -180.0)
                            rectangle (axis cs:170.0, -175.0);
                        \path[,fill,
                              black!22]
                            (axis cs:165.0, -175.0)
                            rectangle (axis cs:170.0, -170.0);
                        \path[,fill,
                              black!8]
                            (axis cs:165.0, -170.0)
                            rectangle (axis cs:170.0, -165.0);
                        \path[,fill,
                              black!8]
                            (axis cs:165.0, -165.0)
                            rectangle (axis cs:170.0, -160.0);
                        \path[,fill,
                              black!7]
                            (axis cs:165.0, -160.0)
                            rectangle (axis cs:170.0, -155.0);
                        \path[,fill,
                              black!4]
                            (axis cs:165.0, -155.0)
                            rectangle (axis cs:170.0, -150.0);
                        \path[,fill,
                              black!7]
                            (axis cs:165.0, -150.0)
                            rectangle (axis cs:170.0, -145.0);
                        \path[,fill,
                              black!7]
                            (axis cs:165.0, -145.0)
                            rectangle (axis cs:170.0, -140.0);
                        \path[,fill,
                              black!4]
                            (axis cs:165.0, -140.0)
                            rectangle (axis cs:170.0, -135.0);
                        \path[,fill,
                              black!5]
                            (axis cs:165.0, -135.0)
                            rectangle (axis cs:170.0, -130.0);
                        \path[,fill,
                              black!7]
                            (axis cs:165.0, -130.0)
                            rectangle (axis cs:170.0, -125.0);
                        \path[,fill,
                              black!4]
                            (axis cs:165.0, -125.0)
                            rectangle (axis cs:170.0, -120.0);
                        \path[,fill,
                              black!2]
                            (axis cs:165.0, -120.0)
                            rectangle (axis cs:170.0, -115.0);
                        \path[,fill,
                              black!4]
                            (axis cs:165.0, -115.0)
                            rectangle (axis cs:170.0, -110.0);
                        \path[,fill,
                              black!1]
                            (axis cs:165.0, -110.0)
                            rectangle (axis cs:170.0, -105.0);
                        \path[,fill,
                              black!1]
                            (axis cs:165.0, -105.0)
                            rectangle (axis cs:170.0, -100.0);
                        \path[,fill,
                              black!1]
                            (axis cs:165.0, -100.0)
                            rectangle (axis cs:170.0, -95.0);
                        \path[,fill,
                              black!1]
                            (axis cs:165.0, -95.0)
                            rectangle (axis cs:170.0, -90.0);
                        \path[,fill,
                              black!1]
                            (axis cs:165.0, -50.0)
                            rectangle (axis cs:170.0, -45.0);
                        \path[,fill,
                              black!1]
                            (axis cs:165.0, 35.0)
                            rectangle (axis cs:170.0, 40.0);
                        \path[,fill,
                              black!1]
                            (axis cs:165.0, 40.0)
                            rectangle (axis cs:170.0, 45.0);
                        \path[,fill,
                              black!2]
                            (axis cs:165.0, 45.0)
                            rectangle (axis cs:170.0, 50.0);
                        \path[,fill,
                              black!2]
                            (axis cs:165.0, 50.0)
                            rectangle (axis cs:170.0, 55.0);
                        \path[,fill,
                              black!3]
                            (axis cs:165.0, 55.0)
                            rectangle (axis cs:170.0, 60.0);
                        \path[,fill,
                              black!2]
                            (axis cs:165.0, 60.0)
                            rectangle (axis cs:170.0, 65.0);
                        \path[,fill,
                              black!2]
                            (axis cs:165.0, 65.0)
                            rectangle (axis cs:170.0, 70.0);
                        \path[,fill,
                              black!4]
                            (axis cs:165.0, 70.0)
                            rectangle (axis cs:170.0, 75.0);
                        \path[,fill,
                              black!2]
                            (axis cs:165.0, 75.0)
                            rectangle (axis cs:170.0, 80.0);
                        \path[,fill,
                              black!2]
                            (axis cs:165.0, 80.0)
                            rectangle (axis cs:170.0, 85.0);
                        \path[,fill,
                              black!1]
                            (axis cs:165.0, 85.0)
                            rectangle (axis cs:170.0, 90.0);
                        \path[,fill,
                              black!2]
                            (axis cs:165.0, 90.0)
                            rectangle (axis cs:170.0, 95.0);
                        \path[,fill,
                              black!2]
                            (axis cs:165.0, 95.0)
                            rectangle (axis cs:170.0, 100.0);
                        \path[,fill,
                              black!4]
                            (axis cs:165.0, 100.0)
                            rectangle (axis cs:170.0, 105.0);
                        \path[,fill,
                              black!2]
                            (axis cs:165.0, 105.0)
                            rectangle (axis cs:170.0, 110.0);
                        \path[,fill,
                              black!4]
                            (axis cs:165.0, 110.0)
                            rectangle (axis cs:170.0, 115.0);
                        \path[,fill,
                              black!5]
                            (axis cs:165.0, 115.0)
                            rectangle (axis cs:170.0, 120.0);
                        \path[,fill,
                              black!4]
                            (axis cs:165.0, 120.0)
                            rectangle (axis cs:170.0, 125.0);
                        \path[,fill,
                              black!2]
                            (axis cs:165.0, 125.0)
                            rectangle (axis cs:170.0, 130.0);
                        \path[,fill,
                              black!5]
                            (axis cs:165.0, 130.0)
                            rectangle (axis cs:170.0, 135.0);
                        \path[,fill,
                              black!11]
                            (axis cs:165.0, 135.0)
                            rectangle (axis cs:170.0, 140.0);
                        \path[,fill,
                              black!14]
                            (axis cs:165.0, 140.0)
                            rectangle (axis cs:170.0, 145.0);
                        \path[,fill,
                              black!14]
                            (axis cs:165.0, 145.0)
                            rectangle (axis cs:170.0, 150.0);
                        \path[,fill,
                              black!40]
                            (axis cs:165.0, 150.0)
                            rectangle (axis cs:170.0, 155.0);
                        \path[,fill,
                              black!54]
                            (axis cs:165.0, 155.0)
                            rectangle (axis cs:170.0, 160.0);
                        \path[,fill,
                              black!73]
                            (axis cs:165.0, 160.0)
                            rectangle (axis cs:170.0, 165.0);
                        \path[,fill,
                              black!89]
                            (axis cs:165.0, 165.0)
                            rectangle (axis cs:170.0, 170.0);
                        \path[,fill,
                              black!62]
                            (axis cs:165.0, 170.0)
                            rectangle (axis cs:170.0, 175.0);
                        \path[,fill,
                              black!39]
                            (axis cs:165.0, 175.0)
                            rectangle (axis cs:170.0, 180.0);
                        \path[,fill,
                              black!50]
                            (axis cs:170.0, -180.0)
                            rectangle (axis cs:175.0, -175.0);
                        \path[,fill,
                              black!26]
                            (axis cs:170.0, -175.0)
                            rectangle (axis cs:175.0, -170.0);
                        \path[,fill,
                              black!18]
                            (axis cs:170.0, -170.0)
                            rectangle (axis cs:175.0, -165.0);
                        \path[,fill,
                              black!16]
                            (axis cs:170.0, -165.0)
                            rectangle (axis cs:175.0, -160.0);
                        \path[,fill,
                              black!15]
                            (axis cs:170.0, -160.0)
                            rectangle (axis cs:175.0, -155.0);
                        \path[,fill,
                              black!11]
                            (axis cs:170.0, -155.0)
                            rectangle (axis cs:175.0, -150.0);
                        \path[,fill,
                              black!8]
                            (axis cs:170.0, -150.0)
                            rectangle (axis cs:175.0, -145.0);
                        \path[,fill,
                              black!5]
                            (axis cs:170.0, -145.0)
                            rectangle (axis cs:175.0, -140.0);
                        \path[,fill,
                              black!10]
                            (axis cs:170.0, -140.0)
                            rectangle (axis cs:175.0, -135.0);
                        \path[,fill,
                              black!4]
                            (axis cs:170.0, -135.0)
                            rectangle (axis cs:175.0, -130.0);
                        \path[,fill,
                              black!8]
                            (axis cs:170.0, -130.0)
                            rectangle (axis cs:175.0, -125.0);
                        \path[,fill,
                              black!4]
                            (axis cs:170.0, -125.0)
                            rectangle (axis cs:175.0, -120.0);
                        \path[,fill,
                              black!3]
                            (axis cs:170.0, -120.0)
                            rectangle (axis cs:175.0, -115.0);
                        \path[,fill,
                              black!2]
                            (axis cs:170.0, -115.0)
                            rectangle (axis cs:175.0, -110.0);
                        \path[,fill,
                              black!1]
                            (axis cs:170.0, -110.0)
                            rectangle (axis cs:175.0, -105.0);
                        \path[,fill,
                              black!1]
                            (axis cs:170.0, -105.0)
                            rectangle (axis cs:175.0, -100.0);
                        \path[,fill,
                              black!1]
                            (axis cs:170.0, -100.0)
                            rectangle (axis cs:175.0, -95.0);
                        \path[,fill,
                              black!1]
                            (axis cs:170.0, -95.0)
                            rectangle (axis cs:175.0, -90.0);
                        \path[,fill,
                              black!1]
                            (axis cs:170.0, -75.0)
                            rectangle (axis cs:175.0, -70.0);
                        \path[,fill,
                              black!1]
                            (axis cs:170.0, -40.0)
                            rectangle (axis cs:175.0, -35.0);
                        \path[,fill,
                              black!1]
                            (axis cs:170.0, 15.0)
                            rectangle (axis cs:175.0, 20.0);
                        \path[,fill,
                              black!1]
                            (axis cs:170.0, 35.0)
                            rectangle (axis cs:175.0, 40.0);
                        \path[,fill,
                              black!2]
                            (axis cs:170.0, 40.0)
                            rectangle (axis cs:175.0, 45.0);
                        \path[,fill,
                              black!4]
                            (axis cs:170.0, 45.0)
                            rectangle (axis cs:175.0, 50.0);
                        \path[,fill,
                              black!3]
                            (axis cs:170.0, 50.0)
                            rectangle (axis cs:175.0, 55.0);
                        \path[,fill,
                              black!4]
                            (axis cs:170.0, 55.0)
                            rectangle (axis cs:175.0, 60.0);
                        \path[,fill,
                              black!4]
                            (axis cs:170.0, 60.0)
                            rectangle (axis cs:175.0, 65.0);
                        \path[,fill,
                              black!3]
                            (axis cs:170.0, 65.0)
                            rectangle (axis cs:175.0, 70.0);
                        \path[,fill,
                              black!3]
                            (axis cs:170.0, 70.0)
                            rectangle (axis cs:175.0, 75.0);
                        \path[,fill,
                              black!2]
                            (axis cs:170.0, 75.0)
                            rectangle (axis cs:175.0, 80.0);
                        \path[,fill,
                              black!3]
                            (axis cs:170.0, 80.0)
                            rectangle (axis cs:175.0, 85.0);
                        \path[,fill,
                              black!3]
                            (axis cs:170.0, 85.0)
                            rectangle (axis cs:175.0, 90.0);
                        \path[,fill,
                              black!4]
                            (axis cs:170.0, 90.0)
                            rectangle (axis cs:175.0, 95.0);
                        \path[,fill,
                              black!3]
                            (axis cs:170.0, 95.0)
                            rectangle (axis cs:175.0, 100.0);
                        \path[,fill,
                              black!4]
                            (axis cs:170.0, 100.0)
                            rectangle (axis cs:175.0, 105.0);
                        \path[,fill,
                              black!4]
                            (axis cs:170.0, 105.0)
                            rectangle (axis cs:175.0, 110.0);
                        \path[,fill,
                              black!2]
                            (axis cs:170.0, 110.0)
                            rectangle (axis cs:175.0, 115.0);
                        \path[,fill,
                              black!5]
                            (axis cs:170.0, 115.0)
                            rectangle (axis cs:175.0, 120.0);
                        \path[,fill,
                              black!2]
                            (axis cs:170.0, 120.0)
                            rectangle (axis cs:175.0, 125.0);
                        \path[,fill,
                              black!5]
                            (axis cs:170.0, 125.0)
                            rectangle (axis cs:175.0, 130.0);
                        \path[,fill,
                              black!10]
                            (axis cs:170.0, 130.0)
                            rectangle (axis cs:175.0, 135.0);
                        \path[,fill,
                              black!7]
                            (axis cs:170.0, 135.0)
                            rectangle (axis cs:175.0, 140.0);
                        \path[,fill,
                              black!4]
                            (axis cs:170.0, 140.0)
                            rectangle (axis cs:175.0, 145.0);
                        \path[,fill,
                              black!10]
                            (axis cs:170.0, 145.0)
                            rectangle (axis cs:175.0, 150.0);
                        \path[,fill,
                              black!17]
                            (axis cs:170.0, 150.0)
                            rectangle (axis cs:175.0, 155.0);
                        \path[,fill,
                              black!35]
                            (axis cs:170.0, 155.0)
                            rectangle (axis cs:175.0, 160.0);
                        \path[,fill,
                              black!49]
                            (axis cs:170.0, 160.0)
                            rectangle (axis cs:175.0, 165.0);
                        \path[,fill,
                              black!69]
                            (axis cs:170.0, 165.0)
                            rectangle (axis cs:175.0, 170.0);
                        \path[,fill,
                              black!80]
                            (axis cs:170.0, 170.0)
                            rectangle (axis cs:175.0, 175.0);
                        \path[,fill,
                              black!64]
                            (axis cs:170.0, 175.0)
                            rectangle (axis cs:175.0, 180.0);
                        \path[,fill,
                              black!55]
                            (axis cs:175.0, -180.0)
                            rectangle (axis cs:180.0, -175.0);
                        \path[,fill,
                              black!43]
                            (axis cs:175.0, -175.0)
                            rectangle (axis cs:180.0, -170.0);
                        \path[,fill,
                              black!23]
                            (axis cs:175.0, -170.0)
                            rectangle (axis cs:180.0, -165.0);
                        \path[,fill,
                              black!17]
                            (axis cs:175.0, -165.0)
                            rectangle (axis cs:180.0, -160.0);
                        \path[,fill,
                              black!13]
                            (axis cs:175.0, -160.0)
                            rectangle (axis cs:180.0, -155.0);
                        \path[,fill,
                              black!8]
                            (axis cs:175.0, -155.0)
                            rectangle (axis cs:180.0, -150.0);
                        \path[,fill,
                              black!8]
                            (axis cs:175.0, -150.0)
                            rectangle (axis cs:180.0, -145.0);
                        \path[,fill,
                              black!8]
                            (axis cs:175.0, -145.0)
                            rectangle (axis cs:180.0, -140.0);
                        \path[,fill,
                              black!8]
                            (axis cs:175.0, -140.0)
                            rectangle (axis cs:180.0, -135.0);
                        \path[,fill,
                              black!7]
                            (axis cs:175.0, -135.0)
                            rectangle (axis cs:180.0, -130.0);
                        \path[,fill,
                              black!8]
                            (axis cs:175.0, -130.0)
                            rectangle (axis cs:180.0, -125.0);
                        \path[,fill,
                              black!5]
                            (axis cs:175.0, -125.0)
                            rectangle (axis cs:180.0, -120.0);
                        \path[,fill,
                              black!5]
                            (axis cs:175.0, -120.0)
                            rectangle (axis cs:180.0, -115.0);
                        \path[,fill,
                              black!4]
                            (axis cs:175.0, -115.0)
                            rectangle (axis cs:180.0, -110.0);
                        \path[,fill,
                              black!1]
                            (axis cs:175.0, -110.0)
                            rectangle (axis cs:180.0, -105.0);
                        \path[,fill,
                              black!2]
                            (axis cs:175.0, -105.0)
                            rectangle (axis cs:180.0, -100.0);
                        \path[,fill,
                              black!1]
                            (axis cs:175.0, 20.0)
                            rectangle (axis cs:180.0, 25.0);
                        \path[,fill,
                              black!1]
                            (axis cs:175.0, 25.0)
                            rectangle (axis cs:180.0, 30.0);
                        \path[,fill,
                              black!2]
                            (axis cs:175.0, 30.0)
                            rectangle (axis cs:180.0, 35.0);
                        \path[,fill,
                              black!2]
                            (axis cs:175.0, 35.0)
                            rectangle (axis cs:180.0, 40.0);
                        \path[,fill,
                              black!3]
                            (axis cs:175.0, 40.0)
                            rectangle (axis cs:180.0, 45.0);
                        \path[,fill,
                              black!2]
                            (axis cs:175.0, 50.0)
                            rectangle (axis cs:180.0, 55.0);
                        \path[,fill,
                              black!5]
                            (axis cs:175.0, 55.0)
                            rectangle (axis cs:180.0, 60.0);
                        \path[,fill,
                              black!3]
                            (axis cs:175.0, 60.0)
                            rectangle (axis cs:180.0, 65.0);
                        \path[,fill,
                              black!3]
                            (axis cs:175.0, 65.0)
                            rectangle (axis cs:180.0, 70.0);
                        \path[,fill,
                              black!2]
                            (axis cs:175.0, 70.0)
                            rectangle (axis cs:180.0, 75.0);
                        \path[,fill,
                              black!2]
                            (axis cs:175.0, 75.0)
                            rectangle (axis cs:180.0, 80.0);
                        \path[,fill,
                              black!2]
                            (axis cs:175.0, 80.0)
                            rectangle (axis cs:180.0, 85.0);
                        \path[,fill,
                              black!1]
                            (axis cs:175.0, 85.0)
                            rectangle (axis cs:180.0, 90.0);
                        \path[,fill,
                              black!1]
                            (axis cs:175.0, 90.0)
                            rectangle (axis cs:180.0, 95.0);
                        \path[,fill,
                              black!3]
                            (axis cs:175.0, 95.0)
                            rectangle (axis cs:180.0, 100.0);
                        \path[,fill,
                              black!1]
                            (axis cs:175.0, 100.0)
                            rectangle (axis cs:180.0, 105.0);
                        \path[,fill,
                              black!2]
                            (axis cs:175.0, 105.0)
                            rectangle (axis cs:180.0, 110.0);
                        \path[,fill,
                              black!2]
                            (axis cs:175.0, 110.0)
                            rectangle (axis cs:180.0, 115.0);
                        \path[,fill,
                              black!2]
                            (axis cs:175.0, 115.0)
                            rectangle (axis cs:180.0, 120.0);
                        \path[,fill,
                              black!2]
                            (axis cs:175.0, 120.0)
                            rectangle (axis cs:180.0, 125.0);
                        \path[,fill,
                              black!4]
                            (axis cs:175.0, 125.0)
                            rectangle (axis cs:180.0, 130.0);
                        \path[,fill,
                              black!4]
                            (axis cs:175.0, 130.0)
                            rectangle (axis cs:180.0, 135.0);
                        \path[,fill,
                              black!4]
                            (axis cs:175.0, 135.0)
                            rectangle (axis cs:180.0, 140.0);
                        \path[,fill,
                              black!6]
                            (axis cs:175.0, 140.0)
                            rectangle (axis cs:180.0, 145.0);
                        \path[,fill,
                              black!8]
                            (axis cs:175.0, 145.0)
                            rectangle (axis cs:180.0, 150.0);
                        \path[,fill,
                              black!15]
                            (axis cs:175.0, 150.0)
                            rectangle (axis cs:180.0, 155.0);
                        \path[,fill,
                              black!22]
                            (axis cs:175.0, 155.0)
                            rectangle (axis cs:180.0, 160.0);
                        \path[,fill,
                              black!27]
                            (axis cs:175.0, 160.0)
                            rectangle (axis cs:180.0, 165.0);
                        \path[,fill,
                              black!56]
                            (axis cs:175.0, 165.0)
                            rectangle (axis cs:180.0, 170.0);
                        \path[,fill,
                              black!61]
                            (axis cs:175.0, 170.0)
                            rectangle (axis cs:180.0, 175.0);
                        \path[,fill,
                              black!79]
                            (axis cs:175.0, 175.0)
                            rectangle (axis cs:180.0, 180.0);
        

        

        

        

        

        

        

    \end{axis}
\end{tikzpicture}
\end{sansmath}}
\caption{Simulated versus reconstructed azimuthal angle for showers of
\SI{1}{\peta\electronvolt} protons with $\theta$ = $\SI{22.5}{\degree}$.
Only events with at least \SI{1}{\mip} in all corner detectors are
included.  There is a strong correlation between simulated and
reconstructed azimuthal angles.}
\label{fig:phi-reconstruction}
\end{figure}

\begin{figure}
\centering
{\pgfkeys{/artist/width/.initial=.45\linewidth}
% \usepackage{tikz}
% \usetikzlibrary{arrows}
% \usepackage{pgfplots}
% \pgfplotsset{compat=1.3}
% \usepackage[detect-family]{siunitx}
% \usepackage[eulergreek]{sansmath}
% \sisetup{text-sf=\sansmath}
% \usepackage{relsize}
%
\pgfkeysifdefined{/artist/width}
    {\pgfkeysgetvalue{/artist/width}{\defaultwidth}}
    {\def\defaultwidth{ .67\linewidth }}
%
%
\begin{sansmath}
\begin{tikzpicture}[
        font=\sffamily,
        every pin/.style={inner sep=2pt, font={\sffamily\smaller}},
        every label/.style={inner sep=2pt, font={\sffamily\smaller}},
        every pin edge/.style={<-, >=stealth', shorten <=2pt},
        pin distance=2.5ex,
    ]
    \begin{axis}[
            width=\defaultwidth,
            axis equal=false,
            %
            title={ $N_\mathrm{MIP} \geq 1$ },
            %
            xlabel={ $\phi_\mathrm{sim}$ [\si{\degree}] },
            ylabel={ $\phi_\mathrm{rec} - \phi_\mathrm{sim}$ [\si{\degree}] },
            %
            xmin={ -180 },
            xmax={ 180 },
            ymin={ -17 },
            ymax={ 17 },
            %
            xtick={ -180, -90, ..., 180 },
            ytick={  },
            %
            tick align=outside,
            max space between ticks=40,
            every tick/.style={},
            axis on top,
        ]

        

        
            \addplot[draw=none, fill=lightgray] coordinates {
                (-169.411764706, -14.1713)
                (-148.235294118, -12.468059063)
                (-127.058823529, -11.0449)
                (-105.882352941, -11.2351412773)
                (-84.7058823529, -11.4317622185)
                (-63.5294117647, -13.3315)
                (-42.3529411765, -13.7586979866)
                (-21.1764705882, -11.2813067436)
                (0.0, -10.658823967)
                (21.1764705882, -10.8869824409)
                (42.3529411765, -12.6394)
                (63.5294117647, -13.9865534306)
                (84.7058823529, -13.6414685249)
                (105.882352941, -11.2984335423)
                (127.058823529, -11.5407104492)
                (148.235294118, -11.6513)
                (169.411764706, -12.6269)
                (169.411764706, 11.1297)
                (148.235294118, 11.4016)
                (127.058823529, 12.5679228306)
                (105.882352941, 13.4742546082)
                (84.7058823529, 11.1536848545)
                (63.5294117647, 10.6660635471)
                (42.3529411765, 10.5689)
                (21.1764705882, 11.9597837925)
                (0.0, 14.9547119141)
                (-21.1764705882, 13.6269369125)
                (-42.3529411765, 10.8548159599)
                (-63.5294117647, 11.2296)
                (-84.7058823529, 12.2276055813)
                (-105.882352941, 12.2186751366)
                (-127.058823529, 14.6418)
                (-148.235294118, 12.9003789425)
                (-169.411764706, 11.4052)
                } -- cycle;
        

        
            
            % Draw series plot
            \addplot[mark=*,mark options=white,only marks] coordinates {
                (-169.411764706, -1.33021)
                (-148.235294118, 0.0632817000151)
                (-127.058823529, 1.63209)
                (-105.882352941, 0.90381154418)
                (-84.7058823529, 0.0545185673982)
                (-63.5294117647, -1.58713)
                (-42.3529411765, -1.18046838045)
                (-21.1764705882, 0.693237)
                (0.0, 1.58421)
                (21.1764705882, 0.564699530602)
                (42.3529411765, -1.22859)
                (63.5294117647, -1.96649330854)
                (84.7058823529, -0.946370452642)
                (105.882352941, 1.24150294065)
                (127.058823529, 1.11114192009)
                (148.235294118, 0.108491)
                (169.411764706, -0.859402)
            };
        
            
            % Draw series plot
            \addplot[mark=o,only marks] coordinates {
                (-169.411764706, -1.33021)
                (-148.235294118, 0.0632817000151)
                (-127.058823529, 1.63209)
                (-105.882352941, 0.90381154418)
                (-84.7058823529, 0.0545185673982)
                (-63.5294117647, -1.58713)
                (-42.3529411765, -1.18046838045)
                (-21.1764705882, 0.693237)
                (0.0, 1.58421)
                (21.1764705882, 0.564699530602)
                (42.3529411765, -1.22859)
                (63.5294117647, -1.96649330854)
                (84.7058823529, -0.946370452642)
                (105.882352941, 1.24150294065)
                (127.058823529, 1.11114192009)
                (148.235294118, 0.108491)
                (169.411764706, -0.859402)
            };
        

        
            \draw[gray]
                ({rel axis cs:0, 0} |- {axis cs:0, 0 }) --
                ({rel axis cs:1, 1} |- {axis cs:0, 0 });
        

        

        

        

    \end{axis}
\end{tikzpicture}
\end{sansmath}
% \usepackage{tikz}
% \usetikzlibrary{arrows}
% \usepackage{pgfplots}
% \pgfplotsset{compat=1.3}
% \usepackage[detect-family]{siunitx}
% \usepackage[eulergreek]{sansmath}
% \sisetup{text-sf=\sansmath}
% \usepackage{relsize}
%
\pgfkeysifdefined{/artist/width}
    {\pgfkeysgetvalue{/artist/width}{\defaultwidth}}
    {\def\defaultwidth{ .67\linewidth }}
%
%
\begin{sansmath}
\begin{tikzpicture}[
        font=\sffamily,
        every pin/.style={inner sep=2pt, font={\sffamily\smaller}},
        every label/.style={inner sep=2pt, font={\sffamily\smaller}},
        every pin edge/.style={<-, >=stealth', shorten <=2pt},
        pin distance=2.5ex,
    ]
    \begin{axis}[
            width=\defaultwidth,
            axis equal=false,
            %
            title={ $N_\mathrm{MIP} \geq 2$ },
            %
            xlabel={ $\phi_\mathrm{sim}$ [\si{\degree}] },
            ylabel={ $\phi_\mathrm{rec} - \phi_\mathrm{sim}$ [\si{\degree}] },
            %
            xmin={ -180 },
            xmax={ 180 },
            ymin={ -17 },
            ymax={ 17 },
            %
            xtick={ -180, -90, ..., 180 },
            ytick={  },
            %
            tick align=outside,
            max space between ticks=40,
            every tick/.style={},
            axis on top,
        ]

        

        
            \addplot[draw=none, fill=lightgray] coordinates {
                (-169.411764706, -7.53644394875)
                (-148.235294118, -6.33169)
                (-127.058823529, -6.18743395805)
                (-105.882352941, -6.3608045578)
                (-84.7058823529, -5.80275774002)
                (-63.5294117647, -7.29410874844)
                (-42.3529411765, -7.13015341759)
                (-21.1764705882, -6.37966)
                (0.0, -6.10157)
                (21.1764705882, -6.24220514297)
                (42.3529411765, -7.04852592945)
                (63.5294117647, -8.03421449661)
                (84.7058823529, -7.14956498146)
                (105.882352941, -6.80750584602)
                (127.058823529, -6.52677822113)
                (148.235294118, -6.21431064606)
                (169.411764706, -6.85467505455)
                (169.411764706, 6.42060303688)
                (148.235294118, 6.47482776642)
                (127.058823529, 6.68469905853)
                (105.882352941, 6.91910088062)
                (84.7058823529, 6.16594612598)
                (63.5294117647, 6.35390615463)
                (42.3529411765, 6.27192318439)
                (21.1764705882, 6.58482098579)
                (0.0, 7.23066)
                (-21.1764705882, 6.88208)
                (-42.3529411765, 6.18765246868)
                (-63.5294117647, 6.09555435181)
                (-84.7058823529, 6.75742006302)
                (-105.882352941, 6.2915122509)
                (-127.058823529, 7.42415571213)
                (-148.235294118, 7.36006)
                (-169.411764706, 6.24600958824)
                } -- cycle;
        

        
            
            % Draw series plot
            \addplot[mark=*,mark options=white,only marks] coordinates {
                (-169.411764706, -0.615878)
                (-148.235294118, 0.465027)
                (-127.058823529, 0.786496)
                (-105.882352941, 0.525726467371)
                (-84.7058823529, 0.285461)
                (-63.5294117647, -0.983492553234)
                (-42.3529411765, -0.567916542292)
                (-21.1764705882, 0.260244)
                (0.0, 0.52128)
                (21.1764705882, 0.278153)
                (42.3529411765, -0.531586766243)
                (63.5294117647, -0.904167)
                (84.7058823529, -0.497907102108)
                (105.882352941, 0.356727093458)
                (127.058823529, 0.634374)
                (148.235294118, 0.111906)
                (169.411764706, -0.184538)
            };
        
            
            % Draw series plot
            \addplot[mark=o,only marks] coordinates {
                (-169.411764706, -0.615878)
                (-148.235294118, 0.465027)
                (-127.058823529, 0.786496)
                (-105.882352941, 0.525726467371)
                (-84.7058823529, 0.285461)
                (-63.5294117647, -0.983492553234)
                (-42.3529411765, -0.567916542292)
                (-21.1764705882, 0.260244)
                (0.0, 0.52128)
                (21.1764705882, 0.278153)
                (42.3529411765, -0.531586766243)
                (63.5294117647, -0.904167)
                (84.7058823529, -0.497907102108)
                (105.882352941, 0.356727093458)
                (127.058823529, 0.634374)
                (148.235294118, 0.111906)
                (169.411764706, -0.184538)
            };
        

        
            \draw[gray]
                ({rel axis cs:0, 0} |- {axis cs:0, 0 }) --
                ({rel axis cs:1, 1} |- {axis cs:0, 0 });
        

        

        

        

    \end{axis}
\end{tikzpicture}
\end{sansmath}
}
\caption{The uncertainty of the reconstruction of the azimuthal angle as a
function of the simulated angle, for $\theta$ = $\SI{22.5}{\degree}$.  The
dots show the median values and the shaded region contains
\SI{50}{\percent} of the events, evenly distributed around the median. On
the left, minimum reconstruction requirements are used.  That is, at least
1 particle in all three corner detectors.  On the right, at least 2
particles in all three corner detectors are selected.  The more stringent
requirements result in a more accurate reconstruction.}
\label{fig:phi-reconstruction-error}
\end{figure}

The uncertainty in the reconstruction of the zenith angle is shown in
\figref{fig:theta-reconstruction-error}, for $\SI{0}{\degree} \leq \theta \leq
\SI{45}{\degree}$.
With increasing $\theta$, the arrival time difference between detectors
increases (\eqref{eq:time_angle_relation}) while the absolute uncertainty in the
measurements, due to instrumentation and the structure of the shower front, is
not changed. The result is that the \emph{relative} timing uncertainty decreases
and the reconstruction becomes more accurate.
The right hand figure shows more stringent reconstruction requirements. When
more particles traverse the detectors, the reconstruction uncertainty reduces.

\begin{figure}
\centering
{\pgfkeys{/artist/width/.initial=.45\linewidth}
% \usepackage{tikz}
% \usetikzlibrary{arrows}
% \usepackage{pgfplots}
% \pgfplotsset{compat=1.3}
% \usepackage[detect-family]{siunitx}
% \usepackage[eulergreek]{sansmath}
% \sisetup{text-sf=\sansmath}
% \usepackage{relsize}
%
\pgfkeysifdefined{/artist/width}
    {\pgfkeysgetvalue{/artist/width}{\defaultwidth}}
    {\def\defaultwidth{ .67\linewidth }}
%
%
\begin{sansmath}
\begin{tikzpicture}[
        font=\sffamily,
        every pin/.style={inner sep=2pt, font={\sffamily\smaller}},
        every label/.style={inner sep=2pt, font={\sffamily\smaller}},
        every pin edge/.style={<-, >=stealth', shorten <=2pt},
        pin distance=2.5ex,
    ]
    \begin{axis}[
            width=\defaultwidth,
            axis equal=false,
            %
            title={ $N_\mathrm{MIP} \geq 1$ },
            %
            xlabel={ $\theta_\mathrm{sim}$ [\si{\degree}] },
            ylabel={ $\theta_\mathrm{rec} - \theta_\mathrm{sim}$ [\si{\degree}] },
            %
            xmin={  },
            xmax={  },
            ymin={ -8 },
            ymax={ 22 },
            %
            xtick={  },
            ytick={  },
            %
            tick align=outside,
            max space between ticks=40,
            every tick/.style={},
            axis on top,
        ]

        

        
            \addplot[draw=none, fill=lightgray] coordinates {
                (0, 3.91251671314)
                (5, 0.838356360793)
                (10, -0.443454489112)
                (15, -1.28439587355)
                (22.5, -2.34810185432)
                (30, -3.46658885479)
                (35, -4.20188522339)
                (45, -5.60682761669)
                (45, 6.06249201298)
                (35, 6.44192755222)
                (30, 6.96421599388)
                (22.5, 8.31906414032)
                (15, 10.3945536613)
                (10, 12.3053934574)
                (5, 15.63880229)
                (0, 20.2432279587)
                } -- cycle;
        

        
            
            % Draw series plot
            \addplot[mark=*,mark options=white,only marks] coordinates {
                (0, 8.85276)
                (5, 5.16502213478)
                (10, 3.35315263271)
                (15, 2.34780311584)
                (22.5, 1.50870335102)
                (30, 0.977362453938)
                (35, 0.648010760546)
                (45, 0.0836100466549)
            };
        
            
            % Draw series plot
            \addplot[mark=o,only marks] coordinates {
                (0, 8.85276)
                (5, 5.16502213478)
                (10, 3.35315263271)
                (15, 2.34780311584)
                (22.5, 1.50870335102)
                (30, 0.977362453938)
                (35, 0.648010760546)
                (45, 0.0836100466549)
            };
        

        
            \draw[gray]
                ({rel axis cs:0, 0} |- {axis cs:0, 0 }) --
                ({rel axis cs:1, 1} |- {axis cs:0, 0 });
        

        

        

        

    \end{axis}
\end{tikzpicture}
\end{sansmath}
% \usepackage{tikz}
% \usetikzlibrary{arrows}
% \usepackage{pgfplots}
% \pgfplotsset{compat=1.3}
% \usepackage[detect-family]{siunitx}
% \usepackage[eulergreek]{sansmath}
% \sisetup{text-sf=\sansmath}
% \usepackage{relsize}
%
\pgfkeysifdefined{/artist/width}
    {\pgfkeysgetvalue{/artist/width}{\defaultwidth}}
    {\def\defaultwidth{ .67\linewidth }}
%
%
\begin{sansmath}
\begin{tikzpicture}[
        font=\sffamily,
        every pin/.style={inner sep=2pt, font={\sffamily\smaller}},
        every label/.style={inner sep=2pt, font={\sffamily\smaller}},
        every pin edge/.style={<-, >=stealth', shorten <=2pt},
        pin distance=2.5ex,
    ]
    \begin{axis}[
            width=\defaultwidth,
            axis equal=false,
            %
            title={ $N_\mathrm{MIP} \geq 2$ },
            %
            xlabel={ $\theta_\mathrm{sim}$ [\si{\degree}] },
            ylabel={ $\theta_\mathrm{rec} - \theta_\mathrm{sim}$ [\si{\degree}] },
            %
            xmin={  },
            xmax={  },
            ymin={ -8 },
            ymax={ 22 },
            %
            xtick={  },
            ytick={  },
            %
            tick align=outside,
            max space between ticks=40,
            every tick/.style={},
            axis on top,
        ]

        

        
            \addplot[draw=none, fill=lightgray] coordinates {
                (0, 2.37448036671)
                (5, -0.258547618985)
                (10, -1.20961660147)
                (15, -1.6652713418)
                (22.5, -2.09249830246)
                (30, -2.49825)
                (35, -2.73837125301)
                (45, -3.38497)
                (45, 3.6834)
                (35, 3.21308481693)
                (30, 3.32601)
                (22.5, 3.39965987206)
                (15, 3.81269168854)
                (10, 4.20867586136)
                (5, 5.51892781258)
                (0, 9.11929249763)
                } -- cycle;
        

        
            
            % Draw series plot
            \addplot[mark=*,mark options=white,only marks] coordinates {
                (0, 4.80685019493)
                (5, 1.79072)
                (10, 0.925116)
                (15, 0.661529392004)
                (22.5, 0.39332)
                (30, 0.297892)
                (35, 0.0979654)
                (45, 0.121622)
            };
        
            
            % Draw series plot
            \addplot[mark=o,only marks] coordinates {
                (0, 4.80685019493)
                (5, 1.79072)
                (10, 0.925116)
                (15, 0.661529392004)
                (22.5, 0.39332)
                (30, 0.297892)
                (35, 0.0979654)
                (45, 0.121622)
            };
        

        
            \draw[gray]
                ({rel axis cs:0, 0} |- {axis cs:0, 0 }) --
                ({rel axis cs:1, 1} |- {axis cs:0, 0 });
        

        

        

        

    \end{axis}
\end{tikzpicture}
\end{sansmath}
}
\caption{The uncertainty of the reconstruction of the zenith angle as a
function of the simulated angle. The dots show the median values and the
shaded region contains \SI{50}{\percent} of the events, evenly distributed
around the median.  On the left, minimum reconstruction requirements are
used.  That is, at least 1 particle in all three corner detectors.  On the
right, at least 2 particles in all three corner detectors are selected.
The more stringent requirements result in more accurate reconstructions.
With increasing zenith angle, the reconstruction uncertainty decreases.}
\label{fig:theta-reconstruction-error}
\end{figure}

The timing accuracy resulting from the plane front approximation and the
transport time of scintillation photons in the detector is given by
\begin{equation}
\sigma_t = \sqrt{\sigma_{t,\, \mathrm{front}}^2 + \sigma_{t,\,
\mathrm{transport}}^2}.
\end{equation}
For all
events, the difference in azimuthal angles $\phi_\mathrm{sim} -
\phi_\mathrm{rec}$ is determined. Then, the value for the angle
difference containing \SI{66}{\percent} of the events will be taken as the
reconstruction uncertainty. This estimator behaves similarly to the standard
deviation, but is more robust for distributions with long tails. The same
procedure is followed for the zenith angle.
The uncertainties will be compared to the results from
\eqtworef{eq:errphi}{eq:errtheta}.

The first particle that traverses the detector determines the measured arrival
time. In \cite{Steijger:2010-firsthit} the distribution of the arrival times of
particles in the shower front is approximated by a normal distribution.
The standard deviation of the distribution that arises from taking $N$ samples
is calculated and is equivalent to detecting $N$ particles in a detector.
The standard deviation is taken as an estimate for the uncertainty on the
arrival time.  The angular uncertainty can then be calculated using
\eqtworef{eq:errphi}{eq:errtheta}.

\figref{fig:results-sim-mip} presents the zenith and azimuthal reconstruction
uncertainties as a function of the \emph{minimum} number of particles in the
three corner detectors, for \SI{1}{\peta\electronvolt} proton showers with a
zenith angle of \SI{22.5}{\degree}. The open circles are the results for the
zenith angle reconstruction, and the closed circles depict the azimuthal angle.
The dashed line is the calculation performed according to
\cite{Steijger:2010-firsthit}.
\begin{figure}
\centering
% \usepackage{tikz}
% \usetikzlibrary{arrows}
% \usepackage{pgfplots}
% \pgfplotsset{compat=1.3}
% \usepackage[detect-family]{siunitx}
% \usepackage[eulergreek]{sansmath}
% \sisetup{text-sf=\sansmath}
% \usepackage{relsize}
%
\pgfkeysifdefined{/artist/width}
    {\pgfkeysgetvalue{/artist/width}{\defaultwidth}}
    {\def\defaultwidth{ .67\linewidth }}
%
%
\begin{sansmath}
\begin{tikzpicture}[
        font=\sffamily,
        every pin/.style={inner sep=2pt, font={\sffamily\smaller}},
        every label/.style={inner sep=2pt, font={\sffamily\smaller}},
        every pin edge/.style={<-, >=stealth', shorten <=2pt},
        pin distance=2.5ex,
    ]
    \begin{axis}[
            width=\defaultwidth,
            %
            title={  },
            %
            xlabel={ Minimum number of particles },
            ylabel={ Reconstruction uncertainty [\si{\degree}] },
            %
            xmin={  },
            xmax={  },
            ymin={ 0 },
            ymax={ 32 },
            %
            xtick={ 1, 2, 3, 4 },
            ytick={  },
            %
            tick align=outside,
            max space between ticks=40,
            every tick/.style={},
        ]

        

        
            
            % Draw series plot
            \addplot[mark=*,mark options={fill=white},only marks] coordinates {
                (1, 8.86094818574)
                (2, 4.30924885605)
                (3, 3.04931200023)
                (4, 2.43898152732)
            };
        
            
            % Draw series plot
            \addplot[mark=*,only marks] coordinates {
                (1, 21.3475953316)
                (2, 10.4979834056)
                (3, 7.4502241806)
                (4, 6.00999457787)
            };
        
            
            % Draw series plot
            \addplot[no markers,dashed,smooth] coordinates {
                (1.0, 11.4338299661)
                (1.0303030303, 11.3476624759)
                (1.06060606061, 11.263402156)
                (1.09090909091, 11.1810202305)
                (1.12121212121, 11.1004879237)
                (1.15151515152, 11.0217764595)
                (1.18181818182, 10.9448570623)
                (1.21212121212, 10.869700956)
                (1.24242424242, 10.796279365)
                (1.27272727273, 10.7245635133)
                (1.30303030303, 10.654524625)
                (1.33333333333, 10.5861339243)
                (1.36363636364, 10.5193626354)
                (1.39393939394, 10.4541819824)
                (1.42424242424, 10.3905631894)
                (1.45454545455, 10.3284774807)
                (1.48484848485, 10.2678960802)
                (1.51515151515, 10.2087902122)
                (1.54545454545, 10.1511311009)
                (1.57575757576, 10.0948899703)
                (1.60606060606, 10.0400380446)
                (1.63636363636, 9.98654654793)
                (1.66666666667, 9.93438670448)
                (1.69696969697, 9.88352973838)
                (1.72727272727, 9.83394687377)
                (1.75757575758, 9.78560933479)
                (1.78787878788, 9.73848834558)
                (1.81818181818, 9.69255513029)
                (1.84848484848, 9.64778091307)
                (1.87878787879, 9.60413691806)
                (1.90909090909, 9.5615943694)
                (1.93939393939, 9.52012449124)
                (1.9696969697, 9.47969850772)
                (2.0, 9.44028764299)
                (2.0303030303, 9.40186399357)
                (2.06060606061, 9.36440314552)
                (2.09090909091, 9.32788155729)
                (2.12121212121, 9.2922756873)
                (2.15151515152, 9.257561994)
                (2.18181818182, 9.22371693583)
                (2.21212121212, 9.19071697122)
                (2.24242424242, 9.15853855862)
                (2.27272727273, 9.12715815646)
                (2.30303030303, 9.09655222318)
                (2.33333333333, 9.06669721723)
                (2.36363636364, 9.03756959703)
                (2.39393939394, 9.00914582104)
                (2.42424242424, 8.98140234768)
                (2.45454545455, 8.95431563541)
                (2.48484848485, 8.92786214264)
                (2.51515151515, 8.90201832784)
                (2.54545454545, 8.87676064943)
                (2.57575757576, 8.85206556585)
                (2.60606060606, 8.82790953554)
                (2.63636363636, 8.80426901695)
                (2.66666666667, 8.7811204685)
                (2.69696969697, 8.75844034865)
                (2.72727272727, 8.73620511582)
                (2.75757575758, 8.71439122846)
                (2.78787878788, 8.69297514501)
                (2.81818181818, 8.6719333239)
                (2.84848484848, 8.65124222358)
                (2.87878787879, 8.63087830248)
                (2.90909090909, 8.61081801904)
                (2.93939393939, 8.5910378317)
                (2.9696969697, 8.57151419891)
                (3.0, 8.55222357909)
                (3.0303030303, 8.53314592022)
                (3.06060606061, 8.51427512838)
                (3.09090909091, 8.4956085992)
                (3.12121212121, 8.47714372828)
                (3.15151515152, 8.45887791125)
                (3.18181818182, 8.44080854372)
                (3.21212121212, 8.4229330213)
                (3.24242424242, 8.40524873962)
                (3.27272727273, 8.38775309429)
                (3.30303030303, 8.37044348093)
                (3.33333333333, 8.35331729516)
                (3.36363636364, 8.33637193258)
                (3.39393939394, 8.31960478882)
                (3.42424242424, 8.3030132595)
                (3.45454545455, 8.28659474023)
                (3.48484848485, 8.27034662662)
                (3.51515151515, 8.2542663143)
                (3.54545454545, 8.23835119888)
                (3.57575757576, 8.22259867598)
                (3.60606060606, 8.20700614121)
                (3.63636363636, 8.19157099019)
                (3.66666666667, 8.17629061854)
                (3.69696969697, 8.16116242187)
                (3.72727272727, 8.1461837958)
                (3.75757575758, 8.13135213595)
                (3.78787878788, 8.11666483794)
                (3.81818181818, 8.10211929737)
                (3.84848484848, 8.08771290987)
                (3.87878787879, 8.07344307105)
                (3.90909090909, 8.05930717654)
                (3.93939393939, 8.04530262194)
                (3.9696969697, 8.03142680287)
                (4.0, 8.01767711496)
            };
        
            
            % Draw series plot
            \addplot[no markers,dashed,smooth] coordinates {
                (1.0, 4.73604744185)
                (1.0303030303, 4.70035569876)
                (1.06060606061, 4.66545393148)
                (1.09090909091, 4.63133022065)
                (1.12121212121, 4.59797264694)
                (1.15151515152, 4.56536929098)
                (1.18181818182, 4.53350823343)
                (1.21212121212, 4.50237755493)
                (1.24242424242, 4.47196533615)
                (1.27272727273, 4.44225965773)
                (1.30303030303, 4.41324860031)
                (1.33333333333, 4.38492024456)
                (1.36363636364, 4.35726267112)
                (1.39393939394, 4.33026396063)
                (1.42424242424, 4.30391219376)
                (1.45454545455, 4.27819545115)
                (1.48484848485, 4.25310181346)
                (1.51515151515, 4.22861936132)
                (1.54545454545, 4.2047361754)
                (1.57575757576, 4.18144033634)
                (1.60606060606, 4.1587199248)
                (1.63636363636, 4.13656302142)
                (1.66666666667, 4.11495770686)
                (1.69696969697, 4.09389206176)
                (1.72727272727, 4.07335416677)
                (1.75757575758, 4.05333210255)
                (1.78787878788, 4.03381394975)
                (1.81818181818, 4.01478778902)
                (1.84848484848, 3.996241701)
                (1.87878787879, 3.97816376635)
                (1.90909090909, 3.96054206571)
                (1.93939393939, 3.94336467975)
                (1.9696969697, 3.9266196891)
                (2.0, 3.91029517443)
                (2.0303030303, 3.89437957772)
                (2.06060606061, 3.8788627864)
                (2.09090909091, 3.86373504924)
                (2.12121212121, 3.84898661499)
                (2.15151515152, 3.83460773242)
                (2.18181818182, 3.82058865031)
                (2.21212121212, 3.80691961741)
                (2.24242424242, 3.7935908825)
                (2.27272727273, 3.78059269433)
                (2.30303030303, 3.76791530168)
                (2.33333333333, 3.75554895331)
                (2.36363636364, 3.74348389798)
                (2.39393939394, 3.73171038447)
                (2.42424242424, 3.72021866154)
                (2.45454545455, 3.70899897795)
                (2.48484848485, 3.69804158248)
                (2.51515151515, 3.68733672388)
                (2.54545454545, 3.67687465093)
                (2.57575757576, 3.66664561239)
                (2.60606060606, 3.65663985702)
                (2.63636363636, 3.6468476336)
                (2.66666666667, 3.63725919089)
                (2.69696969697, 3.62786477765)
                (2.72727272727, 3.61865464265)
                (2.75757575758, 3.60961903466)
                (2.78787878788, 3.60074820244)
                (2.81818181818, 3.59203239476)
                (2.84848484848, 3.58346186038)
                (2.87878787879, 3.57502684808)
                (2.90909090909, 3.56671760661)
                (2.93939393939, 3.55852438475)
                (2.9696969697, 3.55043743126)
                (3.0, 3.54244699491)
                (3.0303030303, 3.53454476986)
                (3.06060606061, 3.52672823195)
                (3.09090909091, 3.5189963024)
                (3.12121212121, 3.51134790244)
                (3.15151515152, 3.5037819533)
                (3.18181818182, 3.4962973762)
                (3.21212121212, 3.48889309238)
                (3.24242424242, 3.48156802307)
                (3.27272727273, 3.47432108949)
                (3.30303030303, 3.46715121288)
                (3.33333333333, 3.46005731446)
                (3.36363636364, 3.45303831546)
                (3.39393939394, 3.44609313712)
                (3.42424242424, 3.43922070065)
                (3.45454545455, 3.43241992729)
                (3.48484848485, 3.42568973827)
                (3.51515151515, 3.41902905482)
                (3.54545454545, 3.41243679817)
                (3.57575757576, 3.40591188954)
                (3.60606060606, 3.39945325017)
                (3.63636363636, 3.39305980128)
                (3.66666666667, 3.3867304641)
                (3.69696969697, 3.38046415987)
                (3.72727272727, 3.37425980981)
                (3.75757575758, 3.36811633514)
                (3.78787878788, 3.36203265711)
                (3.81818181818, 3.35600769694)
                (3.84848484848, 3.35004037585)
                (3.87878787879, 3.34412961508)
                (3.90909090909, 3.33827433585)
                (3.93939393939, 3.3324734594)
                (3.9696969697, 3.32672590696)
                (4.0, 3.32103059974)
            };
        

        

        

        
    \end{axis}
\end{tikzpicture}
\end{sansmath}
\caption{Dependence of reconstruction accuracy on the minimum number of
particles in all three corner detectors.  Data for the most probable
zenith angle ($\theta$ = $\SI{22.5}{\degree}$) is shown.  The open circles
are the uncertainties for the zenith angle, the closed circles for the
azimuthal angle. The dashed lines show the estimates for the uncertainties
as calculated in \cite{Steijger:2010-firsthit}, with a standard deviation
of \SI{1.8}{\nano\second}.}
\label{fig:results-sim-mip}
\end{figure}
The results clearly show an increase in accuracy for larger particle numbers.
When more than one particle hits a detector, the shower front is better defined.
The probability that the first particle travels close to the causal front
increases with the number of particles considered. However, the data points fall
off much steeper than the calculated uncertainty.
This is attributed to the fact that the shower front time structure can not be
approximated by a normal distribution. As seen in
\figref{fig:front-arrival-time}, the distribution has long tails. To better
understand the effect of the front structure on the presented results, the
simulated arrival time distribution is used as the basis for a Monte Carlo
procedure. First, consider that the time structure of the shower front is a
function of the core distance (\figref{fig:arrival-time-distributions}). Therefore, for each
value of the minimum number of particles ($N_\mathrm{\mip}$) in the detectors,
the median core distance $R_0$ is determined (\figref{fig:mip-core-dists}).
\begin{figure}
\centering
% \usepackage{tikz}
% \usetikzlibrary{arrows}
% \usepackage{pgfplots}
% \pgfplotsset{compat=1.3}
% \usepackage[detect-family]{siunitx}
% \usepackage[eulergreek]{sansmath}
% \sisetup{text-sf=\sansmath}
% \usepackage{relsize}
%
\pgfkeysifdefined{/artist/width}
    {\pgfkeysgetvalue{/artist/width}{\defaultwidth}}
    {\def\defaultwidth{ .67\linewidth }}
%
%
\begin{sansmath}
\begin{tikzpicture}[
        font=\sffamily,
        every pin/.style={inner sep=2pt, font={\sffamily\smaller}},
        every label/.style={inner sep=2pt, font={\sffamily\smaller}},
        every pin edge/.style={<-, >=stealth', shorten <=2pt},
        pin distance=2.5ex,
    ]
    \begin{axis}[
            width=\defaultwidth,
            axis equal=false,
            %
            title={  },
            %
            xlabel={ Minimum number of particles },
            ylabel={ Core distance [\si{\meter}] },
            %
            xmin={  },
            xmax={  },
            ymin={ 0 },
            ymax={  },
            %
            xtick={ 0, 1, 2, 3, 4 },
            ytick={  },
            %
            tick align=outside,
            max space between ticks=40,
            every tick/.style={},
            axis on top,
        ]

        

        
            \addplot[draw=none, fill=lightgray] coordinates {
                (1, 18.1133012772)
                (2, 12.751144886)
                (3, 10.0411520004)
                (4, 8.55329823494)
                (4, 18.8099541664)
                (3, 22.1670970917)
                (2, 27.8337697983)
                (1, 40.3014287949)
                } -- cycle;
        

        
            
            % Draw series plot
            \addplot[mark=*,mark options=white,only marks] coordinates {
                (1, 28.1280117035)
                (2, 19.7581)
                (3, 15.7564)
                (4, 13.2933115959)
            };
        
            
            % Draw series plot
            \addplot[mark=o,only marks] coordinates {
                (1, 28.1280117035)
                (2, 19.7581)
                (3, 15.7564)
                (4, 13.2933115959)
            };
        

        

        

        

        

    \end{axis}
\end{tikzpicture}
\end{sansmath}
\caption{Distribution of core distances as a function of the minimum
number of particles in all three corner detectors.  Events with a larger
number of detected particles are generally found closer to the shower
core.  The results are obtained from showers produced by
\SI{1}{\peta\electronvolt} protons with $\theta$ = $\SI{22.5}{\degree}$.}
\label{fig:mip-core-dists}
\end{figure}
The arrival time of
leptons in the simulated \SI{1}{\peta\electronvolt} proton showers ($\theta =
\SI{22.5}{\degree}$) is determined for core distances $R = R_0 \pm
\Delta R$, with $\Delta R = \SI{5}{\meter}$ chosen in accordance with the size
of the \hisparc station. The arrival times are corrected for the passage of the causal
front (\eqref{eq:time_angle_relation}). The arrival times of particles in
different parts of the shower can thus be compared to each other. For events with at least two
particles in the three corner detectors, the median core distance is $R \approx
\SI{20}{\meter}$. The arrival time distribution is given in
\figref{fig:montecarlo-arrival-times} (gray line). Using this distribution as
the basis for a Monte Carlo, we can simulate the arrival times of any number of
particles in a detector. Drawing two random
numbers from this distribution and taking the smallest value as the arrival time
measurement in the detector, and repeating this process, one obtains the
distribution depicted as the black line in
\figref{fig:montecarlo-arrival-times}. The median value of this distribution is
used as the uncertainty in the timing measurements.
\begin{figure}
\centering
% \usepackage{tikz}
% \usetikzlibrary{arrows}
% \usepackage{pgfplots}
% \pgfplotsset{compat=1.3}
% \usepackage[detect-family]{siunitx}
% \usepackage[eulergreek]{sansmath}
% \sisetup{text-sf=\sansmath}
% \usepackage{relsize}
%
\pgfkeysifdefined{/artist/width}
    {\pgfkeysgetvalue{/artist/width}{\defaultwidth}}
    {\def\defaultwidth{ .67\linewidth }}
%
%
\begin{sansmath}
\begin{tikzpicture}[
        font=\sffamily,
        every pin/.style={inner sep=2pt, font={\sffamily\smaller}},
        every label/.style={inner sep=2pt, font={\sffamily\smaller}},
        every pin edge/.style={<-, >=stealth', shorten <=2pt},
        pin distance=2.5ex,
    ]
    \begin{axis}[
            width=\defaultwidth,
            %
            title={  },
            %
            xlabel={ Arrival time [\si{\nano\second}] },
            ylabel={ Number of events },
            %
            xmin={ 0 },
            xmax={ 20 },
            ymin={ 0 },
            ymax={  },
            %
            xtick={  },
            ytick={  },
            %
            tick align=outside,
            max space between ticks=40,
            every tick/.style={},
            axis on top,
        ]

        

        

        
            
            % Draw series plot
            \addplot[no markers,black!50,const plot] coordinates {
                (0.0, 2568)
                (0.2, 4608)
                (0.4, 5328)
                (0.6, 5029)
                (0.8, 4545)
                (1.0, 4146)
                (1.2, 3589)
                (1.4, 3368)
                (1.6, 3051)
                (1.8, 2904)
                (2.0, 2651)
                (2.2, 2502)
                (2.4, 2300)
                (2.6, 2135)
                (2.8, 2131)
                (3.0, 1887)
                (3.2, 1802)
                (3.4, 1746)
                (3.6, 1619)
                (3.8, 1405)
                (4.0, 1404)
                (4.2, 1289)
                (4.4, 1246)
                (4.6, 1155)
                (4.8, 1148)
                (5.0, 1123)
                (5.2, 1025)
                (5.4, 1041)
                (5.6, 1016)
                (5.8, 869)
                (6.0, 796)
                (6.2, 869)
                (6.4, 879)
                (6.6, 765)
                (6.8, 726)
                (7.0, 720)
                (7.2, 656)
                (7.4, 660)
                (7.6, 628)
                (7.8, 599)
                (8.0, 598)
                (8.2, 532)
                (8.4, 542)
                (8.6, 600)
                (8.8, 546)
                (9.0, 499)
                (9.2, 435)
                (9.4, 445)
                (9.6, 391)
                (9.8, 444)
                (10.0, 419)
                (10.2, 388)
                (10.4, 393)
                (10.6, 389)
                (10.8, 379)
                (11.0, 359)
                (11.2, 340)
                (11.4, 305)
                (11.6, 322)
                (11.8, 315)
                (12.0, 319)
                (12.2, 273)
                (12.4, 288)
                (12.6, 321)
                (12.8, 291)
                (13.0, 243)
                (13.2, 276)
                (13.4, 260)
                (13.6, 237)
                (13.8, 225)
                (14.0, 233)
                (14.2, 230)
                (14.4, 221)
                (14.6, 237)
                (14.8, 241)
                (15.0, 194)
                (15.2, 188)
                (15.4, 183)
                (15.6, 227)
                (15.8, 242)
                (16.0, 176)
                (16.2, 195)
                (16.4, 180)
                (16.6, 154)
                (16.8, 161)
                (17.0, 165)
                (17.2, 140)
                (17.4, 160)
                (17.6, 170)
                (17.8, 164)
                (18.0, 144)
                (18.2, 165)
                (18.4, 142)
                (18.6, 137)
                (18.8, 138)
                (19.0, 137)
                (19.2, 112)
                (19.4, 112)
                (19.6, 114)
                (19.8, 100)
                (20.0, 100)
            };
        
            
            % Draw series plot
            \addplot[no markers,solid,const plot] coordinates {
                (0.0, 4995)
                (0.2, 8722)
                (0.4, 9288)
                (0.6, 8638)
                (0.8, 7415)
                (1.0, 6133)
                (1.2, 5347)
                (1.4, 4515)
                (1.6, 4074)
                (1.8, 3646)
                (2.0, 3097)
                (2.2, 2794)
                (2.4, 2542)
                (2.6, 2263)
                (2.8, 2170)
                (3.0, 1887)
                (3.2, 1700)
                (3.4, 1504)
                (3.6, 1347)
                (3.8, 1181)
                (4.0, 1100)
                (4.2, 989)
                (4.4, 972)
                (4.6, 901)
                (4.8, 772)
                (5.0, 814)
                (5.2, 724)
                (5.4, 650)
                (5.6, 631)
                (5.8, 504)
                (6.0, 481)
                (6.2, 505)
                (6.4, 462)
                (6.6, 405)
                (6.8, 407)
                (7.0, 359)
                (7.2, 353)
                (7.4, 290)
                (7.6, 302)
                (7.8, 326)
                (8.0, 255)
                (8.2, 249)
                (8.4, 227)
                (8.6, 192)
                (8.8, 215)
                (9.0, 177)
                (9.2, 177)
                (9.4, 178)
                (9.6, 151)
                (9.8, 117)
                (10.0, 120)
                (10.2, 137)
                (10.4, 130)
                (10.6, 124)
                (10.8, 100)
                (11.0, 89)
                (11.2, 100)
                (11.4, 89)
                (11.6, 81)
                (11.8, 83)
                (12.0, 81)
                (12.2, 91)
                (12.4, 78)
                (12.6, 87)
                (12.8, 81)
                (13.0, 54)
                (13.2, 53)
                (13.4, 58)
                (13.6, 46)
                (13.8, 59)
                (14.0, 41)
                (14.2, 49)
                (14.4, 54)
                (14.6, 39)
                (14.8, 47)
                (15.0, 41)
                (15.2, 39)
                (15.4, 39)
                (15.6, 41)
                (15.8, 50)
                (16.0, 23)
                (16.2, 34)
                (16.4, 30)
                (16.6, 29)
                (16.8, 21)
                (17.0, 20)
                (17.2, 20)
                (17.4, 17)
                (17.6, 19)
                (17.8, 26)
                (18.0, 21)
                (18.2, 15)
                (18.4, 12)
                (18.6, 17)
                (18.8, 15)
                (19.0, 16)
                (19.2, 12)
                (19.4, 10)
                (19.6, 15)
                (19.8, 12)
                (20.0, 12)
            };
        

        

        

        

        

    \end{axis}
\end{tikzpicture}
\end{sansmath}
\caption{Time structure of the shower front (gray). This distribution is used as
the basis for a Monte Carlo. The black line shows the distribution resulting
from taking the first particle (out of two) arriving in a detector.}
\label{fig:montecarlo-arrival-times}
\end{figure}
For $N_\mathrm{\mip} \geq 2$, one finds $\sigma_{t,\, \mathrm{front}} =
\SI{1.4}{\nano\second}$. The total timing uncertainty is thus given by:
\begin{equation}
\sigma_t = \sqrt{\sigma_{t,\, \mathrm{front}}^2 + \sigma_{t,\,
\mathrm{transport}}^2} = \sqrt{1.4^2 + 1.2^2} = \SI{1.8}{\nano\second}.
\label{eq:ch4-timing1}
\end{equation}
Repeating this procedure, the solid line in \figref{fig:results-sim-mip-full} is
obtained. These results reproduce the uncertainties more accurately than the
approximation discussed in \cite{Steijger:2010-firsthit}. For
$N_\mathrm{\mip} \geq 1$, however, the reconstruction uncertainties are
smaller than estimated. This is attributed to the fact that events with
more than one particle in some (but not all) detectors have a better
timing accuracy, which is not taken into account in the estimation. The
discrepancy is large for $N_\mathrm{\mip} \geq 1$, since the slope in the
uncertainty is very large for small numbers of particles.
\begin{figure}
\centering
% \usepackage{tikz}
% \usetikzlibrary{arrows}
% \usepackage{pgfplots}
% \pgfplotsset{compat=1.3}
% \usepackage[detect-family]{siunitx}
% \usepackage[eulergreek]{sansmath}
% \sisetup{text-sf=\sansmath}
% \usepackage{relsize}
%
\pgfkeysifdefined{/artist/width}
    {\pgfkeysgetvalue{/artist/width}{\defaultwidth}}
    {\def\defaultwidth{ .67\linewidth }}
%
%
\begin{sansmath}
\begin{tikzpicture}[
        font=\sffamily,
        every pin/.style={inner sep=2pt, font={\sffamily\smaller}},
        every label/.style={inner sep=2pt, font={\sffamily\smaller}},
        every pin edge/.style={<-, >=stealth', shorten <=2pt},
        pin distance=2.5ex,
    ]
    \begin{axis}[
            width=\defaultwidth,
            %
            title={  },
            %
            xlabel={ Minimum number of particles },
            ylabel={ Reconstruction uncertainty [\si{\degree}] },
            %
            xmin={  },
            xmax={  },
            ymin={ 0 },
            ymax={ 32 },
            %
            xtick={ 1, 2, 3, 4 },
            ytick={  },
            %
            tick align=outside,
            max space between ticks=40,
            every tick/.style={},
        ]

        

        
            
            % Draw series plot
            \addplot[mark=*,mark options={fill=white},only marks] coordinates {
                (1, 8.86094818574)
                (2, 4.30924885605)
                (3, 3.04931200023)
                (4, 2.43898152732)
            };
        
            
            % Draw series plot
            \addplot[mark=*,only marks] coordinates {
                (1, 21.3475953316)
                (2, 10.4979834056)
                (3, 7.4502241806)
                (4, 6.00999457787)
            };
        
            
            % Draw series plot
            \addplot[no markers,dashed,smooth] coordinates {
                (1.0, 11.4338299661)
                (1.0303030303, 11.3476624759)
                (1.06060606061, 11.263402156)
                (1.09090909091, 11.1810202305)
                (1.12121212121, 11.1004879237)
                (1.15151515152, 11.0217764595)
                (1.18181818182, 10.9448570623)
                (1.21212121212, 10.869700956)
                (1.24242424242, 10.796279365)
                (1.27272727273, 10.7245635133)
                (1.30303030303, 10.654524625)
                (1.33333333333, 10.5861339243)
                (1.36363636364, 10.5193626354)
                (1.39393939394, 10.4541819824)
                (1.42424242424, 10.3905631894)
                (1.45454545455, 10.3284774807)
                (1.48484848485, 10.2678960802)
                (1.51515151515, 10.2087902122)
                (1.54545454545, 10.1511311009)
                (1.57575757576, 10.0948899703)
                (1.60606060606, 10.0400380446)
                (1.63636363636, 9.98654654793)
                (1.66666666667, 9.93438670448)
                (1.69696969697, 9.88352973838)
                (1.72727272727, 9.83394687377)
                (1.75757575758, 9.78560933479)
                (1.78787878788, 9.73848834558)
                (1.81818181818, 9.69255513029)
                (1.84848484848, 9.64778091307)
                (1.87878787879, 9.60413691806)
                (1.90909090909, 9.5615943694)
                (1.93939393939, 9.52012449124)
                (1.9696969697, 9.47969850772)
                (2.0, 9.44028764299)
                (2.0303030303, 9.40186399357)
                (2.06060606061, 9.36440314552)
                (2.09090909091, 9.32788155729)
                (2.12121212121, 9.2922756873)
                (2.15151515152, 9.257561994)
                (2.18181818182, 9.22371693583)
                (2.21212121212, 9.19071697122)
                (2.24242424242, 9.15853855862)
                (2.27272727273, 9.12715815646)
                (2.30303030303, 9.09655222318)
                (2.33333333333, 9.06669721723)
                (2.36363636364, 9.03756959703)
                (2.39393939394, 9.00914582104)
                (2.42424242424, 8.98140234768)
                (2.45454545455, 8.95431563541)
                (2.48484848485, 8.92786214264)
                (2.51515151515, 8.90201832784)
                (2.54545454545, 8.87676064943)
                (2.57575757576, 8.85206556585)
                (2.60606060606, 8.82790953554)
                (2.63636363636, 8.80426901695)
                (2.66666666667, 8.7811204685)
                (2.69696969697, 8.75844034865)
                (2.72727272727, 8.73620511582)
                (2.75757575758, 8.71439122846)
                (2.78787878788, 8.69297514501)
                (2.81818181818, 8.6719333239)
                (2.84848484848, 8.65124222358)
                (2.87878787879, 8.63087830248)
                (2.90909090909, 8.61081801904)
                (2.93939393939, 8.5910378317)
                (2.9696969697, 8.57151419891)
                (3.0, 8.55222357909)
                (3.0303030303, 8.53314592022)
                (3.06060606061, 8.51427512838)
                (3.09090909091, 8.4956085992)
                (3.12121212121, 8.47714372828)
                (3.15151515152, 8.45887791125)
                (3.18181818182, 8.44080854372)
                (3.21212121212, 8.4229330213)
                (3.24242424242, 8.40524873962)
                (3.27272727273, 8.38775309429)
                (3.30303030303, 8.37044348093)
                (3.33333333333, 8.35331729516)
                (3.36363636364, 8.33637193258)
                (3.39393939394, 8.31960478882)
                (3.42424242424, 8.3030132595)
                (3.45454545455, 8.28659474023)
                (3.48484848485, 8.27034662662)
                (3.51515151515, 8.2542663143)
                (3.54545454545, 8.23835119888)
                (3.57575757576, 8.22259867598)
                (3.60606060606, 8.20700614121)
                (3.63636363636, 8.19157099019)
                (3.66666666667, 8.17629061854)
                (3.69696969697, 8.16116242187)
                (3.72727272727, 8.1461837958)
                (3.75757575758, 8.13135213595)
                (3.78787878788, 8.11666483794)
                (3.81818181818, 8.10211929737)
                (3.84848484848, 8.08771290987)
                (3.87878787879, 8.07344307105)
                (3.90909090909, 8.05930717654)
                (3.93939393939, 8.04530262194)
                (3.9696969697, 8.03142680287)
                (4.0, 8.01767711496)
            };
        
            
            % Draw series plot
            \addplot[no markers,dashed,smooth] coordinates {
                (1.0, 4.73604744185)
                (1.0303030303, 4.70035569876)
                (1.06060606061, 4.66545393148)
                (1.09090909091, 4.63133022065)
                (1.12121212121, 4.59797264694)
                (1.15151515152, 4.56536929098)
                (1.18181818182, 4.53350823343)
                (1.21212121212, 4.50237755493)
                (1.24242424242, 4.47196533615)
                (1.27272727273, 4.44225965773)
                (1.30303030303, 4.41324860031)
                (1.33333333333, 4.38492024456)
                (1.36363636364, 4.35726267112)
                (1.39393939394, 4.33026396063)
                (1.42424242424, 4.30391219376)
                (1.45454545455, 4.27819545115)
                (1.48484848485, 4.25310181346)
                (1.51515151515, 4.22861936132)
                (1.54545454545, 4.2047361754)
                (1.57575757576, 4.18144033634)
                (1.60606060606, 4.1587199248)
                (1.63636363636, 4.13656302142)
                (1.66666666667, 4.11495770686)
                (1.69696969697, 4.09389206176)
                (1.72727272727, 4.07335416677)
                (1.75757575758, 4.05333210255)
                (1.78787878788, 4.03381394975)
                (1.81818181818, 4.01478778902)
                (1.84848484848, 3.996241701)
                (1.87878787879, 3.97816376635)
                (1.90909090909, 3.96054206571)
                (1.93939393939, 3.94336467975)
                (1.9696969697, 3.9266196891)
                (2.0, 3.91029517443)
                (2.0303030303, 3.89437957772)
                (2.06060606061, 3.8788627864)
                (2.09090909091, 3.86373504924)
                (2.12121212121, 3.84898661499)
                (2.15151515152, 3.83460773242)
                (2.18181818182, 3.82058865031)
                (2.21212121212, 3.80691961741)
                (2.24242424242, 3.7935908825)
                (2.27272727273, 3.78059269433)
                (2.30303030303, 3.76791530168)
                (2.33333333333, 3.75554895331)
                (2.36363636364, 3.74348389798)
                (2.39393939394, 3.73171038447)
                (2.42424242424, 3.72021866154)
                (2.45454545455, 3.70899897795)
                (2.48484848485, 3.69804158248)
                (2.51515151515, 3.68733672388)
                (2.54545454545, 3.67687465093)
                (2.57575757576, 3.66664561239)
                (2.60606060606, 3.65663985702)
                (2.63636363636, 3.6468476336)
                (2.66666666667, 3.63725919089)
                (2.69696969697, 3.62786477765)
                (2.72727272727, 3.61865464265)
                (2.75757575758, 3.60961903466)
                (2.78787878788, 3.60074820244)
                (2.81818181818, 3.59203239476)
                (2.84848484848, 3.58346186038)
                (2.87878787879, 3.57502684808)
                (2.90909090909, 3.56671760661)
                (2.93939393939, 3.55852438475)
                (2.9696969697, 3.55043743126)
                (3.0, 3.54244699491)
                (3.0303030303, 3.53454476986)
                (3.06060606061, 3.52672823195)
                (3.09090909091, 3.5189963024)
                (3.12121212121, 3.51134790244)
                (3.15151515152, 3.5037819533)
                (3.18181818182, 3.4962973762)
                (3.21212121212, 3.48889309238)
                (3.24242424242, 3.48156802307)
                (3.27272727273, 3.47432108949)
                (3.30303030303, 3.46715121288)
                (3.33333333333, 3.46005731446)
                (3.36363636364, 3.45303831546)
                (3.39393939394, 3.44609313712)
                (3.42424242424, 3.43922070065)
                (3.45454545455, 3.43241992729)
                (3.48484848485, 3.42568973827)
                (3.51515151515, 3.41902905482)
                (3.54545454545, 3.41243679817)
                (3.57575757576, 3.40591188954)
                (3.60606060606, 3.39945325017)
                (3.63636363636, 3.39305980128)
                (3.66666666667, 3.3867304641)
                (3.69696969697, 3.38046415987)
                (3.72727272727, 3.37425980981)
                (3.75757575758, 3.36811633514)
                (3.78787878788, 3.36203265711)
                (3.81818181818, 3.35600769694)
                (3.84848484848, 3.35004037585)
                (3.87878787879, 3.34412961508)
                (3.90909090909, 3.33827433585)
                (3.93939393939, 3.3324734594)
                (3.9696969697, 3.32672590696)
                (4.0, 3.32103059974)
            };
        
            
            % Draw series plot
            \addplot[no markers,smooth] coordinates {
                (1.0, 29.3707292622)
                (1.0303030303, 28.4106975407)
                (1.06060606061, 27.4820535983)
                (1.09090909091, 26.5842434814)
                (1.12121212121, 25.7167132362)
                (1.15151515152, 24.878908909)
                (1.18181818182, 24.0702765463)
                (1.21212121212, 23.2902621943)
                (1.24242424242, 22.5383118994)
                (1.27272727273, 21.8138717079)
                (1.30303030303, 21.1163876662)
                (1.33333333333, 20.4453058205)
                (1.36363636364, 19.8000722172)
                (1.39393939394, 19.1801329026)
                (1.42424242424, 18.5849339231)
                (1.45454545455, 18.013921325)
                (1.48484848485, 17.4665411546)
                (1.51515151515, 16.9422394582)
                (1.54545454545, 16.4404622822)
                (1.57575757576, 15.960655673)
                (1.60606060606, 15.5022656767)
                (1.63636363636, 15.0647383399)
                (1.66666666667, 14.6475197087)
                (1.69696969697, 14.2500558296)
                (1.72727272727, 13.8717927489)
                (1.75757575758, 13.5121765128)
                (1.78787878788, 13.1706531678)
                (1.81818181818, 12.8466687601)
                (1.84848484848, 12.5396693361)
                (1.87878787879, 12.2491009421)
                (1.90909090909, 11.9744096245)
                (1.93939393939, 11.7150414295)
                (1.9696969697, 11.4704424036)
                (2.0, 11.2400585929)
                (2.0303030303, 11.0233551925)
                (2.06060606061, 10.8198739915)
                (2.09090909091, 10.6291759274)
                (2.12121212121, 10.4508219381)
                (2.15151515152, 10.284372961)
                (2.18181818182, 10.129389934)
                (2.21212121212, 9.98543379467)
                (2.24242424242, 9.85206548067)
                (2.27272727273, 9.72884592969)
                (2.30303030303, 9.61533607938)
                (2.33333333333, 9.51109686743)
                (2.36363636364, 9.41568923149)
                (2.39393939394, 9.32867410924)
                (2.42424242424, 9.24961243835)
                (2.45454545455, 9.17806515648)
                (2.48484848485, 9.1135932013)
                (2.51515151515, 9.05575751048)
                (2.54545454545, 9.00411902169)
                (2.57575757576, 8.95823867259)
                (2.60606060606, 8.91767740086)
                (2.63636363636, 8.88199614416)
                (2.66666666667, 8.85075584016)
                (2.69696969697, 8.82351742653)
                (2.72727272727, 8.79984184094)
                (2.75757575758, 8.77929002105)
                (2.78787878788, 8.76142290453)
                (2.81818181818, 8.74580142905)
                (2.84848484848, 8.73198653229)
                (2.87878787879, 8.7195391519)
                (2.90909090909, 8.70802022556)
                (2.93939393939, 8.69699069093)
                (2.9696969697, 8.68601148568)
                (3.0, 8.67464354749)
                (3.0303030303, 8.66252440824)
                (3.06060606061, 8.64959797672)
                (3.09090909091, 8.63588475596)
                (3.12121212121, 8.62140524898)
                (3.15151515152, 8.60617995879)
                (3.18181818182, 8.59022938841)
                (3.21212121212, 8.57357404087)
                (3.24242424242, 8.55623441917)
                (3.27272727273, 8.53823102634)
                (3.30303030303, 8.5195843654)
                (3.33333333333, 8.50031493936)
                (3.36363636364, 8.48044325125)
                (3.39393939394, 8.45998980409)
                (3.42424242424, 8.43897510088)
                (3.45454545455, 8.41741964466)
                (3.48484848485, 8.39534393844)
                (3.51515151515, 8.37276848523)
                (3.54545454545, 8.34971378807)
                (3.57575757576, 8.32620034996)
                (3.60606060606, 8.30224867392)
                (3.63636363636, 8.27787926298)
                (3.66666666667, 8.25311262015)
                (3.69696969697, 8.22796924845)
                (3.72727272727, 8.2024696509)
                (3.75757575758, 8.17663433052)
                (3.78787878788, 8.15048379033)
                (3.81818181818, 8.12403853335)
                (3.84848484848, 8.09731906258)
                (3.87878787879, 8.07034588107)
                (3.90909090909, 8.04313949181)
                (3.93939393939, 8.01572039784)
                (3.9696969697, 7.98810910216)
                (4.0, 7.96032610781)
            };
        
            
            % Draw series plot
            \addplot[no markers,smooth] coordinates {
                (1.0, 12.1657543972)
                (1.0303030303, 11.7680962379)
                (1.06060606061, 11.3834393223)
                (1.09090909091, 11.0115541954)
                (1.12121212121, 10.6522114021)
                (1.15151515152, 10.3051814872)
                (1.18181818182, 9.97023499555)
                (1.21212121212, 9.64714247212)
                (1.24242424242, 9.33567446174)
                (1.27272727273, 9.0356015093)
                (1.30303030303, 8.74669415966)
                (1.33333333333, 8.46872295771)
                (1.36363636364, 8.20145844832)
                (1.39393939394, 7.94467117638)
                (1.42424242424, 7.69813168676)
                (1.45454545455, 7.46161052433)
                (1.48484848485, 7.23487823397)
                (1.51515151515, 7.01770536056)
                (1.54545454545, 6.80986244899)
                (1.57575757576, 6.61112004411)
                (1.60606060606, 6.42124869082)
                (1.63636363636, 6.24001893398)
                (1.66666666667, 6.06720131849)
                (1.69696969697, 5.9025663892)
                (1.72727272727, 5.74588469101)
                (1.75757575758, 5.59692676878)
                (1.78787878788, 5.4554631674)
                (1.81818181818, 5.32126443174)
                (1.84848484848, 5.19410110668)
                (1.87878787879, 5.07374373709)
                (1.90909090909, 4.95996286786)
                (1.93939393939, 4.85252904386)
                (1.9696969697, 4.75121280997)
                (2.0, 4.65578471106)
                (2.0303030303, 4.56602322361)
                (2.06060606061, 4.48173855045)
                (2.09090909091, 4.402748826)
                (2.12121212121, 4.3288721847)
                (2.15151515152, 4.25992676097)
                (2.18181818182, 4.19573068924)
                (2.21212121212, 4.13610210393)
                (2.24242424242, 4.08085913948)
                (2.27272727273, 4.02981993031)
                (2.30303030303, 3.98280261086)
                (2.33333333333, 3.93962531553)
                (2.36363636364, 3.90010617877)
                (2.39393939394, 3.86406333501)
                (2.42424242424, 3.83131491866)
                (2.45454545455, 3.80167906416)
                (2.48484848485, 3.77497390593)
                (2.51515151515, 3.7510175784)
                (2.54545454545, 3.72962821601)
                (2.57575757576, 3.71062395316)
                (2.60606060606, 3.6938229243)
                (2.63636363636, 3.67904326386)
                (2.66666666667, 3.66610310625)
                (2.69696969697, 3.6548205859)
                (2.72727272727, 3.64501383725)
                (2.75757575758, 3.63650099472)
                (2.78787878788, 3.62910019274)
                (2.81818181818, 3.62262956574)
                (2.84848484848, 3.61690724813)
                (2.87878787879, 3.61175137436)
                (2.90909090909, 3.60698007885)
                (2.93939393939, 3.60241149602)
                (2.9696969697, 3.5978637603)
                (3.0, 3.59315500612)
                (3.0303030303, 3.58813509428)
                (3.06060606061, 3.58278079103)
                (3.09090909091, 3.57710058901)
                (3.12121212121, 3.57110298084)
                (3.15151515152, 3.56479645915)
                (3.18181818182, 3.55818951658)
                (3.21212121212, 3.55129064574)
                (3.24242424242, 3.54410833926)
                (3.27272727273, 3.53665108978)
                (3.30303030303, 3.52892738993)
                (3.33333333333, 3.52094573233)
                (3.36363636364, 3.51271460961)
                (3.39393939394, 3.50424251439)
                (3.42424242424, 3.49553793932)
                (3.45454545455, 3.486609377)
                (3.48484848485, 3.47746532009)
                (3.51515151515, 3.46811426119)
                (3.54545454545, 3.45856469295)
                (3.57575757576, 3.44882510799)
                (3.60606060606, 3.43890399893)
                (3.63636363636, 3.42880985841)
                (3.66666666667, 3.41855117906)
                (3.69696969697, 3.4081364535)
                (3.72727272727, 3.39757417436)
                (3.75757575758, 3.38687283427)
                (3.78787878788, 3.37604092586)
                (3.81818181818, 3.36508694175)
                (3.84848484848, 3.35401937458)
                (3.87878787879, 3.34284671698)
                (3.90909090909, 3.33157746157)
                (3.93939393939, 3.32022010098)
                (3.9696969697, 3.30878312783)
                (4.0, 3.29727503477)
            };
        

        

        

        
    \end{axis}
\end{tikzpicture}
\end{sansmath}
\caption{Dependence of reconstruction accuracy on the minimum number of
particles in all three corner detectors.  Data for the most probable
zenith angle ($\theta$ = $\SI{22.5}{\degree}$) is shown.  The open circles
are the uncertainties for the zenith angle, the closed circles for the
azimuthal angle. The dashed lines show the estimates for the uncertainties
as calculated in \cite{Steijger:2010-firsthit}, with a standard deviation
of \SI{1.8}{\nano\second}. The solid lines show the estimates from the
Monte Carlo distributions.}
\label{fig:results-sim-mip-full}
\end{figure}

In the following discussion the reconstruction will be
restricted to at least 2 particles in each detector ($N_\mathrm{\mip} \geq 2$).
This reduces the effect of fluctuations in the shower front for large core
distances. The timing uncertainty will thus be taken to be $\sigma_t =
\SI{1.8}{\nano\second}$. This value is of the order of the
uncertainties discussed in \secref{sec:uncertainty-sources}. It is therefore
justified to use the plane front approximation.

\figref{fig:results-sim-zenith} shows the reconstruction uncertainty as a
function of zenith angle, for \SI{1}{\peta\electronvolt} proton showers. While
the uncertainty in the zenith angle does not differ significantly for vertical
and inclined showers, the uncertainty in the azimuthal angle diverges. It should
be noted that for small zenith angles, two points with a large difference in
azimuthal angle may actually be close together in terms of angular distance.
Estimates for the reconstruction uncertainties are calculated using
\eqtworef{eq:errphi}{eq:errtheta}. The estimates are calculated by averaging over all
azimuthal angles. Data and simulation are in good agreement.

\begin{figure}
\centering
% \usepackage{tikz}
% \usetikzlibrary{arrows}
% \usepackage{pgfplots}
% \pgfplotsset{compat=1.3}
% \usepackage[detect-family]{siunitx}
% \usepackage[eulergreek]{sansmath}
% \sisetup{text-sf=\sansmath}
% \usepackage{relsize}
%
\pgfkeysifdefined{/artist/width}
    {\pgfkeysgetvalue{/artist/width}{\defaultwidth}}
    {\def\defaultwidth{ .67\linewidth }}
%
%
\begin{sansmath}
\begin{tikzpicture}[
        font=\sffamily,
        every pin/.style={inner sep=2pt, font={\sffamily\smaller}},
        every label/.style={inner sep=2pt, font={\sffamily\smaller}},
        every pin edge/.style={<-, >=stealth', shorten <=2pt},
        pin distance=2.5ex,
    ]
    \begin{axis}[
            width=\defaultwidth,
            %
            title={  },
            %
            xlabel={ Shower zenith angle [\si{\degree}] },
            ylabel={ Angle reconstruction uncertainty [\si{\degree}] },
            %
            xmin={  },
            xmax={  },
            ymin={ 0 },
            ymax={ 60 },
            %
            xtick={  },
            ytick={  },
            %
            tick align=outside,
            max space between ticks=40,
            every tick/.style={},
        ]

        

        
            
            % Draw series plot
            \addplot[mark=*,mark options={fill=white},only marks] coordinates {
                (0, 5.17501542656)
                (5, 4.44839114755)
                (10, 4.18195762577)
                (15, 4.30102421546)
                (22.5, 4.30924885605)
                (30, 4.44958164953)
                (35, 4.4777832274)
                (45, 5.03719894307)
            };
        
            
            % Draw series plot
            \addplot[mark=*,only marks] coordinates {
                (5, 45.9835213086)
                (10, 23.7913628408)
                (15, 16.157309856)
                (22.5, 10.4979834056)
                (30, 7.81845497838)
                (35, 6.48766242314)
                (45, 5.54102784719)
            };
        
            
            % Draw series plot
            \addplot[no markers,solid] coordinates {
                (0.0, inf)
                (0.918367346939, 272.995937913)
                (1.83673469388, 136.515504946)
                (2.75510204082, 91.0298257374)
                (3.67346938776, 68.2928402281)
                (4.59183673469, 54.655338095)
                (5.51020408163, 45.5675839744)
                (6.42857142857, 39.0796924954)
                (7.34693877551, 34.2167220392)
                (8.26530612245, 30.4370393264)
                (9.18367346939, 27.4156652522)
                (10.1020408163, 24.9457957218)
                (11.0204081633, 22.8895620682)
                (11.9387755102, 21.1515174319)
                (12.8571428571, 19.6634861212)
                (13.7755102041, 18.3754732288)
                (14.693877551, 17.2499831428)
                (15.612244898, 16.2583432892)
                (16.5306122449, 15.3782532946)
                (17.4489795918, 14.5921080977)
                (18.3673469388, 13.8858241272)
                (19.2857142857, 13.2480008591)
                (20.2040816327, 12.6693110392)
                (21.1224489796, 12.1420499794)
                (22.0408163265, 11.6597975298)
                (22.9591836735, 11.2171611783)
                (23.8775510204, 10.8095784347)
                (24.7959183673, 10.43316313)
                (25.7142857143, 10.0845846526)
                (26.6326530612, 9.76097216957)
                (27.5510204082, 9.45983800443)
                (28.4693877551, 9.17901584548)
                (29.387755102, 8.91661054033)
                (30.306122449, 8.67095701908)
                (31.2244897959, 8.44058646671)
                (32.1428571429, 8.22419829493)
                (33.0612244898, 8.02063678568)
                (33.9795918367, 7.82887152265)
                (34.8979591837, 7.64798091282)
                (35.8163265306, 7.47713824371)
                (36.7346938776, 7.31559983224)
                (37.6530612245, 7.16269490851)
                (38.5714285714, 7.01781694501)
                (39.4897959184, 6.88041619618)
                (40.4081632653, 6.74999325577)
                (41.3265306122, 6.62609347356)
                (42.2448979592, 6.5083021009)
                (43.1632653061, 6.39624005626)
                (44.0816326531, 6.28956022072)
                (45.0, 6.18794418741)
            };
        
            
            % Draw series plot
            \addplot[no markers,solid] coordinates {
                (0.0, inf)
                (0.918367346939, 4.37609942487)
                (1.83673469388, 4.37778653232)
                (2.75510204082, 4.38060078831)
                (3.67346938776, 4.38454581614)
                (4.59183673469, 4.3896267044)
                (5.51020408163, 4.39585002297)
                (6.42857142857, 4.40322384385)
                (7.34693877551, 4.4117577668)
                (8.26530612245, 4.42146295009)
                (9.18367346939, 4.43235214632)
                (10.1020408163, 4.44443974379)
                (11.0204081633, 4.45774181348)
                (11.9387755102, 4.47227616202)
                (12.8571428571, 4.48806239092)
                (13.7755102041, 4.50512196252)
                (14.693877551, 4.52347827298)
                (15.612244898, 4.54315673285)
                (16.5306122449, 4.56418485575)
                (17.4489795918, 4.5865923557)
                (18.3673469388, 4.61041125382)
                (19.2857142857, 4.63567599514)
                (20.2040816327, 4.66242357624)
                (21.1224489796, 4.69069368483)
                (22.0408163265, 4.72052885207)
                (22.9591836735, 4.75197461903)
                (23.8775510204, 4.78507971827)
                (24.7959183673, 4.81989627223)
                (25.7142857143, 4.85648000991)
                (26.6326530612, 4.89489050358)
                (27.5510204082, 4.9351914276)
                (28.4693877551, 4.97745084157)
                (29.387755102, 5.02174150035)
                (30.306122449, 5.06814119362)
                (31.2244897959, 5.11673311851)
                (32.1428571429, 5.16760628846)
                (33.0612244898, 5.22085598278)
                (33.9795918367, 5.27658424125)
                (34.8979591837, 5.33490040903)
                (35.8163265306, 5.39592173794)
                (36.7346938776, 5.45977405066)
                (37.6530612245, 5.52659247567)
                (38.5714285714, 5.59652226169)
                (39.4897959184, 5.66971968178)
                (40.4081632653, 5.74635303849)
                (41.3265306122, 5.82660378363)
                (42.2448979592, 5.9106677678)
                (43.1632653061, 5.99875663768)
                (44.0816326531, 6.0910994016)
                (45.0, 6.18794418741)
            };
        

        

        

        
    \end{axis}
\end{tikzpicture}
\end{sansmath}
\caption{Dependence of reconstruction uncertainty on zenith angle, for
\SI{1}{\peta\electronvolt} proton showers.  A minimum of 2 particles is required
in all three corner detectors.  The data points show the uncertainties in
the reconstruction of zenith angle (open circles) and azimuthal angle
(closed circles).  Estimates for the reconstruction uncertainties (solid
lines) are calculated using \eqtworef{eq:errphi}{eq:errtheta}, by
averaging over all azimuthal angles.}
\label{fig:results-sim-zenith}
\end{figure}

\figref{fig:results-size} shows the dependence of the angular resolution on
station size for \SI{1}{\peta\electronvolt} proton showers with $\theta =
\SI{22.5}{\degree}$.   The resolution increases with increasing size of the
station.
Data and calculations are in agreement to better than \SI{20}{\percent}. The
results show that a detector distance smaller than \SI{10}{\meter} would have a
detrimental effect on the reconstruction uncertainties. The calculated
uncertainty distribution
flattens.

\begin{figure}
\centering
% \usepackage{tikz}
% \usetikzlibrary{arrows}
% \usepackage{pgfplots}
% \pgfplotsset{compat=1.3}
% \usepackage[detect-family]{siunitx}
% \usepackage[eulergreek]{sansmath}
% \sisetup{text-sf=\sansmath}
% \usepackage{relsize}
%
\pgfkeysifdefined{/artist/width}
    {\pgfkeysgetvalue{/artist/width}{\defaultwidth}}
    {\def\defaultwidth{ .67\linewidth }}
%
%
\begin{sansmath}
\begin{tikzpicture}[
        font=\sffamily,
        every pin/.style={inner sep=2pt, font={\sffamily\smaller}},
        every label/.style={inner sep=2pt, font={\sffamily\smaller}},
        every pin edge/.style={<-, >=stealth', shorten <=2pt},
        pin distance=2.5ex,
    ]
    \begin{axis}[
            width=\defaultwidth,
            axis equal=false,
            %
            title={  },
            %
            xlabel={ Station size [\si{\meter}] },
            ylabel={ Angle reconstruction uncertainty [\si{\degree}] },
            %
            xmin={  },
            xmax={  },
            ymin={ 0 },
            ymax={ 25 },
            %
            xtick={  },
            ytick={  },
            %
            tick align=outside,
            max space between ticks=40,
            every tick/.style={},
            axis on top,
        ]

        

        

        
            
            % Draw series plot
            \addplot[mark=*,mark options=white,only marks] coordinates {
                (5, 7.4434071737)
                (10, 4.30924885605)
                (20, 2.81110505482)
            };
        
            
            % Draw series plot
            \addplot[mark=o,only marks] coordinates {
                (5, 7.4434071737)
                (10, 4.30924885605)
                (20, 2.81110505482)
            };
        
            
            % Draw series plot
            \addplot[mark=*,only marks] coordinates {
                (5, 18.4856105584)
                (10, 10.4979834056)
                (20, 6.90668892796)
            };
        
            
            % Draw series plot
            \addplot[no markers,solid] coordinates {
                (5.0, 22.8676599323)
                (5.30612244898, 21.5483718593)
                (5.61224489796, 20.3730061215)
                (5.91836734694, 19.3192299428)
                (6.22448979592, 18.36910388)
                (6.5306122449, 17.5080521357)
                (6.83673469388, 16.7241095027)
                (7.14285714286, 16.0073619526)
                (7.44897959184, 15.34952516)
                (7.75510204082, 14.7436228511)
                (8.0612244898, 14.183738439)
                (8.36734693878, 13.6648211791)
                (8.67346938776, 13.1825333727)
                (8.97959183673, 12.7331288259)
                (9.28571428571, 12.3133553482)
                (9.59183673469, 11.9203759222)
                (9.89795918367, 11.5517045019)
                (10.2040816327, 11.2051533668)
                (10.5102040816, 10.8787896765)
                (10.8163265306, 10.5708994027)
                (11.1224489796, 10.2799572173)
                (11.4285714286, 10.0046012204)
                (11.7346938776, 9.74361162332)
                (12.0408163265, 9.49589268375)
                (12.3469387755, 9.26045732795)
                (12.6530612245, 9.0364140055)
                (12.9591836735, 8.82295540695)
                (13.2653061224, 8.61934874371)
                (13.5714285714, 8.42492734348)
                (13.8775510204, 8.23908335796)
                (14.1836734694, 8.06126141498)
                (14.4897959184, 7.89095307523)
                (14.7959183673, 7.72769197712)
                (15.1020408163, 7.57104957218)
                (15.4081632653, 7.42063136876)
                (15.7142857143, 7.27607361482)
                (16.0204081633, 7.13704036103)
                (16.3265306122, 7.00322085426)
                (16.6326530612, 6.87432721891)
                (16.9387755102, 6.75009238965)
                (17.2448979592, 6.63026826439)
                (17.5510204082, 6.51462405048)
                (17.8571428571, 6.40294478104)
                (18.1632653061, 6.29502998136)
                (18.4693877551, 6.19069246786)
                (18.7755102041, 6.08975726458)
                (19.0816326531, 5.99206062397)
                (19.387755102, 5.89744914043)
                (19.693877551, 5.80577894654)
                (20.0, 5.71691498307)
            };
        
            
            % Draw series plot
            \addplot[no markers,solid] coordinates {
                (5.0, 9.47209488369)
                (5.30612244898, 8.92562787117)
                (5.61224489796, 8.43877544183)
                (5.91836734694, 8.00228705691)
                (6.22448979592, 7.60873195575)
                (6.5306122449, 7.25207264533)
                (6.83673469388, 6.92735297464)
                (7.14285714286, 6.63046641858)
                (7.44897959184, 6.35798149727)
                (7.75510204082, 6.10700854343)
                (8.0612244898, 5.87509682659)
                (8.36734693878, 5.66015425977)
                (8.67346938776, 5.46038410942)
                (8.97959183673, 5.27423465115)
                (9.28571428571, 5.10035878353)
                (9.59183673469, 4.93758137554)
                (9.89795918367, 4.78487267321)
                (10.2040816327, 4.64132649301)
                (10.5102040816, 4.50614222622)
                (10.8163265306, 4.37860989906)
                (11.1224489796, 4.25809770001)
                (11.4285714286, 4.14404151162)
                (11.7346938776, 4.03593608088)
                (12.0408163265, 3.93332753645)
                (12.3469387755, 3.83580701902)
                (12.6530612245, 3.7430052363)
                (12.9591836735, 3.65458778977)
                (13.2653061224, 3.57025114847)
                (13.5714285714, 3.48971916768)
                (13.8775510204, 3.41274006839)
                (14.1836734694, 3.33908380792)
                (14.4897959184, 3.26853978381)
                (14.7959183673, 3.20091482276)
                (15.1020408163, 3.1360314142)
                (15.4081632653, 3.07372615431)
                (15.7142857143, 3.01384837208)
                (16.0204081633, 2.95625891274)
                (16.3265306122, 2.90082905813)
                (16.6326530612, 2.84743956626)
                (16.9387755102, 2.79597981507)
                (17.2448979592, 2.74634703728)
                (17.5510204082, 2.69844563547)
                (17.8571428571, 2.65218656743)
                (18.1632653061, 2.60748679383)
                (18.4693877551, 2.56426878067)
                (18.7755102041, 2.52246005055)
                (19.0816326531, 2.48199277701)
                (19.387755102, 2.44280341737)
                (19.693877551, 2.4048323798)
                (20.0, 2.36802372092)
            };
        

        

        

        

        

    \end{axis}
\end{tikzpicture}
\end{sansmath}
\caption{Reconstruction uncertainty as a function of station size.  A
minimum of 2 particles is required in all three corner detectors.  Data
for the most probable zenith angle ($\theta$ = $\SI{22.5}{\degree}$) is
shown.  The data points show the uncertainties in the reconstruction of
zenith angle (open circles) and azimuthal angle (closed circles).  The
estimates for the uncertainties (solid lines) are calculated by averaging
over all azimuthal angles.}
\label{fig:results-size}
\end{figure}

Sampling frequency limits the accuracy with which the analog signal can be
digitized.
To simulate the effect of sampling, discrete arrival times have been chosen
corresponding with the sampling rate.
The following sampling times are used: \SI{0}{\nano\second} (no sampling),
\SIlist{1;2.5;5}{\nano\second}.
The reconstruction uncertainty is shown in
\figref{fig:results-binsize}.  The width of the bins introduce an uncertainty
$\sigma_{\mathrm{bin}}^2 = \frac{w^2}{12}$, with $w$ the width of the bin.  The
total uncertainty is then given by
\begin{equation}
\sigma_t = \sqrt{\sigma_{t,\, \mathrm{front}}^2 + \sigma_{t,\, \mathrm{transport}}^2 + \frac{w^2}{12}}.
\label{eq:ch4-timing2}
\end{equation}
For the \SI{2.5}{\nano\second} resolution of the \hisparc hardware, one obtains
$\sigma_t = \SI{2.0}{\nano\second}$, only slightly larger than the total
uncertainty with perfect time resolution.  Consequently, the reconstruction
uncertainties are only slightly affected. The results show that the
\SI{2.5}{\nano\second} sampling time is an excellent choice.

\begin{figure}
\centering
% \usepackage{tikz}
% \usetikzlibrary{arrows}
% \usepackage{pgfplots}
% \pgfplotsset{compat=1.3}
% \usepackage[detect-family]{siunitx}
% \usepackage[eulergreek]{sansmath}
% \sisetup{text-sf=\sansmath}
% \usepackage{relsize}
%
\pgfkeysifdefined{/artist/width}
    {\pgfkeysgetvalue{/artist/width}{\defaultwidth}}
    {\def\defaultwidth{ .67\linewidth }}
%
%
\begin{sansmath}
\begin{tikzpicture}[
        font=\sffamily,
        every pin/.style={inner sep=2pt, font={\sffamily\smaller}},
        every label/.style={inner sep=2pt, font={\sffamily\smaller}},
        every pin edge/.style={<-, >=stealth', shorten <=2pt},
        pin distance=2.5ex,
    ]
    \begin{axis}[
            width=\defaultwidth,
            axis equal=false,
            %
            title={  },
            %
            xlabel={ Sampling time [\si{\nano\second}] },
            ylabel={ Angle reconstruction uncertainty [\si{\degree}] },
            %
            xmin={  },
            xmax={  },
            ymin={ 0 },
            ymax={ 20 },
            %
            xtick={  },
            ytick={  },
            %
            tick align=outside,
            max space between ticks=40,
            every tick/.style={},
            axis on top,
        ]

        

        

        
            
            % Draw series plot
            \addplot[mark=*,mark options=white,only marks] coordinates {
                (0, 4.30924885605)
                (1, 4.46697144776)
                (2.5, 5.24851898542)
                (5, 7.33933913301)
            };
        
            
            % Draw series plot
            \addplot[mark=o,only marks] coordinates {
                (0, 4.30924885605)
                (1, 4.46697144776)
                (2.5, 5.24851898542)
                (5, 7.33933913301)
            };
        
            
            % Draw series plot
            \addplot[mark=*,only marks] coordinates {
                (0, 10.4979834056)
                (1, 10.925650373)
                (2.5, 12.8186573822)
                (5, 17.9288340279)
            };
        
            
            % Draw series plot
            \addplot[no markers,solid] coordinates {
                (0.0, 11.4338299661)
                (0.102040816327, 11.4353608923)
                (0.204081632653, 11.4399524417)
                (0.30612244898, 11.4476009314)
                (0.408163265306, 11.4583002398)
                (0.510204081633, 11.4720418311)
                (0.612244897959, 11.4888147887)
                (0.714285714286, 11.5086058587)
                (0.816326530612, 11.5313995013)
                (0.918367346939, 11.5571779511)
                (1.02040816327, 11.5859212847)
                (1.12244897959, 11.6176074956)
                (1.22448979592, 11.652212576)
                (1.32653061224, 11.6897106036)
                (1.42857142857, 11.7300738345)
                (1.5306122449, 11.7732727996)
                (1.63265306122, 11.8192764057)
                (1.73469387755, 11.8680520381)
                (1.83673469388, 11.9195656671)
                (1.9387755102, 11.9737819543)
                (2.04081632653, 12.0306643613)
                (2.14285714286, 12.090175257)
                (2.24489795918, 12.1522760256)
                (2.34693877551, 12.2169271729)
                (2.44897959184, 12.2840884309)
                (2.55102040816, 12.3537188609)
                (2.65306122449, 12.4257769535)
                (2.75510204082, 12.5002207255)
                (2.85714285714, 12.5770078144)
                (2.95918367347, 12.656095568)
                (3.0612244898, 12.7374411313)
                (3.16326530612, 12.8210015291)
                (3.26530612245, 12.9067337437)
                (3.36734693878, 12.9945947893)
                (3.4693877551, 13.0845417815)
                (3.57142857143, 13.1765320025)
                (3.67346938776, 13.2705229618)
                (3.77551020408, 13.3664724529)
                (3.87755102041, 13.4643386052)
                (3.97959183673, 13.564079932)
                (4.08163265306, 13.6656553745)
                (4.18367346939, 13.7690243413)
                (4.28571428571, 13.8741467449)
                (4.38775510204, 13.980983033)
                (4.48979591837, 14.0894942183)
                (4.59183673469, 14.199641903)
                (4.69387755102, 14.3113883011)
                (4.79591836735, 14.4246962575)
                (4.89795918367, 14.5395292642)
                (5.0, 14.6558514736)
            };
        
            
            % Draw series plot
            \addplot[no markers,solid] coordinates {
                (0.0, 4.73604744185)
                (0.102040816327, 4.73668157224)
                (0.204081632653, 4.73858345424)
                (0.30612244898, 4.7417515624)
                (0.408163265306, 4.74618336108)
                (0.510204081633, 4.75187531456)
                (0.612244897959, 4.75882290109)
                (0.714285714286, 4.76702063069)
                (0.816326530612, 4.7764620666)
                (0.918367346939, 4.78713985011)
                (1.02040816327, 4.7990457287)
                (1.12244897959, 4.81217058701)
                (1.22448979592, 4.82650448062)
                (1.32653061224, 4.84203667222)
                (1.42857142857, 4.85875566987)
                (1.5306122449, 4.87664926713)
                (1.63265306122, 4.89570458467)
                (1.73469387755, 4.91590811314)
                (1.83673469388, 4.93724575689)
                (1.9387755102, 4.95970287838)
                (2.04081632653, 4.98326434281)
                (2.14285714286, 5.00791456293)
                (2.24489795918, 5.03363754352)
                (2.34693877551, 5.06041692553)
                (2.44897959184, 5.08823602946)
                (2.55102040816, 5.11707789793)
                (2.65306122449, 5.14692533715)
                (2.75510204082, 5.17776095717)
                (2.85714285714, 5.20956721079)
                (2.95918367347, 5.24232643094)
                (3.0612244898, 5.27602086653)
                (3.16326530612, 5.31063271656)
                (3.26530612245, 5.34614416258)
                (3.36734693878, 5.38253739928)
                (3.4693877551, 5.41979466335)
                (3.57142857143, 5.45789826049)
                (3.67346938776, 5.49683059058)
                (3.77551020408, 5.53657417108)
                (3.87755102041, 5.57711165864)
                (3.97959183673, 5.61842586893)
                (4.08163265306, 5.66049979482)
                (4.18367346939, 5.70331662283)
                (4.28571428571, 5.74685974808)
                (4.38775510204, 5.79111278759)
                (4.48979591837, 5.83605959221)
                (4.59183673469, 5.88168425706)
                (4.69387755102, 5.9279711307)
                (4.79591836735, 5.97490482298)
                (4.89795918367, 6.02247021177)
                (5.0, 6.07065244849)
            };
        

        

        

        

        

    \end{axis}
\end{tikzpicture}
\end{sansmath}
\caption{Reconstruction uncertainty as a function of bin size.  A minimum
of 2 particles is required in all three corner detectors.  Data for the
most probable zenith angle ($\theta$ = $\SI{22.5}{\degree}$) is shown.  The
data points show the uncertainties in the reconstruction of zenith angle
(open circles) and azimuthal angle (closed circles).  Estimates (solid
lines) are calculated using $\sigma_{\mathrm{bin}}^2$ = $\frac{w^2}{12}$,
with $w$ the width of the bin.}
\label{fig:results-binsize}
\end{figure}


\section{Discussion and Conclusions}
\label{sec:reconstruction-conclusions}

A \SI{1}{\peta\electronvolt} proton shower has particle densities higher than
\SI{1}{\per\square\meter} up to \SI{30}{\meter}
from the core (\figref{fig:ldf-threshold}).
Defining the \emph{footprint} of an EAS as the area containing particle
densities higher than \SI{1}{\per\square\meter}, such a shower has a footprint
of \SI{2.8e3}{\square\meter}.
A \hisparc station has an area of \SI{43}{\square\meter}, which is
\SI{1.5}{\percent} of this footprint. The detection area of the three corner
detectors is only \SI{1.5}{\square\meter} (\SI{0.053}{\percent} of the
footprint).
Using the limited information available, the reconstruction of shower direction
is performed with surprisingly good accuracy. In some cases, the direction can
be reconstructed for core distances significantly larger than \SI{30}{\meter}.

It is shown that the uncertainty in the reconstruction of the azimuthal angle
can be quite large.  It is important to realize, however, that while the
azimuthal angle may not be very accurately known, the effect of this on the
position on the celestial sphere decreases with decreasing $\theta$.  For small
zenith angles, all possible directions on the celestial sphere are close
together.
A small uncertainty in the direction can thus result in a large uncertainty of
the azimuthal angle. For two vectors which only differ in their azimuthal
angles, $\Delta\phi = \phi_1 - \phi_2$, the angular distance is given by
\begin{equation}
\phi_\mathrm{dist} = \Delta\phi \sin\theta.
\label{eq:angular-distance}
\end{equation}
For \SI{1}{\peta\electronvolt} proton showers with a zenith angle of
\SI{22.5}{\degree}, and $N_\mathrm{MIP} \geq 2$, the uncertainty in the
reconstruction of shower direction is determined to be $\sigma_\theta =
\SI{4.3}{\degree}$ and $\sigma_\phi = \SI{11}{\degree}$.
Using \eqref{eq:angular-distance}, the angular distance between two directions
with $\theta = \SI{22.5}{\degree}$ and separated by $\Delta\phi =
\SI{11}{\degree}$ is only $\phi_\mathrm{dist} = \SI{4.0}{\degree}$. The angular
distance between two directions separated by
$\Delta\phi = \SI{11}{\degree}$ and $\Delta\theta = \SI{4.3}{\degree}$, then
becomes $d = \SI{5.9}{\degree}$. In other words, for $\theta =
\SI{22.5}{\degree}$ and $N_\mathrm{MIP} \geq 2$, \SI{66}{\percent} of all EAS
directions are accurately reconstructed to within \SI{5.9}{\degree} on the
celestial sphere. \figref{fig:sim-angular-distance} shows the angular distance
as a function of zenith angle, with the direction uncertainties calculated using
\eqtworef{eq:errphi}{eq:errtheta} with $\sigma_t = \SI{1.8}{\nano\second}$. The
results from the simulation are more accurate than the calculated
estimates.
\begin{figure}
\centering
% \usepackage{tikz}
% \usetikzlibrary{arrows}
% \usepackage{pgfplots}
% \pgfplotsset{compat=1.3}
% \usepackage[detect-family]{siunitx}
% \usepackage[eulergreek]{sansmath}
% \sisetup{text-sf=\sansmath}
% \usepackage{relsize}
%
\pgfkeysifdefined{/artist/width}
    {\pgfkeysgetvalue{/artist/width}{\defaultwidth}}
    {\def\defaultwidth{ .67\linewidth }}
%
%
\begin{sansmath}
\begin{tikzpicture}[
        font=\sffamily,
        every pin/.style={inner sep=2pt, font={\sffamily\smaller}},
        every label/.style={inner sep=2pt, font={\sffamily\smaller}},
        every pin edge/.style={<-, >=stealth', shorten <=2pt},
        pin distance=2.5ex,
    ]
    \begin{axis}[
            width=\defaultwidth,
            %
            title={  },
            %
            xlabel={ Shower zenith angle [\si{\degree}] },
            ylabel={ Angular distance [\si{\degree}] },
            %
            xmin={  },
            xmax={  },
            ymin={ 6 },
            ymax={  },
            %
            xtick={  },
            ytick={  },
            %
            tick align=outside,
            max space between ticks=40,
            every tick/.style={},
        ]

        

        
            
            % Draw series plot
            \addplot[no markers,solid] coordinates {
                (0.0, nan)
                (0.918367346939, 6.18834168495)
                (1.83673469388, 6.18953484164)
                (2.75510204082, 6.19152565203)
                (3.67346938776, 6.19431744804)
                (4.59183673469, 6.19791491046)
                (5.51020408163, 6.20232408519)
                (6.42857142857, 6.20755240431)
                (7.34693877551, 6.21360871203)
                (8.26530612245, 6.22050329574)
                (9.18367346939, 6.22824792235)
                (10.1020408163, 6.23685588012)
                (11.0204081633, 6.24634202625)
                (11.9387755102, 6.25672284048)
                (12.8571428571, 6.26801648515)
                (13.7755102041, 6.28024287193)
                (14.693877551, 6.29342373589)
                (15.612244898, 6.30758271706)
                (16.5306122449, 6.32274545041)
                (17.4489795918, 6.33893966453)
                (18.3673469388, 6.35619528985)
                (19.2857142857, 6.37454457708)
                (20.2040816327, 6.39402222686)
                (21.1224489796, 6.41466553128)
                (22.0408163265, 6.43651452857)
                (22.9591836735, 6.45961217204)
                (23.8775510204, 6.48400451445)
                (24.7959183673, 6.50974090947)
                (25.7142857143, 6.53687423162)
                (26.6326530612, 6.56546111673)
                (27.5510204082, 6.59556222473)
                (28.4693877551, 6.62724252714)
                (29.387755102, 6.66057162183)
                (30.306122449, 6.69562407784)
                (31.2244897959, 6.73247981351)
                (32.1428571429, 6.77122451155)
                (33.0612244898, 6.81195007514)
                (33.9795918367, 6.85475512971)
                (34.8979591837, 6.89974557556)
                (35.8163265306, 6.9470351975)
                (36.7346938776, 6.99674633795)
                (37.6530612245, 7.0490106416)
                (38.5714285714, 7.10396988021)
                (39.4897959184, 7.16177686773)
                (40.4081632653, 7.22259647746)
                (41.3265306122, 7.2866067744)
                (42.2448979592, 7.35400027839)
                (43.1632653061, 7.42498537584)
                (44.0816326531, 7.49978790057)
                (45.0, 7.57865290799)
            };
        

        

        

        
    \end{axis}
\end{tikzpicture}
\end{sansmath}
\caption{Calculated estimate of EAS direction accuracy expressed as
angular distance, as a function of zenith angle. The reconstructions are
restricted to $N_\mathrm{MIP} \geq 2$, and all showers are generated by
\SI{1}{\peta\electronvolt} protons. \SI{66}{\percent} of all EAS
directions are accurately reconstructed to within the angular distance
shown in the figure.}
\label{fig:sim-angular-distance}
\end{figure}

There is a slight quantitative difference between calculations and simulations
which suggests that a detailed analysis of the shower front is required.  The
arrival time uncertainty is assumed to be \SI{1.8}{\nano\second} for
$N_\mathrm{\mip} \geq 2$.  In reality, this value depends on the distance of the
detector to the shower core, as the shower front becomes thicker farther away
from the core. To improve the understanding of the reconstruction uncertainties,
the shower front arrival time distribution should be parametrized for several
zenith angles, as a function of the core distance.
This distribution should then be propagated using the detection efficiency to
obtain an estimate which is approximately correct for all core distances and
also yields a correct average over all events (i.e. detected showers).
\figref{fig:results-sim-mip} shows the reconstruction uncertainty as a function
of the minimum number of particles in the detectors. In the analysis presented
in this chapter, a value for the timing accuracy is determined for a median core
distance $R_0$, which is taken from the simulation results. This value is used
for the calculations. The discrepancy between calculation and simulation
demonstrates that the uncertainties in the data cannot be fully described in
this way and indicates that there must be another contribution which depends on
the minimum number of particles per detector. This is most likely the core
distance, which is, on average, smaller for large numbers of particles per
detector (\figref{fig:mip-core-dists}). This explains the discrepancy for larger
numbers of particles. The first datapoint, at $N_\mathrm{MIP} \geq 1$, is also
better than expected from calculations, but this was explained to be the result
of more than one particle in some, but not all, detectors. This results in a
more accurate reconstruction, not taken into account by the calculations.

The energy of the primary particle is also important. The particle density on
the ground scales linearly with the primary energy and thus the maximum (and
median) core distance for which EAS are detected depends on this energy.
Furthermore, the particle density determines the number of particles in the
detectors. It has been demonstrated that this is of great importance to the
accuracy of the reconstructions.

The nature of the primary particle is also expected to play a role. The early
development of EAS differs for protons and heavy nuclei. The effect of this on
the arrival time distribution of particles in the shower front has not been
studied. However, as discussed in \secref{sec:solar-neighborhood},
\SI{84}{\percent} of the primary particles are protons. An additional
\SI{12}{\percent} is made up of helium nuclei. Therefore, the effect is expected
to be minor.

Restricting the analysis to at least two particles in all three corner detectors
significantly reduces the uncertainty in the arrival time distribution.  The
reconstruction uncertainty as a function of zenith angle shows an excellent
result (\figref{fig:results-sim-zenith}).

The \SI{10}{\meter} detector
separation distance is a proper choice to obtain a good accuracy
(\figref{fig:results-size}).
Increasing the size of the station would make it difficult to fit the station on
a typical roof, while the accuracy only slowly improves with size.
A smaller station size, however, would be detrimental.

The \SI{400}{\mega\hertz} sampling rate of the analog signal is also a proper
choice.  Sampling accuracy better than \SI{2.5}{\nano\second} only decreases the
uncertainties slightly.  However, \SI{5}{\nano\second} resolution is
significantly worse.

A single \hisparc station can be used to reconstruct the direction of EAS.
In this chapter, an algorithm was developed to calculate the direction of
EAS from relative differences in the measured particle arrival times in
the individual detectors.  The accuracy of the presented algorithm has
been determined from the simulation results.  To better understand the
accuracy of the reconstruction, several sources of measurement
uncertainties have been identified.  The resulting timing uncertainties
have been propagated through the analysis to obtain an estimate for the
size of the reconstruction uncertainties.  These calculations have been
compared to the uncertainties determined using the simulation.  Errors on
errors were not considered.  Howerver, the uncertainties are reasonably
well understood.  

Considering the cost of a single \hisparc station (\EUR{10000}), the
precision with which it determines the direction of EAS can compete with
an air shower array.  The reconstruction performs well for all azimuthal
angles and all zenith angles considered in this analysis ($\theta <
\SI{45}{\degree}$).

The next chapters will elaborate on the present analysis.  Integrating a
\hisparc station in the \kascade array enables a detailed performance study.
In \chref{ch:kascade} simulation results of the direction reconstruction
accuracy are compared to experimental data. This analysis will be
extended in \chref{ch:sciencepark}, in which a triangular cluster of three
stations in the Science Park Array will be discussed.
