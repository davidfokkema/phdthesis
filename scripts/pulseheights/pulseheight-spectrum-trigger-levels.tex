% \usepackage{tikz}
% \usetikzlibrary{arrows}
% \usepackage{pgfplots}
% \pgfplotsset{compat=1.3}
% \usepackage[detect-family]{siunitx}
% \usepackage[eulergreek]{sansmath}
% \sisetup{text-sf=\sansmath}
% \usepackage{relsize}
%
\pgfkeysifdefined{/artist/width}
    {\pgfkeysgetvalue{/artist/width}{\defaultwidth}}
    {\def\defaultwidth{ .67\linewidth }}
%
%
\begin{sansmath}
\begin{tikzpicture}[
        font=\sffamily,
        every pin/.style={inner sep=2pt, font={\sffamily\smaller}},
        every label/.style={inner sep=2pt, font={\sffamily\smaller}},
        every pin edge/.style={<-, >=stealth', shorten <=2pt},
        pin distance=2.5ex,
    ]
    \begin{axis}[
            xmode=normal,
            ymode=log,
            width=\defaultwidth,
            axis equal=false,
            %
            title={  },
            %
            xlabel={ Pulseheight [\si{\milli\volt}] },
            ylabel={ Count },
            %
            xmin={ 0 },
            xmax={ 600 },
            ymin={  },
            ymax={  },
            %
            xtick={  },
            ytick={  },
            %
            tick align=outside,
            max space between ticks=40,
            every tick/.style={},
            axis on top,
        ]

        




    
    % Draw series plot
    \addplot[no markers,solid,const plot] coordinates {
        (0.0, 25214)
        (5.7, 586)
        (11.4, 282)
        (17.1, 180)
        (22.8, 128)
        (28.5, 87)
        (34.2, 96)
        (39.9, 65)
        (45.6, 77)
        (51.3, 72)
        (57.0, 69)
        (62.7, 61)
        (68.4, 375)
        (74.1, 685)
        (79.8, 736)
        (85.5, 777)
        (91.2, 892)
        (96.9, 1026)
        (102.6, 1084)
        (108.3, 1236)
        (114.0, 1233)
        (119.7, 1349)
        (125.4, 1261)
        (131.1, 1223)
        (136.8, 1158)
        (142.5, 1134)
        (148.2, 1068)
        (153.9, 948)
        (159.6, 897)
        (165.3, 872)
        (171.0, 713)
        (176.7, 660)
        (182.4, 631)
        (188.1, 535)
        (193.8, 501)
        (199.5, 447)
        (205.2, 440)
        (210.9, 420)
        (216.6, 394)
        (222.3, 388)
        (228.0, 338)
        (233.7, 295)
        (239.4, 282)
        (245.1, 241)
        (250.8, 224)
        (256.5, 232)
        (262.2, 212)
        (267.9, 225)
        (273.6, 212)
        (279.3, 173)
        (285.0, 182)
        (290.7, 169)
        (296.4, 146)
        (302.1, 110)
        (307.8, 157)
        (313.5, 128)
        (319.2, 113)
        (324.9, 100)
        (330.6, 108)
        (336.3, 109)
        (342.0, 83)
        (347.7, 95)
        (353.4, 105)
        (359.1, 78)
        (364.8, 96)
        (370.5, 83)
        (376.2, 82)
        (381.9, 65)
        (387.6, 53)
        (393.3, 74)
        (399.0, 56)
        (404.7, 50)
        (410.4, 58)
        (416.1, 46)
        (421.8, 54)
        (427.5, 49)
        (433.2, 50)
        (438.9, 43)
        (444.6, 33)
        (450.3, 54)
        (456.0, 36)
        (461.7, 29)
        (467.4, 24)
        (473.1, 34)
        (478.8, 32)
        (484.5, 40)
        (490.2, 33)
        (495.9, 26)
        (501.6, 26)
        (507.3, 30)
        (513.0, 19)
        (518.7, 25)
        (524.4, 16)
        (530.1, 28)
        (535.8, 22)
        (541.5, 21)
        (547.2, 14)
        (552.9, 20)
        (558.6, 24)
        (564.3, 18)
        (570.0, 16)
        (575.7, 28)
        (581.4, 15)
        (587.1, 14)
        (592.8, 14)
        (598.5, 14)
    };





    \draw[gray]
        ({rel axis cs:0, 0} -| {axis cs:30, 0 }) --
        ({rel axis cs:1, 1} -| {axis cs:30, 0 });

    \draw[gray]
        ({rel axis cs:0, 0} -| {axis cs:70, 0 }) --
        ({rel axis cs:1, 1} -| {axis cs:70, 0 });



    \node[coordinate,,
          pin={ above:{ \SI{1}{\mip} }}]
        at (axis cs:122.5, 1305.77192982) {};

    \node[coordinate,pin distance=4.5ex,
          pin={ right:{ \SI{30}{\milli\volt} }}]
        at (axis cs:30, 10000.0) {};

    \node[coordinate,,
          pin={ right:{ \SI{70}{\milli\volt} }}]
        at (axis cs:70, 20000.0) {};




    \end{axis}
\end{tikzpicture}
\end{sansmath}