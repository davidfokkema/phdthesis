% \usepackage{tikz}
% \usetikzlibrary{arrows}
% \usepackage{pgfplots}
% \pgfplotsset{compat=1.3}
% \usepackage[detect-family]{siunitx}
% \usepackage[eulergreek]{sansmath}
% \sisetup{text-sf=\sansmath}
% \usepackage{relsize}
%
\pgfkeysifdefined{/artist/width}
    {\pgfkeysgetvalue{/artist/width}{\defaultwidth}}
    {\def\defaultwidth{ .5\linewidth }}
%
%
\begin{sansmath}
\begin{tikzpicture}[
        font=\sffamily,
        every pin/.style={inner sep=2pt, font={\sffamily\smaller}},
        every label/.style={inner sep=2pt, font={\sffamily\smaller}},
        every pin edge/.style={<-, >=stealth', shorten <=2pt},
        pin distance=2.5ex,
    ]
    \begin{semilogyaxis}[
            width=\defaultwidth,
            %
            title={  },
            %
            xlabel={ Atmospheric depth [\si{\gram\per\centi\meter\squared}] },
            ylabel={ Number of particles },
            %
            xmin={  },
            xmax={  },
            ymin={  },
            ymax={  },
            %
            xtick={  },
            ytick={  },
            %
            tick align=outside,
            max space between ticks=40,
            every tick/.style={},
        ]

        

        

        
            
            % Draw series plot
            \addplot[no markers,smooth] coordinates {
                (51.8062, 9251.61)
                (103.611, 99545.5)
                (155.416, 500195.0)
                (207.221, 1419235.0)
                (259.026, 2796906.0)
                (310.831, 4280570.0)
                (362.636, 5438811.0)
                (414.441, 6086355.0)
                (466.246, 6196028.0)
                (518.051, 5946996.0)
                (569.856, 5418979.0)
                (621.661, 4811922.0)
                (673.465, 4148159.0)
                (725.27, 3506044.0)
                (777.075, 2921480.0)
                (828.88, 2405365.0)
                (880.685, 1960832.0)
                (932.49, 1573544.0)
                (984.295, 1238545.0)
            };
        
            
            % Draw series plot
            \addplot[no markers,smooth] coordinates {
                (51.8062, 841.588)
                (103.611, 10758.4)
                (155.416, 51631.1)
                (207.221, 139188.0)
                (259.026, 261561.0)
                (310.831, 383117.0)
                (362.636, 470382.0)
                (414.441, 511453.0)
                (466.246, 515179.0)
                (518.051, 485411.0)
                (569.856, 439551.0)
                (621.661, 385842.0)
                (673.465, 330769.0)
                (725.27, 277578.0)
                (777.075, 230140.0)
                (828.88, 188636.0)
                (880.685, 153176.0)
                (932.49, 121973.0)
                (984.295, 96092.5)
            };
        
            
            % Draw series plot
            \addplot[no markers,smooth] coordinates {
                (51.8062, 163.019)
                (103.611, 827.295)
                (155.416, 2032.32)
                (207.221, 3524.59)
                (259.026, 5188.91)
                (310.831, 6709.14)
                (362.636, 8062.59)
                (414.441, 9088.34)
                (466.246, 9843.41)
                (518.051, 10294.0)
                (569.856, 10499.7)
                (621.661, 10565.8)
                (673.465, 10499.0)
                (725.27, 10400.4)
                (777.075, 10212.0)
                (828.88, 9923.25)
                (880.685, 9590.22)
                (932.49, 9215.53)
                (984.295, 8799.93)
            };
        

        

        

        
            \node[coordinate,
                  label={[] right:{ $\gamma$ }}]
                at (axis cs:984.295, 1238545.0) {};
        
            \node[coordinate,
                  label={[] right:{ e }}]
                at (axis cs:984.295, 96092.5) {};
        
            \node[coordinate,
                  label={[] right:{ $\mu$ }}]
                at (axis cs:984.295, 8799.93) {};
        

        

    \end{semilogyaxis}
\end{tikzpicture}
\end{sansmath}