\documentclass{article}
\usepackage[papersize={17cm,24cm},
%\usepackage[a4paper,layoutsize={17cm,24cm},layoutoffset={2cm,2.85cm},
            total={13cm,21cm},
            showcrop]{geometry}

\usepackage[T1]{fontenc}
\usepackage[utf8]{inputenc}
\usepackage[dutch]{babel}
\usepackage{fouriernc}
\usepackage{microtype}

\usepackage{enumerate}
\usepackage[range-phrase={ tot }]{siunitx}
\usepackage{relsize}
\usepackage{xspace}
\usepackage{setspace}

\newcommand{\hisparc}{\textsmaller{HiSPARC}\xspace}

\pagestyle{empty}
\onehalfspacing


\begin{document}

\noindent
Stellingen behorende bij het proefschrift

\vfill

\begin{center}
\larger
\emph{The \hisparc experiment\\data acquisition and reconstruction of
shower direction}
\end{center}

\vfill

\begin{enumerate}[I]
\item \label{modellen} Numerieke modellen zijn potentieel gevaarlijk: ze
leveren altijd een (verklaarbaar) antwoord, maar per definitie nooit het
juiste.
\item Het onder \ref{modellen} gestelde werd duidelijk na zowel de
ingebruikname van de Erasmusbrug, alswel de recent vastgestelde interactie
tussen de glaswasinstallatie en de vloeren en meubilair van het gebouw
Westraven.
\item Het onbegrip van modelafgeleide observabelen als \emph{regenkans}
ligt wellicht ten grondslag aan klimaatscepsis.
\item In het huidige economische klimaat is het opportuun geld te steken
in nieuwe experimenten die gebruik maken van creatieve toepassingen van
bewezen (en goedkope) technologie.
\item \emph{Physicists generally do not spend their working days
contemplating flowers in a state of cosmic awe. --- Brian Greene}.  Dit
zegt echter niets over de vrijetijdsbesteding van fysici.
\item Als iemand een boek kwijt is, had hij beter op moeten ruimen.  Als
iemand een bestand kwijt is, is het de schuld van de computer.
\item De ervaring van veel mensen dat een computer een moeilijk te
gebruiken apparaat is en `het gewoon niet doet' heeft als oorzaak dat de
meeste gereedschappen bij onkundig gebruik toch redelijk functioneren.
Men kan stellen dat het resultaat als functie van de afwijking van het
juiste gebruik bij benadering een normale verdeling vertoont.  Voor
computers geldt echter een delta-functie.  Het verdient aanbeveling het
geleerde uit vakgebieden als \emph{natural language processing} toe te
passen op het \emph{gebruik} van een computer.
\item De zogenaamde \emph{arrow of time} voorkomt dat de Aardse atmosfeer
de \emph{bron} is van kosmische straling, hoewel de energie ruim voor
handen is.
\item Met zeer beperkte middelen -- drie \SI{0.5}{\square\meter}
scintillatoren met een onderlinge afstand van \SI{10}{\meter}
(\SI{150}{\meter}) -- is het mogelijk de richting van kosmische straling
te bepalen met een onzekerheid van $\approx \SI{8.6}{\degree}$
(\SI{1.5}{\degree}).
\item In de resultaten van een \hisparc detector zijn duidelijk de twee
verschillende bijdragen van geladen deeltjes en $\gamma$'s te
onderscheiden.
\end{enumerate}

\end{document}
