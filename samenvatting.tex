\chapter{Samenvatting}
\markboth{Samenvatting}{}

\begin{otherlanguage}{dutch}

De ontdekking van kosmische straling wordt algemeen toegeschreven aan
Victor Francis Hess, die in 1911 en 1912 een serie ballonvluchten ondernam
om voor eens en altijd vast te stellen of een mysterieuze en overal
aanwezige vorm van straling \emph{vanuit} de Aarde, of van \emph{buiten}
de Aarde afkomstig was.  Tijdens deze vluchten, tot een hoogte van ruim 5
kilometer, stelde hij vast dat de straling sterker werd naarmate je hoger
in de atmosfeer kwam.  Het bleek dat de straling daadwerkelijk vanuit de
ruimte kwam en daarvoor ontving hij in 1936 de Nobelprijs.  Het was echter
Robert Andrews Millikan die de term \emph{kosmische straling} bedacht.
Een term die vandaag de dag nog gangbaar is, ondanks het feit dat het een
onjuiste benaming betreft.  Wat we aanduiden met kosmische straling zijn
feitelijk deeltjes.

De aard van de deeltjes is vergelijkbaar met de aard van de elementen
waaruit ons zonnestelsel is opgebouwd (Figuur \ref{fig:composition} op
pagina \pageref{fig:composition}).  Er zijn verschillen die verklaard
kunnen worden door aan te nemen dat de deeltjes een lange weg hebben
afgelegd voordat ze de Aarde bereiken.  De deeltjes in de kosmische
straling hebben echter een enorme hoeveelheid bewegingsenergie.  Het
betreft hier dus voornamelijk `gewone' deeltjes, die op de één of andere
manier versneld zijn door natuurlijke deeltjesversnellers.

De bekendste deeltjesversneller is wellicht de \emph{Large Hadron Collider
(LHC)} bij Genève.  Deze cirkelvormige versneller is 27 kilometer lang en
versnelt deeltjes door middel van sterke elektrische velden en houdt ze in
een cirkelvormige baan door middel van sterke magnetische velden.  Dit
geeft de LHC de mogelijkheid de deeltjes rondje na rondje, en beetje bij
beetje, meer energie te geven, waarna de energie constant blijft bij een
waarde van \SI{7}{\tera\electronvolt}.  De LHC stelt fysici in staat om
door middel van botsingen het gedrag van materie te bestuderen bij extreem
hoge energieën, zoals die ook kort na de oerknal bestaan moeten hebben.
En, niet te vergeten, nog dagelijks vóórkomen búiten het laboratorium, een
kilometer of twintig boven ons hoofd.  Maar dan duizendmaal sterker,
wanneer kosmische straling onze atmosfeer binnendringt en botst op
luchtmoleculen.

Het is op zich niet verwonderlijk dat de natuurlijke deeltjesversnellers
veel hogere energieën bereiken dan mogelijk zijn met de LHC.  De LHC is
slechts~(!) 27 kilometer lang.  Elektrische en magnetische velden komen
ook voor in het heelal, maar dan op schalen van miljoenen of zelfs
miljarden kilometers.  In veel gevallen zijn deze velden veel minder sterk
dan we kunnen opwekken in laboratoria, maar de grootte maakt dat meer dan
goed.  Elektrisch geladen deeltjes kunnen lange tijd grillige, maar
min-of-meer cirkelvormige, banen beschrijven en langzaam, beetje bij
beetje, versneld worden tot ongekende energieën.  Mogelijke kandidaten
zijn de uitgestrekte resten van sterren die in de vorm van een supernova
aan hun eind zijn gekomen.  Zeer waarschijnlijk zijn er plaatsen in het
heelal die niet alleen veel groter zijn dan de LHC, maar ook veel sterkere
elektromagnetische velden bezitten: pulsars en zwarte gaten, bijvoorbeeld.

Wanneer een deeltje uit de kosmische straling de Aarde op haar pad vindt,
vindt er onherroepelijk een botsing plaats tussen het deeltje en moleculen
uit de atmosfeer.  Bij deze botsing wordt er, volgens $E = mc^2$, een deel
van de bewegingsenergie omgezet in massa, in nieuwe deeltjes.  Deze
deeltjes hebben nog steeds erg veel energie, en botsen weer.  En weer.  Er
ontstaat zo een soort lawine (\emph{air shower}) waarbij het aantal
deeltjes in eerste instantie blijft toenemen.  Wanneer de energie per
deeltje te laag wordt om nieuwe deeltjes te maken, kan het aantal deeltjes
niet langer toenemen.  De deeltjes in de lawine verliezen langzaamaan hun
energie en worden geabsorbeerd in de atmosfeer, of vervallen tot andere
deeltjes.  Bij hoog-energetische kosmische straling is het aantal deeltjes
echter zo groot dat een aanzienlijk deel de grond bereikt, over een
oppervlak dat wel enkele vierkante kilometers groot kan zijn.  Het
\hisparc experiment bestudeert de resten van air showers zodra de deeltjes
hiervan de grond bereiken.

VWO-leerlingen bouwen, onder leiding van onderzoekers, detectoren
bestaande uit een scintillator en een fotobuis en installeren deze in een
skibox op het dak van hun school.  De scintillator bestaat uit een
materiaal dat zwak oplicht zodra elektrisch geladen deeltjes (of ook
hoog-energetische fotonen) uit de air shower door de plaat gaan.  De
fotobuis kan dit zwakke lichtsignaal versterken tot een meetbaar
elektrisch signaal, dat door middel van een kabel naar een meetcomputer
wordt gebracht.  Iedere school beschikt over twee of vier detectoren en
wanneer meerdere detectoren tegelijkertijd deeltjes waarnemen, is de kans
groot dat dit een air shower betreft.  De metingen worden dan via internet
verstuurd naar het Nikhef.  Leerlingen en onderzoekers kunnen vervolgens
analyseren of de vermoedelijke air shower tegelijkertijd is geobserveerd
door meerdere naburige scholen.  Voor de benodigde precieze tijdmetingen
wordt gebruik gemaakt van het GPS-systeem.  GPS is vooral bekend als
navigatiemiddel, maar naast positie geeft het systeem juist ook een heel
nauwkeurige tijd.

Dit proefschrift geeft in de eerste hoofdstukken een beschrijving van
kosmische straling (Hoofdstuk 1) en van het \hisparc experiment (Hoofdstuk
2).  Vervolgens richt het zich op de software die de gegevens opstuurt en
beheert, en op de software die de `gezondheid' van het \hisparc netwerk in
de gaten houdt (Hoofdstuk 3).  Ook wordt daar kort gesproken over
\sapphire, een zogeheten \emph{software framework}.  Het stelt
onderzoekers en leerlingen in staat om eenvoudig simulaties en analyses te
ontwikkelen binnen een uniforme opzet.  Alle software binnen \hisparc
wordt op een open en voor iedereen toegankelijke manier ontwikkeld.

Dit proefschrift richt zich verder op de reconstructie van de richting, de
herkomst, van een kosmisch deeltje dat een air shower veroorzaakt.  In
Hoofdstuk 4 wordt de mogelijkheid onderzocht dat één enkel vier-plaats
station de herkomstrichting van een air shower bepaalt.  De deeltjes in
een air shower reizen als het ware in een `platte pannenkoek', het
\emph{shower front}, naar de grond.  Dat betekent dat een shower recht van
boven de verschillende detectoren in een station vrijwel gelijktijdig
raakt.  Komt de shower echter onder een hoek binnen, dan zullen er
tijdverschillen optreden in de metingen.  Naast vergelijkingen die uit de
tijdverschillen de hoek reconstrueren worden er ook vergelijkingen
afgeleid die iets zeggen over de \emph{nauwkeurigheid}\footnote{Het is
natuurkundig beter om te spreken over de \emph{onzekerheid} van de
resultaten.} waarmee dat kan.  Deze vergelijkingen worden getoetst aan een
simulatie van air showers die een \hisparc station raken.  Het blijkt dat
het mogelijk moet zijn om met slechts één station iets zinnigs te zeggen
over de herkomst van een kosmisch deeltje.  Sterker, de resultaten worden
ondersteund door de vergelijkingen die de nauwkeurigheden beschrijven, wat
aantoont dat we het resultaat grotendeels begrijpen.  Protonen met een
energie van \SI{1}{\peta\electronvolt} die binnenkomen onder een hoek van
\SI{22.5}{\degree} genereren showers die een detectiekans hebben van meer
dan \SI{50}{\percent} tot een afstand van \SI{30}{\meter}.  Worden deze
showers gedetecteerd, dan geeft de reconstructie een fysisch resultaat in
meer dan \SI{95}{\percent} van de gevallen.

De dikte van het shower front introduceert onzekerheden in de tijdmeting.
Dit leidt tot onnauwkeurigheden in de reconstructie.  Aangezien de
deeltjes met bijna de lichtsnelheid reizen en niets sneller kan dan het
licht, worden de onzekerheden alléén veroorzaakt door `trage' deeltjes.
Wanneer meerdere deeltjes een detector doorkruisen wordt de onzekerheid
echter kleiner.  Dit is puur statistiek: neem het \emph{eerste} deeltje
dat de detector doorkruist en de tijdmeting zit per definitie dichter op
de werkelijke waarde.  Dit effect vlakt echter af bij meer dan twee
deeltjes per detector.  De tijdsonzekerheid voor 2 deeltjes is door
simulatie bepaalt op \SI{1.4}{\nano\second}.  Een tweede meetonzekerheid,
veroorzaakt door de tijd die fotonen nodig hebben om vanuit de detector
naar de fotobuis te reizen is door simulatie bepaalt op
\SI{1.2}{\nano\second}.  De totale onzekerheid bedraagt dan
\SI{1.8}{\nano\second}.  Voor \SI{1}{\peta\electronvolt} proton showers
die binnenkomen onder een hoek van \SI{22.5}{\degree} en in alle
hoekdetectoren 2 of meer deeltjes deponeren is de onzekerheid in de zenith
hoek $\sigma_\theta = \SI{4.3}{\degree}$ en in de azimuthale hoek
$\sigma_\phi = \SI{11}{\degree}$.  De onnauwkeurigheid in de azimuthale
hoek is getalmatig groter, maar dit komt alleen door de keuze van het
coördinatensysteem.  De werkelijke hoekafstand is kleiner en de totale
onzekerheid (de hoekafstand tussen de werkelijke en de gereconstrueerde
richting) komt uit op \SI{5.9}{\degree}.

In Hoofdstuk 5 worden de methodes uit Hoofdstuk 4 getoetst aan een
experiment.  Eén \hisparc station is geïnstalleerd binnen de \kascade
detector in Karlsruhe, Duitsland.  Deze detector bestaat uit 252
afzonderlijke subdetectoren die ieder vergelijkbaar zijn met een \hisparc
detector.  Dit veel grotere (en duurdere) experiment verzorgt voor ons een
onafhankelijke meting van de richting van iedere air shower die zowel door
\kascade als door \hisparc waargenomen wordt.  De resultaten worden
besproken en ook hier blijkt, na correctie voor experimentele onzekerheden
die niet voorkomen in de simulatie, dat we de resultaten goed begrijpen.
De totale onzekerheid in de tijdmeting wordt beschreven door:
\begin{equation*}
\sigma_t = \sqrt{\sigma_{t,\, \mathrm{front}}^2 + \sigma_{t,\,
\mathrm{transport}}^2 + \sigma_{t,\, \mathrm{sampling}}^2 + \sigma_{t,\,
\mathrm{other}}^2}\,,
\end{equation*}
waarin $\sigma_{t,\, \mathrm{front}} = \SI{1.4}{\nano\second}$ veroorzaakt
wordt door de dikte van het shower front, zoals eerder beschreven,
$\sigma_{t,\, \mathrm{transport}} = \SI{1.2}{\nano\second}$ een
beschrijving geeft van de fotonen die onderweg zijn naar de fotobuis,
$\sigma_{t,\, \mathrm{sampling}} =
\frac{\SI{2.5}{\nano\second}}{\sqrt{12}}$ de onzekerheid veroorzaakt door
de digitalisering van het signaal weergeeft en $\sigma_{t,\,
\mathrm{other}} = \SI{1.6}{\nano\second}$ een onbekende experimentele
bijdrage beschrijft.  Deze laatste onzekerheid is bepaald uit de data.  De
totale onzekerheid komt dan uit op \SI{2.4}{\nano\second}.  Dit is iets
meer dan de waarde die verklaard werd vanuit de simulatie, en de
experimentele resolutie komt dan uit op een hoekafstand van
\SI{8.6}{\degree} voor \SI{1}{\peta\electronvolt} proton showers onder een
hoek van \SI{22.5}{\degree}, met $N_\mathrm{MIP} \geq 2$.

In Hoofdstuk 6 wordt de methode uitgebreid naar een combinatie van
stations in het \emph{Science Park Array} in Amsterdam.  In theorie moet
een netwerk met grotere afstanden en meer detectoren een beter resultaat
geven voor de reconstructie van de richting van een air shower.  Een
complicatie is nu dat ieder station zijn eigen GPS-tijd moet meten en dat
daar een onnauwkeurigheid in zit.  Omdat we een onafhankelijke tweede
meting missen (er is geen `\kascade' experiment op het Science Park) wordt
er eerst een aantal controles uitgevoerd tussen de stations onderling.  We
begrijpen immers wél de nauwkeurigheid van individuele stations!  De
resultaten komen overeen met de verwachtingen.  Vervolgens worden de
richtingen gereconstrueerd door een cluster van drie stations vergeleken
met de richtingen gereconstrueerd door een enkel station.  Hoewel de
resultaten grotendeels voldoen aan de verwachtingen en een optimistisch
beeld geven van de nauwkeurigheid van de reconstructies, is er een aantal
\emph{systematische onzekerheden} die verder moeten worden onderzocht.
Door de onnauwkeurigheid in de tijdsmeting (GPS) komt de totale
tijdsonzekerheid uit op \SI{5.5}{\nano\second}.  De grotere afstanden
tussen de stations (\SIrange[range-phrase={ tot }]{122}{151}{\meter})
maken deze waarde meer dan goed.  Het gevonden resultaat betekent dat we,
weer voor \SI{1}{\peta\electronvolt} proton showers onder
\SI{22.5}{\degree} met $N_\mathrm{MIP} \geq 2$, de richting van een air
shower kunnen reconstrueren binnen een cirkel met een straal van
\SI{1.5}{\degree} aan de hemelboog.  Dat is slechts een paar keer de
grootte van de maan!  Dit is een mooi resultaat, zeker met de beperkte
kosten van \hisparc stations in het achterhoofd.  Aangezien het Science
Park Array bestaat uit méér dan drie stations, kan dit resultaat nog
worden verbeterd.

Het laatste hoofdstuk, Hoofdstuk 7, vormt de conclusies van dit
proefschrift.  Tevens wordt daarin kort vooruit geblikt naar de
mogelijkheid de energie van het oorspronkelijke kosmische deeltje te
bepalen.  Dit blijkt echter bijzonder moeilijk wanneer slechts een beperkt
aantal detectoren voorhanden is.

Het doel van het \hisparc experiment is tweeledig.  Enerzijds is het een
`traditionele' onderzoeksgroep bestaande uit stafleden, promovendi en
studenten, dat onderzoek doet naar verschillende facetten van kosmische
straling.  Anderzijds is het experiment ook juist bedoeld om leerlingen
uit het voortgezet onderwijs deel te laten nemen aan wetenschappelijk
onderzoek.  Jaarlijks komt een groot aantal scholen bij elkaar voor het
landelijke \hisparc symposium, waar onderzoekers en leerlingen hun werk
presenteren.

\end{otherlanguage}
